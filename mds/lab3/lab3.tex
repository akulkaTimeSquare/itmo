\documentclass[a4paper,hidelinks,14pt]{extarticle}

\usepackage[utf8]{inputenc}
\usepackage[T2A]{fontenc}
\usepackage[english, russian]{babel}
\usepackage{lipsum}
\usepackage{amsmath}
\usepackage{amssymb}
\usepackage{amsfonts}
\usepackage{mathtools}
\usepackage{datetime}
\usepackage[pdftex]{graphicx}
\usepackage{indentfirst}
\usepackage{asymptote}
\usepackage{systeme}
\usepackage[dvipsnames]{xcolor}
\usepackage{lastpage}
\usepackage{fancybox,fancyhdr}
\usepackage{hyperref}
\usepackage[font={small,it}]{caption}

\usepackage{minted, xcolor}
\usemintedstyle{monokai}
\definecolor{bg}{HTML}{282828}
\setminted{bgcolor=bg}
\setminted{fontsize=\scriptsize}
\setminted{}

\fancyhead[R]{\textit{Бифуракции в системе}}
\fancyhead[L]{Практическая работа №3} 
\fancyfoot[C]{Страница \thepage\space из \pageref{LastPage}}
\fancyfoot[R]{}
\fancyfoot[L]{}
\pagestyle{fancy}
\setlength{\headheight}{17.0pt}

\newcommand{\anonsection}[1]{\section*{#1}\addcontentsline{toc}{section}{#1}}

\begin{document}
	\begin{titlepage}
		\setlength{\parindent}{0ex}
		
		\begin{center}
			\textsc{
				\vspace{1ex}
				Научно исследовательский университет ИТМО \\
				\vspace{0.5ex}
				Факультет систем управления и робототехники \\
				\vspace{0.5ex}
			}
		\end{center}
		
		\vspace{50mm}
		
		\begin{center}
			Отчет\\
            по дисциплине\\
            <<Моделирование динамических систем>>\\
            \vspace{2mm}
            Практическая работа №3 \\
            \vspace{2mm}
            Вариант №3
		\end{center}
		
		\vspace{28mm}
		
		\begin{minipage}{.29\linewidth}
			Выполнили
			
			Преподаватель
		\end{minipage}
		\hfill
		\begin{minipage}{.70\linewidth}
			\begin{flushright}
				Воротников А.А., Гридусов Д.Д., Мовчан И.Е.
				\\
				Семенов Д.М.
			\end{flushright}
		\end{minipage}
		
		\vfill
		\begin{center}
			Санкт-Петербург
			\\
			2024
		\end{center}
	
	\end{titlepage}

    \tableofcontents
	\newpage
	\section{Задание}
    Пусть дана нелинейная система:
    \begin{equation*}
        \begin{cases*}
            \dot{x}_1 = -x_1, \\
            \dot{x}_2 = rx_2 - x_2^3 + x_2^5.
        \end{cases*}
    \end{equation*}

	Необходимо
    \begin{itemize}
        \item Найти возможные бифуракции в системе, положения равновесия, определить их тип в зависимости от параметра $r$.
        \item Построить фазовые портреты при различных значениях этого параметра.
    \end{itemize}

	\newpage
	\section{Решение}

	Найдём положения равновесия, решив систему на равенство нулю производных:

	\begin{equation*}
		\begin{cases*}
			-x_1 = 0, \\
			rx_2 - x_2^3 + x_2^5 = 0
		\end{cases*}
		\Rightarrow
		\begin{cases*}
			x_1 = 0, \\
			x_2(r - x_2^2 + x_2^4) = 0,
		\end{cases*}
	\end{equation*}
	откуда положения равновесия:

	\begin{equation}
		\label{1}
		\begin{cases}
			x_1^* = 0,\\
			x_2^* = 0;
		\end{cases}
	\end{equation}
	\begin{equation}
		\label{2}
		\begin{cases}
			x_1^* = 0,\\
			x_2^* = \pm\sqrt{\frac{1+\sqrt{1-4r}}{2}};
		\end{cases}
	\end{equation}
	\begin{equation}
		\label{3}
		\begin{cases}
			x_1^* = 0,\\
			x_2^* = \pm\sqrt{\frac{1-\sqrt{1-4r}}{2}}.
		\end{cases}
	\end{equation}

	Первое (\ref{1}) положение существует при любых параметрах $r \in \mathbb{R}$, второе и третье (\ref{2}) - при $r \le \frac{1}{4}$, четвертое и пятое (\ref{3}) - при $0 \le r \le \frac{1}{4}$.
	
	Исследуем на устойчивость (линеаризовав в окрестности положения равновесия), используя:
	$$
	\dot{x} = Ax, \text{ где  } A = \begin{bmatrix}
		-1 & 0 \\
		0 & r-3x_2^2+5x_2^4
	\end{bmatrix} \Bigg|_{x=x^*}.
	$$

	Видим, что устойчивость пар точек \ref{2} одинакова (для \ref{3} то же самое), поэтому дальше будем исследовать поведение только одного элемента пары.

	Подставим в матрицу $A$ точки, найдём собственные числа. Начнём с \ref{1}:
	$$
	A = \begin{bmatrix}
		-1 & 0 \\
		0 & r-3x_2^2+5x_2^4
	\end{bmatrix} \Bigg|_{x=x^*}
	=
	\begin{bmatrix}
		-1 & 0 \\
		0 & r
	\end{bmatrix}
	\Rightarrow
	\left[
		\begin{array}{l}
			\lambda = -1, \\
			\lambda = r. \\
		\end{array}
	\right.
	$$
	
	При $r > 0$ положение равновесия является седлом, при $r < 0$ - устойчивым узлом.

	Рассмотрим теперь точку из пары \ref{2}:
	$$
	A
	=
	\begin{bmatrix}
		-1 & 0 \\
		0 & 5r-5x_2^2+5x_2^4 - 4r + 2x_2^2
	\end{bmatrix}
	=
	\begin{bmatrix}
		-1 & 0 \\
		0 & - 4r + 2x_2^2
	\end{bmatrix},
	$$
	так как точки были найдены из решения уравнения $r - x_2^2 + x_2^4 = 0$.
	
	Получаем:
	$$
	\left[
		\begin{array}{l}
			\lambda = -1, \\
			\lambda = - 4r + 1 + \sqrt{1 - 4r} = \sqrt{1 - 4r}(\sqrt{1-4r} + 1) > 0.
		\end{array}
	\right.
	$$

	Отсюда положения равновесия из пары \ref{2} являются седлом.

	Аналогичные рассуждения над матрицей $A$ дают собственные числа матрицы $A$ для положения равновесия \ref{3}:
	$$
	\left[
		\begin{array}{l}
			\lambda = -1, \\
			\lambda = -4r + 1 - \sqrt{1-4r} = \sqrt{1-4r}(\sqrt{1-4r} - 1) < 0.
		\end{array}
	\right.
	$$

	Итак, положение равновесия является устойчивым узлом. Отметим также, что здесь и в прошлом случае мы делали выводы о $\lambda$ из условий на существование положений равновесия.

	Проведём дополнительное исследование, построив фазовый портрет системы для различных значений параметра $r$. Результаты на рис. \ref{4}, \ref{5} и \ref{6}. Как можно видеть, численные результаты дают те же типы положений, что и были найдены.
	\begin{figure}[h]
		\begin{center}
			\includegraphics[scale=0.65]{images/1.png}
			\caption{Фазовый портрет системы при $r \in (-\infty, 0)$}
			\label{4}
		\end{center}
	\end{figure}
	\begin{figure}[h]
		\begin{center}
			\includegraphics[scale=0.65]{images/2.png}
			\caption{Фазовый портрет системы при $r \in (0, \frac{1}{4})$}
			\label{5}
		\end{center}
	\end{figure}
	\begin{figure}[h]
		\begin{center}
			\includegraphics[scale=0.65]{images/3.png}
			\caption{Фазовый портрет системы при $r \in (\frac{1}{4}, +\infty)$}
			\label{6}
		\end{center}
	\end{figure}
	\section{Выводы}
	В результате выполнения лабораторной работы было исследовано влияние бифуракционных параметров на свойства системы (её положения равновесия, а также их типы), а также построены фазовые портреты для различных значений этих параметров для численных проверок результатов.

\end{document}