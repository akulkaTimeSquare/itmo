\documentclass[a4paper,hidelinks,14pt]{extarticle}

\usepackage[utf8]{inputenc}
\usepackage[T2A]{fontenc}
\usepackage[english, russian]{babel}
\usepackage{lipsum}
\usepackage{amsmath}
\usepackage{amssymb}
\usepackage{amsfonts}
\usepackage{mathtools}
\usepackage{datetime}
\usepackage[pdftex]{graphicx}
\usepackage{indentfirst}
\usepackage{asymptote}
\usepackage{systeme}
\usepackage[dvipsnames]{xcolor}
\usepackage{lastpage}
\usepackage{fancybox,fancyhdr}
\usepackage{hyperref}
\usepackage[font={small,it}]{caption}

\usepackage{minted, xcolor}
\usemintedstyle{monokai}
\definecolor{bg}{HTML}{282828}
\setminted{bgcolor=bg}
\setminted{fontsize=\scriptsize}
\setminted{}

\fancyhead[R]{\textit{Системы с задержками}}
\fancyhead[L]{Практическая работа №5}
\fancyfoot[C]{Страница \thepage\space из \pageref{LastPage}} 
\fancyfoot[R]{}
\fancyfoot[L]{}
\pagestyle{fancy}
\setlength{\headheight}{17.0pt}

\newcommand{\anonsection}[1]{\section*{#1}\addcontentsline{toc}{section}{#1}}

\begin{document}
	\begin{titlepage}
		\setlength{\parindent}{0ex}
		
		\begin{center}
			\textsc{
				\vspace{1ex}
				Научно исследовательский университет ИТМО \\
				\vspace{0.5ex}
				Факультет систем управления и робототехники \\
				\vspace{0.5ex}
			}
		\end{center}
		
		\vspace{50mm}
		
		\begin{center}
			Отчет\\
            по дисциплине\\
            <<Моделирование динамических систем>>\\
            \vspace{2mm}
            Практическая работа №5 \\
            \vspace{2mm}
            Вариант №3
		\end{center}
		
		\vspace{28mm}
		
		\begin{minipage}{.29\linewidth}
			Выполнили
			
			Преподаватель
		\end{minipage}
		\hfill
		\begin{minipage}{.70\linewidth}
			\begin{flushright}
				Воротников А.А., Гридусов Д.Д., Мовчан И.Е.
				\\
				Семенов Д.М.
			\end{flushright}
		\end{minipage}
		
		\vfill
		\begin{center}
			Санкт-Петербург
			\\
			2024
		\end{center}
	
	\end{titlepage}

    \tableofcontents
	\newpage
	\section{Задание 1}
    
    \subsection{Условие}
    Пусть задана система с задержкой:
    $$
    \dot{x}(t) = -\text{sign} (x (t-h)), \hspace{3mm} t \ge 0,\ h = 2 > 0,
    $$
    $h$ - постоянная задержка, $x(t) = \varphi (t)$ на $[-h, 0]$,
    $$
    \varphi(t) =
    \begin{cases*}
        0.5, \hspace{15mm} t\in[-2, -1),\\
        -t-0.5, \hspace{3mm} t\in[-1, 0].   
    \end{cases*}
    $$

    Необходимо построить решение системы методом шагов.
    \subsection{Решение}

    Так как производная $\dot{x}$ как бы запаздывает в своих значениях, мы можем определить значения исходной функции $x$ по ней, используя предыдущие. Итак, пусть $t \in [0, 2]$, из начальных условий знаем $x(0) = \varphi(0) = -0.5$, тогда
    \begin{equation*}
        \begin{cases*}
            \dot{x}(t) = -1,\ \varphi (t - 2) > 0, \\
            \dot{x}(t) = 1,\hspace{5mm} \varphi (t - 2) < 0
        \end{cases*}
        \Rightarrow
        \begin{cases*}
            \dot{x}(t) = -1,\ t\in [0, 1.5), \\
            \dot{x}(t) = 1,\hspace{5mm}t\in [1.5, 2].
        \end{cases*}
    \end{equation*}
    
    Теперь если $\dot{x}(t) = -1$ на $[0, 1.5)$, то $x(t) = -t + C$, где $-0.5 = 0 + C \Rightarrow C = -0.5$, и $x(1.5) = -1.5 - 0.5 = -2$.

    А для $[1.5, 2)\text{: }\dot{x}(t) = -1$, откуда $x(t) = t + C$, где $-2 = 1.5 + C \Rightarrow C = -3.5$, и $x(2) = 2 - 3.5 = -1.5$.
	
    Аналогично можем распространить решение и на отрезок $[2, 4]$. В данном случае $\dot{x} = 1$ на всём $[2, 4]$, так как на предыдущем шаге получили отрицательную функцию на промежутке $(0, 2]$, но тогда $x(t) = t + C, \text{ где } 2+C = -1.5 \Rightarrow C = -3.5.$
    
    А также на отрезок $[4, 6]\text{: } x(t) = t - 3.5$ на $[4, 5.5)$, $x(t) = -t + 7.5$ на $[5.5, 6]$.
    
    График системы выходит следующим:
    \begin{figure}[h]
		\begin{center}
			\includegraphics[scale=0.75]{images/1.png}
            \caption{График системы при $h = 2$}
            \label{1}
		\end{center}
	\end{figure}
    
    \section{Задание 2}
    \subsection{Условие}
    Пусть дана система с произвольной задержкой $\tau (t):$

    $$
    \dot{x} = -2x(t) - 0.1x(t-\tau(t))
    $$
    
    Необходимо построить функцию Ляпунова и с помощью метода Разумихина доказать устойчивость данной системы.

    \subsection{Решение}
    
    Функция Ляпунова задаётся выражением
    $$
    V(x) = x^2.
    $$

    Её производная тогда
    $$
    V'(x) = 2x\dot{x} = 2x(-2x+0.1x(t - \tau (t))) = -4x^2 + 0.2x(t)x(t - \tau (t)).
    $$

    Для устойчивости системы необходимо, чтобы $\dot{V}(x(t)) \le 0$ для всех $x(t+\theta)\text{ }(\theta \in [-h, 0])$, удовлетворяющих $V(x(t+\theta)) <= V(x(t))$. По методу же Разумихина данное выполнено, если
    $$
    \Psi = \begin{bmatrix}
        A^T P + PA + qP & P A_1 \\
        A_1^T P & -q P
    \end{bmatrix} < 0, \text{  } q > 0,\text{ }P > 0
    $$

    В нашем случае $A = -2, A_1 = -0.1$, а это значит, что
    $$
    \Psi = \begin{bmatrix}
        -4P + qP & -0.1 P \\
        -0.1 P & -q P
    \end{bmatrix} < 0, \text{  } q > 0,\text{ }P > 0
    $$

    Решим с помощью критерия Сильвестра:
    \begin{equation*}
        \begin{cases*}
            -4P + qP = -P(4-q) < 0, \\
            -P^2((-4+q)q + 0.01) > 0, \\
            q > 0, \\
            P > 0.
        \end{cases*}
    \end{equation*}

    Откуда получаем, что система разрешима при $q \in (0.0025, 3.9975)$, а значит, она асимптотически устойчива.

    \section{Задание 3}
    \subsection{Условие}
    Пусть дана система с постоянной задержкой $h=3$:
    $$
    \dot{x} = Ax(t)+A_1x(t-h), \hspace{1mm} A = \begin{bmatrix}
        -4 & 1 \\
        -2 & -4
    \end{bmatrix}, \hspace{1mm} A_1 = \begin{bmatrix}
        1 & 2 \\
        1 & -1
    \end{bmatrix}
    $$

    Необходимо построить графики системы (смоделировать её) и доказать её устойчивость.
    \subsection{Решение}
    По методу функционалов Ляпунова-Красовского, система устойчива, если разрешимо
    $$
    \Psi =
    \begin{bmatrix}
        A^T P + PA + Q & P A_1 \\
        A_1^T P & -(1-\dot{\tau}(t))Q
    \end{bmatrix} =
    \begin{bmatrix}
        A^T P + PA + Q & P A_1 \\
        A_1^T P & -Q
    \end{bmatrix} < 0
    $$
    для $P > 0, \hspace{1mm} Q > 0$.

    Решение в MATLAB даёт
    $$
    P = \begin{bmatrix}
        0.3183 &   -0.0205 \\
        -0.0205 &   0.3154        
    \end{bmatrix}, \hspace{1mm}
    Q = \begin{bmatrix}
        1.0558 &   0.1102 \\
        0.1102 &   1.2894.
    \end{bmatrix}
    $$

    Неравенство разрешилось, а значит, система устойчива. Посмотрим на графиках:
    \begin{figure}[h]
		\begin{center}
			\includegraphics[scale=0.5]{images/2.png}
			\caption{Моделирование системы}
			\label{2}
		\end{center}
	\end{figure}

    \section{Выводы}
	По результатам выполнения лабораторной работы был изучен способ построения графика решения системы с постоянной задержкой $h$ (метод шагов), исследованы способы доказать устойчивость системы с задержкой (методы Лянуова-Красовского и Разумихина, использующие матричные неравенства для достижения результатов).


\end{document}