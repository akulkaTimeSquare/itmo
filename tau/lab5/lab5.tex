\documentclass[a4paper,hidelinks,14pt]{extarticle}

\usepackage[utf8]{inputenc}
\usepackage[T2A]{fontenc}
\usepackage[english, russian]{babel}
\usepackage{lipsum}
\usepackage{amsmath}
\usepackage{amssymb}
\usepackage{amsfonts}
\usepackage{mathtools}
\usepackage{datetime}
\usepackage[pdftex]{graphicx}
\usepackage{indentfirst}
\usepackage{asymptote}
\usepackage{systeme}
\usepackage[dvipsnames]{xcolor}
\usepackage{lastpage}
\usepackage{fancybox,fancyhdr}
\usepackage{hyperref}
\usepackage[font={small,it}]{caption}
\usepackage{titlesec}
\titleformat{\section}
  {\normalfont\large\bfseries}
  {\thesection}{1em}{}
\fancyhead[L]{Лабораторная работа A}
\fancyhead[C]{}
\fancyhead[R]{\textit{Линейно-квадратичные радости}}
\fancyfoot[L]{}
\fancyfoot[C]{Страница \thepage\space из \pageref{LastPage}}
\fancyfoot[R]{}
\pagestyle{fancy}
\newcommand{\gt}{\textgreater}
\newcommand{\lt}{\textless}
\let\oldemptyset\emptyset
\let\emptyset\varnothing

\begin{document}
	\begin{titlepage}
		\setlength{\parindent}{0ex}
		
		\begin{center}
			\textsc{
				\vspace{1ex}
				Научно исследовательский университет ИТМО \\
				\vspace{0.5ex}
				Факультет систем управления и робототехники \\
				\vspace{0.5ex}
			}
		\end{center}
		
		\vspace{50mm}
		
		\begin{center}
			Отчет по лабораторной работе A \\
			Линейно-квадратичные радости \\
            Вариант 11
		\end{center}
		
		\vspace{50mm}
		
		\begin{minipage}{.48\linewidth}
			Выполнил студент группы R3380
			
			Преподаватель
		\end{minipage}
		\hfill
		\begin{minipage}{.5\linewidth}
			\begin{flushright}
				Мовчан Игорь Евгеньевич
				\\
				Пашенко Артем Витальевич
			\end{flushright}
		\end{minipage}
		
		\vfill
		\begin{center}
			Санкт-Петербург
			\\
			2025
		\end{center}
		
	\end{titlepage}

	\tableofcontents
	\clearpage
	
	\section{Исследование LQR}
    Рассмотрим линейную систему
    \[
        \dot{x} = Ax + Bu, \quad x_0 = x(0) = \begin{bmatrix}
            1 & 1 & 1
        \end{bmatrix}^T
    \]

    В соответствии с вариантом, матрицы $A$ и $B$ имеют вид:
    \[
        A = \begin{bmatrix}
            11 & -2 & 13 \\
            6 & -1 & 6 \\
            -6 & -1 & -8
        \end{bmatrix}, \quad
        B = \begin{bmatrix}
            2 \\
            0 \\
            0
        \end{bmatrix}
    \]

	Проверим система на стабилизируемость. Для этого перейдем к Жордановой форме системы с матрицами $\hat{A}$, имеющей собственными числами $\lambda_1 = 1$, $\lambda_2 = 2$ и $\lambda_3 = 3$, $\hat{B}$ и матрицей $T$ перехода:
	\[
		\hat{A} = T^{-1}AT = \begin{bmatrix}
            -2 & 0 & 0 \\
            0 & 2 & 3 \\
            0 & -3 & 2
        \end{bmatrix}, \quad
    T = \begin{bmatrix}
        -1 & -1.5 & -0.5 \\
        0 & -1 & 0 \\
        1 & 1 & 0
    \end{bmatrix}
    \]
    
    Можно получить обратную матрицу к $T$:
    \[
        T^{-1} = \begin{bmatrix}
            0 & 1 & 1 \\
            0 & -1 & 0 \\
            -2 & 1 & -2
        \end{bmatrix}
    \]

    Откуда:
    \[
        \hat{B} = T^{-1}B = \begin{bmatrix}
            0 \\
            0 \\
            -4
        \end{bmatrix}
    \]

	Таким образом, обнулилась первая строка матрицы $\hat{B}$, связанная с собственным числом $\lambda_1 = -2$, значит, оно не управляемо. Для пары компексно-сопряженных $\lambda_{23} = 2 \pm 3i$ хотя бы одна из двух соответствующих строчек матрицы $\hat{B}$ не обнулена ($\hat{B}_2 = 0$, но $\hat{B}_3 \neq 0$), значит, они управляемы.

	По итогу система является частично управляемой, но стабилизируемой, так как единственное неуправляемое собственное число имеет отрицательную вещественную часть ($\Re(\lambda_1) = -2 < 0$) и является устойчивым, а значит, возможно создать стабилизирующее управление, сводящее вектор состояния к нулю.

	Замкнем систему регулятором $u = Kx$. Соответствующая схема приведена на рисунке:
	\begin{figure}[h]
		\centering
		\includegraphics[width=0.65\textwidth]{images/lqr_scheme.png}
		\caption{Схема замкнутой системы при LQR}
		\label{fig:lqr_scheme}
	\end{figure}

	Зададимся значениями матриц:
	\[
		Q^* = I = \begin{bmatrix} 1 & 0 & 0 \\ 0 & 1 & 0 \\ 0 & 0 & 1 \end{bmatrix} \succ 0, \quad R^* = 1 \succ 0
	\]

	а также значением параметра $\alpha=15 > 0$. С их помощью сформируем четыре набора матриц $(Q, R)$:
    \[
        (Q_1, R_1) = (Q^*, R^*), \quad (Q_2, R_2) = (\alpha Q^*, R^*)
	\]
	\[
		(Q_3, R_3) = (Q^*, \alpha R^*), \quad (Q_4, R_4) = (\alpha Q^*, \alpha R^*)
    \]

	или:
	\[
		(Q_1, R_1) = (I, 1), \quad (Q_2, R_2) = (15I, 1)
	\]
	\[
		(Q_3, R_3) = (I, 15), \quad (Q_4, R_4) = (15I, 15)
	\]

	Для каждого из наборов матриц $(Q, R)$ синтезируем регулятор, минимизирующий функционал качества:
	\[
		J = \int_0^{+ \infty} (x^T Q x + u^T R u) dt
	\]
	
	Для выполнения поставленной задачи будем решать матричное уравнение Риккати относительно матрицы $P \succ 0$ при $\nu = 1$:
	\[
		A^T P + P A + Q - \nu P B R^{-1} B^T P = 0
	\]
	
	и вычислять матрицу $K$ обратной связи как:
	\[
		K = -R^{-1} B^T P
	\]
	
	Отметим, что в LQR-регуляторах матрицы $Q$ и $R$ играют роль штрафов за медленные процессы и большие управления соответственно. Изучим их влияние на вид синтезируемых регуляторов.

	\textbf{Начнём с пары $(Q_1, R_1) = (I, 1)$.}

	Решением соответствующего уравнения Риккати при $Q = Q_1$ и $R = R_1$ является матрица $P_1$:
	\[
		P_1 = \begin{bmatrix}
			2.6791  & -0.2724  & 2.3301  \\
			-0.2724  & 1.0726  & -0.1762  \\
			2.3301  & -0.1762  & 2.3594 
		\end{bmatrix} \succ 0
	\]

	Тогда матрица регулятора $K_1$ равна:
	\[
		K_1 = -R_1^{-1} B^T P_1 = \begin{bmatrix}
			-5.3581 & 0.5448 & -4.6602 
		\end{bmatrix}
	\]

	На основе $P_1$, можно вычислить минимальное значение функционала качества, которого и достигает найденный регулятор при интегрировании на бесконечном промежутке времени ($x_0 = \begin{bmatrix} 1 & 1 & 1 \end{bmatrix}^T$):
	\[
		J_{min1} = x_0^T P_1 x_0 = 9.87405
	\]

	Также можно найти экспериментальное значением функционала качества, то есть ограничиться временем $t_1 = 2$:
	\[
		J_{exp1} = \int_0^{t_1} (x^T Q_1 x + (K_1 x)^T R_1 (K_1 x)) dt = 9.87383
	\]

	Видим, что значения $J_{min1}$ и $J_{exp1}$ \textit{очень} близки друг к другу. При ещё большем увеличении времени за счёт положительной определенности матриц $Q$ и $R$ значение $J_{exp1}$ возрастёт, придя <<на бесконечности>> к минимальному $J_{min1}$.

	\textbf{Теперь рассмотрим пару $(Q_2, R_2) = (15I, 1)$.}
	
	Решением всё того же уравнения Риккати при $Q = Q_2$ и $R = R_2$ будет
	\[
		P_2 = \begin{bmatrix}
			4.7281 & 0.3869 & 2.8534  \\
			0.3869 & 4.5266 & 1.9003  \\
			2.8534 & 1.9003 & 4.9640 
		\end{bmatrix} \succ 0
	\]
	
	Тогда обратная связь $K_2$ равна:
	\[
		K_2 = -R_2^{-1} B^T P_2 = \begin{bmatrix}
			-9.4563 & -0.7738 & -5.7068
		\end{bmatrix}
	\]

	Аналогично предыдущему случаю, можно найти минимальное значение функционала качества (условия те же):
	\[
		J_{min2} = x_0^T P_2 x_0 = 20.03906
	\]
	
	и его экспериментальное значение при $t_2 = 2$:
	\[
		J_{exp2} = \int_0^{t_1} (x^T Q_2 x + (K_2 x)^T R_2 (K_2 x)) dt = 20.0369
	\]

	\textbf{Далее возьмём пару $(Q_3, R_3) = (I, 15)$.}

	Решим уравнение Риккати при $Q = Q_3$ и $R = R_3$:
	\[
		P_3 = \begin{bmatrix}
			31.1904 & -5.0159 & 30.6312  \\
			-5.0159 & 12.0429 & -4.8657  \\
			30.6312 & -4.8657 & 30.5511
		\end{bmatrix} \succ 0
	\]
	
	Матрица регулятора $K_3$ равна:
	\[
		K_3 = -R_3^{-1} B^T P_3 = \begin{bmatrix}
			-4.1587 & 0.6688 & -4.0842 
		\end{bmatrix}
	\]
	
	Минимальное значение функционала качества при начальном состоянии $x_0$:
	\[
		J_{min3} = x_0^T P_3 x_0 = 77.971
	\]
	
	Его экспериментальное значение при $t_3 = 2$:
	\[
		J_{exp3} = \int_0^{t_3} (x^T Q_3 x + (K_3 x)^T R_3 (K_3 x)) dt = 77.9652
	\]

	\textbf{Наконец, рассмотрим пару $(Q_4, R_4) = (15I, 15)$.}

	Решением соответствующего уравнения Риккати при $Q = Q_4$ и $R = R_4$ является матрица $P_4$:
	\[
		P_4 = \begin{bmatrix}
			40.1859 & -4.0857 & 34.9516  \\
			-4.0857 & 16.0893 & -2.6436  \\
			34.9516 & -2.6436 & 35.3909 
		\end{bmatrix} \succ 0
	\]
	
	Матрица регулятора $K_4$ равна:
	\[
		K_4 = -R_4^{-1} B^T P_4 = \begin{bmatrix}
			-5.3581  & 0.5448  & -4.6602
		\end{bmatrix}
	\]
	
	Минимальное значение функционала качества при $x_0$:
	\[
		J_{min4} = x_0^T P_4 x_0 = 98.7405
	\]
	
	Его экспериментальное значение при $t_4 = 2$:
	\[
		J_{exp4} = \int_0^{t_4} (x^T Q_4 x + (K_4 x)^T R_4 (K_4 x)) dt = 98.7383
	\]

	Проведем компьютерное моделирование системы с найденными регуляторами при начальном состоянии $x_0 = \begin{bmatrix} 1 & 1 & 1 \end{bmatrix}^T$. На рисунках \ref{fig:x1_lqr}-\ref{fig:x4_lqr} изображены графики состояний системы при соответствующих им парах $(Q_i, R_i)$. На рисунке \ref{fig:controls_lqr} изображены графики синтезированных управлений. На рисунке \ref{fig:j_lqr} - графики экспериментальных и минимальных значений функционалов качества.
	
	\begin{figure}
		\centering
		\includegraphics[width=0.8\textwidth]{images/x1_lqr_plot1.png}
		\caption{Графики состояний системы с $(Q_1, R_1) = (I, 1)$ при LQR}
		\label{fig:x1_lqr}
	\end{figure}
	\begin{figure}
		\centering
		\includegraphics[width=0.8\textwidth]{images/x2_lqr_plot2.png}
		\caption{Графики состояний системы с $(Q_2, R_2) = (\alpha I, 1)$ при LQR}
		\label{fig:x2_lqr}
	\end{figure}
	\begin{figure}
		\centering
		\includegraphics[width=0.8\textwidth]{images/x3_lqr_plot3.png}
		\caption{Графики состояний системы с $(Q_3, R_3) = (I, \alpha)$ при LQR}
		\label{fig:x3_lqr}
	\end{figure}
	\begin{figure}
		\centering
		\includegraphics[width=0.8\textwidth]{images/x4_lqr_plot4.png}
		\caption{Графики состояний системы с $(Q_4, R_4) = (\alpha I, \alpha)$ при LQR}
		\label{fig:x4_lqr}
	\end{figure}
	\begin{figure}
		\centering
		\includegraphics[width=0.8\textwidth]{images/controls_lqr.png}
		\caption{Графики управлений с соответствующими парами $(Q_i, R_i)$ при LQR}
		\label{fig:controls_lqr}
	\end{figure}
	\begin{figure}
		\centering
		\includegraphics[width=0.9\textwidth]{images/jexp_plot.png}
		\caption{Графики $J_{exp}(t)$ и $J_{min}(t)$ с соответствующими парами $(Q_i, R_i)$ при LQR}
		\label{fig:j_lqr}
	\end{figure}

	Сравним полученные результаты. Для этого также дополнительно вычислим получающиеся в результате синтеза регулятора значения собственных чисел матрицы замкнутой системы $A + BK_i$:
	\[
		A+BK_1 = \begin{bmatrix}
			0.2838 & -0.9105 & 3.6796  \\
			6.0000 & -1.0000 & 6.0000  \\
			-6.0000 & -1.0000 & -8.0000
		\end{bmatrix}
	\]
	\[
		\sigma(A+BK_1) = \{-3.3581\pm3.7785i, -2\}
	\]
	\[
		A+BK_2 = \begin{bmatrix}
			-7.9125 & -3.5476 & 1.5864  \\
			6.0000 & -1.0000 & 6.0000  \\
			-6.0000 & -1.0000 & -8.0000
		\end{bmatrix}
	\]
	\[
		\sigma(A+BK_2) = \{-7.4563\pm5.5313i, -2\}
	\]
	\[
		A+BK_3 = \begin{bmatrix}
			2.6826 & -0.6624 & 4.8317  \\
			6.0000 & -1.0000 & 6.0000  \\
			-6.0000 & -1.0000 & -8.0000
		\end{bmatrix}
	\]
	\[
		\sigma(A+BK_3) = \{-2.1587\pm3.0865i, -2\}
	\]
	\[
		A+BK_4 = \begin{bmatrix}
			0.2838 & -0.9105 & 3.6796  \\
			6.0000 & -1.0000 & 6.0000  \\
			-6.0000 & -1.0000 & -8.0000
		\end{bmatrix}
	\]
	\[
		\sigma(A+BK_4) = \{-3.3581\pm3.7785i, -2\}
	\]

	Итак, наиболее <<мягко>> ведёт себя система при $(Q_3, R_3) = (I, \alpha)$, то есть когда штраф за большие управления велик. Здесь имеем самые близкие к нулю вещественные части собственных чисел, соответственно, и медленные процессы - управление низкое.

	При $(Q_2, R_2) = (\alpha I, 1)$ ситуация противоположная - собственные числа системы достигают наиболее отдаленных от нуля значений, что соответствует наиболее <<жесткому>> регулятору. Можем видеть резкие изменения состояний системы (рисунок \ref{fig:x2_lqr}) по сравнению с другими случаями (хотя неуправляемое собственное число и несколько ограничивает <<быстродействие>>)

	При $(Q_1, R_1) = (I, 1)$ и $(Q_4, R_4) = (\alpha I, \alpha)$ замкнутые системы, их собственные числа и управления \textit{идентичны}. По сути единственное отличие этих случаев - экспериментальные и минимальные значения функционалов качества, при $(Q_4, R_4)$ они больше в $\alpha$ раз (рисунок \ref{fig:j_lqr} - там же можем видеть, что при $t = 2$ все $J_{exp}(t)$ приходят к соответствующим $J_{min}(t)$, что было получено ранее численно). Всё это говорит о том, что на вид синтезируемого регулятора влияет именно \textit{соотношение} матриц $Q$ и $R$. Соответственно, чем больше $Q$ относительно $R$, тем большее быстродействие получается. Чем больше $R$ относительно $Q$, тем меньше быстродействие, но и значения управления. При равных $Q$ и $R$ получается тогда некий <<баланс>>.

	\section{Исследование LQE}
	Рассмотрим линейную систему
	\[
		\begin{cases}
			\dot{x} = Ax + f \\
			y = Cx + \xi
		\end{cases} \quad x_0 = x(0) = \begin{bmatrix} 1 & 1 & 1 & 1\end{bmatrix}^T
	\]
	
	В соотвествие с вариантом, матрицы $A, C$ имеют вид:
	\[
		A = \begin{bmatrix}
			20 & 5 & -16 & 9 \\
			6 & 1 & -4 & 1 \\
			32 & 9 & -25 & 14 \\
			8 & 4 & -6 & 4
		\end{bmatrix}, \quad C = \begin{bmatrix} -1 & 0 & 1 & -1 \end{bmatrix}
	\]

	Также зададимся \textit{детерминированными} сигналами $f(t)$ и $\xi(t)$, представляющими собой гармонические колебания:
	\[
		f(t) = \frac{1}{4}\begin{bmatrix} \cos(2t) \\ 2\sin(4t) \\ 3\cos(t) \\ 4\sin(3t) \end{bmatrix}, \quad 
		\xi(t) = \sin(5t)
	\]
	
	Перед синтезом наблюдателей проверим, можно ли вообще их синтезировать. Для этого проверим, является ли система обнаруживаемой. Опять-таки используем Жорданову форму системы с матрицами $\hat{A}$, имеющей собственными числами $\lambda_{12} = \pm i$ и $\lambda_{34} = \pm 2i$, $\hat{C}$ и матрицей $T$ перехода:
	\[
		\hat{A} = T^{-1}AT = \begin{bmatrix}
            0 & 1 & 0 & 0 \\
            -1 & 0 & 0 & 0 \\
            0 & 0 & 0 & 2 \\
            0 & 0 & -2 & 0
        \end{bmatrix}, \quad
        T = \begin{bmatrix}
            0.5 & -0.5 & -0.5 & 0.5 \\
            -0.5 & 0.5 & 0 & 1 \\
            1 & -0.5 & 0 & 1 \\
            1 & 0 & 1 & 0
        \end{bmatrix}
	\]

	Откуда матрица $C$ в жордановом базисе равна:
	\[
		\hat{C} = C T = \begin{bmatrix}
			-0.5 & 0 & -0.5 & 0.5
		\end{bmatrix}
	\]

	Для каждой из пар комплексно-сопряженных чисел хотя бы один из соответствующих им столбцов матрицы $\hat{C}$ не обнуляется ($\hat{C}_2=0$, но $\hat{C}_1 = -0.5 \neq 0$ для $\lambda_{12}$, $\hat{C}_3 = -0.5 \neq 0$ и $\hat{C}_4 = 0.5 \neq 0$ для $\lambda_{34}$). Таким образом, каждое из собственных чисел наблюдаемо, следовательно, вся система является полностью наблюдаемой, соответственно, и обнаруживаемой, так как вообще нет ненаблюдаемых собственных чисел. Можем синтезировать наблюдатель.

	Итак, схема наблюдателя состояния $\hat{x} = A\hat{x} + L(C\hat{x} - y)$ приведена на рисунке ниже:
	\begin{figure}[h]
		\centering
		\includegraphics[width=0.65\textwidth]{images/lqe_scheme.png}
		\caption{Схема наблюдателя состояния при LQE}
		\label{fig:lqe_scheme}
	\end{figure}

	Теперь зададимся матрицами $Q^*$ и $R^*$:
	\[
		Q^* = I = \begin{bmatrix} 1 & 0 & 0 & 0 \\ 0 & 1 & 0 & 0 \\ 0 & 0 & 1 & 0 \\ 0 & 0 & 0 & 1 \end{bmatrix} \succ 0, \quad R^* = 1 \succ 0
	\]

	а также параметром $\alpha=25 > 0$. С их помощью сформируем четыре набора матриц $(Q, R)$:
	\[
		(Q_1, R_1) = (Q^*, R^*), \quad (Q_2, R_2) = (\alpha Q^*, R^*)
	\]
	\[
		(Q_3, R_3) = (Q^*, \alpha R^*), \quad (Q_4, R_4) = (\alpha Q^*, \alpha R^*)
	\]

	или:
	\[
		(Q_1, R_1) = (I, 1), \quad (Q_2, R_2) = (25I, 1)
	\]
	\[
		(Q_3, R_3) = (I, 25), \quad (Q_4, R_4) = (25I, 25)
	\]

	Для каждой из пар значений $(Q, R)$ синтезируем наблюдатель с <<критерием доверия>>:
	\[
		J = \int_0^{+ \infty} (x^T Q^{-1} x + \xi^T R^{-1} \xi) dt
	\]
	
	Для этого решим матричное уравнение Риккати относительно матрицы $P \succ 0$ при $\nu = 1$:
	\[
		AP + P A^T + Q - \nu P C^T R^{-1} C P = 0
	\]

	После чего вычислим матрицу коррекции наблюдателя $L$ как:
	\[
		L = -P C^T R^{-1}
	\]

	Отметим, что в LQE матрцы $Q$ (ковариация шума процесса) и $R$ (ковариация шума измерений) играют роль доверительных характеристик к модели и датчикам соответственно. Изучим их влияние на синтезируемый наблюдатель.

	\textbf{Опять-таки, начнём с $(Q_1, R_1) = (I, 1)$.}

	Решением уравнения Риккати для наблюдателя при $Q = Q_1$ и $R = R_1$ является положительно-определённая $P_1$:
	\[
		P_1 = \begin{bmatrix}
			128.3827 & 43.1957 & 236.9805 & 122.1419 \\
			43.1957 & 19.2208 & 78.5788 & 37.3454 \\
			236.9805 & 78.5788 & 449.4129 & 242.4209 \\
			122.1419 & 37.3454 & 242.4209 & 142.2572
		\end{bmatrix} \succ 0
	\]

	\begin{figure}
		\centering
		\includegraphics[width=0.8\textwidth]{images/x1_1_lqe_plot1.png}
		\caption{Графики первого состояния с $(Q_1, R_1) = (I, 1)$ при LQE}
		\label{fig:x1_1_lqe}
	\end{figure}
	\begin{figure}
		\centering
		\includegraphics[width=0.8\textwidth]{images/x1_2_lqe_plot1.png}
		\caption{Графики второго состояния с $(Q_1, R_1) = (I, 1)$ при LQE}
		\label{fig:x1_2_lqe}
	\end{figure}
	\begin{figure}
		\centering
		\includegraphics[width=0.8\textwidth]{images/x1_3_lqe_plot1.png}
		\caption{Графики третьего состояния с $(Q_1, R_1) = (I, 1)$ при LQE}
		\label{fig:x1_3_lqe}
	\end{figure}
	\begin{figure}
		\centering
		\includegraphics[width=0.8\textwidth]{images/x1_4_lqe_plot1.png}
		\caption{Графики четвертого состояния с $(Q_1, R_1) = (I, 1)$ при LQE}
		\label{fig:x1_4_lqe}
	\end{figure}
	\begin{figure}
		\centering
		\includegraphics[width=0.8\textwidth]{images/err1_lqe_plot1.png}
		\caption{График ошибкок $e_i(t) = x_i(t) - \hat{x}_i(t)$ с $(Q_1, R_1) = (I, 1)$ при LQE}
		\label{fig:err1_lqe1}
	\end{figure}

	Матрица коррекции наблюдателя $L_1$ тогда равна:
	\[
		L_1 = -P_1 C^T R_1^{-1} = \begin{bmatrix}
			13.5441  \\
			1.9623  \\
			29.9885  \\
			21.9782
		\end{bmatrix}
	\]

	Выполним компьютерное моделирование с $L_1$ и нулевыми начальными условиями наблюдателя $\hat{x}(0) = \begin{bmatrix} 0 & 0 & 0 & 0 \end{bmatrix}^T$. На рисунках \ref{fig:x1_1_lqe}-\ref{fig:x1_4_lqe} изображены графики состояний системы и наблюдателя. На рисунке \ref{fig:err1_lqe1} представлен график ошибки оценки $e_i(t) = x_i(t) - \hat{x}_i(t)$.

	\textbf{Теперь рассмотрим пару $(Q_2, R_2) = (25I, 1)$.}

	Решением рассматриваемого в пункте уравнения Риккати при $Q = Q_2$ и $R = R_2$ будет:
	\[
		P_2 = \begin{bmatrix}
			2041.4721 & 748.8486 & 3512.9148 & 1538.1147  \\
			748.8486 & 294.7013 & 1286.3000 & 558.1005  \\
			3512.9148 & 1286.3000 & 6173.4061 & 2792.9160  \\
			1538.1147 & 558.1005 & 2792.9160 & 1333.8949  
		\end{bmatrix} \succ 0
	\]

	\begin{figure}
		\centering
		\includegraphics[width=0.8\textwidth]{images/x2_1_lqe_plot2.png}
		\caption{Графики первого состояния с $(Q_2, R_2) = (25I, 1)$ при LQE}
		\label{fig:x2_1_lqe}
	\end{figure}
	
	\begin{figure}
		\centering
		\includegraphics[width=0.8\textwidth]{images/x2_2_lqe_plot2.png}
		\caption{Графики второго состояния с $(Q_2, R_2) = (25I, 1)$ при LQE}
		\label{fig:x2_2_lqe}
	\end{figure}
	\begin{figure}
		\centering
		\includegraphics[width=0.8\textwidth]{images/x2_3_lqe_plot2.png}
		\caption{Графики третьего состояния с $(Q_2, R_2) = (25I, 1)$ при LQE}
		\label{fig:x2_3_lqe}
	\end{figure}
	\begin{figure}
		\centering
		\includegraphics[width=0.8\textwidth]{images/x2_4_lqe_plot2.png}
		\caption{Графики четвертого состояния с $(Q_2, R_2) = (25I, 1)$ при LQE}
		\label{fig:x2_4_lqe}
	\end{figure}
	\begin{figure}
		\centering
		\includegraphics[width=0.8\textwidth]{images/err2_lqe_plot2.png}
		\caption{График ошибкок $e_i(t) = x_i(t) - \hat{x}_i(t)$ с $(Q_2, R_2) = (25I, 1)$ при LQE}
		\label{fig:err2_lqe}
	\end{figure}

	Из найденного можно вычислить матрицу $L_2$:
	\[
		L_2 = -P_2 C^T R_2^{-1} = \begin{bmatrix}
			66.6721  \\
			20.6491  \\
			132.4246  \\
			79.0937
		\end{bmatrix}
	\]

	Промоделируем систему с матрицей $L_2$ при нулевых начальных условиях наблюдателя $\hat{x}(0) = \begin{bmatrix} 0 & 0 & 0 & 0 \end{bmatrix}^T$. На рисунках \ref{fig:x2_1_lqe}-\ref{fig:x2_4_lqe} изображены состояния системы и найденного наблюдателя. На рисунке \ref{fig:err2_lqe} представлен график ошибки оценки $e_i(t) = x_i(t) - \hat{x}_i(t)$.



	\textbf{Далее возьмём пару $(Q_3, R_3) = (I, 25)$.}

	Решением соответствующего уравнения Риккати при $Q = Q_3$ и $R = R_3$ будет матрица $P_3$:
	\[
		P_3 = \begin{bmatrix}
			216.6412 & 38.2700 & 424.4798 & 257.3331  \\
			38.2700 & 84.5543 & 76.9858 & 9.8890  \\
			424.4798 & 76.9858 & 898.6119 & 612.0852  \\
			257.3331 & 9.8890 & 612.0852 & 502.3009
		\end{bmatrix} \succ 0
	\]

	\begin{figure}
		\centering
		\includegraphics[width=0.8\textwidth]{images/x3_1_lqe_plot3.png}
		\caption{Графики первого состояния с $(Q_3, R_3) = (I, 25)$ при LQE}
		\label{fig:x3_1_lqe}
	\end{figure}
	
	\begin{figure}
		\centering
		\includegraphics[width=0.8\textwidth]{images/x3_2_lqe_plot3.png}
		\caption{Графики второго состояния с $(Q_3, R_3) = (I, 25)$ при LQE}
		\label{fig:x3_2_lqe}
	\end{figure}
	\begin{figure}
		\centering
		\includegraphics[width=0.8\textwidth]{images/x3_3_lqe_plot3.png}
		\caption{Графики третьего состояния с $(Q_3, R_3) = (I, 25)$ при LQE}
		\label{fig:x3_3_lqe}
	\end{figure}
	\begin{figure}
		\centering
		\includegraphics[width=0.8\textwidth]{images/x3_4_lqe_plot3.png}
		\caption{Графики четвертого состояния с $(Q_3, R_3) = (I, 25)$ при LQE}
		\label{fig:x3_4_lqe}
	\end{figure}
	\begin{figure}
		\centering
		\includegraphics[width=0.8\textwidth]{images/err3_lqe_plot3.png}
		\caption{График ошибкок $e_i(t) = x_i(t) - \hat{x}_i(t)$ с $(Q_3, R_3) = (I, 25)$ при LQE}
		\label{fig:err3_lqe}
	\end{figure}

	Аналогично предыдущему случаю, матрица $L_3$ можно найти:
	\[
		L_3 = -P_3 C^T R_3^{-1} = \begin{bmatrix}
			1.9798  \\
			-1.1531  \\
			5.5181  \\
			5.9020
		\end{bmatrix}
	\]

	Проведём моделирование системы и наблюдателя с матрицей коррекции $L_3$ при нулевых начальных условиях $\hat{x}(0) = \begin{bmatrix} 0 & 0 & 0 & 0 \end{bmatrix}^T$. На рисунках \ref{fig:x3_1_lqe}-\ref{fig:x3_4_lqe} изображены графики состояний. На рисунке \ref{fig:err3_lqe} представлен график ошибки оценки наблюдения $e_i(t) = x_i(t) - \hat{x}_i(t)$.

	\textbf{Наконец, рассмотрим пару $(Q_4, R_4) = (25I, 25)$.}
	
	Решим уравнение Риккати при $Q = Q_4$ и $R = R_4$:
	\[
		P_4 = \begin{bmatrix}
			3209.5667 & 1079.8935 & 5924.5132 & 3053.5487  \\
			1079.8935 & 480.5200 & 1964.4706 & 933.6352  \\
			5924.5132 & 1964.4706 & 11235.3231 & 6060.5235  \\
			3053.5487 & 933.6352 & 6060.5235 & 3556.4294
		\end{bmatrix} \succ 0
	\]


	\begin{figure}
		\centering
		\includegraphics[width=0.8\textwidth]{images/x4_1_lqe_plot4.png}
		\caption{Графики первого состояния с $(Q_4, R_4) = (25I, 25)$ при LQE}
		\label{fig:x4_1_lqe}
	\end{figure}
	\begin{figure}
		\centering
		\includegraphics[width=0.8\textwidth]{images/x4_2_lqe_plot4.png}
		\caption{Графики второго состояния с $(Q_4, R_4) = (25I, 25)$ при LQE}
		\label{fig:x4_2_lqe}
	\end{figure}
	\begin{figure}
		\centering
		\includegraphics[width=0.8\textwidth]{images/x4_3_lqe_plot4.png}
		\caption{Графики третьего состояния с $(Q_4, R_4) = (25I, 25)$ при LQE}
		\label{fig:x4_3_lqe}
	\end{figure}
	\begin{figure}
		\centering
		\includegraphics[width=0.8\textwidth]{images/x4_4_lqe_plot4.png}
		\caption{Графики четвертого состояния с $(Q_4, R_4) = (25I, 25)$ при LQE}
		\label{fig:x4_4_lqe}
	\end{figure}
	\begin{figure}
		\centering
		\includegraphics[width=0.8\textwidth]{images/err4_lqe_plot4.png}
		\caption{График ошибкок $e_i(t) = x_i(t) - \hat{x}_i(t)$ с $(Q_4, R_4) = (25I, 25)$ при LQE}
		\label{fig:err4_lqe}
	\end{figure}

	Откуда матрица $L_4$ равна:
	\[
		L_4 = -P_4 C^T R_4^{-1} = \begin{bmatrix}
			13.5441  \\
			1.9623  \\
			29.9885  \\
			21.9782
		\end{bmatrix}
	\]

	Аналогично предыдущим случаям, замоделируем систему и наблюдатель с матрицей коррекции $L_4$ при нулевых начальных условиях $\hat{x}(0) = \begin{bmatrix} 0 & 0 & 0 & 0 \end{bmatrix}^T$. На рисунках \ref{fig:x4_1_lqe}-\ref{fig:x4_4_lqe} изображены графики состояний. На рисунке \ref{fig:err4_lqe} представлен график ошибки оценки наблюдения $e_i(t) = x_i(t) - \hat{x}_i(t)$.

	Сравним полученные результаты. При $(Q_3, R_3) = (I, 25)$ ошибки оценки $e_i(t)$ имеют малые амплитуды колебаний, однако наблюдатель <<заторможен>> (рисунок \ref{fig:x1_3_lqe} - шум от выхода визуально отсутствует, но имеются проблемы с реагированием на быстрые изменения в системе). Наблюдатель больше доверяет модели, чем датчикам.

	При $(Q_2, R_2) = (25I, 1)$ ошибки $e_i(t)$ огромны, состояния $\hat{x}_i(t)$ постоянно скачут, однако наблюдатель достаточно быстро адаптируется к возникающим изменениям в системе, сильно доверяет датчикам.

	При $(Q_1, R_1) = (I, 1)$ и $(Q_4, R_4) = (25I, 25)$ ошибки идентичны, то есть опять приходим к выводу, что важно именно соотношение матриц $Q$ и $R$, а не их значения. В этом случае также виден <<баланс>> между шумом и реагированием.

	Здесь важно отметить, что при правильном выборе матриц $Q$ и $R$, отражающих действительное распределение между доверием к модели и измерениям, наблюдатель идеально сойдется к состояниям системы, без шума, и будет быстро реагировать на изменения в системе. К <<правильному>> выбору пары $(Q, R)$ и стремимся, делая верные или не очень предположения о взаимоотношении доверий и находя наблюдатель, минимизируя соответствующие <<критерии доверия>>. Отметим также, что в случае верных матриц $Q$ и $R$ достигается минимальное значение функционала $J$ для наблюдателя. 

	\section{Исследование LQG}
	Рассмотрим систему
	\[
		\begin{cases}
			\dot{x} = Ax + Bu + f \\
			y = Cx + Du + \xi
		\end{cases} \quad x_0 = x(0) = \begin{bmatrix} 1 & 1 & 1 & 1\end{bmatrix}^T
	\]

	В соответствии с вариантом, матрицы $A$ и $B$ имеют вид:
	\[
		A = \begin{bmatrix}
			5 & -5 & -9 & 3 \\
			-5 & 5 & -3 & 9 \\
			-9 & -3 & 5 & 5 \\
			3 & 9 & 5 & 5
		\end{bmatrix}, \quad
		B = \begin{bmatrix}
			2 & 0 \\
			6 & 0 \\
			6 & 0 \\
			2 & 0
		\end{bmatrix}
	\]

	Матрицы $C$ и $D$ же:
	\[
		C = \begin{bmatrix}
			1 & -1 & 1 & 1 \\
			1 & 3 & -1 & 3
		\end{bmatrix}, \quad
		D = \begin{bmatrix}
			0 & 2 \\
			0 & 1
		\end{bmatrix}
	\]

	Также зададимся \textit{случайными} сигналами $f(t)$ и $\xi(t)$ в виде гауссовских белых шумов, то есть сигналы подчиняются нормальному распределению с нулевым математическим ожиданием, с диагональными матрицами ковариации $F$ и $E$ соответственно:
	\[
		F = \begin{bmatrix}
			1 & 0 & 0 & 0 \\
			0 & 2 & 0 & 0 \\
			0 & 0 & 4 & 0 \\
			0 & 0 & 0 & 3
		\end{bmatrix}, \quad
		E = \begin{bmatrix}
			2 & 0 \\
			0 & 5
		\end{bmatrix}
	\]

	Прежде чем переходить к синтезам LQR-регулятора и LQE, в паре и представляющих LQG, проверим систему на стабилизируемость и обнаруживаемость и поймём, возможно ли вообще создать управление и наблюдение для рассматриваемой системы. Для этого перейдём к Жордановой форме системы с матрицами $\hat{A}$, $\hat{B}$ и $\hat{C}$, а также матрицей $T$ для перехода к базису Жордана:
	\[
		\hat{A} = T^{-1}AT = \begin{bmatrix}
			16 & 0 & 0 & 0 \\
			0 & 12 & 0 & 0 \\
			0 & 0 & 4 & 0 \\
			0 & 0 & 0 & -12
		\end{bmatrix}, \quad
		T = \begin{bmatrix}
			-1 & 1 & 1 & -1 \\
			1 & 1 & -1 & -1 \\
			1 & -1 & 1 & -1 \\
			1 & 1 & 1 & 1
		\end{bmatrix}
	\]

	Откуда можно найти матрицы $\hat{B}$ и $\hat{C}$:
	\[
		\hat{B} = T^{-1}B = \begin{bmatrix}
			3 & 0 \\
			1 & 0 \\
			1 & 0 \\
			-3 & 0
		\end{bmatrix}, \quad
		\hat{C} = CT = \begin{bmatrix}
			0 & 0 & 4 & 0 \\
			4 & 8 & 0 & 0
		\end{bmatrix}
	\] 

	Можем видеть, что каждому собственному числу в матрице $\hat{A}$ соответствует ненулевая строка в матрице $\hat{B}$, а значит, все собственные числа системы управляема. Система в такой случае является полностью управляемой, следовательно, и стабилизируемой (так как вообще нет неуправляемых собственных чисел).

	Также заметим, что четвертый столбец в матрице $\hat{C}$ состоит из нулей, а значит, собственное число $\lambda_4 = -12$ является ненаблюдаемым. Всем остальным $\lambda_i$  соответствует ненулевая строка в матрице $\hat{C}$ - они наблюдаемы. Система тогда является частично наблюдаемой, но обнаруживаемой, так как единственное ненаблюдаемое $\lambda_4$ имеет отрицательную вещественную часть ($\text{Re}(\lambda_4) = -12 < 0$).

	Таким образом, система является стабилизируемой и обнаруживаемой, а значит, для неё возможно создать управление и наблюдение. Схема моделирования, состоящая из регулятора и наблюдателя состояния, управления $u = K\hat{x}$, изображена на рисунке \ref{fig:lqg_scheme}.
	\begin{figure}[h]
		\centering
		\includegraphics[width=0.65\textwidth]{images/lqg_scheme.png}
		\caption{Схема моделирования системы при LQG}
		\label{fig:lqg_scheme}
	\end{figure}

	Теперь зададимся значениями пары матриц $(Q_k, R_k)$ регулятора:
	\[
		Q_k = I = \begin{bmatrix}
			1 & 0 & 0 & 0 \\
			0 & 1 & 0 & 0 \\
			0 & 0 & 1 & 0 \\
			0 & 0 & 0 & 1
		\end{bmatrix} \succ 0, \quad
		R_k = \begin{bmatrix}
			1 & 0 \\
			0 & 1
		\end{bmatrix} \succ 0
	\]

	А также для пары $(Q_l, R_l)$ для наблюдателя:
	\[
		Q_l = F = \begin{bmatrix}
			1 & 0 & 0 & 0 \\
			0 & 2 & 0 & 0 \\
			0 & 0 & 4 & 0 \\
			0 & 0 & 0 & 3
		\end{bmatrix} \succ 0, \quad
		R_l = E = \begin{bmatrix}
			2 & 0 \\
			0 & 5
		\end{bmatrix} \succ 0
	\]

	Выбор именно таких матриц наблюдателя опирается на теорию фильтра Калмана: при верно выбранной паре $(Q_l, R_l)$, то есть при выборе реальной ковариации шумов модели и измерений, математическое ожидание ошибки оценки наблюдения будет наименьшим.

	Итак, синтезируем матрицу регулятора, решая уравнение Риккати при $\nu = 1$, $Q = Q_k$ и $R = R_k$:
	\[
		A^T P + P A + Q - \nu P B R^{-1} B^T P = 0
	\]

	Откуда матрица $P_k$ равна:
	\[
		P_k = \begin{bmatrix}
			526.9134 & 354.0717 & -632.6571 & 248.4114 \\  
			354.0717 & 251.0646 & -436.6226 & 168.5971 \\ 
			-632.6571 & -436.6226 & 770.0936 & -299.3527 \\ 
			248.4114 & 168.5971 & -299.3527 & 117.8224
		\end{bmatrix} \succ 0
	\]
		
	Тогда матрица $K$ обратной связи:
	\[
		K = -R_k^{-1} B^T P_k = \begin{bmatrix}
			120.8624 & 68.0104 & -136.8063 & 52.0664 \\
			0.0000 & 0.0000 & 0.0000 & 0.0000
		\end{bmatrix}
	\]

	Далее, решая уравнение Риккати для наблюдателя при $\nu = 1$, $Q = Q_l$ и $R = R_l$:
	\[
		AP + P A^T + Q - \nu P C^T R^{-1} C P = 0
	\]

	Откуда матрица $P_l$ равна:
	\[
		P_l = \begin{bmatrix}
			526.9134 & 354.0717 & -632.6571 & 248.4114  \\
			354.0717 & 251.0646 & -436.6226 & 168.5971  \\
			-632.6571 & -436.6226 & 770.0936 & -299.3527  \\
			248.4114 & 168.5971 & -299.3527 & 117.8224
		\end{bmatrix} \succ 0
	\]

	Вычислим матрицу коррекции наблюдателя $L$ как:
	\[
		L = -P_l C^T R_l^{-1} = \begin{bmatrix}
			-2.0433 & 78.0928  \\
			2.1077 & -35.4308  \\
			-2.2480 & -78.1021  \\
			-2.1836 & -35.4307
		\end{bmatrix}
	\]

	Синтез проведен, перейдем к моделированию. На рисунке \ref{fig:controls_lqg} изображены графики управлений, на рисунках \ref{fig:x_lqg_plot1}-\ref{fig:x_lqg_plot4} - графики состояний и оценок, а на рисунке \ref{fig:err_lqg_plot1} - графики ошибок оценки.

	\begin{figure}
		\centering
		\includegraphics[width=0.8\textwidth]{images/controls_lqg.png}
		\caption{Графики управлений с $(Q_k, R_k)$ регулятора и $(Q_l, R_l)$ наблюдателя}
		\label{fig:controls_lqg}
	\end{figure}
	\begin{figure}
		\centering
		\includegraphics[width=0.8\textwidth]{images/x_lqg_plot1.png}
		\caption{Графики первой компоненты состояний и оценок при LQG}
		\label{fig:x_lqg_plot1}
	\end{figure}
	\begin{figure}
		\centering
		\includegraphics[width=0.8\textwidth]{images/x_lqg_plot2.png}
		\caption{Графики второй компоненты состояний и оценок при LQG}
		\label{fig:x_lqg_plot2}
	\end{figure}
	\begin{figure}
		\centering
		\includegraphics[width=0.8\textwidth]{images/x_lqg_plot3.png}
		\caption{Графики третьей компоненты состояний и оценок при LQG}
		\label{fig:x_lqg_plot3}
	\end{figure}
	\begin{figure}
		\centering
		\includegraphics[width=0.8\textwidth]{images/x_lqg_plot4.png}
		\caption{Графики четвертой компоненты состояний и оценок при LQG}
		\label{fig:x_lqg_plot4}
	\end{figure}
	\begin{figure}
		\centering
		\includegraphics[width=0.8\textwidth]{images/err_lqg_plot1.png}
		\caption{График ошибки оценки состояний при LQG}
		\label{fig:err_lqg_plot1}
	\end{figure}

	Таким образом, наблюдатель дал очень качественную оценку состояний системы, ошибки находятся в пределах от -1 до 1, в районе 0. Регулятор также справляется со своей задачей, успешно стабилизирует систему, хотя и остаются остаточные скачки от воздействия $f(t)$, но они в данном случае неискоренимы.

    \newpage
    \section{Общие выводы}
	В ходе выполнения лабораторной работы были изучены линейно-квадратичные регуляторы, наблюдатели и их комбинация - LQG.
	
	Было получено, что задача синтеза LQR полностью опирается на выбор матриц $Q$ и $R$ и того, что хочется от системы получить (так, при более быстрых желаемых процессах матрица $R$ должна быть меньше, а матрица $Q$ - больше, и наоборот). 
	
	В случае LQE матрицы $Q$ и $R$ сыграли играют ключевую роль и определяют качество оценки состояний системы (при верно расставленных приоритетах в доверии к модели и измерениям можно получить идеальную оценку состояний системы).

	Все выводы вышесказанные выводы также подтвердились и моделированиями.
    
\end{document}