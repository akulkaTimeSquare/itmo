\documentclass[a4paper,hidelinks,14pt]{extarticle}

\usepackage[utf8]{inputenc}
\usepackage[T2A]{fontenc}
\usepackage[english, russian]{babel}
\usepackage{lipsum}
\usepackage{amsmath}
\usepackage{amssymb}
\usepackage{amsfonts}
\usepackage{mathtools}
\usepackage{datetime}
\usepackage[pdftex]{graphicx}
\usepackage{indentfirst}
\usepackage{asymptote}
\usepackage{systeme}
\usepackage[dvipsnames]{xcolor}
\usepackage{lastpage}
\usepackage{fancybox,fancyhdr}
\usepackage{hyperref}
\usepackage[font={small,it}]{caption}
\fancyhead[L]{ЛР №2}
\fancyhead[C]{}
\fancyhead[R]{\textit{Модальные регуляторы и наблюдатели}}
\fancyfoot[L]{}
\fancyfoot[C]{Страница \thepage\space из \pageref{LastPage}}
\fancyfoot[R]{}
\pagestyle{fancy}
\newcommand{\gt}{\textgreater}
\newcommand{\lt}{\textless}
\let\oldemptyset\emptyset
\let\emptyset\varnothing

\begin{document}
	\begin{titlepage}
		\setlength{\parindent}{0ex}
		
		\begin{center}
			\textsc{
				\vspace{1ex}
				Научно исследовательский университет ИТМО \\
				\vspace{0.5ex}
				Факультет систем управления и робототехники \\
				\vspace{0.5ex}
			}
		\end{center}
		
		\vspace{50mm}
		
		\begin{center}
			Отчет по лабораторной работе №2 \\
			Модальные регуляторы и наблюдатели \\
            Вариант 11
		\end{center}
		
		\vspace{50mm}
		
		\begin{minipage}{.48\linewidth}
			Выполнил студент группы R3380
			
			Преподаватель
		\end{minipage}
		\hfill
		\begin{minipage}{.5\linewidth}
			\begin{flushright}
				Мовчан Игорь Евгеньевич
				\\
				Пашенко Артем Витальевич
			\end{flushright}
		\end{minipage}
		
		\vfill
		\begin{center}
			Санкт-Петербург
			\\
			2025
		\end{center}
		
	\end{titlepage}

	\tableofcontents
	\clearpage
	
	\section{Модальный регулятор}
    Исследуем линейную систему
    \[
        \dot{x} = Ax + Bu
    \]
    
    В соответствии с вариантом задания, матрицы $A$ и $B$ имеют вид
    \[
        A = \begin{bmatrix}
            11 & -2 & 13 \\
            6 & -1 & 6 \\
            -6 & -1 & -8
        \end{bmatrix}, \quad
        B = \begin{bmatrix}
            2 \\
            0 \\
            0
        \end{bmatrix}
    \]

    Выполним анализ управляемости системы и её собственных чисел. Для этого используем Жорданову форму $\hat{A} = T^{-1}AT$ матрицы $A$, имеющей собственными числами $\lambda_1 = -2$ и $\lambda_{23} = 2 \pm 3i$, а также вспомогательную матрицу $T$ для перехода к ней:
    \[
        \hat{A} = \begin{bmatrix}
            -2 & 0 & 0 \\
            0 & 2 & 3 \\
            0 & -3 & 2
        \end{bmatrix}, \quad
    T = \begin{bmatrix}
        -1 & -1.5 & -0.5 \\
        0 & -1 & 0 \\
        1 & 1 & 0
    \end{bmatrix}
    \]
    
    Можно получить обратную матрицу к $T$:
    \[
        T^{-1} = \begin{bmatrix}
            0 & 1 & 1 \\
            0 & -1 & 0 \\
            -2 & 1 & -2
        \end{bmatrix}
    \]

    Откуда:
    \[
        \hat{B} = T^{-1}B = \begin{bmatrix}
            0 \\
            0 \\
            -4
        \end{bmatrix}
    \]

    Заметим, что хоть элемент $\hat{B}_2$ матрицы $\hat{B}$, соответствующий одному из мнимых собственных чисел, и равен нулю, он не влияет на управляемость $\lambda_2$, так как та достигается через сопряженное $\lambda_3$.

    Также отметим, что $\hat{B}_1 = 0$, а значит, $\lambda_1$ не управляемо. Вся же система является частично управляемой, но стабилизируемой, так как единственное неуправляемое собственное число лежит в левой полуплоскости, то есть имеет отрицательную вещественную часть, а значит, является асимптотически устойчивым.

    Замкнем систему модальным регулятором вида:
    \[
        u = Kx
    \]

    Матрица $K$ коэффициентов обратной связи выбирается так, чтобы собственные числа уже замкнутой системы с матрицей $A + BK$ были устойчивыми.

    Схема моделирования системы с таким регулятором приведена на рисунке \ref{fig:model}.
    \begin{figure}[h]
        \centering
        \includegraphics[width=0.9\textwidth]{images/model.png}
        \caption{Схема моделирования системы с модальным регулятором}
        \label{fig:model}
    \end{figure}

    Проведем анализ предлагаемых спектров замкнутой системы:
    \[\sigma_1 = \{ -1, -1, -1 \}, \quad \sigma_2 = \{ -2, -2, -2 \}\]
    \[\sigma_3 = \{ -1, -10, -100 \}, \quad \sigma_4 = \{ -2, -20, -200 \}\]
    \[\sigma_5 = \{ -1, -1 \pm 3i \}, \quad \sigma_6 = \{ -2, -2 \pm 6i \}\]

    Отсекая те, в которых не возникает неуправляемого собственного числа $\lambda_1$, получаем достижимые спектры $\sigma_2$, $\sigma_4$ и $\sigma_6$ (все оставшиеся оказались недостижимыми). Дело в том, что наличие неуправляемых собственных чисел в спектре замкнутой системы необходимо, так как влияние на них матрицей $B$ вне зависимости от задаваемого управления $u$ невозможно (данное хорошо прослеживается в эквивалентой Жордановой форме системы, где будет происходит обнуление управления при умножении на $B$ - Жордановы блоки неуправляемых собственных чисел останутся неизменными, а значит, те попадут в спектр замкнутой системы).

    Для каждого оставшегося спектра найдем матрицу $K$ коэффициентов обратной связи. Начнём с $\sigma_2$.

    Для синтеза соответствующей матрицы $K$ регулятора необходимо решить уравнение Сильвестра относительно $P$:
    \[
        AP - PG  = BY
    \]

    Однако обратимого решения уравнения при неуправляемой паре $(A, B)$ не существует, поэтому воспользуемся вышесказанным про неизменность той части матрицы $\hat{A}$, которая соотвествует неуправляемым собственным числам, и рассмотрим её сужение $\hat{A}'$ (cоотвествующие неуправляемым собственным числам элементы $\hat{B}$ отбрасываются, как и в $\hat{A}$, образуя $\hat{B}'$) на управляемые собственные числа. Вот это сужение в процессе подачи управления $\hat{K}'$ и должно обладать необходимым спектром, исключая неуправляемые собственные числа. В итоге решаем уравнение:
    \[
        \hat{A}'P - PG = \hat{B}'Y
    \]
    
    Пару $(Y, G)$ сделаем наблюдаемой, усеченную матрицу обратной связи найдем по формуле $\hat{K}' = -YP^{-1}$, после чего, опять учитывая <<неприкосновенность>> неуправляемых собственных чисел, расширимся до $\hat{K}$ любыми числами и перейдем в базис исходной системы, используя $u = Kx = KT\hat{x} = \hat{K}\hat{x} \Rightarrow K = \hat{K}T^{-1}$.

    Итак, для спектра $\sigma_2$ получаем:
    \[
        \hat{A}' = \begin{bmatrix}
            2 & 3 \\
            -3 & 2
        \end{bmatrix}, \quad
        \hat{B}' = \begin{bmatrix}
            0 \\
            -4
        \end{bmatrix}, \quad
        T = \begin{bmatrix}
            -1 & -1.5 & -0.5 \\
            0 & -1 & 0 \\
            1 & 1 & 0
        \end{bmatrix}
    \]

    Выберем
    \[
    G_{\sigma_2} = \begin{bmatrix}
       -2 & 1 \\
       0 & -2
    \end{bmatrix}, \quad Y_{\sigma_2} = \begin{bmatrix}
        1 & 1
    \end{bmatrix}
    \]

    Тогда матрица наблюдаемости $Y_{\sigma_2}G_{\sigma_2} = \begin{bmatrix}
        Y_{\sigma_2} & Y_{\sigma_2}G_{\sigma_2}
    \end{bmatrix}^T$ имеет ранг 2, значит, пара $(Y_{\sigma_2}, G_{\sigma_2})$ наблюдаема.

    Таким образом, пара $(\hat{A}', \hat{B}')$ управляема, $(Y_{\sigma_2}, G_{\sigma_2})$ - наблюдаема, $\sigma(\hat{A}') \cap \sigma(G_{\sigma_2}) = \emptyset$, а значит уравнение Сильвестра имеет единственное обратимое решение
    \[
        P_{\sigma_2} = \begin{bmatrix}
            0.48 & 0.6336 \\
            -0.64 & -0.6848
        \end{bmatrix}
    \]

    Исходя из этого, матрица регулятора в базисе Жордана:
    \[
        \hat{K}'_{\sigma_2} = -Y_{\sigma_2}P_{\sigma_2}^{-1} = \begin{bmatrix}
            0.583 & 2
        \end{bmatrix} \Rightarrow
        \hat{K}_{\sigma_2} = \begin{bmatrix}
            2 & 0.583 & 2
        \end{bmatrix}
    \]

    В исходном базисе статический регулятор тогда задаётся через
    \[
        K_{\sigma_2} = \hat{K}_{\sigma_2}T^{-1} = \begin{bmatrix}
            -4 & 3.4167 & -2
        \end{bmatrix}
    \]

    Теперь выполним проверку корректности синтеза, для этого найдем матрицу замкнутой системы:
    \[
        A + BK_{\sigma_2} = \begin{bmatrix}
            3 & 4.83 & 9 \\
            6 & -1 & 6 \\
            -6 & -1 & -8
        \end{bmatrix}
    \]

    Она имеет собственными числами
    \[
        \lambda_1 = \lambda_2 = \lambda_3 = -2
    \]

    Следовательно, желаемый спектр $\sigma_2$ действительно был достигнут, а значит, задача синтеза регулятора полностью решена.

    \begin{figure}
        \centering
        \includegraphics[width=0.8\textwidth]{images/u.png}
        \caption{График формируемого регулятором управления $u(t)$ при $\sigma_2$}
        \label{fig:u}
    \end{figure}
    \begin{figure}
        \centering
        \includegraphics[width=0.8\textwidth]{images/x.png}
        \caption{График вектора состояния замкнутой системы $x(t)$ при $\sigma_2$}
        \label{fig:x}
    \end{figure}
    Выполним моделирование и построим графики формируемого регулятором управления $u(t) = Kx(t)$ и вектора состояния замкнутой системы $x(t)$ при начальных условиях $x(0) = [1, 1, 1]^T$. Графики приведены на рисунках \ref{fig:u} и \ref{fig:x} соответственно.

    Перейдем к спектру $\sigma_4$. Выберем для него матрицу $G_{\sigma_4}$ и $Y_{\sigma_4}$ так, чтобы пара $(Y_{\sigma_4}, G_{\sigma_4})$ была наблюдаемой:
    \[
        G_{\sigma_4} = \begin{bmatrix}
            -20 & 0 \\
            0 & -200
        \end{bmatrix}, \quad
        Y_{\sigma_4} = \begin{bmatrix}
            1 & 1
        \end{bmatrix}, \quad
        \text{rank}(Y_{\sigma_4}G_{\sigma_4}) = 2
    \]

    Тогда уравнение Сильвестра с теми же матрицами $\hat{A}'$ и $\hat{B}'$ имеет единственное обратимое решение
    \[
        P_{\sigma_4} = \begin{bmatrix}
            0.0243 & 0.0003 \\
            -0.1785 & -0.0198
        \end{bmatrix}
    \]

    Исходя из этого, найдем матрицу регулятора в базисе Жордана:
    \[
        \hat{K}'_{\sigma_4} = -Y_{\sigma_4}P_{\sigma_4}^{-1} = \begin{bmatrix}
            369.583 & 56
        \end{bmatrix} \Rightarrow
        \hat{K}_{\sigma_4} = \begin{bmatrix}
            4 & 369.583 & 56
        \end{bmatrix}
    \]

    В исходном базисе статический регулятор тогда задаётся через
    \[
        K_{\sigma_4} = \hat{K}_{\sigma_4}T^{-1} = \begin{bmatrix}
            -112 & -309.583 & -108
        \end{bmatrix}
    \]

    Теперь выполним проверку корректности синтеза, для этого найдем матрицу замкнутой системы:
    \[
        A + BK_{\sigma_4} = \begin{bmatrix}
            -213 & -621.167 & -203 \\
            6 & -1 & 6 \\
            -6 & -1 & -8
        \end{bmatrix}
    \]

    Она имеет собственными числами
    \[
        \lambda_1 = -2, \quad \lambda_2 = -20, \quad \lambda_3 = -200
    \]

    Следовательно, желаемый спектр $\sigma_4$ был достигнут, а значит, задача синтеза регулятора полностью решена.

    \begin{figure}
        \centering
        \includegraphics[width=0.8\textwidth]{images/u1.png}
        \caption{График формируемого регулятором управления $u(t)$ при $\sigma_4$}
        \label{fig:u1}
    \end{figure}
    \begin{figure}
        \centering
        \includegraphics[width=0.8\textwidth]{images/x1.png}
        \caption{График вектора состояния замкнутой системы $x(t)$ при $\sigma_4$}
        \label{fig:x1}
    \end{figure}
    Графики формируемого регулятором управления $u(t) = Kx(t)$ и вектора состояния замкнутой системы $x(t)$ при начальных условиях $x(0) = [1, 1, 1]^T$ приведены на рисунках \ref{fig:u1} и \ref{fig:x1} соответственно.

    Рассмотрим теперь спектр $\sigma_6$. Выберем для него матрицу $G_{\sigma_6}$ и $Y_{\sigma_6}$ так, чтобы пара $(Y_{\sigma_6}, G_{\sigma_6})$ была наблюдаемой:
    \[
        G_{\sigma_6} = \begin{bmatrix}
            -2 & 6 \\
            -6 & -2
        \end{bmatrix}, \quad
        Y_{\sigma_6} = \begin{bmatrix}
            1 & 1
        \end{bmatrix}, \quad
        \text{rank}(Y_{\sigma_6}G_{\sigma_6}) = 2
    \]

    Рассматриваемое уравнение Сильвестра тогда имеет единственное обратимое решение
    \[
        P_{\sigma_6} = \begin{bmatrix}
            -0.292 & 0.1831 \\
            0.0231 & -0.828
        \end{bmatrix}
    \]

    Исходя из этого, матрица регулятора в базисе Жордана:
    \[
        \hat{K}'_{\sigma_6} = -Y_{\sigma_6}P_{\sigma_6}^{-1} = \begin{bmatrix}
            3.583 & 2
        \end{bmatrix} \Rightarrow
        \hat{K}_{\sigma_6} = \begin{bmatrix}
            6 & 3.583 & 2
        \end{bmatrix}
    \]

    В исходном базисе статический регулятор тогда задаётся через
    \[
        K_{\sigma_6} = \hat{K}_{\sigma_6}T^{-1} = \begin{bmatrix}
            -4 & 4.4167 & 2
        \end{bmatrix}
    \]

    Теперь выполним проверку корректности синтеза, для этого найдем матрицу замкнутой системы:
    \[
        A + BK_{\sigma_6} = \begin{bmatrix}
            3 & 6.833 & 17 \\
            6 & -1 & 6 \\
            -6 & -1 & -8
        \end{bmatrix}
    \]

    Получившаяся матрица имеет собственными числами
    \[
        \lambda_1 = -2, \quad \lambda_{23} = -2 \pm 6i
    \]

    Значит, желаемый спектр $\sigma_6$ был достигнут - задача синтеза регулятора полностью решена.

    \begin{figure}
        \centering
        \includegraphics[width=0.8\textwidth]{images/u2.png}
        \caption{График формируемого регулятором управления $u(t)$ при $\sigma_6$}
        \label{fig:u2}
    \end{figure}
    \begin{figure}
        \centering
        \includegraphics[width=0.8\textwidth]{images/x2.png}
        \caption{График вектора состояния замкнутой системы $x(t)$ при $\sigma_6$}
        \label{fig:x2}
    \end{figure}
    Графики управления и вектора состояния замкнутой системы при начальных условиях $x(0) = [1, 1, 1]^T$ приведены на рисунках \ref{fig:u2} и \ref{fig:x2} соответственно.

    Расположим графики полученных управлений для спектров $\sigma_2$, $\sigma_4$ и $\sigma_6$ вместе (на рисунке \ref{fig:u_all} приведены графики при $t \in [0, 4]$ - на рисунке \ref{fig:u_all1} приведены графики при $t \in [0.01, 4]$).

    \begin{figure}
        \centering
        \includegraphics[width=0.8\textwidth]{images/u_all.png}
        \caption{Графики полученных управлений $u(t)$ при $t \in [0, 4]$}
        \label{fig:u_all}
    \end{figure}
    \begin{figure}
        \centering
        \includegraphics[width=0.8\textwidth]{images/u_all1.png}
        \caption{Графики полученных управлений $u(t)$ при $t \in [0.01, 4]$}
        \label{fig:u_all1}
    \end{figure}

    Сравнивая управления систем, можем видеть, что чем более быстрых переходных процессов хочется получить (чем дальше собственные числа замнкутой системы от мнимой оси), тем больше по величине и управление в первые моменты времени, появляются <<всплески>>, растет перерегулирование (сравнение управлений у спектров $\sigma_4$ и $\sigma_2$ или $\sigma_6$ на рисунке \ref{fig:u_all}). Соответственно, чем медленнее - тем менее скачкообразно (спектр $\sigma_2$ без выраженных колебаний и скачков в управлении на рисунке \ref{fig:u_all1}). Живя в математической абстракции нет никаких ограничений на величину управления, оно может быть сколько угодно великим (как, соответственно, и скорость затухания) - главное не бесконечным, однако в реальной жизни проявляется физическая часть объекта - мы не хотим получать тысячи ампер в проводах! Поэтому при синтезе уже физически существующих регуляторов необходимо учитывать это, находя некий компромисс между быстротой затухания и возникающим перерегулированием.
    
    Также отметим, что при спектре $\sigma_6$ все переходные процессы происходят быстрее (в сравнении с $\sigma_2$), однако управление при этом менее плавное, система совершает частые колебания, связанные с наличием мнимых собственных чисел в спектре.
    
    Немаловажно и то, что у спектра $\sigma_4$ возникают <<выбросы>> и в координатах вектора состояния (рисунок \ref{fig:x1}, координата $x_3$ в начальные моменты времени), что в целом тоже нежелательно, так как может негативно сказаться на физике объекта.

    Спектр $\sigma_2$ же на рисунках \ref{fig:u} и \ref{fig:x} наоборот даёт \textit{медленные}, но монотонные затухания, не обладающие ярко выраженными колебаниями или <<всплесками>>.
    
    В связи со всем вышесказанным выбор желаемого спектра для синтеза регулятора - задача выбора чисел с учетом всех неуправляемостей, физических ограничений на воздействия, а также желаемых переходных характеристик системы. Важным будет упомянуть, что решаемую задачу синтеза отчасти могут облегчить стандартные полиномы типа Ньютона, Баттерворта или Чебышева, которые позволяют, зная желаемые переходные характеристики системы, выбрать и соответствующее управление системы.

    \section{Наблюдатель полного порядка}
    Рассмотрим линейную систему
    \[
    \begin{cases}
        \dot{x} = Ax \\
        y = Cx
    \end{cases}
    \]

    В соответствии с вариантом задания, матрицы $A$ и $C$ имеют вид
    \[
        A = \begin{bmatrix}
            20 & 5 & -16 & 9 \\
            6 & 1 & -4 & 1 \\
            32 & 9 & -25 & 14 \\
            8 & 4 & -6 & 4
        \end{bmatrix}, \quad C^T = \begin{bmatrix}
            -1 \\ 0 \\ 1 \\ -1
        \end{bmatrix}
    \]

    Выполним анализ наблюдаемости системы. Собственные числа матрицы $A$ равны
    \[
        \lambda_{12} = \pm i, \quad \lambda_{34} = \pm 2i
    \]

    Используем Жорданову форму $\hat{A} = T^{-1}AT$ и найдем $\hat{C}$:
    \[
        \hat{A} = \begin{bmatrix}
            0 & 1 & 0 & 0 \\
            -1 & 0 & 0 & 0 \\
            0 & 0 & 0 & 2 \\
            0 & 0 & -2 & 0
        \end{bmatrix}, \quad
        T = \begin{bmatrix}
            0.5 & -0.5 & -0.5 & 0.5 \\
            -0.5 & 0.5 & 0 & 1 \\
            1 & -0.5 & 0 & 1 \\
            1 & 0 & 1 & 0
        \end{bmatrix}
    \]
    
    Откуда матрица $C$ в жордановом базисе равна
    \[
        \hat{C} = C T = \begin{bmatrix}
            -0.5 & 0 & -0.5 & 0.5
        \end{bmatrix}
    \]

    Лишь элемент $\hat{C}_2$ обнулился, однако его пара $\hat{C}_1 \neq 0$ это компенсирует. Для второй пары $\hat{C}_{34} = \pm 0.5 \neq 0$, а значит, все собственные числа оказались наблюдаемыми, сама система тогда является полностью наблюдаемой и, следовательно, обнаруживаемой (так как вообще нет ненаблюдаемых собственных чисел). Можем спокойно строить наблюдатель состояния $\dot{\hat{x}} = A \hat{x} + L (C\hat{x} - y)$ - схема приведена на рисунке \ref{fig:observer}.
    \begin{figure}
        \centering
        \includegraphics[width=\textwidth]{images/observer.png}
        \caption{Схема наблюдателя состояния}
        \label{fig:observer}
    \end{figure}

    Рассмотрим каждый из предложенных спектров $\sigma_1$, $\sigma_2$ и $\sigma_3$:
    \[
        \sigma_1 = \{-6, -6, -6, -6\}, \quad \sigma_2 = \{-6, -60, -600, -6000\}
    \]
    \[
        \sigma_3 = \{-6 \pm 7i, -6 \pm 8i\}
    \]

    Так как система полностью наблюдаема, с помощью матрицы коррекции наблюдателя $L$ можно достичь любого желаемого спектра у матрицы наблюдателя $A + LC$. Начнём с синтеза наблюдателя для спектра $\sigma_1$. Для этого воспользуемся уравнением Сильвестра:
    \[
        GQ - QA = YC
    \]

    Зададимся матрицей $G_{\sigma_1}$ и $Y_{\sigma_1}$ так, чтобы пара $(G_{\sigma_1}, Y_{\sigma_1})$ была управляемой, а $G_{\sigma_1}$ имела необходимый спектр $\sigma_1$:
    \[
        G_{\sigma_1} = \begin{bmatrix}
            -6 & 1 & 0 & 0 \\
            0 & -6 & 1 & 0 \\
            0 & 0 & -6 & 1 \\
            0 & 0 & 0 & -6
        \end{bmatrix}, \quad
        Y_{\sigma_1} = \begin{bmatrix}
            1 \\ 1 \\ 1 \\ 1
        \end{bmatrix}
    \]


    Таким образом, пара $(A, C)$ наблюдаема, $(G_{\sigma_1}, Y_{\sigma_1})$ управляема, а $\sigma(A) \cap \sigma(G_{\sigma_1}) = \emptyset$. Эти условия дают единственное обратимое решение уравнения Сильвестра:
    \[
        Q_{\sigma_1} = \begin{bmatrix}
            0.3657 & 0.0119 & -0.3154 & 0.2346 \\
            0.3636 & 0.0118 & -0.3139 & 0.2338 \\
            0.3512 & 0.0111 & -0.3042 & 0.2281 \\
            0.2865 & 0.0081 & -0.2514 & 0.1932
        \end{bmatrix}
    \]

    Теперь можем синтезировать матрицу коррекции наблюдателя $L_{\sigma_1}$ с использованием полученого:
    \[
        L_{\sigma_1} = Q_{\sigma_1}^{-1}Y_{\sigma_1} \approx \begin{bmatrix}
            175.667 \\
            261.667 \\
            381.000 \\
            229.333
        \end{bmatrix}
    \]

    Матрица наблюдателя $A + L_{\sigma_1}C_{\sigma_1}$ тогда принимает вид
    \[
        A + L_{\sigma_1}C_{\sigma_1} \approx \begin{bmatrix}
            -155.667 & 5.000 & 159.667 & -166.667 \\
            -255.667 & 1.000 & 257.667 & -260.667 \\
            -349.000 & 9.000 & 356.000 & -367.000 \\
            -221.333 & 4.000 & 223.333 & -225.333 \\
        \end{bmatrix}
    \]
    
    И имеет собственными числами
    \[
        \lambda_1 = \lambda_2 = \lambda_3 = \lambda_4 = -6
    \]

    Следовательно, желаемый спектр $\sigma_1$ был достигнут, а значит, задача синтеза наблюдателя полностью решена.

    Проведем компьютерное моделирование с начальными условиями системы $x(0) = [1, 1, 1, 1]^T$ и наблюдателя $\hat{x}(0) = [2, 0, 0, -1]^T$. Cравнительные графики вектора состояний $x(t)$ и наблюдателя $\hat{x}(t)$ при $\sigma_1$ приведены на рисунках \ref{fig:x_obs1}-\ref{fig:x_obs4}, а график ошибки наблюдателя $e(t) = \hat{x}(t) - x(t)$ приведен на рисунке \ref{fig:x_obs_err}.

    \begin{figure}
        \centering
        \includegraphics[width=0.8\textwidth]{images/x1_1_observer.png}
        \caption{Векторы состояния системы $x_1(t)$ и наблюдателя $\hat{x}_1(t)$ при $\sigma_1$}
        \label{fig:x_obs1}
    \end{figure}
    \begin{figure}
        \centering
        \includegraphics[width=0.8\textwidth]{images/x1_2_observer.png}
        \caption{Векторы состояния системы $x_2(t)$ и наблюдателя $\hat{x}_2(t)$ при $\sigma_1$}
        \label{fig:x_obs2}
    \end{figure}
    \begin{figure}
        \centering
        \includegraphics[width=0.8\textwidth]{images/x1_3_observer.png}
        \caption{Векторы состояния системы $x_3(t)$ и наблюдателя $\hat{x}_3(t)$ при $\sigma_1$}
        \label{fig:x_obs3}
    \end{figure}
    \begin{figure}
        \centering
        \includegraphics[width=0.8\textwidth]{images/x1_4_observer.png}
        \caption{Векторы состояния системы $x_4(t)$ и наблюдателя $\hat{x}_4(t)$ при $\sigma_1$}
        \label{fig:x_obs4}
    \end{figure}
    \begin{figure}
        \centering
        \includegraphics[width=0.8\textwidth]{images/x1_observer_error.png}
        \caption{График ошибки наблюдателя $e(t) = \hat{x}(t) - x(t)$ при $\sigma_1$}
        \label{fig:x_obs_err}
    \end{figure}

    Перейдем к синтезу наблюдателя для спектра $\sigma_2$. Отметим, что для $\sigma_2$ и $\sigma_3$ будем решать всё то же уравнение Сильвестра:
    \[
        GQ - QA = YC
    \]

    Зададимся такими матрицами $G_{\sigma_2}$ и $Y_{\sigma_2}$, что пара $(G_{\sigma_2}, Y_{\sigma_2})$ является управляемой, а $G_{\sigma_2}$ имеет необходимый спектр $\sigma_2$:
    \[
        G_{\sigma_2} = \begin{bmatrix}
            -6 & 0 & 0 & 0 \\
            0 & -60 & 0 & 0 \\
            0 & 0 & -600 & 0 \\
            0 & 0 & 0 & -6000
        \end{bmatrix}, \quad
        Y_{\sigma_2} = \begin{bmatrix}
            1 \\ 1 \\ 1 \\ 1
        \end{bmatrix}
    \]

    С заданными матрицами существует единственное и обратимое решение рассматриваемого уравнения Сильвестра:
    \[
        Q_{\sigma_2} = \begin{bmatrix}
            0.2865 & 0.0081 & -0.2514 & 0.1932 \\
            0.0178 & 0.0000 & -0.0175 & 0.0170 \\
            0.0017 & 0.0000 & -0.0017 & 0.0017 \\
            0.0002 & 0.0000 & -0.0002 & 0.0002
        \end{bmatrix}
    \]

    Теперь можем синтезировать матрицу коррекции наблюдателя $L_{\sigma_2}$ с использованием полученого:
    \[
        L_{\sigma_2} = Q_{\sigma_2}^{-1}Y_{\sigma_2} \approx \begin{bmatrix}
            -270681790 \\
            217338652 \\
            -404001376 \\
            -133312919
        \end{bmatrix}
    \]

    Матрица наблюдателя $A + L_{\sigma_2}C_{\sigma_2}$ тогда принимает вид
    \[
        A + L_{\sigma_2}C_{\sigma_2} \approx \begin{bmatrix}
            270681810.33 & 5 & -270681806.33 & 270681799.33 \\
            -217338645.66 & 1 & 217338647.66 & -217338650.66 \\
            404001407.00 & 9 & -404001400.00 & 404001389.00 \\
            133312926.67 & 4 & -133312924.67 & 133312922.67
        \end{bmatrix}
    \]

    И имеет собственными числами
    \[
        \lambda_1 = -6, \quad \lambda_2 = -60, \quad \lambda_3 = -600, \quad \lambda_4 = -6000
    \]

    Следовательно, желаемый спектр $\sigma_2$ был достигнут, а значит, задача синтеза наблюдателя полностью решена.

    Проведем компьютерное моделирование с начальными условиями системы $x(0) = [1, 1, 1, 1]^T$ и наблюдателя $\hat{x}(0) = [2, 0, 0, -1]^T$. Cравнительные графики вектора состояний $x(t)$ и наблюдателя $\hat{x}(t)$ при $\sigma_2$ приведены на рисунках \ref{fig:x_2_obs1}-\ref{fig:x_2_obs4}, а график ошибки наблюдателя $e(t) = \hat{x}(t) - x(t)$ приведен на рисунке \ref{fig:x_2_obs_err}.

    \begin{figure}
        \centering
        \includegraphics[width=0.8\textwidth]{images/x2_1_observer.png}
        \caption{Векторы состояния системы $x_1(t)$ и наблюдателя $\hat{x}_1(t)$ при $\sigma_2$}
        \label{fig:x_2_obs1}
    \end{figure}
    \begin{figure}
        \centering
        \includegraphics[width=0.8\textwidth]{images/x2_2_observer.png}
        \caption{Векторы состояния системы $x_2(t)$ и наблюдателя $\hat{x}_2(t)$ при $\sigma_2$}
        \label{fig:x_2_obs2}
    \end{figure}
    \begin{figure}
        \centering
        \includegraphics[width=0.8\textwidth]{images/x2_3_observer.png}
        \caption{Векторы состояния системы $x_3(t)$ и наблюдателя $\hat{x}_3(t)$ при $\sigma_2$}
        \label{fig:x_2_obs3}
    \end{figure}
    \begin{figure}
        \centering
        \includegraphics[width=0.8\textwidth]{images/x2_4_observer.png}
        \caption{Векторы состояния системы $x_4(t)$ и наблюдателя $\hat{x}_4(t)$ при $\sigma_2$}
        \label{fig:x_2_obs4}
    \end{figure}
    \begin{figure}
        \centering
        \includegraphics[width=0.8\textwidth]{images/x2_observer_error.png}
        \caption{График ошибки наблюдателя $e(t) = \hat{x}(t) - x(t)$ при $\sigma_2$}
        \label{fig:x_2_obs_err}
    \end{figure}
    
    Перейдем к синтезу наблюдателя для спектра $\sigma_3$. Выберем матрицы $G_{\sigma_3}$ и $Y_{\sigma_3}$ так, чтобы пара $(G_{\sigma_3}, Y_{\sigma_3})$ была управляемой, а $G_{\sigma_3}$ имела необходимый спектр $\sigma_3$:
    \[
        G_{\sigma_3} = \begin{bmatrix}
            -6 & 7 & 0 & 0 \\
            -7 & -6 & 0 & 0 \\
            0 & 0 & -6 & 8 \\
            0 & 0 & -8 & -6
        \end{bmatrix}, \quad
        Y_{\sigma_3} = \begin{bmatrix}
            1 \\ 1 \\ 1 \\ 1
        \end{bmatrix}
    \]

    С этими матрицами существует единственное и обратимое решение рассматриваемого уравнения Сильвестра:
    \[
        Q_{\sigma_3} = \begin{bmatrix}
            0.1956 & 0.0001 & -0.1855 & 0.1644 \\
            -0.0764 & -0.0048 & 0.0583 & -0.0275 \\
            0.1677 & -0.0008 & -0.1615 & 0.1476 \\
            -0.0790 & -0.0038 & 0.0629 & -0.0346
        \end{bmatrix}
    \]

    Используя полученную матрицу, синтезируем матрицу коррекции наблюдателя $L_{\sigma_3}$:
    \[
        L_{\sigma_3} = Q_{\sigma_3}^{-1}Y_{\sigma_3} = \begin{bmatrix}
            -1284 \\
            1500 \\
            -1752 \\
            -444
        \end{bmatrix}
    \]

    Проверим корректность полученного наблюдателя. Найдем матрицу наблюдателя $A + L_{\sigma_3}C_{\sigma_3}$:
    \[
        A + L_{\sigma_3}C_{\sigma_3} = \begin{bmatrix}
            1304 & 5 & -1300 & 1293 \\
            -1494 & 1 & 1496 & -1499 \\
            1784 & 9 & -1777 & 1766 \\
            452 & 4 & -450 & 448
        \end{bmatrix}
    \]

    Она имеет собственными числами
    \[
        \lambda_{12} = 6 \pm 7i, \quad \lambda_{34} = 6 \pm 8i
    \]

    Желаемый спектр $\sigma_3$ был достигнут, а значит, задача синтеза наблюдателя полностью решена.

    Проведем компьютерное моделирование с начальными условиями системы $x(0) = [1, 1, 1, 1]^T$ и наблюдателя $\hat{x}(0) = [2, 0, 0, -1]^T$. Cравнительные графики вектора состояний $x(t)$ и наблюдателя $\hat{x}(t)$ при $\sigma_3$ приведены на рисунках \ref{fig:x_3_obs1}-\ref{fig:x_3_obs4}, а график ошибки наблюдателя $e(t) = \hat{x}(t) - x(t)$ приведен на рисунке \ref{fig:x_3_obs_err}.

    \begin{figure}
        \centering
        \includegraphics[width=0.8\textwidth]{images/x3_1_observer.png}
        \caption{Векторы состояния системы $x_1(t)$ и наблюдателя $\hat{x}_1(t)$ при $\sigma_3$}
        \label{fig:x_3_obs1}
    \end{figure}
    \begin{figure}
        \centering
        \includegraphics[width=0.8\textwidth]{images/x3_2_observer.png}
        \caption{Векторы состояния системы $x_2(t)$ и наблюдателя $\hat{x}_2(t)$ при $\sigma_3$}
        \label{fig:x_3_obs2}
    \end{figure}
    \begin{figure}
        \centering
        \includegraphics[width=0.8\textwidth]{images/x3_3_observer.png}
        \caption{Векторы состояния системы $x_3(t)$ и наблюдателя $\hat{x}_3(t)$ при $\sigma_3$}
        \label{fig:x_3_obs3}
    \end{figure}
    \begin{figure}
        \centering
        \includegraphics[width=0.8\textwidth]{images/x3_4_observer.png}
        \caption{Векторы состояния системы $x_4(t)$ и наблюдателя $\hat{x}_4(t)$ при $\sigma_3$}
        \label{fig:x_3_obs4}
    \end{figure}
    \begin{figure}
        \centering
        \includegraphics[width=0.8\textwidth]{images/x3_observer_error.png}
        \caption{График ошибки наблюдателя $e(t) = \hat{x}(t) - x(t)$ при $\sigma_3$}
        \label{fig:x_3_obs_err}
    \end{figure}

    Сопоставим полученные результаты. Наблюдатель $\hat{x}(t)$ сходится к вектору состояний $x(t)$ при всех трех спектрах, однако наблюдается ситуация, подобная рассматриваемой в пункте 1, - при быстрых желаемых переходных процессах наблюдателя появляются <<всплекски>> в моделируемом векторе состояний $\hat{x}(t)$ в начальные моменты времени (наглядно это видно на графиках спектра $\sigma_2$, представленных на рисунках \ref{fig:x_2_obs1}-\ref{fig:x_2_obs4} и \ref{fig:x_2_obs_err}). В реальной жизни наблюдатели используются, например, в задачах построения управления при неполном знании вектора состояний системы. Естественно предположить, что увидев такие сильные отклонения, как при спектре $\sigma_2$, управление, основанное на таком наблюдателе, сразу же задаст большое воздействие на объект, что, конечно же, нежелательно, так как сильно <<расшатывает>> физику, приводит к поломкам и возможным скачкам ещё и в системе.

    Спектр $\sigma_3$ даёт выраженные колебания в ошибке наблюдателя, однако сходится немного быстрее, чем при $\sigma_1$, поведение которого является наиболее плавным, но в то же время относительно медлительным из всех рассматриваемых случаев.

    В итоге синтез наблюдателя, как и регулятора, полностью основывается на том, какое качество переходных процессов является допустимым, а какое - нет. С учётом всех желаний и нужно выбирать оптимальный спектр для синтеза наблюдателей.

    \section{Модальное управление по выходу}
    Рассмотрим систему
    \[
        \begin{cases}
            \dot{x} = Ax + Bu \\
            y = Cx + Du
        \end{cases}
    \]

    В соответствии с вариантом, матрицы $A$ и $B$ имеют вид
    \[
        A = \begin{bmatrix}
            5 & -5 & -9 & 3 \\
            -5 & 5 & -3 & 9 \\
            -9 & -3 & 5 & 5 \\
            3 & 9 & 5 & 5
        \end{bmatrix}, \quad
        B = \begin{bmatrix}
            2 \\
            6 \\
            6 \\
            2
        \end{bmatrix}
    \]
    
    Матрицы $C$ и $D$ задаются же как:
    \[
        C = \begin{bmatrix}
            1 & -1 & 1 & 1 \\
            1 & 3 & -1 & 3
        \end{bmatrix}, \quad
        D = \begin{bmatrix}
            2 \\
            1
        \end{bmatrix}
    \]

    Исследуем управляемость и наблюдаемость системы. Для этого используем Жорданову форму матрицы $A$, имеющей собственными числами $\lambda_1 = 16$, $\lambda_2 = 12$, $\lambda_3 = 4$ и $\lambda_4 = -12$:
    \[
        \hat{A} = T^{-1}AT = \begin{bmatrix}
            16 & 0 & 0 & 0 \\
            0 & 12 & 0 & 0 \\
            0 & 0 & 4 & 0 \\
            0 & 0 & 0 & -12
        \end{bmatrix}, \quad
        T = \begin{bmatrix}
            -1 & 1 & 1 & -1 \\
            1 & 1 & -1 & -1 \\
            1 & -1 & 1 & -1 \\
            1 & 1 & 1 & 1
        \end{bmatrix}
    \]

    Откуда:
    \[
        \hat{B} = T^{-1}B = \begin{bmatrix}
            3 \\
            1 \\
            1 \\
            -3
        \end{bmatrix}, \quad
        \hat{C} = CT = \begin{bmatrix}
            0 & 0 & 4 & 0 \\
            4 & 8 & 0 & 0
        \end{bmatrix}
    \]

    Видим, что в матрице управления $\hat{B}$ в жордановом базисе нет нулей, а значит, все собственные числа $\lambda_{1-4}$ управляемы, система - полностью управляемая, следовательно, и стабилизируемая.

    Также отметим, что в матрице наблюдения $\hat{C}$ в жордановом базисе имеет четвертый нулевой столбец $\hat{C}_4$, а значит, собственное число $\lambda_4=-12$ не наблюдаемо. Для остальных собственных чисел же все столбцы $\hat{C}_1 = [0, 4]^T \neq 0$, $\hat{C}_2 = [0, 8]^T \neq 0$ и $\hat{C}_3 = [4, 0]^T \neq 0$ ненулевые - $\lambda_{1-3}$ наблюдаемы. В итоге получилось, что система является частично наблюдаемой, но обнаруживаемой, так как $\Re(\lambda_4) < 0$.

    Итак, перейдем к задаче модального управления по выходу. Для начала построим схему моделирования системы, замкнутой регулятором, состоящем из наблюдателя состояния $\dot{\hat{x}} = A\hat{x} + (B + LD)u + L(C\hat{x} - y)$ и закона управления $u = K\hat{x}$. Рисунок \ref{fig:model_output} как раз представляет собой данную схему в среде Simulink.

    \begin{figure}
        \centering
        \includegraphics[width=\textwidth]{images/observer_and_regulator.png}
        \caption{Схема моделирования модального управления по выходу}
        \label{fig:model_output}
    \end{figure}

    Теперь зададимся парой \textit{достижимых} спектров $\sigma_r$ и $\sigma_n$ для регулятора и наблюдателя соответственно, обеспечивающих асимптотическую устойчивость замкнутой системы. Спектр $\sigma_r$ можно брать любым, так как система является полностью управляемой. На $\sigma_n$ же действует ограничение в виде ненаблюдаемости $\lambda_4=-12$ - его необходимо включить в спектр (здесь работает та же логика, что и при неуправляемом собственном числе - какую бы матрицу $L$ наблюдателя не сформировали, в матрице наблюдения $\hat{C}$ будет нулевой столбец, соответствующий $\lambda_4$ и оставляющий его жорданову клетку в матрице наблюдателя $\hat{A} + L\hat{C}$ неизменной - значит, спектр содержит $\lambda_4$), остальные собственные числа же можно брать любыми. По итогу возьмем, например, такие спектры:
    \[
        \sigma_r = \{ -13, -13, -14, -14 \}, \quad \sigma_n = \{ -11, -11, -12, -12 \}
    \]

    Модальное управление по выходу по своей сути является регулятором, объединенным с наблюдателем. Их также можно синтезировать по-отдельности. Начнём с регулятора. Зададимся матрицей $G_r$, имеющей спектр $\sigma_r$:   
    \[
        G_r = \begin{bmatrix}
            -13 &	1 &	0	&0 \\
            0   &	-13 &	0 &	0 \\
            0 &	0 &	-14	&1 \\ 
            0	&0	&0	&-14
        \end{bmatrix}
    \]

    Теперь возьмём матрицу $Y_r$ таким образом, чтобы пара $(Y_r, G_r)$ была наблюдаемой:
    \[
        Y_r = \begin{bmatrix}
            1 \\
            1 \\
            1 \\
            1
        \end{bmatrix}
    \]

    Итак, решим уравнение Сильвестра относительно матрицы $P$:
    \[
        AP - PG_r = BY_r
    \]

    Так как пара $(A, B)$ управляема, $(Y_r, G_r)$ наблюдаема, а спектры матриц $A$ и $G_r$ не пересекаются, то существует единственное и обратимое решение уравнения Сильвестра. Найдем его:
    \[
        P \approx \begin{bmatrix}
            2.9954&  5.9969&  1.4940&  2.2452\\  
            3.0846&  6.0863&  1.5829&  2.3346\\  
            3.1223&  6.1277&  1.6171&  2.3720  \\
            -2.7977&  -5.7891&  -1.3060&  -2.0481
        \end{bmatrix}
    \]

    Итак, матрица $K$ коэффициентов обратной связи регулятора:
    \[
        K = -Y_rP^{-1} = \begin{bmatrix}
            169.2232  & 105.8420 &  -199.7036 &  75.3615 
        \end{bmatrix}
    \]

    Проверим корректность полученного регулятора. Найдем матрицу замкнутой системы $A + BK$:
    \[
        A + BK \approx \begin{bmatrix}
            343.4463 & 206.6840 & -408.4071 & 153.7230  \\
            1010.3389 & 640.0521 & -1201.2214 & 461.1689  \\
            1006.3389 & 632.0521 & -1193.2214 & 457.1689  \\
            341.4463 & 220.6840 & -394.4071 & 155.7230
        \end{bmatrix}
    \]

    Она имеет собственными числами
    \[
        \lambda_1 = \lambda_2 = -13, \quad \lambda_3 = \lambda_4 = -14
    \]

    Желаемый спектр $\sigma_r$ был достигнут, а значит, задача синтеза регулятора полностью решена. Перейдем к наблюдателю. Так как пара $(A, C)$ является частично наблюдаемой, то обратимого решения уравнения Сильвестра
    \[
        GA - QA = YC
    \]
    не существует, поэтому воспользуемся усечением Жордановой формы системы $\hat{A}$ и $\hat{C}$ на наблюдаемые собственные числа, а после дополним найденную матрицу коррекции наблюдателя нулевыми строками, соответствующим ненаблюдаемым собственным числам, и перейдем к исходному базису. В итоге получаем:
    \[
        \hat{A}' = \begin{bmatrix}
            16 & 0 & 0 \\
            0 & 12 & 0 \\
            0 & 0 & 4
        \end{bmatrix}, \quad
        \hat{C}' = \begin{bmatrix}
            0 & 0 & 4 \\
            4 & 8 & 0
        \end{bmatrix}
    \]

    Теперь зададимся матрицей $G_n$, имеющей спектр $\sigma_n$:
    \[
        G_n = \begin{bmatrix}
            -11 & 1 & 0 \\
            0 & -11 & 0 \\
            0 & 0 & -12
        \end{bmatrix}
    \]

    Также возьмём матрицу $Y_n$ таким образом, чтобы пара $(G_n, Y_n)$ была управляемой:
    \[
        Y_n = \begin{bmatrix}
            1 &	1 \\
            1 & 1 \\
            1 & 1
        \end{bmatrix}
    \]

    Теперь пара $(\hat{A}', \hat{C}')$ управляема, спектры матриц $G_n$ и $\hat{A}'$ не пересекаются, и пара $(G_n, Y_n)$ управляема, а значит, существует единственное и обратимое решение уравнения Сильвестра:
    \[
        G_n Q - Q\hat{A}' = Y_n\hat{C}'
    \]

    Итак, матрица $Q$:
    \[
        Q \approx \begin{bmatrix}
            -0.1536 & -0.3629 & -0.2844  \\
            -0.1481 & -0.3478 & -0.2667  \\
            -0.1429 & -0.3333 & -0.2500
        \end{bmatrix}
    \]

    Откуда усеченная матрица коррекции наблюдателя в жордановом базисе принимает вид:
    \[
        \hat{L}' = Q^{-1}Y_n = \begin{bmatrix}
            -106.312 & -106.312  \\
            49.594 & 49.594  \\
            -9.375 & -9.375 
        \end{bmatrix}
    \]

    Дополним её нулевыми строками, соответствующими ненаблюдаемым собственным числам, и перейдем к исходному базису:
    \[
        \hat{L} = \begin{bmatrix}
            -106.312 & -106.312  \\
            49.594 & 49.594  \\
            -9.375 & -9.375  \\
            0 & 0
        \end{bmatrix} \Rightarrow L = T\hat{L} = \begin{bmatrix}
            146.5312 & 146.5312  \\
            -47.3437 & -47.3437  \\
            -165.2812 & -165.2812  \\
            -66.0937 & -66.0937
        \end{bmatrix}
    \]
    
    Проверим теперь корректность полученного наблюдателя. Найдем матрицу замкнутой системы $A + LC$:
    \[
        A + LC = \begin{bmatrix}
            298.0625 & 288.0625 & -9.0000 & 589.1250  \\
            -99.6875 & -89.6875 & -3.0000 & -180.3750  \\
            -339.5625 & -333.5625 & 5.0000 & -656.1250  \\
            -129.1875 & -123.1875 & 5.0000 & -259.3750
        \end{bmatrix}
    \]

    Она имеет собственными числами
    \[
        \lambda_1 = \lambda_2 = -11, \quad \lambda_3 = \lambda_4 = -12
    \]
    
    Желаемый спектр $\sigma_n$ был достигнут, а значит, задача синтеза наблюдателя полностью решена. 
    
    Перейдем к компьютерному моделированию системы с начальными условиями $x(0) = [1, 1, 1, 1]^T$ и $\hat{x}(0) = [0, 0, 0, 0]^T$ для наблюдателя. На рисунке \ref{fig:u_ex} изображен график управления, на рисунках \ref{fig:x1_ex}-\ref{fig:x4_ex} - графики состояния системы и наблюдателя, на рисунке \ref{fig:e_ex} - график ошибки оценок.

    \begin{figure}
        \centering
        \includegraphics[width=0.8\textwidth]{images/u_ex.png}
        \caption{Управление системы при модальном управлении по выходу}
        \label{fig:u_ex}
    \end{figure}
    \begin{figure}
        \centering
        \includegraphics[width=0.8\textwidth]{images/x1_ex.png}
        \caption{Первая компонента состояния при модальном управлении по выходу}
        \label{fig:x1_ex}
    \end{figure}
    \begin{figure}
        \centering
        \includegraphics[width=0.8\textwidth]{images/x2_ex.png}
        \caption{Вторая компонента состояния при модальном управлении по выходу}
        \label{fig:x2_ex}
    \end{figure}
    \begin{figure}
        \centering
        \includegraphics[width=0.8\textwidth]{images/x3_ex.png}
        \caption{Третья компонента состояния при модальном управлении по выходу}
        \label{fig:x3_ex}
    \end{figure}
    \begin{figure}
        \centering
        \includegraphics[width=0.8\textwidth]{images/x4_ex.png}
        \caption{Четвертая компонента $x(t)$ при модальном управлении по выходу}
        \label{fig:x4_ex}
    \end{figure}
    \begin{figure}
        \centering
        \includegraphics[width=0.8\textwidth]{images/e_ex.png}
        \caption{График ошибки оценки при модальном управлении по выходу}
        \label{fig:e_ex}
    \end{figure}

    Таким образом, задача модального управления по выходу решена. Регулятор успешно стабилизировал систему, сведя все компоненты состояния к нулю, используя оценку наблюдателя, всё является асимптотически устойчивым. Ошибка оценки со временем также стремится к нулю, вначале же она примерно равна 25 (в пике).

    \section{Наблюдатель пониженного порядка}
    Рассмотрим следующую систему:
    \[
    \begin{cases}
        \dot{x} = Ax + Bu \\
        y = Cx + Du
    \end{cases}
    \]

    В соответствии с заданием, матрицы $A$, $B$ и $D$ остались теми же:
    \[
        A = \begin{bmatrix}
            5 & -5 & -9 & 3 \\
            -5 & 5 & -3 & 9 \\
            -9 & -3 & 5 & 5 \\
            3 & 9 & 5 & 5
        \end{bmatrix}, \quad
        B = \begin{bmatrix}
            2 \\
            6 \\
            6 \\
            2
        \end{bmatrix}, \quad
        D = \begin{bmatrix}
            2 \\
            1
        \end{bmatrix}
    \]

    Матрица $C$ же изменилась и теперь имеет вид:
    \[
        C = \begin{bmatrix}
            0 & 0 & 0 & 1 \\
            0 & 1 & 0 & 0
        \end{bmatrix}
    \]
    
    Выходит, напрямую собираются компоненты состояния $x_2$ и $x_4$, задач наблюдателя в данном случае будет оценить $x_1$ и $x_3$, беря значения остальных компонент как данность из выхода.

    Проведем анализ управляемости и наблюдаемости системы и её собственных чисел. Для этого используем Жорданову форму системы с матрицей $A$, имеющей собственными числами
    \[
        \lambda_1 = 16, \quad \lambda_2 = 12, \quad \lambda_3 = 4, \quad \lambda_4 = -12
    \]

    Откуда Жорданова форма $\hat{A}$ и матрица перехода $T$ к ней:
    \[
        \hat{A} = T^{-1}AT = \begin{bmatrix}
            16 & 0 & 0 & 0 \\
            0 & 12 & 0 & 0 \\
            0 & 0 & 4 & 0 \\
            0 & 0 & 0 & -12
        \end{bmatrix}, \quad
        T = \begin{bmatrix}
            -1 & 1 & 1 & -1 \\
            1 & 1 & -1 & -1 \\
            1 & -1 & 1 & -1 \\
            1 & 1 & 1 & 1
        \end{bmatrix}
    \]

    Получаем:
    \[
        \hat{B} = T^{-1}B = \begin{bmatrix}
            3 \\
            1 \\
            1 \\
            -3
        \end{bmatrix}, \quad
        \hat{C} = CT = \begin{bmatrix}
            1	&1	&1	&1 \\
            1	&1	&-1	&-1
        \end{bmatrix}
    \]

    Так как пара $(A, B)$ осталась той же, что и в предыдущем пункте, то и система имеет то же качество управляемости: так как матрица управления $\hat{B}$ не имеет нулевых столбцов, то все собственные числа управляемы, а значит, система является полностью управляемой.

    Матрица $C$ видоизменилась, в Жордановом базисе теперь отсутствуют нулевые столбцы, а значит, все собственные числа $\lambda_{1-4}$ системы теперь наблюдаемы, сама система же полностью наблюдаема.

    Перейдем к синтезу наблюдателя пониженного порядка. Сначала построим схему моделирования системы, замкнутую регулятором, состоящем из наблюдателя \textit{пониженного порядка}
    \[
    \dot{\hat{z}} = G\hat{z} - Yy + (QB + YD)u, \quad
    \hat{x} = \begin{bmatrix} C \\ Q \end{bmatrix}^{-1} \begin{bmatrix} y - Du \\ \hat{z} \end{bmatrix} =
    \begin{bmatrix} C \\ Q \end{bmatrix}^{-1} \begin{bmatrix} Cx \\ \hat{z} \end{bmatrix}
    \]

    Схема моделирования приведена на рисунке \ref{fig:model_reduced_order_observer}. В качестве модального регулятора $K$ используем матрицу, найденную в предыдущем пункте (задача управления $u(t) = K\hat{x}(t)$ здесь, как и прежде, стабилизировать систему):
    \begin{figure}
        \centering
        \includegraphics[width=\textwidth]{images/observer_down.png}
        \caption{Схема моделирования наблюдателя пониженного порядка}
        \label{fig:model_reduced_order_observer}
    \end{figure}
    \[
        K = \begin{bmatrix}
            169.2232  & 105.8420 &  -199.7036 &  75.3615 
        \end{bmatrix}
    \]

    Зададимся желаемым спектром $\sigma = \{ -5, -6\}$ матрицы наблюдателя пониженного порядка $G$, обеспечивающим асимптотическую устойчивость замкнутой системы. Матрица $G$ в этом случае:
    \[
        G = \begin{bmatrix}
            -5 & 0 \\
            0 & -6
        \end{bmatrix}
    \]

    Также выберем матрицу $Y$ таким образом, чтобы пара $(G, Y)$ была полностью управляемой:
    \[
        Y = \begin{bmatrix}
            1 & 1 \\
            1 & 0
        \end{bmatrix}
    \]

    Наконец, синтезируем матрицу $Q$ путем решения соответствующего уравнения Сильвестра:
    \[
        GA - QA = YC
    \]

    Откуда:
    \[
        Q = \begin{bmatrix}
            0.3315 & 0.1727 & 0.3035 & 0.0934 \\
            -0.3641 & 0.1120 & -0.3978 & 0.2073
        \end{bmatrix}
    \]

    В итоге пара $(A, C)$ является полностью наблюдаемой, $(G, Y)$ - управляемой, а собственные числа матриц $A$ и $G$ не пересекаются, следовательно, существует обратная от матрицы $N = \begin{bmatrix} C & Q \end{bmatrix}^T$. Также $G$ гурвицева, и выполнен успешный синтез матрицы $Q$, решающей уравнение Сильвестра, а значит, димамическая система с уравнением $\dot{\hat{z}} = G\hat{z} - Yy + (QB + YD)u$ будет выполнять функции наблюдателя. Таким образом, синтезирован наблюдатель пониженного порядка:
    \[
        \hat{x} = N^{-1} \begin{bmatrix} y \\ \hat{z} \end{bmatrix} = \begin{bmatrix} C \\ Q \end{bmatrix}^{-1} \begin{bmatrix} y \\ \hat{z} \end{bmatrix}
    \]

    \begin{figure}
        \centering
        \includegraphics[width=0.8\textwidth]{images/control_input.png}
        \caption{Управление системы при наблюдателе пониженного порядка}
        \label{fig:control_input}
    \end{figure}
    \begin{figure}
        \centering
        \includegraphics[width=0.8\textwidth]{images/z_plot.png}
        \caption{Вектор состояния наблюдателя пониженной размерности}
        \label{fig:z_plot}
    \end{figure}
    \begin{figure}
        \centering
        \includegraphics[width=0.8\textwidth]{images/x1_observer_down.png}
        \caption{Первая компонента $x_1$ при наблюдателе пониженного порядка}
        \label{fig:x1_observer_down}
    \end{figure}
    \begin{figure}
        \centering
        \includegraphics[width=0.8\textwidth]{images/x2_observer_down.png}
        \caption{Вторая компонента $x_2$ при наблюдателе пониженного порядка}
        \label{fig:x2_observer_down}
    \end{figure}
    \begin{figure}
        \centering
        \includegraphics[width=0.8\textwidth]{images/x3_observer_down.png}
        \caption{Третья компонента $x_3$ при наблюдателе пониженного порядка}
        \label{fig:x3_observer_down}
    \end{figure}
    \begin{figure}
        \centering
        \includegraphics[width=0.8\textwidth]{images/x4_observer_down.png}
        \caption{Четвертая компонента $x_4$ при наблюдателе пониженного порядка}
        \label{fig:x4_observer_down}
    \end{figure}
    \begin{figure}
        \centering
        \includegraphics[width=0.8\textwidth]{images/estimation_error_observer_down.png}
        \caption{График ошибки оценки при наблюдателе пониженного порядка}
        \label{fig:estimation_error_observer_down}
    \end{figure}

    Выполним моделирование полученной с начальными условиями системы $x(0) = [1, 1, 1, 1]^T$ и наблюдателя $\hat{z}(0) = [0, 0]^T$. На рисунке \ref{fig:control_input} изображен график управления, на рисунках \ref{fig:z_plot}-\ref{fig:x4_observer_down} - графики состояния системы и наблюдателя, на рисунке \ref{fig:estimation_error_observer_down} - график ошибки оценки состояния.
    
    Таким образом, задача синтеза наблюдателя пониженного порядка решена. Моделирование подтвердило все полученные результаты: наблюдатель успешно оценил состояние $x_1$ и $x_3$, которые не наблюдаются напрямую через матрицу наблюдения $C$, быстро свёл первоначальную ошибку к нулю (визуально при одной секунде ошибка оценки уже равна нулю), а регулятор стабилизировал систему, сведя все компоненты состояния к нулю.

    \newpage
    \section{Общие выводы}
    В результате выполнения лабораторной работы были исследованы модальные регуляторы и наблюдатели разных типов, был проведён процесс их синтеза с использованием уравнений Сильвестра, получены возможные ограничения на желаемые значения спектров замкнутых систем.

    Также была получена обратная взаимосвязь между перерегулированием и скоростью сходимости оценок (для наблюдателей) и состояний системы (для регуляторов).

    Все задачи выполнены успешно, регуляторы и наблюдатели работают корректно как по-отдельности, так и в паре, а всё изученное подтверждается моделированием.


    
\end{document}