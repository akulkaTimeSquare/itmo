\documentclass[a4paper,hidelinks,14pt]{extarticle}

\usepackage[utf8]{inputenc}
\usepackage[T2A]{fontenc}
\usepackage[english, russian]{babel}
\usepackage{lipsum}
\usepackage{amsmath}
\usepackage{amssymb}
\usepackage{amsfonts}
\usepackage{mathtools}
\usepackage{datetime}
\usepackage[pdftex]{graphicx}
\usepackage{indentfirst}
\usepackage{asymptote}
\usepackage{systeme}
\usepackage[dvipsnames]{xcolor}
\usepackage{lastpage}
\usepackage{fancybox,fancyhdr}
\usepackage{hyperref}
\usepackage[font={small,it}]{caption}
\fancyhead[L]{Лабораторная работа №1}
\fancyhead[C]{}
\fancyhead[R]{\textit{Управляемость и наблюдаемость}}
\fancyfoot[L]{}
\fancyfoot[C]{Страница \thepage\space из \pageref{LastPage}}
\fancyfoot[R]{}
\pagestyle{fancy}
\newcommand{\gt}{\textgreater}
\newcommand{\lt}{\textless}

\begin{document}
	\begin{titlepage}
		\setlength{\parindent}{0ex}
		
		\begin{center}
			\textsc{
				\vspace{1ex}
				Научно исследовательский университет ИТМО \\
				\vspace{0.5ex}
				Факультет систем управления и робототехники \\
				\vspace{0.5ex}
			}
		\end{center}
		
		\vspace{50mm}
		
		\begin{center}
			Отчет по лабораторной работе №1 \\
			Управляемость и наблюдаемость \\
            Вариант 11
		\end{center}
		
		\vspace{50mm}
		
		\begin{minipage}{.48\linewidth}
			Выполнил студент группы R3380
			
			Преподаватель
		\end{minipage}
		\hfill
		\begin{minipage}{.5\linewidth}
			\begin{flushright}
				Мовчан Игорь Евгеньевич
				\\
				Пашенко Артем Витальевич
			\end{flushright}
		\end{minipage}
		
		\vfill
		\begin{center}
			Санкт-Петербург
			\\
			2025
		\end{center}
		
	\end{titlepage}

	\tableofcontents
	\clearpage
	
	\section{Исследование управляемости}
    Исследуем линейную систему
    \begin{equation*}
        \dot{x} = Ax + Bu
    \end{equation*}

    В соответствии с вариантом задания, матрицы $A$ и $B$ имеют вид:
    \begin{equation*}
    A = \begin{bmatrix}
        7 & -6 & 9 \\
        6 & -5 & 6 \\
        -6 & 3 & -8
    \end{bmatrix}, \quad
    B = \begin{bmatrix}
        -2 \\
        -1 \\
        2
    \end{bmatrix}
    \end{equation*}

    Выполним анализ управляемости. Для этого сначала вычислим матрицу управляемости:
    \begin{equation*}
        U = \begin{bmatrix}
            B & AB & A^2B
        \end{bmatrix}
        = \begin{bmatrix}
            -2 & 10 & -23 \\
            -1 & 5 & -7 \\
            2 & -7 & 11
        \end{bmatrix}
    \end{equation*}

    Ранг матрицы управляемости равен размерности пространства состояний (то есть $\text{rank}(U) = 3$), а значит, система является полностью управляемой по критерию Калмана.

    Можно и иначе! Найдем собственные значения матрицы $A$ и каждое из них проверим на управляемость:
    \[
    \lambda_1 = -2, \quad \lambda_2 = -2 + 3i, \quad \lambda_3 = -2 - 3i
    \]

    Для каждого собственного числа $\lambda_i$ построим матрицу Хаутуса:
    \[
    H_U(\lambda_i) = \begin{bmatrix} A - \lambda_i I & B \end{bmatrix}
    \]

    Вычислим соотвествующие матрицы и найдем их ранги:
    \[
    H_U(\lambda_1) =
    \begin{bmatrix}
    7 - (-2) & -6 & 9 & -2 \\
    6 & -5 - (-2) & 6 & -1 \\
    -6 & 3 & -8 - (-2) & 2
    \end{bmatrix} =
    \]
    \[
    = \begin{bmatrix}
    9 & -6 & 9 & -2 \\
    6 & -3 & 6 & -1 \\
    -6 & 3 & -6 & 2
    \end{bmatrix}
    \]
    \[
    H_U(\lambda_2) =
    \begin{bmatrix}
    7 - (-2 + 3i) & -6 & 9 & -2 \\
    6 & -5 - (-2 + 3i) & 6 & -1 \\
    -6 & 3 & -8 - (-2 + 3i) & 2
    \end{bmatrix} =
    \]
    \[
    = \begin{bmatrix}
    9 - 3i & -6 & 9 & -2 \\
    6 & -3 - 3i & 6 & -1 \\
    -6 & 3 & -6 - 3i & 2
    \end{bmatrix}
    \]
    \[
    H_U(\lambda_3) =
    \begin{bmatrix}
    7 - (-2 - 3i) & -6 & 9 & -2 \\
    6 & -5 - (-2 - 3i) & 6 & -1 \\
    -6 & 3 & -8 - (-2 - 3i) & 2
    \end{bmatrix} =
    \]
    \[
    = \begin{bmatrix}
    9 + 3i & -6 & 9 & -2 \\
    6 & -3 + 3i & 6 & -1 \\
    -6 & 3 & -6 + 3i & 2
    \end{bmatrix}
    \]

    Для всех трёх $\lambda_i$ выполнено:
    \[
    \operatorname{rank}(H_U(\lambda_i)) = 3
    \]

    А значит, все собственные числа (соответственно, и сама система) являются управляемыми.

    Найдем \textit{вещественную} Жорданову форму $\hat{A} = T^{-1}AT$ матрицы $A$ и соответствующую матрицу перехода $T$ к ней:
    \[ \hat{A} = \begin{bmatrix} 
    -2 & 0 & 0 \\ 
    0 & -2 & 3 \\ 
    0 & -3 & -2 
    \end{bmatrix}, \quad 
    T = \begin{bmatrix} 
        -1 & -1.5 & -0.5 \\
        0 & -1 & 0 \\
        1 & 1 & 0
    \end{bmatrix}
    \]

    Можно получить обратную матрицу $T$:
    \[ T^{-1} = \begin{bmatrix}   
        0 & 1 & 1 \\
        0 & -1 & 0 \\
        -2 & 1 & -2
    \end{bmatrix} \]

    Откуда:
    $$
    \hat{B} = T^{-1}B = \begin{bmatrix}
    1 \\ 1 \\ -1
    \end{bmatrix}
    $$

    В итоге система в жордановом базисе задается уравнением:
    \[
    \dot{\hat{x}} = \hat{A} \hat{x} + \hat{B} u
    \]

    Для каждой жордановой клетки последняя строка (а для мнимых чисел - хотя бы одна из двух последних строк) соответствующего блока в $\hat{B}$ ненулевая: элементы $\hat{B}_1 = \hat{B}_2 = 1 \neq 0$ и $\hat{B}_3 = -1 \neq 0$, а значит, система является полностью управляемой, равно как и все собственные числа матрицы $A$.

    Полная управляемость для линейных систем даёт нам возможность для любых состояний $x(0)$ и $x(t_1) = x_1$ найти такое управление $u(t)$, что система перейдёт из начальных условий $x(0)$ в $x_1$ за время $t = t_1$. Воспользуемся этим, чтобы перевести систему из состояния $x(0) = 0$ в $x(t_1) = x_1$ за время $t = t_1 = 3$. Из варианта задания получаем точку
    $$
    x_1 =
    \begin{bmatrix}
        -5 \\
        -3 \\
        3
    \end{bmatrix}
    $$

    Найдем грамиан управляемости относительно времени $t = t_1 = 3$:
    \[
    P(t_1) = \int_0^{t_1} e^{A t} B B^{T} e^{A^T t} dt =
    \]
    \[
    = \begin{bmatrix}
        0.3688436 & 0.20346076 & -0.37345989 \\
        0.20346076 & 0.13461396 & -0.17461489 \\
        -0.37345989 & -0.17461489 & 0.46461429
    \end{bmatrix} \approx
    \]
    \[
    \approx
    \begin{bmatrix}
        0.369 & 0.203 & -0.373 \\
        0.203 & 0.135 & -0.175 \\
        -0.373 & -0.175 & 0.465
    \end{bmatrix}
    \]

    В качестве дополнительного анализа управляемости найдем собственные числа матрицы $P(t_1)$:
    \[
    \lambda_1 = 0.88712357 \approx 0.887, \quad
    \lambda_2 = 0.00521594 \approx 0.005
    \]
    \[
    \lambda_3 = 0.07573234 \approx 0.076
    \]

    Так как Грамиан управляемости - симметричная матрица, а её собственные числа положительны, то $P(t_1)$ положительно определён (что также следует из полной управляемости), а значит, рассматриваемое состояние $x_1$ достижимо за время $t = t_1 = 3$, притом управление $u(t)$ можно найти по формуле:
    \[
    u(t) = B^T e^{A^T (t_1 - t)} (P(t_1))^{-1} x_1
    \]

    Для нахождения управления воспользуемся численными средствами. Среда MATLAB даёт следующие результаты:
    \[
        u(t) = \frac{11121409572095649}{281474976710656} e^{2t - 6} -
    \]
    \[   
    - \frac{1863583075832407 - 20684618866560801i}{562949953421312} e^{(2 - 3i)t - 6 + 9i}  - 
    \]
    \[
    - \frac{1863583075832407 + 20684618866560801i}{562949953421312} e^{(2 + 3i)t - 6 - 9i}
    \]

    График соответсвующего управления $u(t)$ при $x(0) = 0$ и $x(t_1) = x_1$ приведен на рисунке \ref{fig:lab1_3_1}, а график состояния системы $x(t)$ при $x(0) = 0$ и найденном $u(t)$ приведен на рисунке \ref{fig:lab1_3_2}. Видим, что система по итогу достигла заданного состояния $x(t_1) = x_1$, управление выполнило свою задачу.
    \begin{figure}
        \centering
        \includegraphics[width=0.95\textwidth]{images/lab1_3_1.png}
        \caption{Управление $u(t)$ при полной управляемости}
        \label{fig:lab1_3_1}
    \end{figure}
    \begin{figure}
        \centering
        \includegraphics[width=0.95\textwidth]{images/lab1_3_2.png}
        \caption{Состояние системы $x(t)$ при полной управляемости}
        \label{fig:lab1_3_2}
    \end{figure}
    
    В итоге система является управляемой, это подтверждается как критерием Калмана, так и исследованиями матриц Хаутуса для всех собственных чисел, которые по итогу аналогично являются управляемыми, и Жордановой формы матрицы системы $\hat{A}$ и матрицы управления $\hat{B}$. Грамиан управляемости в исследуемый момент времени $t = t_1 = 3$ также оказался положительно определен, что ещё раз подтвердило управляемость системы и позволило найти создать программное управление $u(t)$ по формуле, позволившее перевести систему из состояния $x(0) = 0$ в состояние $x(t_1) = x_1$ за необходимое время. Все результаты подтвердились моделированием.

    \section{Ещё одно исследование управляемости}
    Зададимся теперь точками 
    \[
    x_1' = \begin{bmatrix}
        -2 \\
        1 \\
        -1
    \end{bmatrix}, \quad x_1'' = \begin{bmatrix}
        -5 \\
        4 \\
        -1
    \end{bmatrix}.
    \]

    Проверим их на принадлежность управляемому подпространству всё той же линейной системы
    \[
    \dot{x} = Ax + Bu
    \]

    Матрицу $A$ оставим той же, $B$ изменим, получив:
        \begin{equation*}
    A = \begin{bmatrix}
        7 & -6 & 9 \\
        6 & -5 & 6 \\
        -6 & 3 & -8
    \end{bmatrix}, \quad
    B = \begin{bmatrix}
        2 \\
        1 \\
        -1
    \end{bmatrix}
    \end{equation*}

    Посмотрим, действительно ли изменения матрицы управления $B$ приводят к изменениям и в качестве управляемости системы. Для этого проведем уже известный анализ управляемости. Начнём с принадлежности заданных точек $x_1'$ и $x_1''$ управляемому подпространству. Для этого найдем матрицу управляемости системы и её ранг:
    \[
    U = \begin{bmatrix}
        B & AB & A^2B
    \end{bmatrix}
    =
    \begin{bmatrix}
        2 & -1 & -22 \\
        1 & 1 & -17 \\
        -1 & -1 & 17
    \end{bmatrix}
    \]
    \[
    \text{rank}(U) = 2
    \]
    
    Таким образом, система уже не является управляемой по критерию Калмана, а значит, существуют направления, на которые прямо воздействовать невозможно. Принадлежность заданных точек $x_1'$ и $x_1''$ управляемому подпространству можно проверить с помощью матрицы управляемости. Если точка является линейной комбинацией столбцов матрицы управляемости $U$ (иначе говоря, принадлежит линейной оболочке, натянутой на эти вектора, называемой управляемым подпространством), то ранг <<расширенной>> $U$ (вместе с рассматриваемой точкой) не изменится, а самих точек тогда можно будет прямо добиться из нулевого состояния некоторым управлением. Итак, нам нужно проверить ранги следующих матриц:
    \[
    U' = \begin{bmatrix}
        B & AB & A^2B & x_1'
    \end{bmatrix}
    =
    \begin{bmatrix}
        2 & -1 & -22 & -2 \\
        1 & 1 & -17 & 1 \\
        -1 & -1 & 17 & -1
    \end{bmatrix}
    \]
    \[
    U'' = \begin{bmatrix}
        B & AB & A^2B & x_1''
    \end{bmatrix}
    =
    \begin{bmatrix}
        2 & -1 & -22 & -5 \\
        1 & 1 & -17 & 4 \\
        -1 & -1 & 17 & -1
    \end{bmatrix}
    \]

    Получаем:
    \[
    \text{ rank}(U') = 2
    \]
    \[
    \text{ rank}(U'') = 3
    \]

    По итогу точка $x_1'$ принадлежит управляемому подпространству, а точка $x_1''$ не принадлежит. Далее будем считать, что $x_1 = x_1'$.

    Исследуем систему далее. Так как матрица $A$ не изменилась, то её собственные числа остались прежними:
    \[
    \lambda_1 = -2, \quad \lambda_2 = -2 + 3i, \quad \lambda_3 = -2 - 3i
    \]

    Построим для них матрицы Хаутуса:
    \[
    H_U(\lambda_1) = \begin{bmatrix} A - \lambda_1 I & B \end{bmatrix} = \begin{bmatrix}
    9 & -6 & 9 & 2 \\
    6 & -3 & 6 & 1 \\
    -6 & 3 & -6 & -1
    \end{bmatrix}
    \]
    \[
    H_U(\lambda_2) = \begin{bmatrix} A - \lambda_2 I & B \end{bmatrix} = \begin{bmatrix}
    9 - 3i & -6 & 9 & 2 \\
    6 & -3 - 3i & 6 & 1 \\
    -6 & 3 & -6 - 3i & -1
    \end{bmatrix}
    \]
    \[
    H_U(\lambda_3) = \begin{bmatrix} A - \lambda_3 I & B \end{bmatrix} = \begin{bmatrix}
    9 + 3i & -6 & 9 & 2 \\
    6 & -3 + 3i & 6 & 1 \\
    -6 & 3 & -6 + 3i & -1
    \end{bmatrix}
    \]

    Их ранги:
    \[
    \text{rank}(H_U(\lambda_1)) = 2
    \]
    \[
    \text{rank}(H_U(\lambda_2)) = \text{rank}(H_U(\lambda_3)) = 3
    \]

    Соотственно, собственное число $\lambda_1 = -2$ не управляемо, причем существование лишь одного такого числа согласуется с соотношением между рангом матрицы управляемости, равному двум, и размерностью пространства состояний, равной трем, - в итоге получилось лишь одно <<неконтролируемое>> направление.

    Вычисление системы в жордановом базисе остаётся прежним, лишь за исключением полученного $\hat{B}$:
    \[ \hat{A} = T^{-1}AT = \begin{bmatrix} 
    -2 & 0 & 0 \\ 
    0 & -2 & 3 \\ 
    0 & -3 & -2 
    \end{bmatrix}, \quad 
    T = \begin{bmatrix} 
        -1 & -1.5 & -0.5 \\
        0 & -1 & 0 \\
        1 & 1 & 0
    \end{bmatrix}
    \]

    Можно получить обратную матрицу $T$:
    \[ T^{-1} = \begin{bmatrix}   
        0 & 1 & 1 \\
        0 & -1 & 0 \\
        -2 & 1 & -2
    \end{bmatrix} \]

    Откуда:
    $$
    \hat{B} = T^{-1}B = \begin{bmatrix}
    0 \\ -1 \\ -1
    \end{bmatrix}
    $$

    В итоге компонент собственного числа $\lambda_1$ в жордановом базисе $\hat{B}_1 = 0$, а значит, это собственное число не управляемо. Компоненты $\hat{B}_2 = \hat{B}_3 = -1$ же отвечают за управляемость комплексно-сопряженных собственных чисел $\lambda_2$ и $\lambda_3$, хотя бы один из элементов $\hat{B}_2$ и $\hat{B}_3$ не равен нулю, а значит, соответсвующие им собственные числа управляемы. Таким образом, опять приходим к тому, что система является частично управляемой.

    Найдем Грамиан управляемости в момент времени $t = t_1 = 3$:
    \[
        P(t_1) = \int_0^{t_1} e^{A t} B B^{T} e^{A^T t} dt \approx
        \begin{bmatrix}
            0.971 & 0.587 & -0.587 \\
            0.587 & 0.365 & -0.365 \\
            -0.587 & -0.365 & 0.365
        \end{bmatrix}
    \]

    Его собственные числа:
    \[
    \lambda_1 = 1.68910631 \approx 1.689, \quad
    \lambda_2 = 0.0128081396 \approx 0.013
    \]
    \[
    \lambda_3 = 0
    \]

    Одно из собственных чисел оказалось равно 0, а значит грамиан управляемости не положительно определен. Это вносит некоторые коррективы в вычисление управления, переводящего систему из нулевых начальных условий $x(0) = 0$ в состояние $x(t_1) = x_1$ за всё то же желаемое время $t = t_1 = 3$. Так как теперь обратной матрицы от $P(t_1)$ не существует, нужно находить псевдообратную и вычислять программное управление уже по формуле:
    \[
    u(t) = B^T e^{A^T (t_1 - t)} (P(t_1))^{+} x_1 =
    \]
    \[
    =
    - \frac{3190087543359489 + 5379344853786521i}{140737488355328} e^{(2 - 3i)t - 6 + 9i} -
    \]
    \[
    -\frac{3190087543359489 - 5379344853786521i}{140737488355328} e^{(2 + 3i)t - 6 - 9i}
    \]


    Все соотвествующие вычисления управления $u(t)$ были произведены в MATLAB. График $u(t)$ при $x(0) = 0$ и $x(t_1) = x_1$ приведен на рисунке \ref{fig:lab1_4_1}, а график состояния системы $x(t)$ при нулевых начальных условиях и найденном управлении $u(t)$ приведен на рисунке \ref{fig:lab1_4_2}. Видим, что по итогу система достигла заданного состояния $x(t_1) = x_1$ за необходимое время $t_1 = 3$.
    \begin{figure}
        \centering
        \includegraphics[width=0.95\textwidth]{images/lab1_4_1.png}
        \caption{Управление $u(t)$ при неполной управляемости}
        \label{fig:lab1_4_1}
    \end{figure}
    \begin{figure}
        \centering
        \includegraphics[width=0.95\textwidth]{images/lab1_4_2.png}
        \caption{Состояние системы $x(t)$ при неполной управляемости}
        \label{fig:lab1_4_2}
    \end{figure}

    Таким образом, показали - изменения матрицы управления $B$ закономерно приводят к изменениям и в качестве управления системы. Изначально полностью управляемая система с матрицей $B = [-2,\ -1,\ 2]^{T}$ при её изменении на $[2,\ 1,\ -1]^{T}$ оказалась уже \textit{частично} управляемой как по критерию Калмана, так и по управлямости собственных чисел, вычислению соотвествующих матриц Хаутуса ($\lambda_1$ оказалось не управляемо, а $\lambda_2$ и $\lambda_3$ - управляемы). Также было получено, что $x_1'$ принадлежит управляемому подпространству, а $x_1''$ - нет. Грамиан управляемости в исследуемый момент времени $t = t_1 = 3$ оказался не положительно определён, что потребовало использования псевдообратной матрицы для вычисления управления $u(t)$, позволившего перевести систему с нулевыми начальными условиями в $x(t_1) = x_1$ за необходимое время. Все результаты подтвердились моделированием.

    \section{Исследование наблюдаемости}
    Рассмотрим систему:
    \[
    \begin{cases}
        \dot{x} = Ax \\
        y = Cx
    \end{cases}
    \]

    В соответствии с вариантом матрицы $A$ и $C$ заданы как:
    \begin{equation*}
        A = \begin{bmatrix}
            -21 & -38 & 6 \\
            8 & 13 & -4 \\
            -6 & -14 & -1
        \end{bmatrix}, \quad
        C = \begin{bmatrix}
            9 & 18 & -2
        \end{bmatrix}
    \end{equation*}

    Выполним анализ наблюдаемости. Для этого сначала вычислим матрицу наблюдаемости:
    \[
    V = \begin{bmatrix}
        C \\
        CA \\
        CA^2
    \end{bmatrix} = 
    \begin{bmatrix}
        9 & 18 & -2 \\
        -33 & -80 & -16 \\
        149 & 438 & 138
    \end{bmatrix}
    \]

    А после $\text{rank}(V) = 3$. Так как ранг матрицы наблюдаемости оказался равен размерности пространства состояний, то система является полностью наблюдаемой по критерию Калмана.

    Можно и иначе! Найдем собственные числа матрицы A:
    \[
    \lambda_1 = 1, \quad \lambda_2 = -5 + 2i, \quad \lambda_3 = -5 - 2i
    \]
    
    Исследуем их на наблюдаемость. Для этого вычислим соответсвующие матрицы Хаутуса:
    \[
    H_V(\lambda_i) = \begin{bmatrix} A - \lambda_i I \\ C \end{bmatrix}
    \]
    \[
    H_V(\lambda_1) = \begin{bmatrix}
        -21 - (1) & -38 & 6 \\
         8 & 13 - (1) & -4 \\
         -6 & -14 & -1 - (1) \\
         9 & 18 & -2 \\
         \end{bmatrix}
        =
        \begin{bmatrix}
        -22 & -38 & 6 \\
         8 & 12 & -4 \\
         -6 & -14 & -2 \\
         9 & 18 & -2 \\
    \end{bmatrix}
    \]
    \[
    H_V(\lambda_2) = \begin{bmatrix}
        -21 - (-5 + 2i) & -38 & 6 \\
         8 & 13 - (-5 + 2i) & -4 \\
         -6 & -14 & -1 - (-5 + 2i) \\
         9 & 18 & -2 \\
         \end{bmatrix} =
    \]
    \[
        =
        \begin{bmatrix}
            -16-2i & -38 & 6 \\
            8 & 18-2i & -4 \\
            -6 & -14 & 4-2i \\
            9 & 18 & -2 \\
    \end{bmatrix}
    \]
    \[
    H_V(\lambda_3) = \begin{bmatrix}
        -21 - (-5 - 2i) & -38 & 6 \\
         8 & 13 - (-5 - 2i) & -4 \\
         -6 & -14 & -1 - (-5 - 2i) \\
         9 & 18 & -2 \\
         \end{bmatrix} =
         \]
         \[
         =
         \begin{bmatrix}
            -16+2i & -38 & 6 \\
            8 & 18+2i & -4 \\
            -6 & -14 & 4+2i \\
            9 & 18 & -2 \\
         \end{bmatrix}
    \]

    Для всех трёх $\lambda_i$ выполнено:
    \[
    \text{rank}(H_V(\lambda_i)) = 3
    \]

    Таким образом, все собственные числа являются наблюдаемыми, а значит, сама система полностью наблюдаема.

    Проверим данное ещё раз через \textit{вещественную} Жорданову форму матрицы $A$ и соответствующую матрицу перехода $T$:
    \[
    \hat{A} = T^{-1}AT = \begin{bmatrix}
    1 & 0 & 0 \\
    0 & -5  & 2 \\
    0 & -2 & -5 \\
    \end{bmatrix}, \quad
    T = \begin{bmatrix}
    2 & 3 & 2 \\
    -1 & -1 & -1 \\
    1 & 1 & 0 \\
    \end{bmatrix}
    \]

    Можно получить обратную матрицу $T$:
    \[
    T^{-1} = \begin{bmatrix}
        -1	&-2	&1 \\
        1	&2	&0 \\
        0	&-1	&-1 \\
    \end{bmatrix}
    \]

    Откуда:
    $$
    \hat{C} = C T = \begin{bmatrix}
    9	&20	&11
    \end{bmatrix}
    $$

    Для каждой жордановой клетки первый столбец (а для мнимых чисел - хотя бы один из первых двух столбцов) соответствующего блока в $\hat{C}$ ненулевой, а значит, все собственные числа являются наблюдаемыми, система - полностью наблюдаемой.

    Для наблюдаемых систем возможно единственным образом восстановить начальное состояние по выходу, матрице наблюдаемости и матрице системы. Считая, что выход системы $y(t) = Cx(t)$ подчиняется закону $y(t) = f(t) = 3e^{-5t} \cos(2t) - e^{-5t} \sin(2t)$ на временном отрезке $t \in [0, t_1] = [0, 3]$, определим начальные условия системы. Для начала вычислим грамиан наблюдаемости относительно времени $t = t_1 = 3$:
    \[
    Q(t_1) = \int_0^{t_1} e^{A^T t} C^T C e^{A t} dt \approx   
    \begin{bmatrix}
    815.02  &1627.79  &-809.9 \\
    1627.79  &3251.44 &-1618.07\\
    -809.9  &-1618.07  &808
    \end{bmatrix}
    \]

    Его собственные числа:
    \[
    \lambda_1 = 4872.01653 \approx 4872.02, \quad
    \lambda_2 = 0.0570726112 \approx 0.057
    \]
    \[
    \lambda_3 = 2.37863082 \approx 2.38
    \]

    Так как все собственные числа положительны, а грамиан наблюдаемости - симметричная матрица, то $Q(t_1)$ положительно определён и существует обратная матрица (что также следует из полной наблюдаемости), а значит, восстановить начальное состояние, соответствующее выходу $y(t) = f(t)$, возможно по формуле:
    \[
    x(0) = (Q(t_1))^{-1} \int_0^{t_1} e^{A^T t} C^T y(t) dt = 
    \begin{bmatrix}
    2.83447514 \\
    5.35515718 \\
    -1.31378549
    \end{bmatrix} \approx
    \begin{bmatrix}
    2.834 \\
    5.355 \\
    -1.314
    \end{bmatrix}
    \]

    \begin{figure}
        \centering
        \includegraphics[width=0.95\textwidth]{images/lab1_5_comparison.png}
        \caption{График функции $y(t)$ и $f(t)$ при полной наблюдаемости}
        \label{fig:lab1_5_comparison}
    \end{figure}
    \begin{figure}
        \centering
        \includegraphics[width=0.95\textwidth]{images/lab1_5_error.png}
        \caption{График ошибки $e(t) = y(t) - f(t)$ при полной наблюдаемости}
        \label{fig:lab1_5_error}
    \end{figure}

    График функции $y(t)$ и $f(t)$ восстановленного начального состояния приведен на рисунке \ref{fig:lab1_5_comparison}, а график ошибки $e(t) = y(t) - f(t)$ восстановленного начального состояния приведен на рисунке \ref{fig:lab1_5_error}. Видим, что моделируемый выход системы $y(t) = Cx(t)$ и заданный $f(t)$ практически совпадают, а ошибка в целом \textit{крайне} мала.

    В итоге система является полностью наблюдаемой, это подтверждается как критерием Калмана, так и исследованиями матриц Хаутуса для всех (наблюдаемых)собственных чисел и Жордановой формы системы с матрицами $\hat{A}$ и $\hat{C}$. Грамиан наблюдаемости в исследуемый момент времени $t = t_1 = 3$ также оказался положительно определен, что ещё раз подтвердило наблюдаемость системы и позволило найти восстановленное начальное состояние системы по выходу. Все результаты подтвердились моделированием.

    \section{Ещё одно исследование наблюдаемости}
    Будем рассматривать всё ту же систему
    \[
    \begin{cases}
    \dot{x} = Ax\\
    y = Cx
    \end{cases}
    \]

    Матрицу $A$ оставим той же, $C$ изменим, получив:
    \[
    A = \begin{bmatrix}
        -21 & -38 & 6 \\
        8 & 13 & -4 \\
        -6 & -14 & -1
    \end{bmatrix}, \quad
    C = \begin{bmatrix}
        7 & 14 & 0
    \end{bmatrix}
    \]

    Проведем анализ наблюдаемости с этими изменениями. Для этого сначала вычислим матрицу наблюдаемости
    \[
    V = \begin{bmatrix}
    C \\
    CA \\
    CA^2
    \end{bmatrix} =
    7  14   0 \\
    -35 -84 -14 \\
    147 434 140
    \]

    Ранг матрицы наблюдаемости $\text{rank}(V) = 2$, а значит, система уже не является полностью наблюдаемой по критерию Калмана. Исследуем систему на наблюдаемость более детально. Найдем собственные числа матрицы $A$ системы:
    \[
    \lambda_1 = 1, \quad \lambda_2 = -5 + 2i, \quad \lambda_3 = -5 - 2i
    \]
    
    Для каждого собственного числа $\lambda_i$ построим матрицу Хаутуса:
    \[
    H_V(\lambda_1) = \begin{bmatrix} A - \lambda_1 I \\ C \end{bmatrix} = 
    \begin{bmatrix}
        -22 & -38 & 6 \\
        8 & 12 & -4 \\
        -6 & -14 & -2 \\
        7 & 14 & 0 \\
    \end{bmatrix}
    \]
    \[
    H_V(\lambda_2) = \begin{bmatrix}
        -16-2i & -38 & 6 \\
        8 & 18-2i & -4 \\
        -6 & -14 & 4-2i \\
        7 & 14 & 0 \\
    \end{bmatrix}
    \]
    \[
    H_V(\lambda_3) = \begin{bmatrix}
        -16+2i & -38 & 6 \\
        8 & 18+2i & -4 \\
        -6 & -14 & 4+2i \\
        7 & 14 & 0 \\
    \end{bmatrix}
    \]
    
    Их ранги:
    \[
    \text{rank}(H_V(\lambda_1)) = 2
    \]
    \[
    \text{rank}(H_V(\lambda_2)) = \text{rank}(H_V(\lambda_3)) = 3
    \]

    Таким образом, собственное число $\lambda_1 = 1$ не наблюдаемо, а $\lambda_{23}$ - наблюдаемы. Это порождает ненаблюдаемое подпространство системы, в данном случае - линейную оболочку над собственным вектором $v = [2, -1, 1]^{T}$, соответствующего $\lambda_1$. Вектор $v$ также порождает ядро матрицы наблюдаемости системы и качественно отличается от остальных векторов тем, что $CA^k v = 0$ для любого $k \le 2 = n - 1$ (или, что эквивалентно, при умножении на $C$ зануляется матричная экспонента от $A$, как раз и порождающая всю динамику системы и в любой момент времени раскладывающаяся в линейную комбинацию степеней матрицы $A$ до $n-1$ включительно, где \textit{$n$ - размерность пространства состояний системы}). 

    Так как существует только одно ненаблюдаемое собственное число, то система уже не является полностью наблюдаемой.

    Найдем теперь \textit{вещественную} Жорданову форму матрицы $A$ и матрицу перехода $T$ (матрица $A$ не изменилась - всё по-прежнему):
    \[
    \hat{A} = T^{-1}AT = \begin{bmatrix}
    1 & 0 & 0 \\
    0 & -5  & 2 \\
    0 & -2 & -5 \\
    \end{bmatrix}, \quad
    T = \begin{bmatrix}
    2 & 3 & 2 \\
    -1 & -1 & -1 \\
    1 & 1 & 0 \\
    \end{bmatrix}
    \]

    Можно получить обратную матрицу $T$:
    \[
    T^{-1} = \begin{bmatrix}
    -1	&-2	&1 \\
    1	&2	&0 \\
    0	&-1	&-1 \\
    \end{bmatrix}
    \]

    Откуда:
    $$
    \hat{C} = C T = \begin{bmatrix}
        0 & 7 & 0
    \end{bmatrix}
    $$

    Получили $\hat{C}_1 = 0$, а значит, собственное число $\lambda_1 = 1$ не наблюдаемо. Далее для двух мнимых собственных чисел $\lambda_2$ и $\lambda_3$ получили $\hat{C}_2 = 7$ и $\hat{C}_3 = 0$. Они являются комплексно-сопряженными, а значит, для наблюдаемости достаточно лишь чтобы хотя бы один из элементов $\hat{C}_2$ и $\hat{C}_3$ был ненулевым. Это выполняется, поэтому $\lambda_2$ и $\lambda_3$ наблюдаемы. Таким образом, система в очередной раз является \textit{частично} наблюдаемой.

    Так как система не является полностью наблюдаемой, то \textit{однозначно} восстановить начальное состояние по выходу, матрице наблюдаемости и матрице системы невозможно. Продемонстрируем это на примере.

    Возьмем всё тот же выход системы $y(t) = f(t) = 3e^{-5t} \cos(2t) - e^{-5t} \sin(2t)$ на временном отрезке $t \in [0, t_1] = [0, 3]$. Определим начальные условия системы, которые могли бы породить такой выход. Для начала вычислим грамиан наблюдаемости системы $Q(t)$ относительно времени $t = t_1 = 3$:
    \[
    Q(t_1) = \int_0^{t_1} e^{A^T t} C^T C e^{A t} dt
    \approx
    \begin{bmatrix}
        4.562 & 8.279 & -0.845 \\
        8.279 & 15.207 & -1.352 \\
        -0.845 & -1.352 & 0.338 \\
    \end{bmatrix}
    \]

    Его собственные числа:
    \[
    \lambda_1 = 19.8567251 \approx 19.857, \quad
    \lambda_2 = 0.250171475 \approx 0.25
    \]
    \[
    \lambda_3 \approx 0
    \]

    Одно из собственных чисел оказалось равно 0, а значит грамиан наблюдаемости не положительно определен. Начальное состояние, порождающее заданный выход $f(t)$, в таком случае можно вычислить через псевдообратную матрицу от $Q(t_1)$ по формуле:
    \[
    x(0) = (Q(t_1))^{+} \int_0^{t_1} e^{A^T t} C^T y(t) dt = \begin{bmatrix}
    0.0952381 \\
    0.16666667 \\
    -0.02380952
    \end{bmatrix}
    \approx
    \begin{bmatrix}
    0.095 \\
    0.167 \\
    -0.024
    \end{bmatrix}
    \]

    Альтернативные состояния системы, порождающие тот же выход $f(t)$, можно строить, добавляя произвольную линейную комбинацию из векторов ненаблюдаемого подпространства системы, так как:
    \[
    CA^k (x(t) + \alpha v) = CA^k x(t) + \alpha CA^k v = CA^k x(t) \text{ для любого } k \le 2
    \]
    
    Здесь работает всё то же зануление умножения матричной экспоненты от $A$ на матрицу $C$ с логикой, озвученной для вектора $v$ выше. Пусть $\alpha = 1$, тогда альтернативное начальное состояние системы при $v = [2, -1, 1]^{T}$, соответствующем ненаблюдаемому $\lambda_1$, будет:
    \[
    x'(0) = x(0) + v = \begin{bmatrix}
    0.095 \\
    0.167 \\
    -0.024
    \end{bmatrix} + \begin{bmatrix}
    2 \\
    -1 \\
    1
    \end{bmatrix} \approx \begin{bmatrix}
    2.095 \\
    -0.833 \\
    0.976
    \end{bmatrix}
    \]

    Во все моменты времени $t \in [0, t_1]$ $x(0)$ и $x'(0)$ будут давать одинаковые выходы $f(t)$, так как динамика от $v$ не будет влиять на выход $y(t) = Cx(t)$ системы.

    Аналогично можно найти:
    \[
    x''(0) = x(0) + 5v = \begin{bmatrix}
    0.095 \\
    0.167 \\
    -0.024
    \end{bmatrix} + 5 \begin{bmatrix}
    2 \\
    -1 \\
    1
    \end{bmatrix} \approx \begin{bmatrix}
    10.095 \\
    -4.833 \\
    4.976
    \end{bmatrix}
    \]

    \begin{figure}
        \centering
        \includegraphics[width=0.95\textwidth]{images/lab1_7_x1_comparison.png}
        \caption{Графики компоненты $x_1(t)$ состояния системы $x(t)$}
        \label{fig:lab1_7_x1_comparison}
    \end{figure}
    \begin{figure}
        \centering
        \includegraphics[width=0.95\textwidth]{images/lab1_7_x2_comparison.png}
        \caption{График компоненты $x_2(t)$ состояния системы $x(t)$}
        \label{fig:lab1_7_x2_comparison}
    \end{figure}
    \begin{figure}
        \centering
        \includegraphics[width=0.95\textwidth]{images/lab1_7_x3_comparison.png}
        \caption{График компоненты $x_3(t)$ состояния системы $x(t)$}
        \label{fig:lab1_7_x3_comparison}
    \end{figure}
    \begin{figure}
        \centering
        \includegraphics[width=0.95\textwidth]{images/lab1_7_all_outputs.png}
        \caption{Графики выхода системы $y(t)$ при различных начальных условиях}
        \label{fig:lab1_7_all_outputs}
    \end{figure}
    Проведем моделирование системы с начальными условиями $x(0)$, $x'(0)$ и $x''(0)$ и сравним выходы системы. Результаты приведены на рисунках \ref{fig:lab1_7_x1_comparison}, \ref{fig:lab1_7_x2_comparison} и \ref{fig:lab1_7_x3_comparison} для компонент состояния $x(t)$, а на рисунке \ref{fig:lab1_7_all_outputs} для выхода системы $y(t)$.


    
    Можем видеть, что графики компонент состояния $x(t)$ при различных начальных условиях полностью различны, а выход системы $y(t)$ при различных начальных условиях практически совпадают.

    Таким образом, система оказалось частично наблюдаемой, причем ненаблюдаемое подпространство порождается собственным вектором $v = [2, -1, 1]^{T}$, соответствующим ненаблюдаемому $\lambda_1 = 1$, что позволиво найти неоднозначное начальное состояние, порождающее тот же выход данный выход $f(t)$. $x(0)$ было найдено через псевдообратную матрицу, так как грамиан наблюдаемости оказался не положительно определен, а $x'(0)$ и $x''(0)$ были найдены путем добавления к нему вектора $v$. Также провелось моделирование системы с различными начальными условиями, подтвердившее все полученные выводы и результаты.

    \section{Управляемость по выходу}
    Рассмотрим систему
    \[
    \begin{cases}
    \dot{x} = Ax + Bu \\
    y = Cx + Du
    \end{cases}
    \]

    В соответствии с вариантом, матрицы $A$, $B$ и $C$ имеют вид:
    \[
    A = \begin{bmatrix}
        7 & -6 & 9 \\
        6 & -5 & 6 \\
        -6 & 3 & -8
    \end{bmatrix}, \quad
    B = \begin{bmatrix}
        1 \\
        0 \\
        0
    \end{bmatrix}, \quad
    C = \begin{bmatrix}
        0 & -1 & 1 \\
        0 & -3 & 0
    \end{bmatrix}
    \]

    Найдем \textit{вещественную} Жорданову форму системы и матрицу $T$ для перехода к ней:
    \[
    \hat{A} = T^{-1}AT = \begin{bmatrix}
    -2 & 0 & 0 \\
    0 & -2 & 3 \\
    0 & -3 & -2 \\
    \end{bmatrix}, \quad
    T = \begin{bmatrix}
    -1 & -1.5 & -0.5 \\
    0 & -1 & 0 \\
    1 & 1 & 0
    \end{bmatrix}
    \]

    Можно получить обратную матрицу $T$:
    \[
    T^{-1} = \begin{bmatrix}
    0 & 1 & 1 \\
    0 & -1 & 0 \\
    -2 & 1 & -2
    \end{bmatrix}
    \]

    Откуда:
    $$
    \hat{B} = T^{-1}B = \begin{bmatrix}
        0 \\
        0 \\
       -2
    \end{bmatrix}, \quad
    \hat{C} = CT = \begin{bmatrix}
        1 & 2 & 0 \\
        0 & 3 & 0 \\
    \end{bmatrix}
    $$

    Тогда система в жордановом базисе задается уравнениями:
    \[
    \begin{cases}
    \dot{\hat{x}} = \hat{A} \hat{x} + \hat{B} u \\
    y = \hat{C} \hat{x} + D u
    \end{cases}
    \]

    Воспользуемся полученными матрицами $\hat{A}$, $\hat{B}$ и $\hat{C}$ для анализа управляемости и наблюдаемости системы.
    
    Исходя из матрицы $\hat{B}$, делаем вывод, что первое собственное число не управляемо, а второе и третье управляемы, так как $\hat{B}_1 = 0$, а для пары комплексно-сопряженных собственных чисел хотя бы один из $\hat{B}_2$ и $\hat{B}_3$ не равен нулю. Это также даёт, что система является лишь частично управляемой.

    Исходя из матрицы $\hat{C}$, делаем вывод, что все собственные числа наблюдаемы, так как столбец $\hat{C}_1 = [1, 0]^T \neq 0$, соответствующий первому собственному числу, а для пары комплексно-сопряженных собственных чисел хотя бы один из столбцов $\hat{C}_2$ и $\hat{C}_3$ не равен нулю ($\hat{C}_2 = [2, 3]^T \neq 0$). Из всего этого ткже делаем вывод о том, что система является полностью наблюдаемой.

    Рассмотрим систему с точки зрения качества управляемости по выходу, означающей способность системы достичь любого выходного сигнала $y(t)$ вне зависимости от того, на какиме составляющие вектора состояния $[x_1, \dots, x_n]^{T}$ мы можем воздействовать, а на какие - нет. То есть уже рассматривается качество управления именно на <<проекцию>> системы на выход, а не на все её внутренности целиком, которые в рассматриваемом случае, как было получено ранее, являются лишь частично управляемыми.

    Проверка управляемости по выходу подразумевает под нахождение ранга матрицы $U_{out} = [CU\ D]$, где $U$ - матрица управляемости. Пусть мы не воздействуем на выход через управление напрямую, тогда $D = 0_{2\times 1}$ и матрица принимает вид:
    \[
    U_{out} = \begin{bmatrix}
        0 &-12 & 48 & 0 \\
        0 &-18 & 72 & 0
    \end{bmatrix}
    \]
    
    Можем видеть, что $\text{rank}(U_{out}) = 1 < 2$ = размерность выхода. Это означает, что система не является управляемой по выходу. Заметим, что данное можно было получить ещё на шаге вычисления матриц в жордановом базисе, так как $\hat{C}_1 \neq 0$, а $\hat{B}_1 = 0$ (наблюдаем первое собственное число, но не влияем на него - не контролируем выход полностью).

    Матрица $D$ помогает воздействовать на выход напрямую, без посредника в виде вектора-состояния. Получается, правильно задав $D$, можно добиться управляемости по выходу, пусть система и будет по-прежнему частично управляемой.

    По критерию управляемости для выхода необходимо, чтобы ранг матрицы $U_{out} = [CU\ D]$ был равен размерности выхода ($= 2$ в нашем случае). То есть достаточно заменить $D = 0_{2\times 1}$, проявляющуюся в последнем столбце матрицы $U_{out}$, на линейно-независимый вектор от $[-12, -18]^T$. Можно взять любой, удовлетворяющий этому условию, так что вектор $D' = [1, 1]^T$ будет отлично подходить, давая по итогу управляемую систему по выходу.

    В итоге для обеспечения полной управляемости по выходу необходимо использовать матрицу $D$, увеличивающую ранг матрицы $U_{out}$ до размерности выхода. При $D \neq 0$ вход системы $u(t)$ начинает оказывать прямое влияние на $y(t)$, помогая увеличивать качество управления по выходу.

    \newpage
	\section{Общие выводы}
	В результате выполнения лабораторной работы были проведены исследования управляемости и наблюдаемости систем. В первом пункте получено, что система является полностью управляемой, что дало возможность построить управление, переводящее систему из любого состояния $x(0)$ в любое другое $x(t_1) = x_1$ за время $t = t_1$. Во втором пункте получено, что изменения матрицы управления $B$ действительно влияют на качество управляемости системы, пришлось изменить подход (использовать псевдообратные матрицы) к вычислению управления для достижения состояния $x_1$ из управляемого подпространства в заданное время $t_1$.

    В третьем пункте получено, что система является полностью наблюдаемой, что дало возможность восстановить начальное состояние системы по выходу и матрице наблюдаемости. В четвертом пункте получено, что система является частично наблюдаемой, что дало возможность найти в каком-то смысле эквивалентные начальные состояния, порождающие один и тот же выход $f(t)$.

    В пятом пункте получено, что система является неуправляемой по выходу, были приведены причины этого и найдена матрица $D'$, всё же позволяющая добиться управляемости по выходу.

    
\end{document}