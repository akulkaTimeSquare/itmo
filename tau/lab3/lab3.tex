\documentclass[a4paper,hidelinks,14pt]{extarticle}

\usepackage[utf8]{inputenc}
\usepackage[T2A]{fontenc}
\usepackage[english, russian]{babel}
\usepackage{lipsum}
\usepackage{amsmath}
\usepackage{amssymb}
\usepackage{amsfonts}
\usepackage{mathtools}
\usepackage{datetime}
\usepackage[pdftex]{graphicx}
\usepackage{indentfirst}
\usepackage{asymptote}
\usepackage{systeme}
\usepackage[dvipsnames]{xcolor}
\usepackage{lastpage}
\usepackage{fancybox,fancyhdr}
\usepackage{hyperref}
\usepackage[font={small,it}]{caption}

\usepackage{titlesec}

% Секция: делаем шрифт не таким большим
\titleformat{\section}[block]
  {\large\bfseries}   % было \LARGE, ставим \Large (чуть меньше, но жирный остался)
  {\thesection}{1em}{} 



\fancyhead[L]{ЛР №3}
\fancyhead[C]{}
\fancyhead[R]{\textit{Регуляторы с заданной степенью устойчивости}}
\fancyfoot[L]{}
\fancyfoot[C]{Страница \thepage\space из \pageref{LastPage}}
\fancyfoot[R]{}
\pagestyle{fancy}
\newcommand{\gt}{\textgreater}
\newcommand{\lt}{\textless}
\let\oldemptyset\emptyset
\let\emptyset\varnothing

\begin{document}
	\begin{titlepage}
		\setlength{\parindent}{0ex}
		
		\begin{center}
			\textsc{
				\vspace{1ex}
				Научно исследовательский университет ИТМО \\
				\vspace{0.5ex}
				Факультет систем управления и робототехники \\
				\vspace{0.5ex}
			}
		\end{center}
		
		\vspace{50mm}
		
		\begin{center}
			Отчет по лабораторной работе №3 \\
			Регуляторы с заданной степенью устойчивости \\
            Вариант 11
		\end{center}
		
		\vspace{50mm}
		
		\begin{minipage}{.48\linewidth}
			Выполнил студент группы R3380
			
			Преподаватель
		\end{minipage}
		\hfill
		\begin{minipage}{.5\linewidth}
			\begin{flushright}
				Мовчан Игорь Евгеньевич
				\\
				Пашенко Артем Витальевич
			\end{flushright}
		\end{minipage}
		
		\vfill
		\begin{center}
			Санкт-Петербург
			\\
			2025
		\end{center}
		
	\end{titlepage}

	\tableofcontents
	\clearpage
	
	\section{Регулятор с заданной степенью устойчивости}
    Рассмотрим линейную систему
    \[
        \dot{x} = Ax + Bu
    \]

    В соответствии с вариантом, матрицы $A$ и $B$ имеют вид:
    \[
        A = \begin{bmatrix}
            11 & -2 & 13 \\
            6 & -1 & 6 \\
            -6 & -1 & -8
        \end{bmatrix}, \quad
        B = \begin{bmatrix}
            2 \\
            0 \\
            0
        \end{bmatrix}
    \]

    Выполним анализ управляемости системы и её собственных чисел. Для этого используем Жорданову форму $\hat{A} = T^{-1}AT$ матрицы $A$, имеющей собственными числами $\lambda_1 = -2$ и $\lambda_{23} = 2 \pm 3i$, а также вспомогательную матрицу $T$ для перехода к ней:
    \[
        \hat{A} = \begin{bmatrix}
            -2 & 0 & 0 \\
            0 & 2 & 3 \\
            0 & -3 & 2
        \end{bmatrix}, \quad
    T = \begin{bmatrix}
        -1 & -1.5 & -0.5 \\
        0 & -1 & 0 \\
        1 & 1 & 0
    \end{bmatrix}
    \]
    
    Можно получить обратную матрицу к $T$:
    \[
        T^{-1} = \begin{bmatrix}
            0 & 1 & 1 \\
            0 & -1 & 0 \\
            -2 & 1 & -2
        \end{bmatrix}
    \]

    Откуда:
    \[
        \hat{B} = T^{-1}B = \begin{bmatrix}
            0 \\
            0 \\
            -4
        \end{bmatrix}
    \]

	Заметим, что хоть элемент $\hat{B}_2$ матрицы $\hat{B}$, соответствующий одному из мнимых собственных чисел, и равен нулю, он не влияет на управляемость $\lambda_2$, так как та достигается через сопряженное $\lambda_3$.

	Также отметим, что $\hat{B}_1 = 0$, а значит, $\lambda_1$ не управляемо. Вся же система, таким образом, является частично управляемой, но стабилизируемой, так как единственное неуправляемое собственное число $\lambda_1$ лежит в левой полуплоскости - имеет отрицательную вещественную часть, а значит, является асимптотически устойчивым.

	Замкнем систему статическим регулятором:
	\[
		u = Kx
	\]

	Матрица $K$ коэффициентов обратной связи выбирается так, чтобы собственные числа уже замкнутой системы с матрицей $A + BK$ были устойчивыми.

	Схема моделирования системы с таким регулятором приведена на рисунке \ref{fig:model}.

    \begin{figure}
        \centering
        \includegraphics[width=0.9\textwidth]{images/reg_scheme.png}
        \caption{Схема моделирования системы с статическим регулятором}
        \label{fig:model}
    \end{figure}

    Неуправляемость $\lambda_1$ накладывает ограничение на максимальную степень устойчивости $\alpha_{max}$, достижимую с помощью статического регулятора. В данном случае, $\alpha_{max}$ равна 2, так как это минимальный модуль вещественных частей неуправляемых собственных чисел ($\lambda_1 = -2 \Rightarrow \alpha_{max} = |\Re(\lambda_1)| = 2$).

    Зададимся теперь \textit{достижимыми} значениями желаемой степени устойчивости, исходя из $\alpha_{max} = 2$:
    \[
        \alpha_1 = 0.5, \quad \alpha_2 = 1.25, \quad \alpha_3 = 2
    \]

    И для каждого из значений $\alpha_i$ выполним синтез регулятора. 
    
    \textbf{Начнём с $\alpha_1 = 0.5$.}
    
    Пусть \textit{не задано ограничений на управление}. Тогда задача сводится к решению матричного неравенства типа Ляпунова относительно $P \succ 0$ и $Y$, а после вычислению матрицы обратной связи $K$:
    \[
        P A^T + AP + 2\alpha P + Y^T B^T + BY \preceq 0, \quad K = YP^{-1}
    \]

    Итак, для $\alpha = \alpha_1 = 0.5$ получаем решение неравенства:
    \[
        P_{1, 1} = \begin{bmatrix}
            5.7102 & 0.4622 & -1.4882  \\
            0.4622 & 4.4696 & -0.6161  \\
            -1.4882 & -0.6161 & 1.7273
        \end{bmatrix}, \quad Y_{1, 1} = \begin{bmatrix}
            -24.0344 \\ -6.7335 \\  8.4948
        \end{bmatrix}^T
    \]

    Откуда:
    \[
        K_{1, 1} = \begin{bmatrix}
            -3.7894 &  -0.9327 & 1.3204
        \end{bmatrix}
    \]

    Выполним проверку корректности синтеза, для этого найдем матрицу замкнутой системы:
    \[
        A + BK_{1, 1} = \begin{bmatrix}
            3.4211 & -3.8653 & 15.6407  \\
            6.0000 & -1.0000 & 6.0000  \\
            -6.0000 & -1.0000 & -8.0000 
        \end{bmatrix}
    \]

    Она имеет спектром
    \[
    \sigma(A + BK_{1, 1}) =  \{-2, -1.7894 \pm 9.4808i\}
    \]

    У всех собственных чисел вещественная часть меньше $\alpha_1$, что соответствует заданной устойчивости - синтез проведен корректно. 
    
    Заметим однако, что значения спектра получились достаточно отдаленными от заданного $\alpha_1$ (слишком <<большими>>, из-за чего переходные процессы протекают быстрее). Как известно из лабораторной работы 2, это приводит к скачкам значений в управлении, чего, возможно, хотелось бы избежать. Поэтому дополнительно исследуем ещё и задачу \textit{минимизации управления}. К вышеперечисленному синтезу добавим условия (параметр $\gamma$ минимизируется, $x_0 = [1, 1, 1]^T$ - начальное состояние системы, если не знаем - любое)
    \[
        \begin{bmatrix}
            P & Y^T \\
            Y & \gamma I
        \end{bmatrix} \succ 0, \quad
        \begin{bmatrix}
            P & x_0 \\
            x_0^T & I
        \end{bmatrix} \succ 0, \quad ||u(t)|| \leq \sqrt{\gamma} \text{\ \ для } x_0
    \]

    Итак, для $\alpha_1 = 0.5$ при решении неравенства Ляпунова совместно с этим условием получаем:
    \[
        P_{2, 1} = \begin{bmatrix}
            1508.3683  & 7.7509  & -1505.3764  \\
            7.7509  & 10.3940  & -6.8852  \\
            -1505.3764  & -6.8852  & 1506.5237
        \end{bmatrix}, \quad Y_{2, 1} = \begin{bmatrix}
            -8.5532 \\ 1.0113  \\ -1.5315
        \end{bmatrix}^T
    \]

    Откуда:
    \[
        K_{2, 1} = \begin{bmatrix}
            -2.5013  &  0.3071  & -2.4990
        \end{bmatrix}
    \]

    Выполним проверку корректности синтеза, для этого найдем матрицу замкнутой системы:
    \[
        A + BK_{2, 1} = \begin{bmatrix}
            5.9974  & -1.3857  & 8.0019  \\
            6.0000  & -1.0000  & 6.0000  \\
            -6.0000  & -1.0000  & -8.0000
        \end{bmatrix}
    \]

    Она имеет спектром
    \[
        \sigma(A + BK_{2, 1}) =  \{-2, -0.5 \pm 3.7541i\}
    \]
    
    Все управляемые собственные числа имеют вещественную часть, \textit{равную} $\alpha_1$, а для неуправляемого $\lambda_1$ выполнено $\Re(\lambda_1) < \alpha_1$. Это соответствует желаемой степени устойчивости, а значит, синтез проведен корректно. Важно, что вследствии решаемой задачи, вещественная часть спектра была <<прижата>> к минимально допустимому значению $\alpha_1$. Отметим, что \textit{минимизация} управления в такой постановке может быть рассмотрена только при известных начальных условиях, в остальном $x_0$ берется любым, решением достигаются близкие к заданной степени устойчивости вещественные части спектра замкнутой системы, и как следствие, более низкие значения управления.

    Проведем моделирование системы с полученными регуляторами при начальных условиях $x(0) = [1, 1, 1]^T$. На рисунках \ref{fig:reg_x_1_1} и \ref{fig:reg_x_2_1} представлены векторы состояния с матрицами обратной связи $K_{1,1}$ и $K_{2,1}$, а на рисунке \ref{fig:reg_u_K11_K21} - соответствующие управления системой.

    \begin{figure}
        \centering
        \includegraphics[width=0.8\textwidth]{images/reg_x_1_1.png}
        \caption{Вектор состояния при заданной устойчивости $\alpha_1 = 0.5$}
        \label{fig:reg_x_1_1}
    \end{figure}
    \begin{figure}
        \centering
        \includegraphics[width=0.8\textwidth]{images/reg_x_2_1.png}
        \caption{Вектор состояния при $\alpha_1 = 0.5$ и минимизации управления}
        \label{fig:reg_x_2_1}
    \end{figure}
    \begin{figure}
        \centering
        \includegraphics[width=0.8\textwidth]{images/reg_u_K11_K21.png}
        \caption{Управления системой с регуляторами $K_{1,1}$ и $K_{2,1}$}
        \label{fig:reg_u_K11_K21}
    \end{figure}
    
    Можем видеть, что при минимизации управления, скорость сходимости к нулю замедляется более чем в $2.5$ раза, при этом величина самого управления уменьшилась почти в $2$ раза. Все ранее сделанные выводы ещё раз подтвердились.

    \newpage
    \textbf{Рассмотрим теперь случай $\alpha_2 = 1.25$.}

    Пусть \textit{не задано ограничений на управление}. Тогда решением матричного неравенства типа Ляпунова относительно $P \succ 0$ и $Y$ при $\alpha = \alpha_2 = 1.25$ является:
    \[
        P_{1, 2} = \begin{bmatrix}
            5.6978  & -0.2749  & -1.6787  \\
            -0.2749  & 5.3479  & -0.3439  \\
            -1.6787  & -0.3439  & 1.9520
        \end{bmatrix}, \quad Y_{1, 2} = \begin{bmatrix}
            -25.5207  \\ -2.7559 \\  8.5408
        \end{bmatrix}^T
    \]

    Откуда:
    \[
        K_{1, 2} = Y P^{-1} = \begin{bmatrix}
            -4.3675   & -0.7080   & 0.4946
        \end{bmatrix}
    \]

    Выполним проверку корректности синтеза, для этого найдем матрицу замкнутой системы:
    \[
        A + BK_{1, 2} = \begin{bmatrix}
            2.2650  & -3.4161  & 13.9893  \\
            6.0000  & -1.0000  & 6.0000  \\
            -6.0000  & -1.0000  & -8.0000
        \end{bmatrix}
    \]

    Она имеет спектром
    \[
        \sigma(A + BK_{1, 2}) =  \{-2, -2.3675 \pm 9.1089i\}
    \]

    Все собственные числа имеют вещественную часть, меньшую $\alpha_2$, а значит, синтез проведен корректно.

    Решим теперь задачу нахождения регулятора, обеспечивающего экспоненциальную устойчивость с $\alpha_2 = 1.25$, совместно с \textit{минимизацией управления}. Матрицы $P$ и $Y$ принимают вид:
    \[
        P_{2, 2} = \begin{bmatrix}
            1416.3035  & 4.9330  & -1413.0401  \\
            4.9330  & 9.6008  & -5.3336  \\
            -1413.0401  & -5.3336  & 1414.4506 
        \end{bmatrix}, \quad Y_{2, 2} = \begin{bmatrix}
            -15.9035  \\ 1.6835  \\ 0.6955
        \end{bmatrix}^T
    \]

    Откуда:
    \[
        K_{2, 2} = Y P^{-1} = \begin{bmatrix}
            -3.2513  & 0.0418  & -3.2474
        \end{bmatrix}
    \]
    
    Выполним проверку корректности синтеза, для этого найдем матрицу замкнутой системы:
    \[
        A + BK_{2, 2} = \begin{bmatrix}
            4.4974  & -1.9163  & 6.5051  \\
            6.0000  & -1.0000  & 6.0000  \\
            -6.0000  & -1.0000  & -8.0000 
        \end{bmatrix}
    \]

    Она имеет спектром
    \[
        \sigma(A + BK_{2, 2}) =  \{-2, -1.25 \pm 4.1811i\}
    \]

    Все управляемые собственные числа имеют вещественную часть, \textit{равную} $\alpha_2$, а для неуправляемого $\lambda_1$ выполнено $\Re(\lambda_1) < \alpha_2$. Это соответствует заданной желаемой степени устойчивости, а значит, синтез проведен корректно.
    
    \begin{figure}
        \centering
        \includegraphics[width=0.8\textwidth]{images/reg_x_1_2.png}
        \caption{Вектор состояния при $\alpha_2 = 1.25$}
        \label{fig:reg_x_1_2}
    \end{figure}
    \begin{figure}
        \centering
        \includegraphics[width=0.8\textwidth]{images/reg_x_2_2.png}
        \caption{Вектор состояния при $\alpha_2 = 1.25$ и минимизации управления}
        \label{fig:reg_x_2_2}
    \end{figure}
    \begin{figure}
        \centering
        \includegraphics[width=0.8\textwidth]{images/reg_u_K12_K22.png}
        \caption{Управления системой с регуляторами $K_{1,2}$ и $K_{2,2}$}
        \label{fig:reg_u_K12_K22}
    \end{figure}

    Проведем моделирование системы с полученными регуляторами при начальных условиях $x(0) = [1, 1, 1]^T$. На рисунках \ref{fig:reg_x_1_2} и \ref{fig:reg_x_2_2} представлены векторы состояния с матрицами обратной связи $K_{1,2}$ и $K_{2,2}$, а на рисунке \ref{fig:reg_u_K12_K22} - соответствующие управления системой.

    \textbf{Наконец, возьмем степень устойчивости $\alpha_3 = \alpha_{max} = 2$.}
    Пусть \textit{не задано ограничений на управление}. Тогда решением матричного неравенства типа Ляпунова относительно $P \succ 0$ и $Y$ при $\alpha = \alpha_3 = 2$ является:
    \[
        P_{1, 3} = \begin{bmatrix}
            7.3122  & -0.5963  & -2.2882  \\
            -0.5963  & 5.9556  & -0.6526  \\
            -2.2882  & -0.6526  & 2.6532
        \end{bmatrix}, \quad Y_{1, 3} = \begin{bmatrix}
            -34.7384 \\ -0.7010 \\  11.7490
        \end{bmatrix}^T
    \]

    Откуда:
    \[
        K_{1, 3} = Y P^{-1} = \begin{bmatrix}
            -4.7323   & -0.5688   & 0.2071
        \end{bmatrix}
    \]

    Выполним проверку корректности синтеза, для этого найдем матрицу замкнутой системы:
    \[
        A + BK_{1, 3} = \begin{bmatrix}
            1.5354  & -3.1377  & 13.4142  \\
            6.0000  & -1.0000  & 6.0000  \\
            -6.0000  & -1.0000  & -8.0000
        \end{bmatrix}
    \]

    Она имеет спектром
    \[
        \sigma(A + BK_{1, 3}) =  \{-2, -2.7323 \pm 9.0054i\}
    \]
    
    Все собственные числа имеют вещественную часть, меньшую либо равную $\alpha_3$, а значит, синтез проведен корректно.
    
    Решим теперь задачу нахождения регулятора, обеспечивающего экспоненциальную устойчивость с $\alpha_3 = 2$ и минимизацию управления. Матрицы $P$ и $Y$ принимают вид:
    \[
        P_{2, 3} = \begin{bmatrix}
            1361.0988  & 2.9343  & -1356.2851  \\
            2.9343  & 8.8454  & -4.4085  \\
            -1356.2851  & -4.4085  & 1357.0198
        \end{bmatrix}, \quad Y_{2, 3} = \begin{bmatrix}
            -28.3550 \\ 2.5196  \\ 6.7239 
        \end{bmatrix}^T
    \]

    Откуда:
    \[
        K_{2, 3} = Y P^{-1} = \begin{bmatrix}
            -4.0000 &  -0.3789 & -3.9941
        \end{bmatrix}
    \]

    Выполним проверку корректности синтеза, для этого найдем матрицу замкнутой системы:
    \[
        A + BK_{2, 3} = \begin{bmatrix}
            3.0000  & -2.7578  & 5.0118  \\
            6.0000  & -1.0000  & 6.0000  \\
            -6.0000  & -1.0000  & -8.0000
        \end{bmatrix}
    \]

    Она имеет спектром
    \[
        \sigma(A + BK_{2, 3}) =  \{-2, -2 \pm 4.6494i\}
    \]
    
    Все собственные числа имеют вещественную часть, \textit{равную} $\alpha_3$. Данное достиглось за счет $\Re(\lambda_1) = \alpha_{max} = \alpha_3$. Это соответствует заданной желаемой степени экспоненциальной устойчивости, а значит, синтез проведен корректно.
    
    \begin{figure}
        \centering
        \includegraphics[width=0.8\textwidth]{images/reg_x_1_3.png}
        \caption{Вектор состояния при $\alpha_3 = 2$}
        \label{fig:reg_x_1_3}
    \end{figure}
    \begin{figure}
        \centering
        \includegraphics[width=0.8\textwidth]{images/reg_x_2_3.png}
        \caption{Вектор состояния при $\alpha_3 = 2$ и минимизации управления}
        \label{fig:reg_x_2_3}
    \end{figure}
    \begin{figure}
        \centering
        \includegraphics[width=0.8\textwidth]{images/reg_u_K13_K23.png}
        \caption{Управления системой с регуляторами $K_{1,3}$ и $K_{2,3}$}
        \label{fig:reg_u_K13_K23}
    \end{figure}

    Проведем моделирование системы с полученными регуляторами при начальных условиях $x(0) = [1, 1, 1]^T$. На рисунках \ref{fig:reg_x_1_3} и \ref{fig:reg_x_2_3} представлены векторы состояния с матрицами обратной связи $K_{1,3}$ и $K_{2,3}$, а на рисунке \ref{fig:reg_u_K13_K23} - соответствующие управления системой. 

    Таким образом, во всех случаях была достигнута желаемая степень устойчивости $\alpha > 0$. Сравнивая полученные результаты, заметим, что при постановке задачи минимизации управления, скорость сходимости к нулю во всех случаях падает за счет уменьшения вещественной части спектра, вследствии величина (норма) самого управления уменьшается. При $\alpha_3$ однако этот эффект менее заметен, что можно объяснить близкими к $\alpha_3$ вещественными частями спектра, получившегося при обычной задаче синтеза регулятора, обеспечивающего экспоненциальную устойчивость.

    Также важно, что увеличение желаемой степени устойчивости приводит к ускорению переходных процессов (для $\alpha_3 = 2$ они более чем в $2$ раза быстрее в сравнении с $\alpha_1 = 0.5$ - рисунки \ref{fig:reg_x_2_1} и \ref{fig:reg_x_2_3}). Таким образом, можно регулировать необходимую степень быстродействия системы, задача же минимизации, если это необходимо, позволит добиться меньших значений управления, оставаясь при этом в контексте нужной степени устойчивости.

    Подойдем к задаче синтеза регуляторов с заданной степенью экспоненциальной устойчивости иначе. Будем решать не неравенства, а \textit{уравнения Риккати} при $\nu = 2$ и $R = 1$ относительно $P \succ 0$, после находить матрицу обратной связи $K$ с вариацией параметров $\alpha$ и $Q$:
    \[
        A^T P + P A + Q - \nu P B R^{-1} B^T P + 2\alpha P = 0, \quad K = R^{-1} B^T P
    \]

    \textbf{Опять-таки, начнём с $\alpha_1 = 0.5$.}

    Пусть $Q = I$. Тогда решением уравнения Риккати при $\alpha = \alpha_1 = 0.5$ будет:
    \[
        P_{3, 1} = \begin{bmatrix}
            1.6725  & -0.0392  & 1.3849  \\
            -0.0392  & 0.8967  & 0.1240  \\
            1.3849  & 0.1240  & 1.5434 
        \end{bmatrix}
    \]

    Откуда:
    \[
        K_{3, 1} = R^{-1} B^T P = \begin{bmatrix}
            -3.3450   & 0.0785   & -2.7698
        \end{bmatrix}
    \]

    Выполним проверку корректности синтеза, для этого найдем матрицу замкнутой системы:
    \[
        A + BK_{3, 1} = \begin{bmatrix}
            4.0000  & -2.0392  & 6.0000  \\
            6.0000  & -1.0000  & 6.0000  \\
            -6.0000  & -1.0000  & -8.0000 
        \end{bmatrix}
    \]

    Она имеет спектром
    \[
        \sigma(A + BK_{3, 1}) =  \{-2, -1.345 \pm 4.8828i\}
    \]
    
    Все собственные числа имеют вещественную часть, меньшую $\alpha_1$. Это соответствует заданной желаемой степени устойчивости, а значит, синтез проведен корректно.

    Пусть теперь $Q = 0$. Тогда решением уравнения Риккати при $\alpha = \alpha_1 = 0.5$ будет:
    \[
        P_{4, 1} = \begin{bmatrix}
            1.2500  & -0.1042  & 1.2500  \\
            -0.1042  & 0.5382  & -0.1042  \\
            1.2500  & -0.1042  & 1.2500   
        \end{bmatrix}
    \]

    Откуда:
    \[
        K_{4, 1} = R^{-1} B^T P = \begin{bmatrix}
            -2.5000  &  0.2083  & -2.5000
        \end{bmatrix}
    \]

    Выполним проверку корректности синтеза, для этого найдем матрицу замкнутой системы:
    \[
        A + BK_{4, 1} = \begin{bmatrix}
            6.0000 &  -1.5833  & 8.0000  \\
            6.0000 & -1.0000  & 6.0000  \\
            -6.0000 & -1.0000  & -8.0000  
        \end{bmatrix}
    \]

    Она имеет спектром
    \[
        \sigma(A + BK_{4, 1}) =  \{-2, -0.5 \pm 3.9051i\}
    \]
    
    Все управляемые собственные числа имеют вещественную часть, \textit{равную} $\alpha_1$, а для неуправляемого $\lambda_1$ выполнено $\Re(\lambda_1) < \alpha_1$. Это соответствует заданной желаемой степени устойчивости, а значит, синтез проведен корректно.
    
    \begin{figure}
        \centering
        \includegraphics[width=0.8\textwidth]{images/reg_x_3_1.png}
        \caption{Вектор состояния при $\alpha_1 = 0.5$}
        \label{fig:reg_x_3_1}
    \end{figure}
    \begin{figure}
        \centering
        \includegraphics[width=0.8\textwidth]{images/reg_x_4_1.png}
        \caption{Вектор состояния при $\alpha_1 = 0.5$ и минимизации управления}
        \label{fig:reg_x_4_1}
    \end{figure}
    \begin{figure}
        \centering
        \includegraphics[width=0.8\textwidth]{images/reg_u_K31_K41.png}
        \caption{Управления системой с регуляторами $K_{3,1}$ и $K_{4,1}$}
        \label{fig:reg_u_K31_K41}
    \end{figure}
    \begin{figure}
        \centering
        \includegraphics[width=0.8\textwidth]{images/reg_u_K11_K21_K31_K41.png}
        \caption{Управления системой с регуляторами $K_{1,1}$, $K_{2,1}$, $K_{3,1}$ и $K_{4,1}$}
        \label{fig:reg_u_K11_K21_K31_K41}
    \end{figure}

    Проведем моделирование системы с полученными регуляторами при начальных условиях $x(0) = [1, 1, 1]^T$. На рисунках \ref{fig:reg_x_3_1} и \ref{fig:reg_x_4_1} представлены векторы состояния с матрицами обратной связи $K_{3,1}$ и $K_{4,1}$, а на рисунках \ref{fig:reg_u_K31_K41} и \ref{fig:reg_u_K11_K21_K31_K41} - соответствующие управления системой.

    Можем видеть, что при $Q = 0$ полученные графики управления практически совпадают с полученными ранее при решении неравенства Ляпунова совместно с задачей минимизации управления. Дело в том, что в уравнениях Риккати пара $(Q, R)$ как раз и определяет важность быстродействия системы и меньших затрат на управления ей соответственно. Соотношение $Q = 0$ и $R = 1$ говорит, что хочется достичь необходимой степени устойчивости, не затрачивая больших усилий на управление (задача минимизации). Пара $Q = I$ и $R = 1$ же даёт отношение, в котором нам важно как быстродействие, так и управление, в равной степени.

    При $Q = 0$ и $R = 1$ также получили вещественные части спектра, равные $\alpha_1$, за счет чего переходные процессы протекают быстрее, а управления затрачивается больше, чем при $Q = I$ и $R = 1$.

    Сходимость к нулю у вектора состояния при $Q = I$ и $R = 1$ несколько медленнее, чем при регуляторе, полученном при решении неравенства Ляпунова ($K_{3,1}$), однако затраты на управление при этом меньше, что можно объяснить большими значениями собственных чисел спектра замкнутой системы у последнего. Ситуацию можно изменить, например, увеличив параметр $Q$ при неизменном $R = 1$.

    \textbf{Теперь рассмотрим степень устойчивости $\alpha_2 = 1.25$.}

    Пусть $Q = I$. Тогда решением уравнения Риккати при $\alpha = \alpha_2 = 1.25$ будет:
    \[
        P_{3, 2} = \begin{bmatrix}
            1.9380 & 0.1062 & 1.6424  \\
            0.1062 & 1.6204 & 0.6475  \\
            1.6424 & 0.6475 & 2.2142  \\
        \end{bmatrix}
    \]

    Откуда:
    \[
        K_{3, 2} = R^{-1} B^T P = \begin{bmatrix}
            -3.8760 &  -0.2125   & -3.2849
        \end{bmatrix}
    \]

    Выполним проверку корректности синтеза, для этого найдем матрицу замкнутой системы:
    \[
        A + BK_{3, 2} = \begin{bmatrix}
            3.2480  & -2.4250  & 6.4303  \\
            6.0000  & -1.0000  & 6.0000  \\
            -6.0000  & -1.0000  & -8.0000  \\
        \end{bmatrix}
    \]

    Она имеет спектром
    \[
        \sigma(A + BK_{3, 2}) =  \{-2, -1.876 \pm 5.1842i\}
    \]
    
    Все собственные числа имеют вещественную часть, меньшую $\alpha_2$. Это соответствует заданной желаемой степени устойчивости, а значит, синтез проведен корректно.
    
    Пусть теперь $Q = 0$. Тогда решением уравнения Риккати при $\alpha = \alpha_2 = 1.25$ будет:
    \[
        P_{4, 2} = \begin{bmatrix}
            1.6250 & 0.0677 & 1.6250  \\
            0.0677 & 0.8859 & 0.0677  \\ 
            1.6250 & 0.0677 & 1.6250   
        \end{bmatrix}
    \]

    Откуда:
    \[
        K_{4, 2} = R^{-1} B^T P = \begin{bmatrix}
            -3.2500 &  -0.1354   & -3.2500
        \end{bmatrix}
    \]

    Выполним проверку корректности синтеза, для этого найдем матрицу замкнутой системы:
    \[
        A + BK_{4, 2} = \begin{bmatrix}
            4.5000  & -2.2708  & 6.5000  \\
            6.0000  & -1.0000  & 6.0000  \\
            -6.0000  & -1.0000  & -8.0000 \\
        \end{bmatrix}
    \]

    Она имеет спектром
    \[
        \sigma(A + BK_{4, 2}) =  \{-2, -1.25 \pm 4.423i\}
    \]
    
    Все управляемые собственные числа имеют вещественную часть, \textit{равную} $\alpha_2$, а для неуправляемого $\lambda_1$ выполнено $\Re(\lambda_1) < \alpha_2$. Это соответствует заданной желаемой степени устойчивости, а значит, синтез проведен корректно.

    Проведем моделирование системы с полученными регуляторами при начальных условиях $x(0) = [1, 1, 1]^T$. На рисунках \ref{fig:reg_x_3_2} и \ref{fig:reg_x_4_2} представлены векторы состояния с матрицами обратной связи $K_{3,2}$ и $K_{4,2}$, а на рисунках \ref{fig:reg_u_K32_K42} и \ref{fig:reg_u_K12_K22_K32_K42} - соответствующие управления системой.

    \begin{figure}
        \centering
        \includegraphics[width=0.8\textwidth]{images/reg_x_3_2.png}
        \caption{Вектор состояния при $\alpha_2 = 1.25$}
        \label{fig:reg_x_3_2}
    \end{figure}
    \begin{figure}
        \centering
        \includegraphics[width=0.8\textwidth]{images/reg_x_4_2.png}
        \caption{Вектор состояния при $\alpha_2 = 1.25$ и минимизации управления}
        \label{fig:reg_x_4_2}
    \end{figure}
    \begin{figure}
        \centering
        \includegraphics[width=0.8\textwidth]{images/reg_u_K32_K42.png}
        \caption{Управления системой с регуляторами $K_{3,2}$ и $K_{4,2}$}
        \label{fig:reg_u_K32_K42}
    \end{figure}
    \begin{figure}
        \centering
        \includegraphics[width=0.8\textwidth]{images/reg_u_K12_K22_K32_K42.png}
        \caption{Управления системой с регуляторами $K_{1,2}$, $K_{2,2}$, $K_{3,2}$ и $K_{4,2}$}
        \label{fig:reg_u_K12_K22_K32_K42}
    \end{figure}

    \textbf{Наконец, примем степень устойчивости $\alpha_3 = \alpha_{max} = 2$.}    

    Пусть $Q = I$. Тогда решением уравнения Риккати при $\alpha = \alpha_3 = 2$ будет:
    \[
        P_{3, 3} = \begin{bmatrix}
            2.2302 & 0.3368 & 1.9363  \\
            0.3368 & 65454.4790 & 65453.1972  \\
            1.9363 & 65453.1972 & 65454.8673
        \end{bmatrix}
    \]

    Откуда:
    \[
        K_{3, 3} = R^{-1} B^T P = \begin{bmatrix}
            -4.4604  & -0.6735   & -3.8727
        \end{bmatrix}
    \]

    Выполним проверку корректности синтеза, для этого найдем матрицу замкнутой системы:
    \[
        A + BK_{3, 3} = \begin{bmatrix}
            2.0791  & -3.3471  & 5.2547  \\
            6.0000  & -1.0000  & 6.0000  \\
            -6.0000  & -1.0000  & -8.0000 \\
        \end{bmatrix}
    \]

    Она имеет спектром
    \[
        \sigma(A + BK_{3, 3}) =  \{-2, -2.4604 \pm 5.568i\}
    \]
    
    Все управляемые собственные числа имеют вещественную часть, меньшую $\alpha_3$, а для неуправляемого $\lambda_1$ выполнено $\Re(\lambda_1) = \alpha_3$. Это соответствует заданной желаемой степени устойчивости, а значит, синтез проведен корректно.

    Пусть теперь $Q = 0$. Тогда решением уравнения Риккати при $\alpha = \alpha_3 = 2$ будет:
    \[
        P_{4, 3} = \begin{bmatrix}
            2.0000 & 0.3333 & 2.0000  \\
            0.3333 & 1.4444 & 0.3333  \\
            2.0000 & 0.3333 & 2.0000   
        \end{bmatrix}
    \]

    Откуда:
    \[
        K_{4, 3} = R^{-1} B^T P = \begin{bmatrix}
            -4.0000   & -0.6667   & -4.0000
        \end{bmatrix}
    \]

    Выполним проверку корректности синтеза, для этого найдем матрицу замкнутой системы:
    \[
        A + BK_{4, 3} = \begin{bmatrix}
            3.0000  & -3.3333  & 5.0000  \\
            6.0000  & -1.0000  & 6.0000  \\
            -6.0000  & -1.0000  & -8.0000
        \end{bmatrix}
    \]

    Она имеет спектром
    \[
        \sigma(A + BK_{4, 3}) =  \{-2, -2 \pm 5i\}
    \]
    
    Все собственные числа имеют вещественную часть, \textit{равную} $\alpha_3 = \alpha_{max} = 2$. Данное достиглось за счет $\Re(\lambda_1) = \alpha_{max} = 2$. Это соответствует заданной желаемой степени устойчивости, а значит, синтез проведен корректно.

    \begin{figure}
        \centering
        \includegraphics[width=0.8\textwidth]{images/reg_x_3_3.png}
        \caption{Вектор состояния при $\alpha_3 = 2$}
        \label{fig:reg_x_3_3}
    \end{figure}
    \begin{figure}
        \centering
        \includegraphics[width=0.8\textwidth]{images/reg_x_4_3.png}
        \caption{Вектор состояния при $\alpha_3 = 2$ и минимизации управления}
        \label{fig:reg_x_4_3}
    \end{figure}
    \begin{figure}
        \centering
        \includegraphics[width=0.8\textwidth]{images/reg_u_K33_K43.png}
        \caption{Управления системой с регуляторами $K_{3,3}$ и $K_{4,3}$}
        \label{fig:reg_u_K33_K43}
    \end{figure}
    \begin{figure}
        \centering
        \includegraphics[width=0.8\textwidth]{images/reg_u_K13_K23_K33_K43.png}
        \caption{Управления системой с регуляторами $K_{1,3}$, $K_{2,3}$, $K_{3,3}$ и $K_{4,3}$}
        \label{fig:reg_u_K13_K23_K33_K43}
    \end{figure}

    Проведем моделирование системы с полученными регуляторами при начальных условиях $x(0) = [1, 1, 1]^T$. На рисунках \ref{fig:reg_x_3_3} и \ref{fig:reg_x_4_3} представлены векторы состояния с матрицами обратной связи $K_{3,3}$ и $K_{4,3}$, а на рисунках \ref{fig:reg_u_K33_K43} и \ref{fig:reg_u_K13_K23_K33_K43} - соответствующие управления системой.

    Таким образом, во всех случаях была достигнута желаемая степень устойчивости $\alpha > 0$. При увеличении желаемой степени устойчивости переходные процессы ускоряются, что видно по графикам вектора состояния (рисунки \ref{fig:reg_x_4_1} и \ref{fig:reg_x_4_3} при $\alpha_1 = 0.5$ и $\alpha_3 = 2$ соответственно). Соотношение $Q$ и $R$ в задаче синтеза регуляторов с помощью уравнений Риккати полностью определяет, насколько важно быстродействие системы по сравнению с управлением. Так, при $Q = 0$ и $R = 1$ при всех значениях $\alpha$ вещественная часть спектра замкнутой системы совпадала с $\alpha$ (спектр был <<прижат>> к $\alpha$), а графики управления совпадали с графиками управления при синтезе регуляторов с помощью неравенства Ляпунова совместно с задачей минимизации управления. 
    
    При сравнении ситуаций с $Q = I$ и $Q = 0$ при $R = 1$ видно, что при $Q = I$ переходные процессы протекают быстрее, чем при $Q = 0$. Данное объясняется тем, что при $Q = 0$ была поставлена задача нахождения регулятора, всё ещё обеспечивающего заданную степень устойчивости $\alpha > 0$, но с минимальными затратами на управление. Тогда как при $Q = I$ был важен некий баланс между быстродействием и затратами на управление.

    В общем, уравнения Риккати позволяют более точно управлять получаемым спектром замкнутой системы, получать различные соотношения между быстродействием и затратами на управление.

    Во всех ситуациях задача минимизации управления позволила добиться меньших значений управления за счет уменьшения скорости сходимости системы к нулю с помощью <<прижатия>> вещественной части спектра замкнутой системы к заданной желаемой степени устойчивости $\alpha > 0$. Всё подтверждается и моделированием.

    \section{Управление по выходу}
    Рассмотрим систему
    \[
        \begin{cases}
        \dot{x} = Ax + Bu \\
        y = Cx
        \end{cases}
    \]

    В соответствии с вариантом, матрицы $A$, $B$ и $C$ имеют вид:
    \[
        A = \begin{bmatrix}
            5 & -5 & -9 & 3 \\
            -5 & 5 & -3 & 9 \\
            -9 & -3 & 5 & 5 \\
            3 & 9 & 5 & 5
        \end{bmatrix}, \quad
        B = \begin{bmatrix}
            2 \\
            6 \\
            6 \\
            2
        \end{bmatrix}, \quad
        C = \begin{bmatrix}
            1 & -1 & 1 & 1 \\
            1 & 3 & -1 & 3
        \end{bmatrix}
    \]

    Выполним анализ управляемости и наблюдаемости системы. Для этого найдем Жорданову форму матрицы $A$ и вспомогательную матрицу $T$ для перехода к ней:
    \[
        \hat{A} = T^{-1}AT = \begin{bmatrix}
            16 & 0 & 0 & 0 \\
            0 & 12 & 0 & 0 \\
            0 & 0 & 4 & 0 \\
            0 & 0 & 0 & -12
        \end{bmatrix}, \quad
                T = \begin{bmatrix}
                    -1 & 1 & 1 & -1 \\
            1 & 1 & -1 & -1 \\
            1 & -1 & 1 & -1 \\
            1 & 1 & 1 & 1
                \end{bmatrix}
            \]

    Откуда:
    \[
        \hat{B} = T^{-1}B = \begin{bmatrix}
            3 \\
            1 \\
            1 \\
            -3
        \end{bmatrix}, \quad
        \hat{C} = CT = \begin{bmatrix}
            0 & 0 & 4 & 0 \\
            4 & 8 & 0 & 0
        \end{bmatrix}
    \]

    В матрице управления $B$ нет нулей, а значит, все собственные числа системы управляемы, следовательно, система является полностью управляемой, а значит, и стабилизируемой (так как вообще нет неуправляемых собственных чисел). 
    
    Отметим, что при помощи регулятора вида $u = Kx$ можно добиться любой желаемой степени устойчивости, так как $\lambda_i$ можно придать любое необходимое значение из комплексной плоскости.

    В матрице наблюдения $C$ обнуляется последний столбец, соответствующий $\lambda_4 = -12$, а значит, собственное число $\lambda_4$ не наблюдаемо. Для остальных $\lambda_{1-3}$ все столбцы $\hat{C}_1 = [0, 4]^T \neq 0$, $\hat{C}_2 = [0, 8]^T \neq 0$ и $\hat{C}_3 = [4, 0]^T \neq 0$ ненулевые, а значиит, все они являются наблюдаемыми. Таким образом, система является частично наблюдаемой, но обнаруживаемой, так как $\Re(\lambda_4) = -12 < 0$.

    В итоге, так как степень устойчивости задаётся минимальной вещественной частью мнимых собственных чисел, а $\lambda_4$ является ненаблюдаемым, то есть оно всегда присутствует в спектре матрицы $A + LC$ наблюдателя полной размерности, то максимальная желаемая степень сходимости $\alpha_{max}$, которую можно достичь с помощью этого наблюдателя, равна 12 ($\lambda_4 = -12 \Rightarrow \alpha_{max} = |\Re(\lambda_4)| = 12$).

    \begin{figure}
        \centering
        \includegraphics[width=0.75\textwidth]{images/observer_and_regulator.png}
        \caption{Схема моделирования системы с управлением по выходу}
        \label{fig:observer_and_regulator}
    \end{figure}

    Построим схему моделирования системы (изображена на рисунке \ref{fig:observer_and_regulator}), замнкутой регулятором, сотосящем из наблюдателя состояния $x = \hat{A}\hat{x} + \hat{B}u + L(\hat{C}\hat{x} - y)$ и регулятора $u = K\hat{x}$.

    Теперь зададися парой значений $\alpha > 0$:
    \[
        \alpha_1 = 1, \quad \alpha_2 = 12
    \]

    Используя их, составим наборы желаемых степеней устойчивости $\alpha_K$ системы и желаемой степени сходимости $\alpha_L$ наблюдателя:
    \[
        \alpha_K = \alpha_L: \quad \alpha_{K1} = \alpha_2=12, \quad \alpha_{L1} = \alpha_2=12
    \]
    \[
        \alpha_K > \alpha_L: \quad \alpha_{K2} = \alpha_2=12, \quad \alpha_{L2} = \alpha_1=1
    \]
    \[
        \alpha_K < \alpha_L: \quad \alpha_{K3} = \alpha_1=1, \quad \alpha_{L3} = \alpha_2=12
    \]

    Условно их можно назвать: наблюдатель и регулятор имеют сопоставимые значения $\alpha$, регулятор <<сильнее>> и наблюдатель <<сильнее>>. Теперь перейдем к синтезу соответствующих управлений по выходу.

    \textbf{Сначала рассмотрим случай $\alpha_K = \alpha_L = \alpha_1 = 12$.}

    Сначала синтезируем регулятор, обеспечивающий желаемую степень устойчивости $\alpha_K = \alpha_2$. Для этого воспользуемся методом уравнений Риккати, при этом будем минимизировать отклонения фактических собственных чисел спектра замкнутой системы от желаемой устойчивости $\alpha_K$ (что, как было выяснено в предыдущем пункте, отчасти эквивалентно минимизации управления с $Q = 0$ и $R = 1$).
    
    Задача была подробно описана в предыдущем пункте, так что вдаваться в подробности не будем. Решением уравнения Риккати при $\alpha = \alpha_{K1} = \alpha_2 = 12$ и заданными матрицами $Q = 0$ и $R = 1$ будет:
    \[
        P_{K1} = \begin{bmatrix}
            6850.5757  & 6301.0517  & -9779.8256  & 3371.8019  \\
            6301.0517  & 5897.2436  & -9091.7833  & 3106.5121  \\
            -9779.8256  & -9091.7833  & 14053.5154  & -4818.0935  \\
            3371.8019  & 3106.5121  & -4818.0935  & 1660.2204   
        \end{bmatrix}
    \]
    
    Откуда:
    \[
        K_{1} = -R^{-1}B^T P_{K1} = \begin{bmatrix}
            427.8880 & 352.1103 & -574.5543 & 205.4440
        \end{bmatrix}
    \]

    Выполним проверку корректности синтеза регулятора, для этого найдем матрицу замкнутой системы:
    \[
        A + BK_{1} = \begin{bmatrix}
            860.7761  & 699.2206  & -1158.1086  & 413.8880  \\
            2562.3282  & 2117.6619  & -3450.3259  & 1241.6641  \\
            2558.3282  & 2109.6619  & -3442.3259  & 1237.6641  \\
            858.7761  & 713.2206  & -1144.1086  & 415.8880  
        \end{bmatrix}
    \]

    Она имеет спектром
    \[
        \sigma(A + BK_{1}) =  \{-12, -12, -12\pm 38.7814i\}
    \]

    Все собственные числа имеют вещественную часть, \textit{равную} $\alpha_K$, а значит, синтез проведен корректно.

    Теперь синтезируем наблюдатель, обеспечивающий степень сходимости $\alpha_{L1} = \alpha_1 = 12$ совместно с минимизацией отклонений фактических собственных чисел спектра наблюдателя от желаемой степени сходимости, что соответствует задаче <<минимизации>> наблюдателя. Опять-таки воспользуемся методом уравнений Риккати, то есть при $\alpha = \alpha_{L1}$, $v = 2$, $R = 1$ и $Q = 0$ относительно $P \succ 0$ будем решать и находить матрицу коррекции наблюдателя $L$:
    \[
        AP + PA^T - 2P C^T R^{-1}C P + 2\alpha P + Q = 0, \quad L = -P C^T R^{-1}
    \]

    Здесь матрицы $Q$ и $R$, как и прежде, показывают, соответственно, насколько важна быстрота сходимости (в данном случае наблюдателя) и возникающее при этом перерегулирование (в начале наблюдения). При взятых $Q = 0$ и $R = 1$ ставится задача обеспечения желаемой степени сходимости $\alpha_{L1}$ с минимально возможными рывками в наблюдении в первые моменты времени. Итак, решением будет:
    \[
        P_{L1} = \begin{bmatrix}
            633.1250  & -233.3750 & -631.1250&  -231.3750  \\
            -233.3750 &  87.1250  &231.3750  &85.1250  \\
            -631.1250 &  231.3750 & 633.1250 & 233.3750  \\
            -231.3750 &  85.1250  &233.3750  &87.1250
        \end{bmatrix}
    \]

    Откуда:
    \[
        L_{1} = -P_{L1} C^T R^{-1} = \begin{bmatrix}
            -4 & 130 \\
            4 & -52\\
           -4 &-130\\
           -4 & -52
        \end{bmatrix}
    \]

    Выполним проверку корректности синтеза наблюдателя, для этого найдем матрицу замкнутой системы:
    \[
        A + L_{1}C = \begin{bmatrix}
            131&  389 & -143 & 389  \\
            -53 & -155 & 53 & -143  \\
            -143 &  -389 & 131 & -389 \\ 
            -53 & -143 & 53 & -155
        \end{bmatrix}
    \]

    Она имеет спектром
    \[
        \sigma(A + L_{1}C) =  \{-12, -12 \pm 25.923i\}
    \]

    Все собственные числа имеют вещественную часть, \textit{равную} $\alpha_{L1}$, а значит, синтез проведен корректно. Перейдем к моделированию системы с полученными регулятором и наблюдателем при начальных условиях $x(0) = [1, 1, 1, 1]^T$ и $x_0 = [0, 0, 0, 0]^T$.

    \begin{figure}
        \centering
        \includegraphics[width=0.8\textwidth]{images/obsreg_u_1.png}
        \caption{Управления системой при $\alpha_{K1} = \alpha_{L1} = 1$}
        \label{fig:obsreg_u_1}
    \end{figure}
    \begin{figure}
        \centering
        \includegraphics[width=0.8\textwidth]{images/obsreg_x1_1.png}
        \caption{Первые компоненты $x(t)$ и $\hat{x}(t)$ при $\alpha_{K1} = \alpha_{L1} = 1$}
        \label{fig:obsreg_x1_1}
    \end{figure}
    \begin{figure}
        \centering
        \includegraphics[width=0.8\textwidth]{images/obsreg_x2_1.png}
        \caption{Вторые компоненты $x(t)$ и $\hat{x}(t)$ при $\alpha_{K1} = \alpha_{L1} = 1$}
        \label{fig:obsreg_x2_1}
    \end{figure}
    \begin{figure}
        \centering
        \includegraphics[width=0.8\textwidth]{images/obsreg_x3_1.png}
        \caption{Третьи компоненты $x(t)$ и $\hat{x}(t)$ при $\alpha_{K1} = \alpha_{L1} = 1$}
        \label{fig:obsreg_x3_1}
    \end{figure}
    \begin{figure}
        \centering
        \includegraphics[width=0.8\textwidth]{images/obsreg_x4_1.png}
        \caption{Четвертые компоненты $x(t)$ и $\hat{x}(t)$ при $\alpha_{K1} = \alpha_{L1} = 1$}
        \label{fig:obsreg_x4_1}
    \end{figure}
    \begin{figure}
        \centering
        \includegraphics[width=0.8\textwidth]{images/obsreg_e_1.png}
        \caption{Ошибка наблюдателя при $\alpha_{K1} = \alpha_{L1} = 1$}
        \label{fig:obsreg_e_1}
    \end{figure}

    На рисунке \ref{fig:obsreg_u_1} представлен график формируемого регулятором управления $u(t)$, на рисунках \ref{fig:obsreg_x1_1} - \ref{fig:obsreg_x4_1} - сравнительные графики $x(t)$ и $\hat{x}(t)$ для каждой из компонент, а на рисунке \ref{fig:obsreg_e_1} - график ошибки наблюдателя $e(t)=x(t)-\hat{x}(t)$. Можем видеть, что всё быстро сходится к нулю, а после первой секунды все приведенные графики визуально равны нулю.

    \textbf{Проделаем то же для $\alpha_{K2} = \alpha_2 = 12$ и $\alpha_L = \alpha_1 = 1$.}

    Синтез регулятора остаётся тем же, так как $\alpha_{K2} = \alpha_{K1} = \alpha_2 = 12$. Поэтому возьмём решение, полученное ранее:
    \[
        P_{K2} = P_{K1} = \begin{bmatrix}
            6850.5757  & 6301.0517  & -9779.8256  & 3371.8019  \\
            6301.0517  & 5897.2436  & -9091.7833  & 3106.5121  \\
            -9779.8256  & -9091.7833  & 14053.5154  & -4818.0935  \\
            3371.8019  & 3106.5121  & -4818.0935  & 1660.2204   
        \end{bmatrix}
    \]

    Откуда:
    \[
        K_{2} = K_{1} = \begin{bmatrix}
            427.8880 & 352.1103 & -574.5543 & 205.4440
        \end{bmatrix}
    \]

    Теперь синтезируем наблюдатель со степенью сходимости $\alpha_{L2} = \alpha_1 = 1$. Аналогично будем использовать метод уравнений Риккати при $\alpha = \alpha_{L2}$, $v = 2$, $R = 1$ и $Q = 0$. Решением в таком случае будет:
    \[
        P_{L2} = \begin{bmatrix}
            123.3008 & -48.6523 & -122.6758 & -48.0273  \\
            -48.6523 & 19.7070 & 48.0273 & 19.0820  \\
            -122.6758 & 48.0273 & 123.3008 & 48.6523  \\
            -48.0273 & 19.0820 & 48.6523 & 19.7070  
        \end{bmatrix}
    \]

    Откуда:
    \[
        L_{2} = -P_{L2} C^T R^{-1} = \begin{bmatrix}
            -1.25  & 44.0625 \\
            1.25  & -19.6875 \\
           -1.25  & -44.0625 \\
           -1.25  & -19.6875
        \end{bmatrix}
    \]

    Выполним проверку корректности синтеза наблюдателя, для этого найдем матрицу замкнутой системы:
    \[
        A + L_{2}C = \begin{bmatrix}
            47.8125  & 128.4375  & -54.3125  & 133.9375  \\
            -23.4375  & -55.3125  & 17.9375  & -48.8125  \\
            -54.3125  & -133.9375  & 47.8125  & -128.4375  \\
            -17.9375  & -48.8125  & 23.4375  & -55.3125  
        \end{bmatrix}
    \]

    Она имеет спектром
    \[
        \sigma(A + L_{2}C) =  \{-1, -1 \pm 14.8661i, -12\}
    \]

    Все собственные числа имеют вещественную часть, меньшую либо равную $\alpha_{K2}$, а значит, синтез проведен корректно. Отметим, что минимизировать расстояние вещественной части спектра наблюдателя до $\alpha_{L2}$ полностью невозможно, так как одно собственное число $\lambda_4 = -12$ является ненаблюдаемым.

    Выполним моделирование системы с полученными регулятором и наблюдателем. Зададим начальные условия $x(0) = [1, 1, 1, 1]^T$ и $x_0 = [0, 0, 0, 0]^T$ и построим графики формируемого регулятором управления $u(t)$ (рисунок \ref{fig:obsreg_u_2}), сравнительные графики $x(t)$ и $\hat{x}(t)$ (рисунки \ref{fig:obsreg_x1_2} - \ref{fig:obsreg_x4_2}) и графики ошибки наблюдателя $e(t)=x(t)-\hat{x}(t)$ (рисунок \ref{fig:obsreg_e_2}).

    \begin{figure}
        \centering
        \includegraphics[width=0.8\textwidth]{images/obsreg_u_2.png}
        \caption{Управления системой при $\alpha_{K2} = 12$ и $\alpha_{L2} = 1$}
        \label{fig:obsreg_u_2}
    \end{figure}
    \begin{figure}
        \centering
        \includegraphics[width=0.8\textwidth]{images/obsreg_x1_2.png}
        \caption{Первые компоненты $x(t)$ и $\hat{x}(t)$ при $\alpha_{K2} = 12$ и $\alpha_{L2} = 1$}
        \label{fig:obsreg_x1_2}
    \end{figure}
    \begin{figure}
        \centering
        \includegraphics[width=0.8\textwidth]{images/obsreg_x2_2.png}
        \caption{Вторые компоненты $x(t)$ и $\hat{x}(t)$ при $\alpha_{K2} = 12$ и $\alpha_{L2} = 1$}
        \label{fig:obsreg_x2_2}
    \end{figure}
    \begin{figure}
        \centering
        \includegraphics[width=0.8\textwidth]{images/obsreg_x3_2.png}
        \caption{Третьи компоненты $x(t)$ и $\hat{x}(t)$ при $\alpha_{K2} = 12$ и $\alpha_{L2} = 1$}
        \label{fig:obsreg_x3_2}
    \end{figure}
    \begin{figure}
        \centering
        \includegraphics[width=0.8\textwidth]{images/obsreg_x4_2.png}
        \caption{Четвертые компоненты $x(t)$ и $\hat{x}(t)$ при $\alpha_{K2} = 12$ и $\alpha_{L2} = 1$}
        \label{fig:obsreg_x4_2}
    \end{figure}
    \begin{figure}
        \centering
        \includegraphics[width=0.8\textwidth]{images/obsreg_e_2.png}
        \caption{Ошибка наблюдателя при $\alpha_{K2} = 12$ и $\alpha_{L2} = 1$}
        \label{fig:obsreg_e_2}
    \end{figure}

    \textbf{Наконец, рассмотрим $\alpha_K = \alpha_1 = 1$ и $\alpha_L = \alpha_2 = 12$.}

    Синтез регулятора наблюдателя остается тем же, что и в первом случае, так как $\alpha_{L3} = \alpha_{L1} = \alpha_2 = 12$. Поэтому возьмём решение, полученное ранее:
    \[
        P_{L3} = P_{L1} = \begin{bmatrix}
            633.1250  & -233.3750 & -631.1250&  -231.3750  \\
            -233.3750 &  87.1250  &231.3750  &85.1250  \\
            -631.1250 &  231.3750 & 633.1250 & 233.3750  \\
            -231.3750 &  85.1250  &233.3750  &87.1250
        \end{bmatrix}
    \]

    Откуда:
    \[
        L_{3} = L_{1} = \begin{bmatrix}
            -4 & 130 \\
            4 & -52\\
           -4 &-130\\
           -4 & -52
        \end{bmatrix}
    \]

    Выполним теперь синтез регулятора, который будет обеспечивать степень устойчивости $\alpha_{K3} = \alpha_1 = 1$. Задействуем метод матричных неравенств типа Ляпунова относительно $P \succ 0$ и $Y$ при $\alpha = \alpha_{K3}$ совместно с минимизацией управления (что позволит <<прижать>> управляемые собственные числа спектра замкнутой системы к $\alpha_{K3}$, минимизировав отклонения между фактическими собственными числами и желаемой степенью устойчивости). Итак, решением будет:
    \[
        P_{K3} = \begin{bmatrix}
            104.01  & 103.96  & 103.97  & -104.04  \\
            103.96  & 103.98  & 103.98  & -103.95  \\
            103.97  & 103.98  & 104.02  & -103.86  \\
            -104.04  & -103.95  & -103.86  & 104.38 
        \end{bmatrix}, \quad Y_{K3} = \begin{bmatrix}
            0.37 \\ 0.04 \\ -0.05  \\ -0.96  
        \end{bmatrix}^T
    \]

    Откуда:
    \[
        K_{3} = \begin{bmatrix}
            46.3177  & 24.0566  & -51.5896  & 18.7837
        \end{bmatrix}
    \]

    Выполним проверку корректности синтеза, для этого найдем матрицу замкнутой системы:
    \[
        A + BK_{3} = \begin{bmatrix}
            97.6355  & 43.1133  & -112.1792  & 40.5674  \\
            272.9064  & 149.3398  & -312.5376  & 121.7022  \\
            268.9064  & 141.3398  & -304.5376  & 117.7022  \\
            95.6355  & 57.1133  & -98.1792  & 42.5674
        \end{bmatrix}
    \]

    Она имеет спектром
    \[
        \sigma(A + BK_{3}) =  \{-1, -1 \pm 13.3236i, -12\}
    \]

    Все собственные числа имеют вещественную часть, меньшую либо равную $\alpha_{K3}$, а значит, синтез проведен корректно.

    \begin{figure}
        \centering
        \includegraphics[width=0.8\textwidth]{images/obsreg_u_3.png}
        \caption{Управления системой при $\alpha_{K3} = 1$ и $\alpha_{L3} = 12$}
        \label{fig:obsreg_u_3}
    \end{figure}
    \begin{figure}
        \centering
        \includegraphics[width=0.8\textwidth]{images/obsreg_x1_3.png}
        \caption{Первые компоненты $x(t)$ и $\hat{x}(t)$ при $\alpha_{K3} = 1$ и $\alpha_{L3} = 12$}
        \label{fig:obsreg_x1_3}
    \end{figure}
    \begin{figure}
        \centering
        \includegraphics[width=0.8\textwidth]{images/obsreg_x2_3.png}
        \caption{Вторые компоненты $x(t)$ и $\hat{x}(t)$ при $\alpha_{K3} = 1$ и $\alpha_{L3} = 12$}
        \label{fig:obsreg_x2_3}
    \end{figure}
    \begin{figure}
        \centering
        \includegraphics[width=0.8\textwidth]{images/obsreg_x3_3.png}
        \caption{Третьи компоненты $x(t)$ и $\hat{x}(t)$ при $\alpha_{K3} = 1$ и $\alpha_{L3} = 12$}
        \label{fig:obsreg_x3_3}
    \end{figure}
    \begin{figure}
        \centering
        \includegraphics[width=0.8\textwidth]{images/obsreg_x4_3.png}
        \caption{Четвертые компоненты $x(t)$ и $\hat{x}(t)$ при $\alpha_{K3} = 1$ и $\alpha_{L3} = 12$}
        \label{fig:obsreg_x4_3}
    \end{figure}
    \begin{figure}
        \centering
        \includegraphics[width=0.8\textwidth]{images/obsreg_e_3.png}
        \caption{Ошибка наблюдателя при $\alpha_{K3} = 1$ и $\alpha_{L3} = 12$}
        \label{fig:obsreg_e_3}
    \end{figure}
    \begin{figure}
        \centering
        \includegraphics[width=0.8\textwidth]{images/obsreg_u_all.png}
        \caption{Управления системой при различных спектрах}
        \label{fig:obsreg_u_all}
    \end{figure}

    Заметим, что одно собственное число $\lambda_4 = -12$ осталось неизменным. Дело в том, что оно уже удовлетворяет условию $\Re(\lambda_4) < \alpha_K$ - регулятору в поставленной задаче нет смысла его изменять, отдавая ещё больше управления, задача минимизации которого ставилась. Таким образом, добиться нулевого отклонения вещественной части спектра замкнутой системы от $\alpha_{K1}$ с помощью минимизации управления при наличии подобных на $\lambda_4$ собственных чисел полностью невозможно. Синтез регулятора при обычной постановке задачи обеспечения заданной экспоненциальной устойчивости же даст отклонения при \textit{всех} собственных числах, так как уведёт $\lambda_{1-3}$ ещё дальше от $\alpha_{K1}$ в сравнении с задачей минимизации управления, где это расстояние как бы оказывается минимальным.

    Выполним моделирование системы с полученными регулятором и наблюдателем. Зададим начальные условия $x(0) = [1, 1, 1, 1]^T$ и $x_0 = [0, 0, 0, 0]^T$ и построим графики формируемого регулятором управления $u(t)$ (рисунок \ref{fig:obsreg_u_3}), сравнительные графики $x(t)$ и $\hat{x}(t)$ (рисунки \ref{fig:obsreg_x1_3} - \ref{fig:obsreg_x4_3}) и графики ошибки наблюдателя $e(t)=x(t)-\hat{x}(t)$ (рисунок \ref{fig:obsreg_e_3}). Для сравнения всех случаев также построен график управления всех трех полученных систем вместе (рисунок \ref{fig:obsreg_u_all}).
    
    Сравним полученные результаты для различных наборов $\alpha_K$ и $\alpha_L$. Наименьшее время стабилизиации системы у регулятора с одинаковыми значениями $\alpha_K$ и $\alpha_L$, в остальных же случаях она примерно одинаковая, но меньшие значения амплитуд возникающих колебаний управлений и состояний достигаются всё же при $\alpha_K > \alpha_L$, то есть при более <<сильном>> регуляторе.

    Ошибка наблюдения сходится быстрее при $\alpha_L > \alpha_K$, так как в этом случае наблюдатель имеет более <<сильную>> степень сходимости, при этом значения достигаются несколько большие, чем при $\alpha_L < \alpha_K$, где, помимо прочего, возникают серьезные колебания в графиках ошибки наблюдения. Последнее несколько <<сковывает>> мощное управление, так как оно ориентируется на слабый наблюдатель, сходящийся достаточно медленно.
    
    Также отметим, что понижение любого из параметров $\alpha_K$ или $\alpha_L$ приводит к серьезному увеличению времени стабилизации системы, переходные процессы состояний и управлений становятся более колебательными и медленными.

    В общем случае оптимальным является вариант, когда наблюдатель имеет несколько более <<сильную>> степень сходимости, чем регулятор, позволяя давать ему своевременные показания о состоянии системы и качественно управлять ей.

    \section{Качественная экспоненциальная устойчивость}
    Наконец, рассмотрим линейную систему
    \[
        \dot{x} = Ax + Bu
    \]

    В соответствии с вариантом, матрицы $A$ и $B$ имеют вид:
    \[
        A = \begin{bmatrix}
            11 & -2 & 13 \\
            6 & -1 & 6 \\
            -6 & -1 & -8
        \end{bmatrix}, \quad B = \begin{bmatrix}
            2 \\
            0 \\
            0
        \end{bmatrix}
    \]

    Выполним анализ управляемости и наблюдаемости системы. Для этого найдем Жорданову форму матрицы $A$ и вспомогательную матрицу $T$ для перехода к ней:
    \[
        \hat{A} = T^{-1}AT = \begin{bmatrix}
            -2 & 0 & 0 \\
            0 & 2 & 3 \\
            0 & -3 & 2
        \end{bmatrix}, \quad
        T = \begin{bmatrix}
            -1 & -1.5 & -0.5 \\
            0 & -1 & 0 \\
            1 & 1 & 0
        \end{bmatrix}
    \]

    Откуда:
    \[
        \hat{B} = T^{-1}B = \begin{bmatrix}
            0 \\
            0 \\
            -4
        \end{bmatrix}
    \]

    Таким образом, система является частично управляемой, но стабилизируемой, так как есть единственное неуправляемое собственное число $\lambda_1 = -2$, которому соответствует нулевая первая строка в матрице $\hat{B}$ и для которого $\Re(\lambda_1) < 0$. Комплексно-сопряженные собственные числа $\lambda_2 = 2 + 3i$ и $\lambda_3 = 2 - 3i$ управляемы, так как хотя бы одна из строк в матрице $\hat{B}$ является ненулевой.
    
    Теперь зададимся значениями параметра $\beta < 0$, характеризующего среднее значение поведения траекторий ($|\beta|$ как <<средняя степень устойчивости>>), и параметра $r > 0$, $\beta + r < 0$, характеризующего разброс поведения траекторий относительно среднего значения $\beta$, причем таким образом, чтобы $r = \frac{|\beta|}{k}$, где $1.5 \le k \le 4$:
    \[
        \beta = -3, \quad r = \frac{3}{1.5} = 2
    \]

    Также отметим, что неуправляемое собственное число $\lambda_1$ лежит на комплексной плоскости в пределах круга радиуса $r$ с центром в точке $(\beta, 0)$. Вообще, вся суть метода как раз и заключаетсяв том, чтобы <<сдвинуть>> спектр замкнутой системы в пределы данного круга при помощи статического регулятора.

    Рассмотрим четыре набора параметров $Q$ и $R$:
    \begin{itemize}
        \item $Q = I$, $R = 1$
        \item $Q = I$, $R = 0$
        \item $Q = 0$, $R = 1$
        \item $Q = 0$, $R = 0$
    \end{itemize}

    Для каждого из наборов синтезируем регулятор вида $u = Kx$, обеспечивающего качественную экспоненциальную устойчивость при помощи матричного уравнения Риккати:
    \[
        (A - B K - \beta I)^T P (A - B K - \beta I) - r^2 P = -Q
    \]

    Матрица $K$ обратной связи тогда находится как:
    \[
        K = -(R + B^T P B)^{-1} B^T P (A - \beta I)
    \]

    \textbf{Начнём с $Q = I$ и $R = 1$.}

    Решением уравнения Риккати при $\beta = -3$, $r = 2$, $Q = I$ и $R = 1$ будет:
    \[
        P_{K1} = \begin{bmatrix}
            6.6038  &  1.3758  & 5.7366 \\
            1.3758  &  0.6638  & 1.3004 \\
            5.7366  &  1.3004 &   5.4981
        \end{bmatrix}
    \]

    Откуда:
    \[
        K_{1} = \begin{bmatrix}
            -4.8359  &  1.1813  & -4.7726
        \end{bmatrix}
    \]

    Найдем матрицу замкнутой с помощью найденного регулятора системы:
    \[
        A + BK_{1} = \begin{bmatrix}
            1.3282  &  0.3626  & 3.4548 \\
            6.0000  &  -1.0000  & 6.0000 \\
           -6.0000  &  -1.0000  & -8.0000
        \end{bmatrix}
    \]

    Она имеет спектром
    \[
        \sigma(A + BK_{1}) =  \{-2, -2.8359 \pm 1.1015i\}
    \]

    Все собственные числа находятся в пределах круга радиуса $r$ с центром в точке $(\beta, 0)$, а значит, синтез проведен корректно. Это подтверждается и рисунком \ref{fig:reg_quality1}.

    \begin{figure}
        \centering
        \includegraphics[width=0.8\textwidth]{reg_quality1.png}
        \caption{Собственные числа при $(Q = I, R = 1)$, регуляторе $K_1$}
        \label{fig:reg_quality1}
    \end{figure}
    \begin{figure}
        \centering
        \includegraphics[width=0.8\textwidth]{images/reg_quality_u_1.png}
        \caption{Управление системой при $(Q = I, R = 1)$, регуляторе $K_1$}
        \label{fig:reg_quality_u_1}
    \end{figure}
    \begin{figure}
        \centering
        \includegraphics[width=0.8\textwidth]{images/reg_quality_x_1.png}
        \caption{Вектор состояния системы при $(Q = I, R = 1)$, регуляторе $K_1$}
        \label{fig:reg_quality_x_1}
    \end{figure}

    Проведем моделирование системы с регулятором $K_1$ при начальных условиях $x(0) = [1, 1, 1]^T$. На рисунках \ref{fig:reg_quality_x_1} и \ref{fig:reg_quality_u_1} представлены векторы состояния и управления с матрицей обратной связи $K_1$.

    \newpage
    \textbf{Рассмотрим случай $Q = I$ и $R = 0$.}

    Решением рассматриваемого уравнения Риккати будет:
    \[
        P_{K2} = \begin{bmatrix}
            5.7062  &  1.2305  & 4.8679 \\
            1.2305  &  0.5897  & 1.1600 \\
            4.8679  &  1.1600  & 4.6553
        \end{bmatrix}
    \]

    Откуда:
    \[
        K_{2} = \begin{bmatrix}
            -5.0876  &  1.2109  & -5.0142
        \end{bmatrix}
    \]

    Найдем матрицу замкнутой с помощью найденного регулятора системы:
    \[
        A + BK_{2} = \begin{bmatrix}
            0.8248  &  0.4218  & 2.9717 \\
            6.0000  &  -1.0000  & 6.0000 \\
           -6.0000  &  -1.0000  & -8.0000
        \end{bmatrix}
    \]

    Она имеет спектром
    \[
        \sigma(A + BK_{2}) =  \{-2, -3, -3.1752\}
    \]

    Все собственные числа лежат в пределах круга радиуса $r$ с центром в точке $(\beta, 0)$, а значит, синтез проведен корректно. Это подтверждается и рисунком \ref{fig:reg_quality2}.
    
    Проведем моделирование системы с регулятором $K_2$ при начальных условиях $x(0) = [1, 1, 1]^T$. На рисунках \ref{fig:reg_quality_x_2} и \ref{fig:reg_quality_u_2} представлены векторы состояния и управления с матрицей обратной связи $K_2$.

    \textbf{Рассмотрим случай $Q = 0$ и $R = 1$.}

    Решением уравнения Риккати будет:
    \[
        P_{K3} = \begin{bmatrix}
            1.8750  &  1.5743  & 1.8750 \\
            0.1457  &  0.5733  & 0.1457 \\
            1.8750  &  1.5743  & 1.8750
        \end{bmatrix}
    \]

    Откуда:
    \[
        K_{3} = \begin{bmatrix}
            -5.7520 &   0.5827 &  -5.7520
        \end{bmatrix}
    \]

    Найдем матрицу замкнутой системы:
    \[
        A + BK_{3} = \begin{bmatrix}
            -0.5040 &  -0.8347 &   1.4960 \\
            6.0000  & -1.0000 &   6.0000 \\
           -6.0000  & -1.0000 &  -8.0000
        \end{bmatrix}
    \]

    Она имеет спектром
    \[
        \sigma(A + BK_{3}) =  \{-2, -3.752 \pm 1.8532i\}
    \]

    \begin{figure}
        \centering
        \includegraphics[width=0.8\textwidth]{reg_quality2.png}
        \caption{Собственные числа при $(Q = I, R = 0)$, регуляторе $K_2$}
        \label{fig:reg_quality2}
    \end{figure}
    \begin{figure}
        \centering
        \includegraphics[width=0.8\textwidth]{images/reg_quality_u_2.png}
        \caption{Управление системой при $(Q = I, R = 0)$, регуляторе $K_2$}
        \label{fig:reg_quality_u_2}
    \end{figure}
    \begin{figure}
        \centering
        \includegraphics[width=0.8\textwidth]{images/reg_quality_x_2.png}
        \caption{Вектор состояния системы при $(Q = I, R = 0)$, регуляторе $K_2$}
        \label{fig:reg_quality_x_2}
    \end{figure}
    
    Все собственные числа лежат в пределах круга радиуса $r$ с центром в точке $(\beta, 0)$, а значит, синтез проведен корректно. Это подтверждается и рисунком \ref{fig:reg_quality3}.

    \begin{figure}
        \centering
        \includegraphics[width=0.8\textwidth]{reg_quality3.png}
        \caption{Собственные числа при $(Q = 0, R = 1)$, регуляторе $K_3$}
        \label{fig:reg_quality3}
    \end{figure}
    \begin{figure}
        \centering
        \includegraphics[width=0.8\textwidth]{images/reg_quality_u_3.png}
        \caption{Управление системой при $(Q = 0, R = 1)$, регуляторе $K_3$}
        \label{fig:reg_quality_u_3}
    \end{figure}
    \begin{figure}
        \centering
        \includegraphics[width=0.8\textwidth]{images/reg_quality_x_3.png}
        \caption{Вектор состояния системы при $(Q = 0, R = 1)$, регуляторе $K_3$}
        \label{fig:reg_quality_x_3}
    \end{figure}

    Проведем моделирование системы с регулятором $K_3$ при начальных условиях $x(0) = [1, 1, 1]^T$. На рисунках \ref{fig:reg_quality_x_3} и \ref{fig:reg_quality_u_3} представлены векторы состояния и управления с матрицей обратной связи $K_3$.

    \textbf{Наконец, примем $Q = 0$ и $R = 0$.}

    Решением уравнения Риккати будет:
    \[
        P_{K4} = 1e^{-38} \begin{bmatrix}
            -0.1131 &   0.0643 &  -0.0069 \\
            0.3260 &   0.1883 &   0.4913 \\
           -0.6704 &   0.2686 &   0.5877
        \end{bmatrix}
    \]

    Откуда:
    \[
        K_{4} = \begin{bmatrix}
            -5.1117  &  1.5990  & -4.6421
        \end{bmatrix}
    \]

    Проверим корректность синтеза, для этого найдем матрицу замкнутой системы:
    \[
        A + BK_{4} = \begin{bmatrix}
            0.7766  &  1.1980  & 3.7157 \\
            6.0000  &  -1.0000  & 6.0000 \\
           -6.0000  &  -1.0000  & -8.0000
        \end{bmatrix}
    \]

    Она имеет спектром
    \[
        \sigma(A + BK_{4}) =  \{-2, -3, -3.2234\}
    \]

    Все собственные числа лежат в пределах круга радиуса $r$ с центром в точке $(\beta, 0)$, а значит, синтез проведен корректно. Это подтверждается и рисунком \ref{fig:reg_quality4}.
    
    Проведем моделирование системы с регулятором $K_4$ при начальных условиях $x(0) = [1, 1, 1]^T$. На рисунках \ref{fig:reg_quality_x_4} и \ref{fig:reg_quality_u_4} представлены векторы состояния и управления с матрицей обратной связи $K_4$.

    \begin{figure}
        \centering
        \includegraphics[width=0.8\textwidth]{reg_quality4.png}
        \caption{Собственные числа при $(Q = 0, R = 0)$, регуляторе $K_4$}
        \label{fig:reg_quality4}
    \end{figure}
    \begin{figure}
        \centering
        \includegraphics[width=0.8\textwidth]{images/reg_quality_u_4.png}
        \caption{Управление системой при $(Q = 0, R = 0)$, регуляторе $K_4$}
        \label{fig:reg_quality_u_4}
    \end{figure}
    \begin{figure}
        \centering
        \includegraphics[width=0.8\textwidth]{images/reg_quality_x_4.png}
        \caption{Вектор состояния системы при $(Q = 0, R = 0)$, регуляторе $K_4$}
        \label{fig:reg_quality_x_4}
    \end{figure}

    Также сравним спектры замкнутых систем, полученных при помощи различных регуляторов, и найденные управления. Графики спектров представлены на рисунке \ref{fig:reg_quality_all}, а графики управлений - на рисунке \ref{fig:reg_quality_u_K1_K2_K3_K4}.
    
    \begin{figure}
        \centering
        \includegraphics[width=0.8\textwidth]{images/reg_quality_all.png}
        \caption{\centering Собственные числа при различных парах $(Q, R)$}
        \label{fig:reg_quality_all}
    \end{figure}
    \begin{figure}
        \centering
        \includegraphics[width=0.8\textwidth]{images/reg_quality_u_K1_K2_K3_K4.png}
        \caption{\centering Управление системой регуляторами с качественной экспоненциальной устойчивостью}
        \label{fig:reg_quality_u_K1_K2_K3_K4}
    \end{figure}

    Все синтезы дали замкнутые системы с собственными числами, лежащими в заданном круге с центром в точке $(\beta, 0)$ и радиусом $r$. При паре $Q = 0$, $R = 1$ все управляемые собственные числа находятся на границе - то есть опять-таки по аналогии с предыдущим пунктом получается как бы <<прижатый>> спектр.
    
    Стоит также отметить, что управление системой при помощи таких регуляторов будет некоторый схожий характер, так как все собственные числа находятся в определенной заданной комплексной области. Это видно по графикам управления на рисунке \ref{fig:reg_quality_u_K1_K2_K3_K4}, которые достаточно близки друг к другу. 

    В итоге с помощью качественной экспоненциальной устойчивости можно контролировать спектр замкнутой системы, а значит, и её переходные процессы и возникающие колебания, держа их в определенных задаваемыми параметрами $\beta$ и $r$ пределах.

    \newpage
    \section{Общие выводы}
    В данной работе были рассмотрены различные методы синтеза регуляторов и наблюдателей с заданной степенью экспоненциальной устойчивостью, все удалось создать и замоделировать, убедившись в работоспособности визуально. 
    
    В первом задании были рассмотрены методы синтеза регуляторов с помощью неравенства Ляпунова и уравнений Риккати, получено, что параметры $Q$ и $R$ в последнем задают связь между желаемым быстродействием системы и затрачиваемым на это управлением.
    
    Во втором задании были синтезированы наблюдатели по выходу, исследована взаимосвязь между степенью устойчивости регулятора и сходимости наблюдателя.

    Наконец, была изучена качественная экспоненциальная устойчивость, позволившая добиться от замкнутой системы средней степени устойчивости $\beta$ мод, задаваемых собственными числами.




    
\end{document}