\documentclass[a4paper,hidelinks,14pt]{extarticle}

\usepackage[utf8]{inputenc}
\usepackage[T2A]{fontenc}
\usepackage[english, russian]{babel}
\usepackage{lipsum}
\usepackage{amsmath}
\usepackage{amssymb}
\usepackage{amsfonts}
\usepackage{mathtools}
\usepackage{datetime}
\usepackage[pdftex]{graphicx}
\usepackage{indentfirst}
\usepackage{asymptote}
\usepackage{systeme}
\usepackage[dvipsnames]{xcolor}
\usepackage{lastpage}
\usepackage{fancybox,fancyhdr}
\usepackage{hyperref}
\usepackage[font={small,it}]{caption}
\fancyhead[L]{Лабораторная работа №1}
\fancyhead[C]{}
\fancyhead[R]{\textit{Задачи}}
\fancyfoot[L]{}
\fancyfoot[C]{\thepage\space}
\fancyfoot[R]{}
\pagestyle{fancy}
\newcommand{\gt}{\textgreater}
\newcommand{\lt}{\textless}
\usepackage{listings}
\usepackage{xcolor}
\lstset{
    basicstyle=\ttfamily\small,
    keywordstyle=\color{blue},
    commentstyle=\color{gray},
    stringstyle=\color{red},
    numbers=left,
    numberstyle=\color[gray]{0.7}\ttfamily\small,
    stepnumber=1,
    numbersep=8pt,
    frame=single,
    showstringspaces=false,
    tabsize=4,
    breaklines=true
}
\usepackage{subcaption}

\begin{document}
	\section{Вычисление длины $C^0$-гладкой кривой}
	Вычисленные через $A^*$ точки:
	\begin{figure}[h]
		\centering
		\includegraphics[width=0.8\textwidth]{points_plot.jpg}
		\caption{Карта и опорные точки}
		\label{12}
	\end{figure}

	Получающаяся траектория задаётся аналитически:
	\[
	S: \bigcup_{i=1}^{n-1}
	\left\{
	\begin{aligned}
	&-\sin\psi_i (x - x_i) + \cos\psi_i (y - y_i) = 0, \\
	&\psi_i = \arctan\left( \frac{y_{i+1} - y_i}{x_{i+1} - x_i} \right), \\
	&r_i = \sqrt{ (x_{i+1} - x_i)^2 + (y_{i+1} - y_i)^2 }, \\
	&Q_i(x, y) = \cos\psi_i (x - x_i) + \sin\psi_i (y - y_i) - r_i, \\
	&\text{Если } Q_i(x, y) > 0, \text{ то } i = i + 1.
	\end{aligned}
	\right.
	\]

	Физически - соединяем опорные точки $(x_i, y_i)$ прямыми, поэтому длину можно вычислить через обычную теорему Пифагора как
	\[
		L = \sum_{i=0}^{n-1} r_i = \sum_{i=0}^{n-1} \sqrt{ (x_{i+1} - x_i)^2 + (y_{i+1} - y_i)^2 } \approx 17.899
	\]

	\section{Вычисление длины $C^1$-гладкой кривой}
	Получающаяся траектория задаётся аналитически:
	\[
	S: \bigcup_{i=1}^{n-1}
	\left\{
	\begin{aligned}
	&-\sin\psi_i (x - x_i) + \cos\psi_i (y - y_i) = 0, \text{ если } Q_{1i} \le 0\\
	&(x - x_{ci})^2 + (y - y_{ci})^2 - R^2 = 0, \text{ если } Q_{2i} \le 0\\
	&\begin{bmatrix}
		x_{ci} \\
		y_{ci}
		\end{bmatrix}
		=
		\begin{bmatrix}
		x_{i+1} \\
		y_{i+1}
		\end{bmatrix}
		+ R_I^P(\psi_i) R_I^P(\delta_i)
		\begin{bmatrix}
		d_{ci} \\
		0
		\end{bmatrix}, \\
	&\psi_i = \arctan\!2\left( \frac{y_{i+1} - y_i}{x_{i+1} - x_i} \right), \\
	&\sigma_i = 
	\begin{cases}
	\pi - (\psi_{i+1} - \psi_i), & \text{если } (\psi_{i+1} - \psi_i) > 0 \\
	-\pi - (\psi_{i+1} - \psi_i), & \text{если } (\psi_{i+1} - \psi_i) \leq 0
	\end{cases}, \\
	&\delta_i = \pi - \frac{\sigma_i}{2}, \quad 
	d_{ci} = \left| \frac{R}{\sin\frac{\sigma_i}{2}} \right|, \quad
	 d_{i} = \left| \frac{R}{\tan\frac{\sigma_i}{2}} \right|, \\
	&r_i = \sqrt{ (x_{i+1} - x_i)^2 + (y_{i+1} - y_i)^2 }, \\
	&Q_{1i} = \cos\psi_i (x - x_i) + \sin\psi_i (y - y_i) - r_i + d_i, \\
	&Q_{2i} = \cos\psi_{i+1} (x - x_i) + \sin\psi_{i+1} (y - y_i) - d_i, \\
	&\text{Если } Q_{2i} > 0, \text{ то } i = i + 1.
	\end{aligned}
	\right.
	\]

	Длина построенной траектории считается суммой прямых и круговых участков, которые переходят друг в друга.
	
	Эту же величину можно получить и эквивалентным способом: сначала суммируем длины всех прямолинейных сегментов $r_i$ между точками $(x_i, y_i)$ и $(x_{i+1}, y_{i+1})$, а после для каждой внутренней точки $(x_i, y_i)$, где происходит изменение направления, вычисляется корректировка на значение $A_i - 2 d_i$ при $A_i = R \theta_i = R (\pi - |\sigma_i|)$ - длина окружности. В итоге для взятого в работе радиуса скругления $R = 0.8$ имеем:
	\[
		L = \sum_{i = 1}^{n-1} r_i + \sum_{j = 2}^{n-1} (A_j - 2 d_j) \approx 17.00623
	\]

	Это чуть меньше, чем у $C^0$-гладкого метода соединения точек, так как длина вписанных окружностей ниже длины прямых.

	\section{Вычисление длины $C^2$-гладкой кривой}
	Траектория задаётся аналитически через различные участки. 
	
	Движение по прямой:
	\begin{align*}
	S_1: 
	\begin{cases}
	\, -\sin\psi_i (x - x_i) + \cos\psi_i (y - y_i) = 0, \\[0.5em]
	\, \psi_i = \arctan\!2 \left( \dfrac{y_{i+1} - y_i}{x_{i+1} - x_i} \right).
	\end{cases}
	\end{align*}

	Движение по окружности:
	\begin{align*}
	S_3:
	\begin{cases}
	\, (x - x_{ci})^2 + (y - y_{ci})^2 - R^2 = 0, \\[0.5em]
	\, \begin{bmatrix} x_{ci} \\ y_{ci} \end{bmatrix} = \begin{bmatrix} x_{i+1} \\ y_{i+1} \end{bmatrix} + R^I_P (\psi_i) R^P_I (\delta_i) \begin{bmatrix} d_{ci} \\ 0 \end{bmatrix}, \\[1em]
	\, d_{ci} = d_{1i} + d_{2i}, \\[0.5em]
	\, d_{1i} = \sqrt{R^2 - (h_i/2)^2}, \\[0.5em]
	\, d_{2i} = \sqrt{h_{i2}^2 - (h_i/2)^2} \\[0.5em]
	\, h_i = \sqrt{(x_{S_{2}e} - x_{S_{1e}})^2 + (y_{S_{2}e} - y_{S_{1e}})^2}, \\[0.5em]
	\, h_{i2} = \sqrt{(x_{S_{2}e} - x_{i+1})^2 + (y_{S_{2}e} - y_{i+1})^2}, \\[0.5em]
	\, \delta_i = \pi - \dfrac{\sigma_i}{2}, \\[1em]
	\, \sigma_i =
	\begin{cases}
	\pi - (\psi_{i+1} - \psi_i), & \text{если } (\psi_{i+1} - \psi_i) > 0, \\[0.5em]
	-\pi - (\psi_{i+1} - \psi_i), & \text{если } (\psi_{i+1} - \psi_i) \leq 0,
	\end{cases} \\[1em]
	\, \psi_i = \arctan\!2 \left( \dfrac{y_{i+1} - y_i}{x_{i+1} - x_i} \right).
	\end{cases}
	\end{align*}

	Движение по <<открывающей>> параболе:
	\begin{align*}
	S_2:
	\begin{cases}
	\, \begin{bmatrix} x_L \\ y_L \end{bmatrix} = \begin{bmatrix} x_{i+1} \\ y_{i+1} \end{bmatrix} + R^P_I(\psi_i) \begin{bmatrix} x - d_i \\ y \end{bmatrix}, \\[1em]
	\, k(x_L)^3 - y_L = 0, \\[0.5em]
	\, d_i = \left| \dfrac{R}{\tan\frac{\theta}{2}} \right|, \\[1em]
	\, \psi_i = \arctan\!2 \left( \dfrac{y_{i+1} - y_i}{x_{i+1} - x_i} \right).
	\end{cases}
	\end{align*}

	Движение по <<закрывающей>> параболе:
	\begin{align*}
	S_4:
	\begin{cases}
	\, \begin{bmatrix} x_L \\ y_L \end{bmatrix} = \begin{bmatrix} x_{i+1} \\ y_{i+1} \end{bmatrix} + R_I^P(\psi_{i+1}) \begin{bmatrix} x + d_i \\ y \end{bmatrix}, \\[1em]
	\, -k(x_L)^3 - y_L = 0, \\[0.5em]
	\, d_i = \left| \dfrac{R}{\tan\frac{\sigma_i}{2}} \right|, \\[1em]
	\, \psi_i = \arctan\!2 \left( \dfrac{y_{i+1} - y_i}{x_{i+1} - x_i} \right).
	\end{cases}
	\end{align*}

	Переключение всё также будет происходить с помощью прямых $Q_i$. Итоговая кривая $S$ определяется как объединение:
	\[
	S = \bigcup_{i=1}^{n-1} 
	\begin{cases}
	S_{1}, \text{ если } Q_{1i} \leq 0, \\
	S_{2}, \text{ если } Q_{2i} \leq 0, \\
	S_{3}, \text{ если } Q_{3i} \leq 0, \\
	S_{4}, \text{ если } Q_{4i} \leq 0, \\
	Q_{1i} = \cos \psi_i (x - x_i) + \sin \psi_i (y - y_i) - r_i + d_i, \\
	Q_{2i} = \cos \psi_{i+1} (x - x_{i+1}) + \sin \psi_{i+1} (y - y_{i+1}) - d_i, \\
	Q_{3i} = \cos \alpha_1 (x - x_i) + \sin \alpha_1 (y - y_i) + t_1, \\
	Q_{4i} = \cos \alpha_2 (x - x_i) + \sin \alpha_2 (y - y_i) - t_2, \\
	\text{Если } Q_{2i} > 0, \text{ то } i = i + 1.
	\end{cases}
	\]

	Введенные вспомогательные параметры:
	\[
		\begin{cases}
		t_1 &= \left[ x_{S_1e} \quad y_{S_1e} \right]^T, \\
		t_2 &= \left[ x_{S_2e} \quad y_{S_2e} \right]^T, \\
		\alpha_1 &= \frac{\psi_{i+1} - \beta_i}{2}, \\
		\alpha_2 &= \psi_{i+1} - \frac{\psi_{i+1} - \beta_i}{2}, \\
		\beta_i &= \arccos \left( 1 - \frac{h_i^2}{2R^2} \right) = 2 \arcsin (\frac{h_i}{2R}).
		\end{cases}
	\]
	
	Геометрические характеристики:
	\[
		\begin{cases}
		h_i &= \sqrt{(x_{S_2e} - x_{S_1e})^2 + (y_{S_2e} - y_{S_1e})^2}, \\
		\psi_i &= \arctan 2 \left( \frac{y_{i+1} - y_i}{x_{i+1} - x_i} \right), \\
		r_i &= \sqrt{(x_{i+1} - x_i)^2 + (y_{i+1} - y_i)^2}, \\
		d_i &= \left| \frac{R}{\tan \frac{\pi}{2}}\right|.
		\end{cases}
	\]

	где $x_{S_1e}$, $x_{S_2e}$, $y_{S_1e}$ и $y_{S_2e}$ — конечные точки парабол $S_2$ и $S_4$.

	Длина построенной траектории считается суммой прямолинейных участков $S_1$, переходных кубических парабол $S_2, S_4$ и круговых сегментов $S_3$, которые последовательно переходят друг в друга.

	Эту же величину можно получить эквивалентным способом через процедуру корректировки. Сначала суммируются длины всех прямолинейных сегментов $r_i$ между узловыми точками $(x_i, y_i)$ и $(x_{i+1}, y_{i+1})$. Затем для каждой внутренней точки, где происходит изменение направления, вычисляется корректирующее значение $\Delta L_j$, учитывающее замену острого угла на систему сопряженных кривых.

	Для принятого в работе радиуса скругления $R = 0.8$ имеем:
	\[
		L = \sum_{i = 1}^{n-1} r_i + \sum_{j = 1}^{n-2} \Delta L_j \approx 17.00662
	\]

	Здесь корректировка $\Delta L_j$ для каждого поворота определяется следующим образом:
	\[
		\Delta L_j = 2 \cdot L^p_{j} + L^a_{j} - 2 d_j
	\]

	Компоненты формулы вычисляются на основе геометрических параметров сопряжения:
	\begin{enumerate}
		\item Криволинейный интеграл $y = kx^3$ на участке $[0, x_L]$:
		\[ L_{j}^p = \int_{0}^{x_L} \sqrt{1 + (3kx^2)^2} \, dx \]
		
		При $x_L = \frac{1}{6kR}$ - проекция параболы, обеспечивающая непрерывность кривизны.
		
		\item Остаточный круговой сегмент между переходными кривыми:
		\[ L_{j}^a = R \cdot \beta_j, \quad \beta_j = 2 \arcsin\left( \frac{h_j}{2R} \right) \]
		здесь $h_j$ - хорда между точками выхода первой параболы и входа во вторую.
		
		\item Расстояние от вершины до точки касания:
		\[ d_j = \left| \frac{R}{\tan\left(\frac{\sigma_j}{2}\right)} \right| \]
	\end{enumerate}

	Итоговое значение длины получается больше, чем у $C^1$-гладкой кривой, из-за более сложной траектории параболического входа и выхода с окружностей.
	
	\section{Длина B-сплайновой кривой}
	Пусть задан неубывающий вектор равеномерно распределенных от 1 до $n-k$ узлов $u_i$
	\[
	U = \{u_0, u_1, \dots, u_m\}
	\]

	А также заданы базисные функции B-сплайна степени $k$, определённые рекурсией:
	\[
	N_{i,0}(t) =
	\begin{cases}
	1, & u_i \le t < u_{i+1}\\
	0, & \text{иначе}
	\end{cases}
	\]
	\[
	N_{i,k}(t) =
	\frac{t-u_i}{u_{i+k}-u_i} N_{i,k-1}(t)
	+
	\frac{u_{i+k+1}-t}{u_{i+k+1}-u_{i+1}} N_{i+1,k-1}(t).
	\]

	Кривая задаётся контрольными точками $P_i = (x_i, y_i)$:
	\[
	C(t) = \sum_i N_{i,k}(t) P_i, \qquad t \in [u_0, u_m]
	\]

	Производная базисной функции B-сплайна имеет вид:
	\[
	N'_{i,k}(t)
	=
	\frac{k}{u_{i+k}-u_i} N_{i,k-1}(t)
	-
	\frac{k}{u_{i+k+1}-u_{i+1}} N_{i+1,k-1}(t).
	\]

	Дифференцируя кривую $C(t)$, получаем:
	\[
		C'(t)
		= \sum_i N'_{i,k}(t) P_i
		= \sum_i N_{i,k-1}(t)
		\frac{k}{u_{i+k}-u_i} (P_i - P_{i-1})
	\]

	Введём контрольные точки производной:
	\[
	D_i = \frac{k}{u_{i+k}-u_i} (P_i - P_{i-1}),
	\]

	Тогда производная кривой представляется в виде
	\[
	C'(t) = \sum_i N_{i,k-1}(t) D_i.
	\]

	Таким образом, производная B-сплайна степени $k$ является B-сплайном степени $k-1$.

	Длина гладкой параметрической кривой определяется как
	\[
	L = \int_{u_0}^{u_m} \left\| C'(t) \right\| \, dt.
	\]

	Поскольку $D_i = (D_i^x, D_i^y)$, имеем:
	\begin{align}
	x'(t) &= \sum_i N_{i,k-1}(t) D_i^x, \\
	y'(t) &= \sum_i N_{i,k-1}(t) D_i^y.
	\end{align}

	Следовательно, длина B-сплайновой кривой равна
	\[
	L
	=
	\int_{u_0}^{u_m}
	\sqrt{
	\left( \sum_i N_{i,k-1}(t) D_i^x \right)^2
	+
	\left( \sum_i N_{i,k-1}(t) D_i^y \right)^2
	}
	\, dt.
	\]

	Для взятого в работе $k = 3$ имеем длину:
	\[
		L \approx 16.6272
	\]

	Несколько меньше, чем у всех ранее вычисленных.

	
	


\end{document}