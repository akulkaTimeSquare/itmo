\documentclass[a4paper,hidelinks,14pt]{extarticle}

\usepackage[utf8]{inputenc}
\usepackage[T2A]{fontenc}
\usepackage[english, russian]{babel}
\usepackage{lipsum}
\usepackage{amsmath}
\usepackage{amssymb}
\usepackage{amsfonts}
\usepackage{mathtools}
\usepackage{datetime}
\usepackage[pdftex]{graphicx}
\usepackage{indentfirst}
\usepackage{asymptote}
\usepackage{systeme}
\usepackage[dvipsnames]{xcolor}
\usepackage{lastpage}
\usepackage{fancybox,fancyhdr}
\usepackage{hyperref}
\usepackage[font={small,it}]{caption}
\fancyhead[L]{ЛР №1}
\fancyhead[C]{}
\fancyhead[R]{\textit{Алгоритмы планирования траекторий}}
\fancyfoot[L]{}
\fancyfoot[C]{\thepage\space}
\fancyfoot[R]{}
\pagestyle{fancy}
\newcommand{\gt}{\textgreater}
\newcommand{\lt}{\textless}
\usepackage{listings}
\usepackage{xcolor}
\lstset{
    basicstyle=\ttfamily\small,
    keywordstyle=\color{blue},
    commentstyle=\color{gray},
    stringstyle=\color{red},
    numbers=left,
    numberstyle=\color[gray]{0.7}\ttfamily\small,
    stepnumber=1,
    numbersep=8pt,
    frame=single,
    showstringspaces=false,
    tabsize=4,
    breaklines=true
}
\usepackage{subcaption}

\begin{document}
	\begin{titlepage}
		\setlength{\parindent}{0ex}
		
		\begin{center}
			\textsc{
				\vspace{1ex}
                Научно-исследовательский университет ИТМО \\
				\vspace{0.5ex}
				Факультет систем управления и робототехники \\
				\vspace{0.5ex}
			}
		\end{center}
		
		\vspace{45mm}
		
		\begin{center}
			Отчет по лабораторной работе №1\\
			Алгоритмы планирования траекторий\\
			Вариант 1
		\end{center}
		
		\vspace{50mm}
		
		\begin{minipage}{.4\linewidth}
			Выполнили студенты
            \\
			\\
			Преподаватель
		\end{minipage}
		\hfill
		\begin{minipage}{.5\linewidth}
			\begin{flushright}
				Мовчан Игорь Евгеньевич
                \\
				Гридусов Денис Дмитриевич
				\\
				Краснов Александр Юрьевич
			\end{flushright}
		\end{minipage}
		
		\vfill
		\begin{center}
			Санкт-Петербург
			\\
			2025
		\end{center}
		
	\end{titlepage}

	\tableofcontents
	\clearpage
	
	\section{Бинарная карта и алгоритм $A^*$}
	Сформируем бинарную карту размером 10 на 10 ячеек, содержащую не менее трети недоступных к посещению клеток, обозначенных серым цветом на рисунке ниже. В качестве начальной ячейки выберем верхний левый угол с координатами $(x, y) = (1, 1)$, в качестве конечной - нижний правый угол с координатами $(x, y) = (10, 10)$. Обозначим их соответственно зеленым и красным цветом.
	\begin{figure}[h]
		\centering
		\includegraphics[width=0.95\textwidth]{images/map_plot.jpg}
		\caption{Сформированная карта движения}
	\end{figure}

	Применим алгоритм $A^*$ к этой карте для нахождения оптимального пути от начальной ячейки до конечной, используя функцию
	\[
		f(n) = g(n) + h(n)
	\]

	Здесь $h(n)$ - эвристическая функция для построения пути, а $g(n)$ - функция подсчета расстояния. Примем их равными соответственно:
	\[
		h(n) = \sqrt{(n_x - \text{goal}_x)^2 + (n_y - \text{goal}_y)^2}
	\]
	\[
		g(n_{\text{neigh}}) = g(n_{\text{cur}}) + \text{cost}_{\text{neigh}} \cdot d(n_{\text{cur}}, n_{\text{neigh}})
	\]

	В рассматриваемом случае доступными к посещению из отдельно взятой клетки примем все 8 её соседей, поэтому $d = \sqrt{2}$, если сосед находится на диагонале, и $d = 1$ иначе. Отметим также, что <<стоимость>> всех клеток будем считать равной 1.
	
	На рисунке \ref{2} синими точками помечены ячейки, через которые проходит найденный эффективный путь, состоящий из 18 клеток и 9 поворотов. Декартовы координаты точек (здесь и в дальнейшем) примем равными номеру ячейки карты по горизонтали и вертикали соответственно.
	\begin{figure}[h]
		\centering
		\includegraphics[width=0.95\textwidth]{images/points_plot.jpg}
		\caption{Бинарная карта с полученной траекторией}
		\label{2}
	\end{figure}

	\newpage
	\section{$C^0$-гладкая траектория}
	Рассмотрим самый очевидный способ соединения полученных точек $(x_i, y_i)$ - прямыми отрезками. Получающаяся при этом траектория аналитически задаётся как:
	\[
	S: \bigcup_{i=1}^{n-1}
	\left\{
	\begin{aligned}
	&-\sin\psi_i (x - x_i) + \cos\psi_i (y - y_i) = 0, \\
	&\psi_i = \arctan\left( \frac{y_{i+1} - y_i}{x_{i+1} - x_i} \right), \\
	&r_i = \sqrt{ (x_{i+1} - x_i)^2 + (y_{i+1} - y_i)^2 }, \\
	&Q_i(x, y) = \cos\psi_i (x - x_i) + \sin\psi_i (y - y_i) - r_i, \\
	&\text{Если } Q_i(x, y) > 0, \text{ то } i = i + 1.
	\end{aligned}
	\right.
	\]

	Её производные не являются непрерывными на всем отрезке движения, поэтому она имеет нулевую гладкость. В целом, для $C^{n}$-гладкой кривой все её $n$ производных являются непрерывными. 
	
	В описании траектории $Q_i$ задают неявные прямые, разделяющие плоскость на две части, таким образом, возможно произвести переключение направления движения и рассчитать относительное положение объекта управления.

	Применение алгоритма для полученных ранее точек приведено на рисунке \ref{3}. Расчёт кривизны кривой производится по формуле
	\[
		k = \frac{x' y'' - y' x''}{((x')^2 + (y')^2)^{3/2}}			
	\]

	Где производные аппроксимируются разностями:
	\[
		x'_i \approx \frac{x_{i+1} - x_{i-1}}{2\Delta t}, \quad
		x''_i \approx \frac{x_{i+1} - 2x_i + x_{i-1}}{2\Delta t^2}
	\]

	Итоговый график приведен на рисунке \ref{4}. Траектория состоит только из прямых линий, для которых кривизна нулевая, поэтому и вся функция кривизны пути выходит константным нулем.

	\begin{figure}
		\centering
		\includegraphics[width=0.95\textwidth]{images/c0.jpg}
		\caption{$C^0$-гладкая траектория}
		\label{3}
	\end{figure}
	\begin{figure}
		\centering
		\includegraphics[width=0.7\textwidth]{images/c0_k.jpg}
		\caption{График кривизны $C^0$-гладкой траектории}
		\label{4}
	\end{figure}

	\newpage
	\section{$C^1$-гладкая траектория}
	Повысим гладкость получаемой в предыдущем пункте траектории с помощью добавления участков, состоящих из дуг окружностей с центрами в точках $(x_{ci}, y_{ci})$ и радиусом окружности $R = 0.8$, на которые будем входить из движения по прямому отрезку. Аналитически алгоритм описывается следующим выражением:
	\[
	S: \bigcup_{i=1}^{n-1}
	\left\{
	\begin{aligned}
	&-\sin\psi_i (x - x_i) + \cos\psi_i (y - y_i) = 0, \text{ если } Q_{1i} \le 0\\
	&(x - x_{ci})^2 + (y - y_{ci})^2 - R^2 = 0, \text{ если } Q_{2i} \le 0\\
	&\begin{bmatrix}
		x_{ci} \\
		y_{ci}
		\end{bmatrix}
		=
		\begin{bmatrix}
		x_{i+1} \\
		y_{i+1}
		\end{bmatrix}
		+ R_I^P(\psi_i) R_I^P(\delta_i)
		\begin{bmatrix}
		d_{ci} \\
		0
		\end{bmatrix}, \\
	&\psi_i = \arctan\!2\left( \frac{y_{i+1} - y_i}{x_{i+1} - x_i} \right), \\
	&\sigma_i = 
	\begin{cases}
	\pi - (\psi_{i+1} - \psi_i), & \text{если } (\psi_{i+1} - \psi_i) > 0 \\
	-\pi - (\psi_{i+1} - \psi_i), & \text{если } (\psi_{i+1} - \psi_i) \leq 0
	\end{cases}, \\
	&\delta_i = \pi - \frac{\sigma_i}{2}, \quad 
	d_{ci} = \left| \frac{R}{\sin\frac{\sigma_i}{2}} \right|, \quad
	 d_{i} = \left| \frac{R}{\tan\frac{\sigma_i}{2}} \right|, \\
	&r_i = \sqrt{ (x_{i+1} - x_i)^2 + (y_{i+1} - y_i)^2 }, \\
	&Q_{1i} = \cos\psi_i (x - x_i) + \sin\psi_i (y - y_i) - r_i + d_i, \\
	&Q_{2i} = \cos\psi_{i+1} (x - x_i) + \sin\psi_{i+1} (y - y_i) - d_i, \\
	&\text{Если } Q_{2i} > 0, \text{ то } i = i + 1.
	\end{aligned}
	\right.
	\]

	В описании использовали матрицу поворота
	\[
	R_I^P(\varphi) = \begin{bmatrix}
		cos(\varphi) & \sin(\varphi) \\
		-\sin(\varphi) & \cos(\varphi)
	\end{bmatrix}
	\]

	Переключение участков движения по прямой и окружности происходит с помощью $Q_{1i}$ и $Q_{2i}$ соответственно.

	Применение алгоритма для полученных ранее точек приведено на рисунке \ref{5}, а график кривизны найденной траектории - на рисунке \ref{6}. На последнем можем видеть резкие скачки на величину $1/R = 1.25$, характеризующие движение по окружности.

	\begin{figure}
		\centering
		\includegraphics[width=0.95\textwidth]{images/c1.jpg}
		\caption{$C^1$-гладкая траектория}
		\label{5}
	\end{figure}
	\begin{figure}
		\centering
		\includegraphics[width=0.7\textwidth]{images/c1_k.jpg}
		\caption{График кривизны $C^1$-гладкой траектории}
		\label{6}
	\end{figure}

	\newpage
	\section{$C^2$-гладкая траектория}
	У предыдущего метода есть серьезный недостаток в виде мгновенного изменения скорости движения объекта, что в реальном мире неосуществимо. В связи с чем актуальной становится задача повышения гладкости траектории до второго порядка с помощью дополнительного соединения прямых и окружностей параболами с $\approx$ линейной производной. Описание кривой представлено ниже.

	Движение по прямой:
	\begin{align*}
	S_1: 
	\begin{cases}
	\, -\sin\psi_i (x - x_i) + \cos\psi_i (y - y_i) = 0, \\[0.5em]
	\, \psi_i = \arctan\!2 \left( \dfrac{y_{i+1} - y_i}{x_{i+1} - x_i} \right).
	\end{cases}
	\end{align*}

	Движение по окружности:
	\begin{align*}
	S_3:
	\begin{cases}
	\, (x - x_{ci})^2 + (y - y_{ci})^2 - R^2 = 0, \\[0.5em]
	\, \begin{bmatrix} x_{ci} \\ y_{ci} \end{bmatrix} = \begin{bmatrix} x_{i+1} \\ y_{i+1} \end{bmatrix} + R^I_P (\psi_i) R^P_I (\delta_i) \begin{bmatrix} d_{ci} \\ 0 \end{bmatrix}, \\[1em]
	\, d_{ci} = d_{1i} + d_{2i}, \\[0.5em]
	\, d_{1i} = \sqrt{R^2 - (h_i/2)^2}, \\[0.5em]
	\, d_{2i} = \sqrt{h_{i2}^2 - (h_i/2)^2} \\[0.5em]
	\, h_i = \sqrt{(x_{S_{2}e} - x_{S_{1e}})^2 + (y_{S_{2}e} - y_{S_{1e}})^2}, \\[0.5em]
	\, h_{i2} = \sqrt{(x_{S_{2}e} - x_{i+1})^2 + (y_{S_{2}e} - y_{i+1})^2}, \\[0.5em]
	\, \delta_i = \pi - \dfrac{\sigma_i}{2}, \\[1em]
	\, \sigma_i =
	\begin{cases}
	\pi - (\psi_{i+1} - \psi_i), & \text{если } (\psi_{i+1} - \psi_i) > 0, \\[0.5em]
	-\pi - (\psi_{i+1} - \psi_i), & \text{если } (\psi_{i+1} - \psi_i) \leq 0,
	\end{cases} \\[1em]
	\, \psi_i = \arctan\!2 \left( \dfrac{y_{i+1} - y_i}{x_{i+1} - x_i} \right).
	\end{cases}
	\end{align*}

	Движение по <<открывающей>> параболе:
	\begin{align*}
	S_2:
	\begin{cases}
	\, \begin{bmatrix} x_L \\ y_L \end{bmatrix} = \begin{bmatrix} x_{i+1} \\ y_{i+1} \end{bmatrix} + R^P_I(\psi_i) \begin{bmatrix} x - d_i \\ y \end{bmatrix}, \\[1em]
	\, k(x_L)^3 - y_L = 0, \\[0.5em]
	\, d_i = \left| \dfrac{R}{\tan\frac{\theta}{2}} \right|, \\[1em]
	\, \psi_i = \arctan\!2 \left( \dfrac{y_{i+1} - y_i}{x_{i+1} - x_i} \right).
	\end{cases}
	\end{align*}

	Движение по <<закрывающей>> параболе:
	\begin{align*}
	S_4:
	\begin{cases}
	\, \begin{bmatrix} x_L \\ y_L \end{bmatrix} = \begin{bmatrix} x_{i+1} \\ y_{i+1} \end{bmatrix} + R_I^P(\psi_{i+1}) \begin{bmatrix} x + d_i \\ y \end{bmatrix}, \\[1em]
	\, -k(x_L)^3 - y_L = 0, \\[0.5em]
	\, d_i = \left| \dfrac{R}{\tan\frac{\pi}{2}} \right|, \\[1em]
	\, \psi_i = \arctan\!2 \left( \dfrac{y_{i+1} - y_i}{x_{i+1} - x_i} \right).
	\end{cases}
	\end{align*}

	Переключение всё также будет происходить с помощью прямых $Q_i$. Итоговая кривая $S$ определяется как объединение:
	\[
	S = \bigcup_{i=1}^{n-1} 
	\begin{cases}
	S_{1}, \text{ если } Q_{1i} \leq 0, \\
	S_{2}, \text{ если } Q_{2i} \leq 0, \\
	S_{3}, \text{ если } Q_{3i} \leq 0, \\
	S_{4}, \text{ если } Q_{4i} \leq 0, \\
	Q_{1i} = \cos \psi_i (x - x_i) + \sin \psi_i (y - y_i) - r_i + d_i, \\
	Q_{2i} = \cos \psi_{i+1} (x - x_{i+1}) + \sin \psi_{i+1} (y - y_{i+1}) - d_i, \\
	Q_{3i} = \cos \alpha_1 (x - x_i) + \sin \alpha_1 (y - y_i) + t_1, \\
	Q_{4i} = \cos \alpha_2 (x - x_i) + \sin \alpha_2 (y - y_i) - t_2, \\
	\text{Если } Q_{2i} > 0, \text{ то } i = i + 1.
	\end{cases}
	\]

	Введенные вспомогательные параметры:
	\[
		\begin{cases}
		t_1 &= \left[ x_{S_1e} \quad y_{S_1e} \right]^T, \\
		t_2 &= \left[ x_{S_2e} \quad y_{S_2e} \right]^T, \\
		\alpha_1 &= \frac{\psi_{i+1} - \beta}{2}, \\
		\alpha_2 &= \psi_{i+1} - \frac{\psi_{i+1} - \beta}{2}, \\
		\beta &= \arccos \left( 1 - \frac{h_i^2}{2R^2} \right).
		\end{cases}
	\]
	
	Геометрические характеристики:
	\[
		\begin{cases}
		h_i &= \sqrt{(x_{S_2e} - x_{S_1e})^2 + (y_{S_2e} - y_{S_1e})^2}, \\
		\psi_i &= \arctan 2 \left( \frac{y_{i+1} - y_i}{x_{i+1} - x_i} \right), \\
		r_i &= \sqrt{(x_{i+1} - x_i)^2 + (y_{i+1} - y_i)^2}, \\
		d_i &= \left| \frac{R}{\tan \frac{\pi}{2}} \right|.
		\end{cases}
	\]

	где $x_{S_1e}$, $x_{S_2e}$, $y_{S_1e}$ и $y_{S_2e}$ — конечные точки парабол $S_2$ и $S_4$.

	Параболы примечательны тем, что их производные можно в определенном порядке представить в виде линейно-нарастающей функции. Они создают дополнительный линейный разгон, предотвращая резкие изменения скорости движения.

	Применение алгоритма для полученных ранее через $A^*$ точек для $k = 20$ приведено на рисунке \ref{7}, а график кривизны полученной траектории - на рисунке \ref{8}. На последнем можем видеть плавный переход на величину $1/R = 1.25$. Резкий скачок уничтожен, движение полностью физически реализуемо!

	Стоит отметить, что при увеличении $k$ траектория второй гладкости всё больше приближается к $C^1$-гладкой. Пожалуй, наиболее заметен данный эффект при сравнении рисунков \ref{11} и \ref{12} при $k_1 = 0.65$ и $k_2 = 100$, где красным цветом обозначен путь первой гладкости.

	\begin{figure}
		\centering
		\includegraphics[width=0.95\textwidth]{images/c2.jpg}
		\caption{$C^2$-гладкая траектория}
		\label{7}
	\end{figure}
	\begin{figure}
		\centering
		\includegraphics[width=0.7\textwidth]{images/c2_k.jpg}
		\caption{График кривизны $C^2$-гладкой траектории}
		\label{8}
	\end{figure}
	\begin{figure}
		\centering
		\includegraphics[width=0.8\textwidth]{images/c2_k1.jpg}
		\caption{$C^2$-гладкая траектория при $k = 0.65$}
		\label{11}
	\end{figure}
	\begin{figure}
		\centering
		\includegraphics[width=0.8\textwidth]{images/c2_k2.jpg}
		\caption{$C^2$-гладкая траектория при $k = 100$}
		\label{12}
	\end{figure}


	\newpage
	\section{Сглаживание с помощью $B$-сплайна}
	Наконец, осуществим сглаживание $C^0$-гладкой траектории при помощи $B$-сплайна. Сперва рекурсивно зададим базисные функции $N_{i, k}(t)$. Для $k = 0$:
	\[
	N_{i,1}(t) =
	\begin{cases}
	1, & u_i \leq t < u_{i+1} \\
	0, & \text{иначе}
	\end{cases}
	\]
	
	Для $k \neq 0$:
	\[
	N_{i,k}(t) = \frac{t - u_i}{u_{i+k} - u_i} N_{i,k-1}(t) + \frac{u_{i+k+1} - t}{u_{i+k+1} - u_{i+1}} N_{i+1,k-1}(t),
	\]
	
	Здесь $u_i$ - узлы, равномерно распределенные от $1$ до $n-k$ с единичным шагом, а $t$ — параметр в пределах той же области.

	Точка кривой вычисляется тогда по формуле:
	\[
	C(t) = \sum_{i=0}^{n} N_{i,k}(n) \cdot CP_i, \quad \text{где } CP_i - \text{ контрольные точки}
	\]

	Контрольными же точками определим ранее вычисленные с помощью $A^*$ алгоритма нахождения оптимального пути. Они зададут опору, но алгоритм, вообще говоря, не обязан через них проходить.
	
	Применение сглаживания приведено на рисунке \ref{9}, а график кривизны получаемой траектории - на рисунке \ref{10}. При изучении последнего можем сказать, что необходимая для физической реализуемости гладкость достигается - нет резких скачков или иных аномалий. Однако в целом график сильно отличается от тех, что изучались нами до этого. Как минимум, не сходятся пики, а движение между участками не вполне предсказуемо.

	\begin{figure}
		\centering
		\includegraphics[width=0.95\textwidth]{images/bsplines.jpg}
		\caption{Сглаживание с помощью B-сплайна}
		\label{9}
	\end{figure}
	\begin{figure}
		\centering
		\includegraphics[width=0.7\textwidth]{images/bsplines_k.jpg}
		\caption{График кривизны при сглаживании с помощью B-сплайна}
		\label{10}
	\end{figure}

	\newpage
	\section{Выводы}
	В результате выполнения лабораторной работы были изучены алгоритмы планирования траекторий движения: $C^n$-гладкие кривые, где $n \le 2$, и сглаживание B-сплайнами. Было получено, что нулевая гладкость физически не является реализуемой - необходимо резко менять и направление движения, и скорость. Повышение $n$ до 1 помогло решить проблему с видом траектории - она лишилась резких поворотов, а до $n = 2$ - проблему с плавным резким скорости. Достигалось всё это за счёт добавления дополнительных участков движения по окружности и кубической параболе соответственно.

	Результаты для сглаживания сплайнами также оказались удовлетворительными - итоговая кривизна не имела резких скачков, а движение являлось плавным.





	
\end{document}