\documentclass[a4paper,hidelinks,14pt]{extarticle}

\usepackage[utf8]{inputenc}
\usepackage[T2A]{fontenc}
\usepackage[english, russian]{babel}
\usepackage{lipsum}
\usepackage{amsmath}
\usepackage{amssymb}
\usepackage{amsfonts}
\usepackage{mathtools}
\usepackage{datetime}
\usepackage[pdftex]{graphicx}
\usepackage{indentfirst}
\usepackage{asymptote}
\usepackage{systeme}
\usepackage[dvipsnames]{xcolor}
\usepackage{lastpage}
\usepackage{fancybox,fancyhdr}
\usepackage{hyperref}
\usepackage[font={small,it}]{caption}
\fancyhead[L]{ЛР №1}
\fancyhead[C]{}
\fancyhead[R]{\textit{Алгоритмы планирования траекторий}}
\fancyfoot[L]{}
\fancyfoot[C]{\thepage\space}
\fancyfoot[R]{}
\pagestyle{fancy}
\newcommand{\gt}{\textgreater}
\newcommand{\lt}{\textless}
\usepackage{listings}
\usepackage{xcolor}
\lstset{
    basicstyle=\ttfamily\small,
    keywordstyle=\color{blue},
    commentstyle=\color{gray},
    stringstyle=\color{red},
    numbers=left,
    numberstyle=\color[gray]{0.7}\ttfamily\small,
    stepnumber=1,
    numbersep=8pt,
    frame=single,
    showstringspaces=false,
    tabsize=4,
    breaklines=true
}
\usepackage{subcaption}

\begin{document}
	\begin{titlepage}
		\setlength{\parindent}{0ex}
		
		\begin{center}
			\textsc{
				\vspace{1ex}
                Научно-исследовательский университет ИТМО \\
				\vspace{0.5ex}
				Факультет систем управления и робототехники \\
				\vspace{0.5ex}
			}
		\end{center}
		
		\vspace{45mm}
		
		\begin{center}
			Отчет по лабораторной работе №1\\
			Алгоритмы планирования траекторий\\
			Вариант 11
		\end{center}
		
		\vspace{50mm}
		
		\begin{minipage}{.4\linewidth}
			Выполнил студент
            \\
			Преподаватель
		\end{minipage}
		\hfill
		\begin{minipage}{.5\linewidth}
			\begin{flushright}
				Мовчан Игорь Евгеньевич
                \\
				Краснов Александр Юрьевич
			\end{flushright}
		\end{minipage}
		
		\vfill
		\begin{center}
			Санкт-Петербург
			\\
			2025
		\end{center}
		
	\end{titlepage}

	\tableofcontents
	\clearpage
	
	\section{Формирование бинарной карты}
	Сформируем бинарную карту размером 10 на 10 клеток, содержащую не менее трети недоступных к посещению клеток, обозначенных черным цветом на рисунке, остальные - доступны, раскрашены серым цветом. В качестве начальной точки выберем верхний левый угол, в качестве конечной - нижний правый угол. Обозначим их соответственно зеленым и красным цветом.
	\begin{figure}[h]
		\centering
		\includegraphics[width=\textwidth]{images/map_plot.jpg}
		\caption{Бинарная карта}
	\end{figure}

	\newpage
	Применим алгоритм A* к этой карте. Итоговая траектория содержит 18 клеток и 10 поворотов. Фиолетовыми точками обозначены клетки, через которые проходит траектория.
	\begin{figure}[h]
		\centering
		\includegraphics[width=\textwidth]{images/points_plot.jpg}
		\caption{Траектория: 18 клеток, 10 поворотов}
	\end{figure}

	\newpage
	\section{$C^0$-гладкая траектория}
	Построим $C^0$-гладкую траекторию через полученные точки.


	



	
\end{document}