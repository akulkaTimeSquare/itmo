\documentclass[a4paper,hidelinks,14pt]{extarticle}

\usepackage[utf8]{inputenc}
\usepackage[T2A]{fontenc}
\usepackage[english, russian]{babel}
\usepackage{amsmath}
\usepackage{amsfonts}
\usepackage{mathtools}
\usepackage{datetime}
\usepackage[pdftex]{graphicx}
\usepackage{indentfirst}
\usepackage{asymptote}
\usepackage{systeme}
\usepackage{listings}
\usepackage[usenames,dvipsnames]{color}	
\usepackage[font={small,it}]{caption}
\usepackage{gensymb}
\usepackage{hyperref}
\usepackage[framed,autolinebreaks,numbered,useliterate]{mcode}
\newcommand{\Span}{\text{Span}}
\newcommand{\Img}{\text{Im}}
\newcommand{\Ker}{\text{Ker}}
\hypersetup{
	colorlinks=true,
	linkcolor=black,
	filecolor=magenta,      
	urlcolor=cyan,
}
\usepackage[dvipsnames]{xcolor}
\usepackage{lastpage}
\usepackage{fancybox,fancyhdr}
\usepackage{hyperref}
\usepackage[font={small,it}]{caption}

\usepackage{minted, xcolor}
\usemintedstyle{monokai} 
\definecolor{bg}{HTML}{282828}
\setminted{bgcolor=bg}
\setminted{fontsize=\scriptsize}
\setminted{}

\fancyhead[R]{\textit{Отчет по лабораторной}}
\fancyfoot[L]{}
\fancyfoot[C]{Страница \thepage\space из \pageref{LastPage}}
\fancyfoot[R]{}
\pagestyle{fancy}
\setlength{\headheight}{17.0pt}

\usepackage{tikz,graphics,color,fullpage,float,epsf,caption,subcaption}
\usepackage{tocloft}

\cftsetindents{section}{0em}{2em}
\cftsetindents{subsection}{0em}{2em}

\renewcommand\cfttoctitlefont{\hfill\Large\bfseries}
\renewcommand\cftaftertoctitle{\hfill\mbox{}}

\begin{document}
	\begin{titlepage}
		
		\setlength{\parindent}{0ex}
		
		\begin{center}
			\textsc{
				\vspace{1ex}
				Научно исследовательский университет ИТМО \\
				\vspace{0.5ex}
				Факультет систем управления и робототехники \\
				\vspace{0.5ex}
			}
		\end{center}
		
		\vspace{50mm}
		
		\begin{center}
			\textbf{Отчет по лабораторной работе №2 \\
			Порты ввода-вывода и условия}
		\end{center}
		
		\vspace{50mm}
		
		\begin{minipage}{.45\linewidth}
			Выполнили:
			
			\smallskip
			
			Преподаватель:
		\end{minipage}
		\hfill
		\begin{minipage}{.5\linewidth}
			\begin{flushright}
				Бухарев С. А. и
				Мовчан И. Е.
				\\
				Громов В. С.
			\end{flushright}
		\end{minipage}
		
		\vfill
		\begin{center}
			Санкт-Петербург
			\\
			2024
		\end{center}

	\end{titlepage}

	\section{Цель работы}

	Изучить работу портов ввода-вывода и условий применительно к программированию промышленных роботов. Написать программу с бесконечным циклом проверки наличий деталей в ряду из трёх единиц, используя манипулятор с установленным на захватном устройстве датчиком обнаружения.
	\newpage

	\section{Описание команд и конечный код}
	Используя следующие команды:
	\begin{itemize}
		\item \textbf{SERVO ON (OFF)} - включение (выключение) двигателей;
		\item \textbf{JOVRD 100} - задание скорости движения в процентах;
		\item \textbf{WAIT M\_IN(3)=1} - ожидание сигнала 1 на 3 порте;
		\item \textbf{DEF ACT 1, M\_IN(1)=0 GOSUB *SUBSTOP} - объявление прерывания №1 при условии наличия 0 на 1 порте с вызовом процедуры \textbf{*SUBSTOP};
		\item \textbf{ACT 1 = 1} - активация прерывания №1;
		\item \textbf{PHELPX=(75, 2, 0, 0, 0, 0)} - объявление вспомогательного массива для смещений;
		\item \textbf{WHILE 1 ... WEND} - бесконечный цикл;
		\item \textbf{FOR IX=0 TO 2 ... NEXT IX} - цикл for, менящий значение переменной IX в цикле от 0 до 2;
		\item \textbf{IF M\_IN(900)=1 THEN .. ENDIF} - проверка сигнала 1 на 900 порте, если есть - заходим в тело цикла;
		\item \textbf{MOV} \textit{точка} - перемещение в указанную в виде массива \textit{точку};
		\item \textbf{MOV} \textit{точка}, \textbf{-100} - перемещение в \textit{точку} с поднятием по OZ на 100 мм;
		\item \textbf{DLY 0.1} - ожидать 0.1 секунду (используется после перемещения для ожидания прекращения инерционного движения);
		\item \textbf{*SUBSTOP} - объявление процедуры;
		\item \textbf{END} - завершение программы.
	\end{itemize}

	Напишем программу, реализующую бесконечную проверку присутствия элементов в ряде из 3 позиций, начинающую свою работу с нажатия кнопки \textit{Reset} (сигнала 1 на 3 порте, изначально - 0) на панели управления и заканчивающейся при нажатии \textit{Stop} (сигнале 0 на 1 порте, изначально - 1), ведущей к вызову процедуры \textit{\textbf{substop}} (в бесконечном цикле проходимся, используя \textit{\textbf{for}}, по ячейкам; в случае присутствия i-ой детали (сигнала 1 на 900 порте, соединённом с датчиком; изначально - 0) - поднимаемся c помощью \textit{\textbf{mov} точка, \textbf{-100}} на 100 мм, а после с помощью \textit{\textbf{mov} точка} опускаемся вниз и двигаемся дальше):
	\begin{lstlisting}
SERVO ON
JOVRD 100

WAIT M_IN(3)=1
DEF ACT 1, M_IN(1)=0 GOSUB *SUBSTOP
ACT 1 = 1

PHELPX=(75, 2, 0, 0, 0, 0)
WHILE 1
	FOR IX=0 TO 2
		MOV P1-PHELPX*IX
		DLY 0.1
		IF M_IN(900)=1 THEN
			MOV P1-PHELPX*IX, -100
			MOV P1-PHELPX*IX
		ENDIF
	NEXT IX
WEND

END

*SUBSTOP
	SERVO OFF
	END
	\end{lstlisting}
	
	Таблица сохранённых точек:
	\begin{center}
		\begin{tabular}{|c|c|c|}
			\hline
			\textbf{No} & \textbf{Position} & \textbf{Orientation} \\
			\hline
			$P_1$ & $219.5, -327.3, 101.6$ & $179, -2, 59, \text{R}, \text{A}, \text{N}$ \\
			\hline
		\end{tabular}
	\end{center}
	\newpage

	\section{Фотографии}
	Необходимые элементы:
	\begin{figure}[h]
		\centering
		\begin{subcaptiongroup}
		  \centering
		  \parbox[b]{.4\textwidth}{
		  \centering
		  \includegraphics[scale=0.18]{./images/dat.jpg}
		  \caption{Датчик обнаружения}}
		  \parbox[b]{.4\textwidth}{
		  \centering
		  \includegraphics[scale=0.18]{./images/knop.jpg}
		  \caption{Панель управления}}
		\end{subcaptiongroup}
	\end{figure}

	Этапы выполнения 1-2:
	\begin{figure}[h]
		\centering
		\begin{subcaptiongroup}
		  \centering
		  \parbox[b]{.4\textwidth}{
		  \centering
		  \includegraphics[scale=0.5]{./images/step1.png}
		  \caption{Нажатие \textit{Reset}, начало работы}}
		  \parbox[b]{.4\textwidth}{
		  \centering
		  \includegraphics[scale=0.5]{./images/step2.png}
		  \caption{Проверка наличия}}
		\end{subcaptiongroup}
	\end{figure}
	\newpage
	Этапы выполнения 3-4:
	\begin{figure}[h]
		\centering
		\begin{subcaptiongroup}
		  \centering
		  \parbox[b]{.4\textwidth}{
		  \centering
		  \includegraphics[scale=0.5]{./images/step3.png}
		  \caption{Реакция на наличие}}
		  \parbox[b]{.4\textwidth}{
		  \centering
		  \includegraphics[scale=0.5]{./images/step4.png}
		  \caption{Нажатие \textit{Stop}, конец работы}}
		\end{subcaptiongroup}
	\end{figure}
	\section{Выводы}
	В результате выполнения лабораторной была изучена работа с портами ввода-вывода в программировании промышленных роботов, реализована через бесконечный цикл программа для проверки наличия деталей в ряду из трёх единиц с использованием датчика обнаружения.
\end{document} 
