\documentclass[a4paper,hidelinks,14pt]{extarticle}

\usepackage[utf8]{inputenc}
\usepackage[T2A]{fontenc}
\usepackage[english, russian]{babel}
\usepackage{lipsum}
\usepackage{amsmath}
\usepackage{amssymb}
\usepackage{amsfonts}
\usepackage{mathtools}
\usepackage{datetime}
\usepackage[pdftex]{graphicx}
\usepackage{indentfirst}
\usepackage{asymptote}
\usepackage{systeme}
\usepackage[dvipsnames]{xcolor}
\usepackage{lastpage}
\usepackage{fancybox,fancyhdr}
\usepackage{hyperref}
\usepackage[font={small,it}]{caption}
\fancyhead[L]{Лабораторная работа №1}
\fancyhead[C]{}
\fancyhead[R]{\textit{Моделирование и устойчивость}}
\fancyfoot[L]{}
\fancyfoot[C]{\thepage\space}
\fancyfoot[R]{}
\pagestyle{fancy}
\newcommand{\gt}{\textgreater}
\newcommand{\lt}{\textless}
\usepackage{listings}
\usepackage{xcolor}
\lstset{
    basicstyle=\ttfamily\small,
    keywordstyle=\color{blue},
    commentstyle=\color{gray},
    stringstyle=\color{red},
    numbers=left,
    numberstyle=\color[gray]{0.7}\ttfamily\small,
    stepnumber=1,
    numbersep=8pt,
    frame=single,
    showstringspaces=false,
    tabsize=4,
    breaklines=true
}
\usepackage{subcaption}

\begin{document}
	\begin{titlepage}
		\setlength{\parindent}{0ex}
		
		\begin{center}
			\textsc{
				\vspace{1ex}
                Научно-исследовательский университет ИТМО \\
				\vspace{0.5ex}
				Факультет систем управления и робототехники \\
				\vspace{0.5ex}
			}
		\end{center}
		
		\vspace{45mm}
		
		\begin{center}
			Отчет по лабораторной работе №1\\
			Моделирование и устойчивость \\
			Вариант 11
		\end{center}
		
		\vspace{50mm}
		
		\begin{minipage}{.4\linewidth}
			Выполнили студенты
            \\
            \\
			Преподаватель
		\end{minipage}
		\hfill
		\begin{minipage}{.5\linewidth}
			\begin{flushright}
				Мовчан Игорь Евгеньевич
                \\                
                Боглачев Артём Сергеевич
                \\
				Краснов Александр Юрьевич
			\end{flushright}
		\end{minipage}
		
		\vfill
		\begin{center}
			Санкт-Петербург
			\\
			2025
		\end{center}
		
	\end{titlepage}

	\tableofcontents
	\clearpage
	
	\section{Влияние дискретного элемента}
    Реализуем схему с дискретным элементом с периодом дискретизации $T = 0.2 \text{ с}$ и коэффициентом передачи ОУ $K_{CO} = 7.2$:
	\begin{figure}[h]
		\centering
		\includegraphics[width=\textwidth]{images/first.png}
		\caption{Система с дискретным элементом}
	\end{figure}
	
    Выведем разностное уравнение этой схемы. На выходе экстрополятора нулевого порядка имеем $u(t) = u[k]$ при $t \in [kT, (k+1)T] $. Тогда интегратору на вход подается постоянная величина, а значит
	\[
		y((k + 1)T) = y(kT) + K_{CO} T u[k]
	\]

	Откуда при компактном обозначении $y[k] = y(kT)$:
	\[
		y[k + 1] = y[k] + K_{CO} T u[k]
	\]

	Далее, учитывая, что $u[k] = K_{FB}(r[k] - y[k])$:
	\[
		y[k + 1] = (1 - K_{FB}K_{CO}T)y[k] + K_{FB}K_{CO}T r[k]
	\]

	В данном случае $r[k]$ - дискретная версия входного сигнала $r(t)$ в моменты времени $t = kT$. Для рассматриваемой схемы
	\[
		r(t) = g(t) = \begin{cases}
			1, & t < 0 \\
			0, & t \geq 0
		\end{cases}
	\]

	Для дальнейшего анализа полезным будет ввести переменную, характеризующую движение системы без входного сигнала $r(t)$:
	\[
		z = 1 - K_{FB}K_{CO}T \Rightarrow y[k + 1] = z y[k] + K_{FB}K_{CO}T r[k]
	\]

	Нейтральная граница $z = 1$, колебательная $z = - 1$.
	
	Затухающие колебания при $-1 \le z < 0$, максимальные при $z = -1$, при $z < -1$ - расходящиеся колебания. При $z \ge 0$ - отсутствие колебаний. При $0 \le z \le 1$ - затухание. При $z > 1$ - расходящиеся экспоненты.

	Оптимальное по быстродействию значение $z = 0$.

	\section{Исследование устойчивости}
	Рассмотрим непрерывный ОУ, заданный уравнением $\ddot{y} = u$, где $u(t)$ - управляющее воздействие, $y(t)$ - выходная величина.

	Зададимся $x_1(t) = y(t)$, $x_2(t) = \dot{y}(t)$, а значит, $\dot{x_1}(t) = x_2(t)$, а $\dot{x_2}(t) = u(t)$. Тогда модель в форме вход-состояние-выход будет:
	\[
		\dot{x} =
		\begin{bmatrix}
			\dot{x_1} \\
			\dot{x_2}
		\end{bmatrix} = \begin{bmatrix}
			0 & 1 \\
			0 & 0
		\end{bmatrix} \begin{bmatrix}
			x_1 \\
			x_2
		\end{bmatrix} + \begin{bmatrix}
			0 \\
			1
		\end{bmatrix} u = A_\text{н} x + B_\text{н} u
	\]

	Дискретизируем ее с использованием выражений:
	\[
		A = e^{A_\text{н}T} = \sum_{i=1}^{\infty} \frac{A^i_\text{н} T^i}{i!}, \quad B = \sum_{i=1}^{\infty} \frac{A_\text{н}^{i-1} T^i}{i!} B_\text{н}
	\]

	Заметим, что $A^2 = 0$, а значит, $A^n = 0$ для любого $n \ge 2$. Тогда
	\[
		A = I + A_\text{н}T = \begin{bmatrix}
			1 & T \\
			0 & 1
		\end{bmatrix}, \quad B = T B_\text{н} + \frac{A_\text{н} T^2 B_\text{н}}{2} = \begin{bmatrix}
			T^2/2 \\
			T
		\end{bmatrix}
	\]

	В итоге, получаем дискретную модель:
	\[
		x(k + 1) = A x(k) + B u(k)
	\]

	Теперь зададим управляющее воздействие в виде 
	\[
	u(k) = -K x(k) = - \begin{bmatrix}
		k_1 & k_2
	\end{bmatrix} \begin{bmatrix}
		x_1(k) \\
		x_2(k)
	\end{bmatrix}
	\]

	А также матрицу динамики замкнутой системы:
	\[
		F = A - BK = \begin{bmatrix}
			1 & T \\
			0 & 1
		\end{bmatrix} - \begin{bmatrix}
			T^2/2 \\
			T
		\end{bmatrix} \begin{bmatrix}
			k_1 & k_2
		\end{bmatrix}  = \begin{bmatrix}
			1 - k_1 T^2/2 & T - k_2 T^2/2 \\
			-k_1 T & 1 - k_2 T
		\end{bmatrix}
	\]

	Характеристический полином для замкнутой системы тогда:
	\[
		P(z) = \det(zI - F) = \begin{vmatrix}
			z - 1 + k_1 T^2/2 & -T + k_2 T^2/2 \\
			k_1 T & z - 1 + k_2 T
		\end{vmatrix} =
	\]
	\[
		= \left(z - 1 + \frac{k_1 T^2}{2}\right)\left(z - 1 + k_2 T\right) + k_1 T \left(T - \frac{k_2 T^2}{2}\right) =
	\]
	\[
		= (z-1)^2 + k_2 T(z-1) + \frac{k_1 T^2}{2}(z-1) + \frac{k_1 k_2 T^3}{2} + k_1 T^2 - \frac{k_1 k_2 T^3}{2} =
	\]
	\[
		= z^2 + \left(-2 + k_2 T + \frac{k_1 T^2}{2}\right)z + \left(1 - k_2 T + \frac{k_1 T^2}{2}\right)
	\]

	C помощью матрицы $M$ размерности $2 \times 2$ такой, что $x = M \xi$ и существует $M^{-1}$, перейдем к базису $\xi$ системы, где матрица $F$ была бы диагональной и имела собственными числами $z_1$ и $z_2$:
	\[
		F_d = M^{-1} F M = \begin{bmatrix}
			z_1 & 0 \\
			0 & z_2
		\end{bmatrix}, \quad \xi (m + 1) = F_d \xi (m), \quad \xi_i (m) = z_i^m \xi(0)
	\]

	Характеристический полином замкнутой системы в этом случае:
	\[
		P(z) = \det(zI - F_d) = (z - z_1)(z - z_2) = z^2 - (z_1 + z_2)z + z_1z_2
	\]

	Зная, что при переходе в новый базис характеристический полином остается тем же:
	\[
		\det(z I - F_d) = \det(M^{-1}(z I - F)M) = \det(z I - F)
	\]
	
	Получаем систему уравнений на коэффициенты $k_1$ и $k_2$:
	\[
		\begin{cases}
			z_1 + z_2 = 2 - k_2 T - \dfrac{k_1 T^2}{2}  \\[1.5ex]
			z_1 z_2 = 1 - k_2 T + \dfrac{k_1 T^2}{2}
		\end{cases}
	\]

	Решим ее аналитически с введенными $s = z_1 + z_2$ и $p = z_1 z_2$:
	\[
			k_1 = \dfrac{1 - s + p}{T^2}, \quad
			k_2 = \dfrac{3 - s - p}{2T}
	\]

	Отлично! Теперь можно синтезировать матрицу $K$ по желаемым корням $z_1$ и $z_2$ замкнутой системы, а также взятому периоду дискретизации $T = 0.2 \text{ с}$. Рассмотрим пять случаев:
	\begin{enumerate}
		\item $z_1 = 0.1$ и $z_2 = 0.7$: обратная связь $K_1 = \begin{bmatrix} 6.75 & 5.325 \end{bmatrix}^T$
		\item $z_1 = -1.2$ и $z_2 = -0.4$: обратная связь $K_2 = \begin{bmatrix} 77 & 10.3 \end{bmatrix}^T$
		\item $z_1 = 0.1$ и $z_2 = 0.5$: обратная связь $K_3 = \begin{bmatrix} 11.25 & 5.875 \end{bmatrix}^T$
		\item $z_{12} = \pm 1.2 j$: обратная связь $K_4 = \begin{bmatrix} 61 & 3.9 \end{bmatrix}^T$
		\item $z_{12} = -0.8 \pm 0.7j$: обратная связь $K_5 = \begin{bmatrix} 93.25 & 8.675 \end{bmatrix}^T$
	\end{enumerate}

	Наконец, осуществим моделирование полученных систем с начальными условиями $y(0) = 1$ и $\dot{y}(0) = 0$. Переходный процесс при входе $r(t) = g(t)$.

	\section{Построение командных генераторов}
	Синтезируем командный генератор гармонического сигнала вида $g(k) = A \sin(k T \omega)$ с параметрами из варианта:
	\[
		T = 0.2 \text{ с}, \quad A = -1.3, \quad \omega = 0.87 \text{ рад/с}
	\]

	Найдем разностное уравнение для этого сигнала:
	\[
		g(k + 1) = A \sin((k + 1) T \omega) = A(\sin(k T \omega) \cos(T \omega) + \cos(k T \omega) \sin(T \omega))
	\]

	Введем переменные состояния:
	\[
		\xi_1(k) = g(k) = A \sin(k T \omega), \quad \xi_2(k) = A \cos(k T \omega)
	\]
	
	Тогда система уравнений для состояний:
	\[
		\begin{cases}
			\xi_1(k + 1) = \phantom{-}\xi_1(k) \cos(T \omega) + \xi_2(k) \sin(T \omega) \\
			\xi_2(k + 1) = -\xi_1(k) \sin(T \omega) + \xi_2(k) \cos(T \omega)
		\end{cases}
	\]

	Запишем эту систему в матричном виде:
	\[
		\xi(k + 1) =
		\begin{bmatrix}
			\xi_1(k + 1) \\
			\xi_2(k + 1)
		\end{bmatrix} = \begin{bmatrix}
			\cos(T \omega) & \sin(T \omega) \\
			-\sin(T \omega) & \cos(T \omega)
		\end{bmatrix} \begin{bmatrix}
			\xi_1(k) \\
			\xi_2(k)
		\end{bmatrix} = \Gamma_d \xi(k)
	\]

	Тогда итоговая модель задается уравнением
	\[
		g(k) = H \xi(k) = H \Gamma_d^k \xi(0)
	\]

	При начальных условиях и матрице $H$:
	\[
		\xi(0) = \begin{bmatrix}
			\xi_1(0) \\
			\xi_2(0)
		\end{bmatrix} = \begin{bmatrix}
			A \sin(0) \\
			A \cos(0)
		\end{bmatrix} = \begin{bmatrix}
			0 \\
			A
		\end{bmatrix}, \quad H = \begin{bmatrix}
			1 & 0
		\end{bmatrix}
	\]

	Далее синтезируем дискретную модель возмущения $g(k) = A + B k T + C (k T)^2$ с параметрами из варианта:
	\[
		A = 5, \quad B = 5.5, \quad C = 1.5
	\]

	Также выведем разностное уравнение для этого сигнала. Зададимся первой переменной состояния:
	\[
		\xi_1(k) = g(k)
	\]

	Возьмем вторую переменную состояния:
	\[
		\xi_2(k) = \xi_1(k + 1) = A + B k T + C (k T)^2 + B T + 2 C k T^2 + C T^2
	\]

	Откуда:
	\[
		\xi_2(k) = \xi_1(k) + B T + 2 C k T^2 + C T^2
	\]
	
	Возьмем третью переменную состояния:
	\[
		\xi_3(k) = \xi_2(k + 1) = \xi_1(k + 1) + B T + 2 C k T^2 + C T^2 + 2 C T^2
	\]

	Откуда:
	\[
		\xi_3(k) = \xi_1(k + 1) + \xi_2(k) - \xi_1(k) + 2 C T^2 = 2 \xi_2(k) - \xi_1(k) + 2 C T^2
	\]

	Далее, имеем:
	\[
		\xi_3(k + 1) = 2 \xi_2 (k + 1) - \xi_1(k + 1) + 2 C T^2
	\]

	Из выражения для $\xi_3(k)$ можно выразить $CT^2$:
	\[
		CT^2 = \xi_3(k) - 2 \xi_2(k) + \xi_1(k)
	\]

	И тогда:
	\[
		\xi_3(k + 1) = 2 \xi_2(k + 1) - \xi_1(k + 1) + \xi_3(k) - 2 \xi_2(k) + \xi_1(k) = 
	\]
	\[
		= 3 \xi_3(k) - 3 \xi_2(k) + \xi_1(k)
	\]

	В матричной форме система записывается как
	\[
		\begin{bmatrix}
			\xi_1(k + 1) \\
			\xi_2(k + 1) \\
			\xi_3(k + 1)
		\end{bmatrix} = \begin{bmatrix}
			0 & 1 & 0 \\
			0 & 0 & 1 \\
			1 & -3 & 3
		\end{bmatrix} \begin{bmatrix}
			\xi_1(k) \\
			\xi_2(k) \\
			\xi_3(k)
		\end{bmatrix} = \Gamma_d \begin{bmatrix}
			\xi_1(k) \\
			\xi_2(k) \\
			\xi_3(k)
		\end{bmatrix}
	\]

	Имеем также
	\[
		g(k) = H \xi(k) = H \Gamma_d^k \xi(0), \quad H = \begin{bmatrix}
			1 & 0 & 0
		\end{bmatrix}
	\]

	Начальные условия задаются через $k = 0$ и $g(0) = A$:
	\[
		\xi(0) = \begin{bmatrix}
			\xi_1(0) \\
			\xi_2(0) \\
			\xi_3(0)
		\end{bmatrix} = \begin{bmatrix}
			A \\
			A + BT + CT^2 \\
			A + 2BT + 3CT^2
		\end{bmatrix}
	\]


	

	
	


	
    
    


	
\end{document}