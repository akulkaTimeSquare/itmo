\documentclass[a4paper, 12pt]{article}
\usepackage[utf8x]{inputenc}
\usepackage[english, russian]{babel}
\usepackage[left=15mm, top=25mm, right=15mm, bottom=25mm]{geometry}
\usepackage{cmap}
\usepackage{indentfirst}
\usepackage{tikz}
\usepackage{float}
\usepackage{amsmath, amsfonts, amssymb}
\usepackage{graphicx}
\usepackage{hyperref}
\usepackage{listings}
\usepackage{caption}
\usepackage{subcaption}
\usepackage{xcolor}
\usepackage{etoolbox}
\usepackage{titlesec}
\pagestyle{plain}

% Выравнивание формул по центру
\patchcmd{\tableofcontents}{\contentsname}{\centering\contentsname}{}{}
\titleformat{\section}[block]{\normalfont\large\bfseries\centering}{}{0pt}{}
\titleformat{\subsection}[block]{\normalfont\normalsize\bfseries\centering}{}{0pt}{}
\allowdisplaybreaks
\graphicspath{{images/}}
\usetikzlibrary{patterns}
\definecolor{LightGray}{gray}{0.95}
\definecolor{LightGray2}{gray}{0.7}
\lstdefinestyle{code}{
    language=MATLAB,
    basicstyle=\footnotesize\ttfamily,
    backgroundcolor=\color{LightGray},
    showspaces=false,
    showstringspaces=false,
    showtabs=false,
    tabsize=4,
    captionpos=b,
    breaklines=true,
    breakatwhitespace=false,
    frame=single,
    rulecolor=\color{LightGray2},
    linewidth=\linewidth,
    keywordstyle=\color{blue}\bfseries,
    commentstyle=\color{green!40!black},
    stringstyle=\color{purple},
    escapeinside={\%*}{*},
    inputencoding=utf8x,
    xleftmargin=0pt,
    framexleftmargin=0pt,
    framexrightmargin=0pt
}
\lstset{style=code}
\hypersetup{
    colorlinks=true,
    linkcolor=blue,
    filecolor=magenta,
    urlcolor=cyan,
    pdftitle={Лабораторная работа №4},
    pdfpagemode=FullScreen,
}

% Для подписей рисунков
\captionsetup[figure]{
  justification=centering,
  labelsep=period,
  name={Рисунок},
  font={small,bf}
}


\begin{document}
    \begin{titlepage}
        \begin{center}
        Федеральное государственное автономное образовательное учреждение высшего образования \\
        «Национальный Исследовательский Университет ИТМО» \\
        \vfill
        {\large\bf ЛАБОРАТОРНАЯ РАБОТА №4} \\
        {\large\bf «Дискретные регуляторы с заданными характеристиками}\\
        {\large\bf переходных процессов»} \\
        Вариант №11 \\
        \vfill
        
        \begin{minipage}{.45\linewidth}
			Выполнили студенты
            \\
			\\[5mm]
			Преподаватель
		\end{minipage}
		\hfill
		\begin{minipage}{.52\linewidth}
			\begin{flushright}                
                Боглачев Артём Сергеевич \\
				Мовчан Игорь Евгеньевич
				\\[5mm]
				Краснов Александр Юрьевич
			\end{flushright}
		\end{minipage}
        \vfill
        Санкт-Петербург, 2025
        \end{center}
    \end{titlepage}
    
    \tableofcontents
    \newpage

\section{Введение}

Целью данной работы является изучение методов синтеза дискретных регуляторов, обеспечивающих заданные характеристики переходных процессов в цифровых системах управления. В работе рассматриваются три типа регуляторов:

\begin{enumerate}
    \item Апериодический регулятор
    \item Регулятор Далина
    \item Регулятор с заданным расположением полюсов
\end{enumerate}

Для варианта 11 определены следующие исходные данные:
\begin{itemize}
    \item Параметры объекта управления: $a = 1.9$, $b = 6.5$
    \item Требования к переходным процессам: $\zeta = 0.45$, $\omega_d = 4$ рад/с
    \item Точность: $K_v = 0.25$
    \item Период дискретизации: $T = 1$ с (для регуляторов 1-2), $T = 0.1$ с (для регулятора 3)
\end{itemize}

\section{Синтез апериодического регулятора}

\subsection{Исходные данные}

\begin{itemize}
    \item Передаточная функция объекта: $G(s) = \dfrac{e^{-as}}{1 + bs}$
    \item Параметры: $a = 1.9$, $b = 6.5$
    \item Период дискретизации: $T = 1$ с
\end{itemize}

\subsection{Приведение параметров}

Так как запаздывание $a = 1.9$ не кратно периоду дискретизации $T = 1$ с, округляем до ближайшего целого значения:
\[
a = 2, \quad b = 6.5, \quad T = 1
\]

\subsection{Дискретизация объекта с ЭНП}

Передаточная функция непрерывной части:
\[
G(s) = \frac{e^{-2s}}{1 + 6.5s}
\]

Дискретная передаточная функция с экстраполятором нулевого порядка (ЭНП):
\[
HG(z) = \mathcal{Z}\left\{ \frac{1 - e^{-sT}}{s} \cdot G(s) \right\} = (1 - z^{-1}) z^{-2} \mathcal{Z}\left\{ \frac{1}{s(1 + 6.5s)} \right\} =
\]

Вычислим Z-преобразование:
\begin{align*}
\mathcal{Z}\left\{ \frac{1}{s(1 + 6.5s)} \right\} &= \mathcal{Z}\left\{ \frac{1}{s} - \frac{1}{s + 1/6.5} \right\} \\
&= \frac{z}{z - 1} - \frac{z}{z - e^{-1/6.5}} \\
&= \frac{z}{z - 1} - \frac{z}{z - 0.857} \\
&= \frac{0.143z}{(z - 1)(z - 0.857)}
\end{align*}

где $e^{-1/6.5} \approx 0.857$.

Тогда:
\[
HG(z) = (1 - z^{-1}) z^{-2} \cdot \frac{0.143z}{(z - 1)(z - 0.857)} = \frac{0.143 z^{-2}}{1 - 0.857 z^{-1}}
\]

\subsection{Синтез регулятора}

Желаемая передаточная функция замкнутой системы для апериодического регулятора:
\[
T(z) = z^{-k}, \quad k \geq 2 \text{ (для физической реализуемости)}
\]

Выбираем $k = 2$.

Формула для расчёта передаточной функции регулятора:
\[
D(z) = \frac{1}{HG(z)} \cdot \frac{T(z)}{1 - T(z)}
\]

Подставляем:
\[
\frac{1}{HG(z)} = \frac{1 - 0.857 z^{-1}}{0.143 z^{-2}}, \quad \frac{T(z)}{1 - T(z)} = \frac{z^{-2}}{1 - z^{-2}}
\]

\begin{align*}
D(z) &= \frac{1 - 0.857 z^{-1}}{0.143 z^{-2}} \cdot \frac{z^{-2}}{1 - z^{-2}} \\
&= \frac{1 - 0.857 z^{-1}}{0.143 (1 - z^{-2})}
\end{align*}

Приводим к положительным степеням $z$:
\[
D(z) = \frac{z^2 - 0.857z}{0.143 (z^2 - 1)}
\]

\subsection{Итоговый результат}

\[
D_{\text{апер}}(z) = \frac{z^2 - 0.857z}{0.143 (z^2 - 1)}
\]

\subsection{Моделирование и результаты}

Моделирование выхода регулятора и системы приведено на рисунках \ref{1} и \ref{2}.

Апериодический регулятор обеспечивает минимальное время переходного процесса (в нашем случае 2 периода дискретизации). Однако он требует больших управляющих воздействий и чувствителен к точности модели объекта.

\section{Синтез регулятора Далина}

\subsection{Исходные данные}

\begin{itemize}
    \item Те же параметры объекта: $a = 2$, $b = 6.5$
    \item Период дискретизации: $T = 1$ с
    \item Параметр $q = b = 6.5$ (постоянная времени объекта)
\end{itemize}

\subsection{Желаемая передаточная функция}

Формула желаемой передаточной функции для регулятора Далина:
\[
T(z) = \frac{z^{-k-1} (1 - e^{-T/q})}{1 - e^{-T/q} z^{-1}}, \quad k = a/T = 2
\]

Вычисляем коэффициенты:
\begin{align*}
e^{-T/q} &= e^{-1/6.5} \approx 0.857 \\
1 - e^{-T/q} &= 0.143
\end{align*}

Тогда:
\[
T(z) = \frac{0.143 z^{-3}}{1 - 0.857 z^{-1}}
\]

\subsection{Синтез регулятора}

Используем ту же дискретную передаточную функцию объекта $HG(z)$:
\[
HG(z) = \frac{0.143 z^{-2}}{1 - 0.857 z^{-1}}
\]

Формула для расчёта передаточной функции регулятора:
\[
D(z) = \frac{1}{HG(z)} \cdot \frac{T(z)}{1 - T(z)}
\]

Вычисляем:
\begin{align*}
\frac{T(z)}{1 - T(z)} &= \frac{0.143 z^{-3}}{1 - 0.857 z^{-1} - 0.143 z^{-3}} \\
D(z) &= \frac{1 - 0.857 z^{-1}}{0.143 z^{-2}} \cdot \frac{0.143 z^{-3}}{1 - 0.857 z^{-1} - 0.143 z^{-3}} \\
&= \frac{z^{-1} - 0.857 z^{-2}}{1 - 0.857 z^{-1} - 0.143 z^{-3}}
\end{align*}

Приводим к положительным степеням $z$:
\[
D(z) = \frac{z^2 - 0.857 z}{z^3 - 0.857 z^2 - 0.143}
\]

\subsection{Итоговый результат}

\[
D_{\text{Далин}}(z) = \frac{z^2 - 0.857 z}{z^3 - 0.857 z^2 - 0.143}
\]

\subsection{Моделирование и результаты}

Моделирование выхода регулятора и системы приведено на рисунках \ref{3} и \ref{4}.

Регулятор Далина обеспечивает более плавный переходный процесс с экспоненциальным характером. Управляющие воздействия имеют приемлемую величину по сравнению с апериодическим регулятором.

\section{Синтез регулятора с заданным расположением полюсов}

\subsection{Исходные данные}

\begin{itemize}
    \item Дискретная передаточная функция объекта с ЭНП:
    \[
    HG(z) = \frac{0.03(z + 0.75)}{z^2 - 1.5z + 0.5}
    \]
    \item Требования к переходным процессам: $\zeta = 0.45$, $\omega_d = 4$ рад/с
    \item Точность: $K_v = 0.25$
    \item Период дискретизации: $T = 0.1$ с
\end{itemize}

\subsection{Расчёт желаемых полюсов}

Собственная частота колебаний:
\[
\omega_n = \frac{\omega_d}{\sqrt{1 - \zeta^2}} = \frac{4}{\sqrt{1 - 0.2025}} \approx 4.48 \text{ рад/с}
\]

Коэффициент затухания:
\[
\sigma = \zeta \omega_n = 0.45 \cdot 4.48 \approx 2.016
\]

Полюса в $z$-плоскости:
\begin{align*}
z_{1,2} &= e^{-\sigma T} e^{\pm j \omega_d T} \\
&= e^{-0.2016} e^{\pm j 0.4} \\
&= 0.8175 \cdot (\cos(0.4) \pm j \sin(0.4)) \\
&= 0.8175 \cdot (0.9211 \pm j 0.3894) \\
&\approx 0.753 \pm j 0.318
\end{align*}

Знаменатель желаемой передаточной функции:
\[
D_{\text{ж}}(z) = (z - z_1)(z - z_2) = z^2 - 1.506z + 0.668
\]

\subsection{Определение коэффициентов передаточной функции}

Задаём форму желаемой передаточной функции:
\[
T(z) = \frac{b_1 z + b_2}{z^2 - 1.506z + 0.668}
\]

\textbf{Условие 1:} Нулевая установившаяся ошибка для ступенчатого воздействия:
\[
T(1) = 1 \Rightarrow \frac{b_1 + b_2}{1 - 1.506 + 0.668} = 1 \Rightarrow b_1 + b_2 = 0.162
\]

\textbf{Условие 2:} Установившаяся ошибка для линейного воздействия:
\[
e_{ss} = \frac{1}{K_v} = \frac{1}{0.25} = 4
\]

Используя теорему о конечном значении для линейного воздействия:
\[
e_{ss} = \lim_{z \to 1} \frac{T}{z-1} [1 - T(z)] = 4
\]
где $T = 0.1$ с.

Вычисляем производную $T(z)$ в точке $z=1$:
\[
T'(1) = \lim_{z \to 1} \frac{1 - T(z)}{z-1}
\]

Производная $T(z)$:
\[
T'(z) = \frac{b_1 D(z) - (b_1 z + b_2) D'(z)}{D^2(z)}, \quad D(z) = z^2 - 1.506z + 0.668
\]

Вычисляем:
\begin{align*}
D(1) &= 0.162 \\
D'(1) &= 2 \cdot 1 - 1.506 = 0.494 \\
T'(1) &= \frac{b_1 \cdot 0.162 - 0.162 \cdot 0.494}{0.162^2} = \frac{0.162b_1 - 0.080028}{0.026244} = -40
\end{align*}

Решая уравнение:
\[
0.162b_1 - 0.080028 = -1.04976 \Rightarrow b_1 \approx -5.987
\]

Тогда:
\[
b_2 = 0.162 - (-5.987) = 6.149
\]

Итоговая желаемая передаточная функция:
\[
T(z) = \frac{-5.987z + 6.149}{z^2 - 1.506z + 0.668}
\]

\subsection{Синтез регулятора}

Формула для расчёта передаточной функции регулятора:
\[
D(z) = \frac{1}{HG(z)} \cdot \frac{T(z)}{1 - T(z)}
\]

Вычисляем:
\[
1 - T(z) = \frac{z^2 + 4.481z - 5.481}{z^2 - 1.506z + 0.668}
\]

\[
\frac{T(z)}{1 - T(z)} = \frac{-5.987z + 6.149}{z^2 + 4.481z - 5.481}
\]

Подставляем $HG(z)$:
\[
D(z) = \frac{z^2 - 1.5z + 0.5}{0.03(z + 0.75)} \cdot \frac{-5.987z + 6.149}{z^2 + 4.481z - 5.481}
\]

Раскрываем скобки:
\[
D(z) = \frac{(z^2 - 1.5z + 0.5)(-5.987z + 6.149)}{0.03(z + 0.75)(z^2 + 4.481z - 5.481)}
\]

Числитель:
\[
-5.987z^3 + 15.072z^2 - 12.424z + 3.0745
\]

Знаменатель:
\[
0.03z^3 + 0.1569z^2 - 0.1088z - 0.1233
\]

\subsection{Итоговый результат}

\[
D_{\text{пол}}(z) \approx \frac{-199.57z^3 + 502.4z^2 - 414.13z + 102.48}{z^3 + 5.23z^2 - 3.627z - 4.11}
\]

\subsection{Моделирование}

Моделирование выхода регулятора и системы при ступенчатом воздействии приведено на рисунках \ref{5} и \ref{6}, а при линейно-нарастающем - на рисунках \ref{7} и \ref{8}.

\section{Выводы}

В ходе выполнения лабораторной работы были синтезированы три типа дискретных регуляторов для заданного объекта управления:

\begin{enumerate}
    \item \textbf{Апериодический регулятор} обеспечивает минимальное время переходного процесса (2 периода дискретизации), но требует больших управляющих воздействий и чувствителен к точности модели объекта.
    
    \item \textbf{Регулятор Далина} обеспечивает более плавный переходный процесс с экспоненциальным характером, управляющие воздействия имеют приемлемую величину.
    
    \item \textbf{Регулятор с заданными полюсами} позволяет точно задать динамические характеристики системы и дать требуемую точность слежения за линейно-нарастающим сигналом.
\end{enumerate}

Все синтезированные регуляторы обеспечивают нулевую установившуюся ошибку для ступенчатого входного воздействия. 

\section{Графики}
\begin{figure}[H]
    \centering
    \includegraphics[width=0.7\textwidth]{images/u.png}
    \caption{Управление апериодического регулятора}
    \label{1}
\end{figure}
\begin{figure}[H]
    \centering
    \includegraphics[width=0.7\textwidth]{images/y.png}
    \caption{Переходный процесс при ступенчатом воздействии (апериодический регулятор)}
    \label{2}
\end{figure}
\begin{figure}[H]
    \centering
    \includegraphics[width=0.7\textwidth]{images/u1.png}
    \caption{Управление регулятора Далина}
    \label{3}
\end{figure}
\begin{figure}[H]
    \centering
    \includegraphics[width=0.7\textwidth]{images/y1.png}
    \caption{Переходный процесс при ступенчатом воздействии (регулятор Далина)}
    \label{4}
\end{figure}
\begin{figure}[H]
    \centering
    \includegraphics[width=0.7\textwidth]{images/u2.png}
    \caption{Управление регулятора с заданным расположением полюсов}
    \label{5}
\end{figure}
\begin{figure}[H]
    \centering
    \includegraphics[width=0.7\textwidth]{images/y2.png}
    \caption{Переходный процесс при ступенчатом воздействии (регулятор с заданным расположением полюсов)}
    \label{6}
\end{figure}
\begin{figure}[H]
    \centering
    \includegraphics[width=0.7\textwidth]{images/u3.png}
    \caption{Управление регулятора с заданным расположением полюсов и линейно-нарастающем воздействии}
    \label{7}
\end{figure}
\begin{figure}[H]
    \centering
    \includegraphics[width=0.7\textwidth]{images/y3.png}
    \caption{Переходный процесс при линейно-нарастающем воздействии (регулятор с заданным расположением полюсов)}
    \label{8}
\end{figure}

\end{document}