\documentclass[a4paper,hidelinks,14pt]{extarticle}

\usepackage[utf8]{inputenc}
\usepackage[T2A]{fontenc}
\usepackage[english, russian]{babel}
\usepackage{lipsum}
\usepackage{amsmath}
\usepackage{amssymb}
\usepackage{amsfonts}
\usepackage{mathtools}
\usepackage{datetime}
\usepackage[pdftex]{graphicx}
\usepackage{indentfirst}
\usepackage{asymptote}
\usepackage{systeme}
\usepackage[dvipsnames]{xcolor}
\usepackage{lastpage}
\usepackage{fancybox,fancyhdr}
\usepackage{hyperref}
\usepackage[font={small,it}]{caption}
\fancyhead[L]{ЛР №3}
\fancyhead[C]{}
\fancyhead[R]{\textit{Система с дискретным ПИД-регулятором}}
\fancyfoot[L]{}
\fancyfoot[C]{\thepage\space}
\fancyfoot[R]{}
\pagestyle{fancy}
\newcommand{\gt}{\textgreater}
\newcommand{\lt}{\textless}
\usepackage{listings}
\usepackage{xcolor}
\lstset{
    basicstyle=\ttfamily\small,
    keywordstyle=\color{blue},
    commentstyle=\color{gray},
    stringstyle=\color{red},
    numbers=left,
    numberstyle=\color[gray]{0.7}\ttfamily\small,
    stepnumber=1,
    numbersep=8pt,
    frame=single,
    showstringspaces=false,
    tabsize=4,
    breaklines=true
}
\usepackage{subcaption}

\begin{document}
	\begin{titlepage}
		\setlength{\parindent}{0ex}
		
		\begin{center}
			\textsc{
				\vspace{1ex}
                Научно-исследовательский университет ИТМО \\
				\vspace{0.5ex}
				Факультет систем управления и робототехники \\
				\vspace{0.5ex}
			}
		\end{center}
		
		\vspace{40mm}
		
		\begin{center}
			Отчет по лабораторной работе №3\\
			\textbf{Исследование системы автоматического}\\
			\textbf{управления с дискретным ПИД-регулятором}\\
			Вариант 11
		\end{center}
		
		\vspace{40mm}
		
		\begin{minipage}{.45\linewidth}
			Выполнили студенты
            \\
			\\
			Преподаватель
		\end{minipage}
		\hfill
		\begin{minipage}{.52\linewidth}
			\begin{flushright}
				Мовчан Игорь Евгеньевич 
				\\
				Боглачев Артём Сергеевич
				\\
				Краснов Александр Юрьевич
			\end{flushright}
		\end{minipage}
		
		\vfill
		\begin{center}
			Санкт-Петербург
			\\
			2025
		\end{center}
		
	\end{titlepage}

	\tableofcontents
	\clearpage
	
	\section{Цель работы}
	Изучение алгоритма цифрового управления, полученного путем аппроксимации непрерывного ПИД-регулятора.

	\section{Составление модели}
	Для начала определим объектом управления электрическую печь с нагревательным элементом, представляющую собой апериодическое звено второго порядка и имеющую передаточную функцию
	\[
		W_o(s) = \frac{k_o}{(T_1 s + 1)(T_2 s + 1)}
	\]

	Тогда приведённая непрерывная часть описывается следующей передаточной функцией
	\[
		W_r(z) = \frac{z - 1}{z} \, \mathcal{Z}\left\{ \frac{W_o(s)}{s} \right\} = \frac{(z - 1)(r_0 z + r_1)}{z (z - d_1)(z - d_2)}
	\]
	
	Здесь \(0 < d_i = e^{-T / T_i} < 1 \) — полюса, лежащие внутри единичного круга, что соответствует устойчивому исходному объекту.

	Для обеспечения астатизма первого порядка и компенсации динамики объекта синтезирован регулятор с передаточной функцией
	\[
		W_c(z) = \frac{U(z)}{E(z)} = \frac{q_0 (z - d_1)(z - d_2)}{z (z - 1)}
			= q_0 \, \frac{z^2 - (d_1 + d_2) z + d_1 d_2}{z (z - 1)}
	\]
	
	Цель регулятора — компенсировать полюса объекта и обеспечить нулевую статическую ошибку слежения \( e[k] = g[k] - y[k] \to 0 \) при постоянных задающем воздействии \( g[k] \) и возможном возмущении.

	Поскольку управляющий сигнал формируется дискретно, а объект непрерывный, в состав системы введён формирующий элемент - экстраполятор нулевого порядка. Его передаточная функция:
	\[
		W_{fe}(s) = \frac{1 - e^{-T s}}{s}.
	\]

	В качестве параметров объекта управления выбраны значения:
	\[
		T_1 = 0.45, \quad T_2 = 1.35
	\]

	А период дискретизации принят равным
	\[
		T = \frac{T_1}{2} = 0.225
	\]

	Финальная модель САУ температуры показана на рисунке~\ref{fig:model}.
	\begin{figure}[h]
		\centering
		\includegraphics[width=\textwidth]{images/model.png}
		\caption{Модель дискретной системы}
		\label{fig:model}
	\end{figure}

	\section{Поиск параметров и моделирование}
	Итак, рассчитаем все передаточные функции. Для этого сперва найдем полюса непрерывной приведенной части:
	\[
		d_1 = e^{-T / T_1} \approx 0.6065, \quad d_2 = e^{-T / T_2} \approx 0.8465
	\]

	Откуда можно найти используемые передаточные функции регулятора и объекта управления с $k_0 = 1$
	\[
		W_c(z) = q_0 \frac{z^2 - 1.453 z + 0.5134}{z(z-1)}, \quad W_r(z) = \frac{1}{(0.45 s + 1)(1.35 s + 1)}
	\]

	На качество переходных процессов теперь влияет только значение коэффициента передачи регулятора $q_0$. Методом проб и ошибок было получено $q_0 = 18$, при котором система устойчива и имеет слабоколебательные переходные процессы на выходах объектов.

	Что ж, промоделируем получаемые выходные процессы дискретного регулятора и системы при \textit{ступенчатом изменении задающего воздействия $g[k]$}, а также \textit{ступенчатом и случайно меняющемся возмущающем воздействии}. Случайный сигнал будем подавать в виде Гауссовского шума с нулевым математическим ожиданием и единичной дисперсией.

	Все соответствующие графики приведены на рисунках~\ref{1}–\ref{6}. На них видно, что построенный регулятор успешно справляется с отслеживанием постоянных задающих сигналов как при отсутствии внешних возмущений, так и при их наличии. Однако при воздействии случайного возмущения регулятор не обеспечивает полного подавления, при этом сохраняется этом корректное среднее.

	Это поведение объясняется структурой регулятора: он обеспечивает астатизм первого порядка, приводя разомкнутую систему к виду интегратора. Такая структура гарантирует нулевую статическую ошибку лишь для постоянных заданий и возмущений, но не компенсирует высокочастотные (случайные) воздействия.
	\begin{figure}[h]
		\centering
		\includegraphics[width=0.8\textwidth]{images/y_1.png}
		\caption{$y(t)$ при \textbf{ступенчатом} изменении \textbf{задающего} воздействия}
		\label{1}
	\end{figure}
	\begin{figure}
		\centering
		\includegraphics[width=0.8\textwidth]{images/u_1.png}
		\caption{$u[k]$ при \textbf{ступенчатом} изменении \textbf{задающего} воздействия}
		\label{2}
	\end{figure}
	\begin{figure}
		\centering
		\includegraphics[width=0.8\textwidth]{images/y1_1.png}
		\caption{$y(t)$ при \textbf{ступенчатом} изменении \textbf{внешнего} воздействия}
		\label{3}
	\end{figure}
	\begin{figure}
		\centering	
		\includegraphics[width=0.8\textwidth]{images/u1_1.png}
		\caption{$u[k]$ при \textbf{ступенчатом} изменении \textbf{внешнего} воздействия}
		\label{4}
	\end{figure}
	\begin{figure}
		\centering
		\includegraphics[width=0.8\textwidth]{images/y2_1.png}
		\caption{$y(t)$ при \textbf{случайном} изменении \textbf{внешнего} воздействия}
		\label{5}
	\end{figure}
	\begin{figure}
		\centering
		\includegraphics[width=0.8\textwidth]{images/u2_1.png}
		\caption{$u[k]$ при \textbf{случайном} изменении \textbf{внешнего} воздействия}
		\label{6}
	\end{figure}

	\section{Влияние периода дискретизации}
	Попробуем теперь уменьшить значение периода дискретизации в модели фиксатора нулевого порядка до
	\[
		T = \frac{T_1}{4}
	\]

	Соответственно, изменятся и параметры системы, так что рассчитаем некоторые из них заново. Для начала найдем полюса $d_i$:
	\[
		d_1 = 0.7788, \quad d_2 = 0.92
	\]

	А после передаточную функцию регулятора:
	\[
		W_c(z) = q_0 \frac{(z-d_1)(z-d_2)}{z(z-1)} = q_0 \frac{z^2 - 1.6988 z + 0.7165}{z(z-1)}
	\]

	Применим полученное к системе, параметр $q_0 = 18$ при этом оставим тем же - все необходимые графики приведены на рисунках \ref{7}-\ref{12}.
	\begin{figure}
		\centering
		\includegraphics[width=0.8\textwidth]{images/y_2.png}
		\caption{$y$ при \textbf{ступенчатом} изменении \textbf{задающего} воздействия и малом $T$}
		\label{7}
	\end{figure}
	\begin{figure}
		\centering
		\includegraphics[width=0.8\textwidth]{images/u_2.png}
		\caption{$u$ при \textbf{ступенчатом} изменении \textbf{задающего} воздействия и малом $T$}
		\label{8}
	\end{figure}
	\begin{figure}
		\centering
		\includegraphics[width=0.8\textwidth]{images/y1_2.png}
		\caption{$y$ при \textbf{ступенчатом} изменении \textbf{внешнего} воздействия и малом $T$}
		\label{9}
	\end{figure}
	\begin{figure}
		\centering	
		\includegraphics[width=0.8\textwidth]{images/u1_2.png}
		\caption{$u$ при \textbf{ступенчатом} изменении \textbf{внешнего} воздействия и малом $T$}
		\label{10}
	\end{figure}
	\begin{figure}
		\centering
		\includegraphics[width=0.8\textwidth]{images/y2_2.png}
		\caption{$y$ при \textbf{случайном} изменении \textbf{внешнего} воздействия и малом $T$}
		\label{11}
	\end{figure}
	\begin{figure}
		\centering
		\includegraphics[width=0.8\textwidth]{images/u2_2.png}
		\caption{$u$ при \textbf{случайном} изменении \textbf{внешнего} воздействия и малом $T$}
		\label{12}
	\end{figure}

	Можем наблюдать заметные улучшения по качеству отслеживания задающих сигналов - уходят колебания, система сходится стабильнее! Однако подавить случайное внешнее воздействие уменьшение $T$, конечно, не способно.

	Всё вышесказанное связано с тем, что малость периода дискретизации $T$ модели напрямую влияет на то, как точно дискретная модель аппроксимирует непрерывную и как часто регулятор видит ошибку и корректирует под неё управление. Соответственно, уменьшение $T$ повышает качество всех процессов, и они становятся менее <<резкими>>.

	\section{Неточная компенсация полюсов}
	А что если, например, параметр $T_2$ был неправильно оценен? Проверим, как на это будет реагировать построенный регулятор при
	\[
		T = \frac{T_1}{2}, \quad q_0 = 18
	\]

	А также неизменных передаточных функциях регулятора и системы из 3 пункта работы и измененом на $20\%$ параметре $T_2$:
	\[
		T'_{2} = 0.8\cdot T_2 = 1.08, \quad T''_{2} = 1.2 \cdot T_2 = 1.62
	\]

	Моделировать случай системы при наличии случайного внешнего воздействия уже не будем, так как опыт показал, что там помогут только фильтры. Для всего остального проведем вычисления и посмотрим на результаты - графики представлены на рисунках \ref{13}-\ref{16} для $T_2 = T'_2$ и \ref{17}-\ref{20} для $T_2 = T''_2$.
	\begin{figure}
		\centering
		\includegraphics[width=0.8\textwidth]{images/y_3.png}
		\caption{$y$ при \textbf{ступенчатом} изменении \textbf{задающего} воздействия и $T_2 = T'_2$}
		\label{13}
	\end{figure}
	\begin{figure}
		\centering	
		\includegraphics[width=0.8\textwidth]{images/u_3.png}
		\caption{$u$ при \textbf{ступенчатом} изменении \textbf{задающего} воздействия и $T_2 = T'_2$}
		\label{14}
	\end{figure}
	\begin{figure}
		\centering
		\includegraphics[width=0.8\textwidth]{images/y1_3.png}
		\caption{$y$ при \textbf{ступенчатом} изменении \textbf{внешнего} воздействия и $T_2 = T'_2$}
		\label{15}
	\end{figure}
	\begin{figure}
		\centering
		\includegraphics[width=0.8\textwidth]{images/u1_3.png}
		\caption{$u$ при \textbf{ступенчатом} изменении \textbf{внешнего} воздействия и $T_2 = T'_2$}
		\label{16}
	\end{figure}

	\begin{figure}
		\centering
		\includegraphics[width=0.8\textwidth]{images/y_4.png}
		\caption{$y$ при \textbf{ступенчатом} изменении \textbf{задающего} воздействия и $T_2 = T''_2$}
		\label{17}
	\end{figure}
	\begin{figure}
		\centering	
		\includegraphics[width=0.8\textwidth]{images/u_4.png}
		\caption{$u$ при \textbf{ступенчатом} изменении \textbf{задающего} воздействия и $T_2 = T''_2$}
		\label{18}
	\end{figure}
	\begin{figure}
		\centering
		\includegraphics[width=0.8\textwidth]{images/y1_4.png}
		\caption{$y$ при \textbf{ступенчатом} изменении \textbf{внешнего} воздействия и $T_2 = T''_2$}
		\label{19}
	\end{figure}
	\begin{figure}
		\centering
		\includegraphics[width=0.8\textwidth]{images/u1_4.png}
		\caption{$u$ при \textbf{ступенчатом} изменении \textbf{внешнего} воздействия и $T_2 = T''_2$}
		\label{20}
	\end{figure}

	Видим, что регулятор успешно справляется с задачей слежения даже при неправильно заданном $T_2$, но делает это несколько хуже, если сравнивать с результатами из 3 пункта, так как переходные процессы стали медленнее и колебательнее.

	\section{Выводы}
	В ходе лабораторной работы был синтезирован и исследован цифровой регулятор, полученный аппроксимацией непрерывного ПИД-регулятора, для управления температурой апериодического объекта второго порядка. Показано, что при корректной компенсации полюсов обеспечивается устойчивость системы и нулевая статическая ошибка при постоянных задающих и ступенчатых возмущающих воздействиях. Получено, что уменьшение периода дискретизации улучшает качество переходных процессов, снижая колебательность и повышая качество слежения. Также выявлено, что небольшая неточность в оценке параметров объекта ухудшает динамические характеристики, но в целом не приводит к потере устойчивости и работоспособности системы.








\end{document}