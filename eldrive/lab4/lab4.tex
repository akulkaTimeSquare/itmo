 \documentclass[a4paper,hidelinks,14pt]{extarticle}

\usepackage[utf8]{inputenc}
\usepackage[T2A]{fontenc}
\usepackage[english, russian]{babel}
\usepackage{lipsum}
\usepackage{amsmath}
\usepackage{amssymb}
\usepackage{amsfonts}
\usepackage{mathtools}
\usepackage{datetime}
\usepackage[pdftex]{graphicx}
\usepackage{indentfirst}
\usepackage{asymptote}
\usepackage{systeme}
\usepackage[dvipsnames]{xcolor}
\usepackage{lastpage}
\usepackage{fancybox,fancyhdr}
\usepackage{hyperref}
\usepackage[font={small,it}]{caption}
\fancyhead[L]{Лабораторная работа №4}
\fancyhead[C]{}
\fancyhead[R]{\textit{Механика электропривода}}
\fancyfoot[L]{}
\fancyfoot[C]{Страница \thepage\space из \pageref{LastPage}}
\fancyfoot[R]{}
\pagestyle{fancy}
\newcommand{\gt}{\textgreater}
\newcommand{\lt}{\textless}

\begin{document}
	\begin{titlepage}
		\setlength{\parindent}{0ex}
		
		\begin{center}
			\textsc{
				\vspace{1ex}
				Научно исследовательский университет ИТМО \\
				\vspace{0.5ex}
				Факультет систем управления и робототехники \\
				\vspace{0.5ex}
			}
		\end{center}
		
		\vspace{50mm}
		
		\begin{center}
			Отчет по лабораторной работе №4 \\
			Механика электропривода
		\end{center}
		
		\vspace{45mm}
		
		\begin{minipage}{.37\linewidth}
			Выполнили студенты \\
			\\
			\\
			\\
			Преподаватель
		\end{minipage}
		\hfill
		\begin{minipage}{.60\linewidth}
			\begin{flushright}
				Мовчан Игорь Евгеньевич \\
				Демкин Алексей Игоревич \\
				Ле Ван Хынг \\
				Будков Ярослав Антонович \\
				Маматов Александр Геннадьевич
			\end{flushright}
		\end{minipage}
		
		\vfill
		\begin{center}
			Санкт-Петербург
			\\
			2025
		\end{center}
		
	\end{titlepage}

	\tableofcontents
	\clearpage
	
	\section{Исходные данные}
	Для начала зададимся трёхфазным асинхронным двигателем из каталога. Выберем модель 5AM112M4, для которого:
	\begin{itemize}
		\item номинальная мощность $P_n = 5.5$ кВт;
		\item Номинальная частота вращения $n_n = 1440$ об/мин;
		\item КПД $\eta = 86\%$
		\item Коэффициент мощности $\cos \phi = 0.83$
		\item Номинальное напряжение $U_n = 380$ В;
		\item Номинальный ток $I_n = 11.7$ А;
		\item Номинальный момент $M_n = 36.5$ Нм;
		\item Отношение пускового момента к номинальному $M_p/M_n = 2.6$
		\item Отношение пускового тока к номинальному $I_p/I_n = 6.7$
		\item Отношение $\lambda = M_{max}/M_n = 2.2$
		\item Момент инерции $J = 0.02$ кгм$^2$
		\item Масса 56.5 кг
		\item Сервис-фактор $F_s = 1.15$
	\end{itemize}

	Также имеем:
	\begin{itemize}
		\item Частота питания $f_s = 50$ Гц;
		\item Число фаз двигателя $m_1 = 3$
	\end{itemize}

	\newpage

	\section{Расчёт параметров схемы замещения}
	Расчитаем дополнительные параметры:
	\[
		U_{1N} = \frac{U_n}{\sqrt{3}} = \frac{380}{\sqrt{3}} \approx 219.393 \text{ В}
	\]
	\[
		I_{1N} = I_n = 11.7 \text{ А}
	\]
	
	Угловая частота сети:
	\[
		\omega_1 = 2\pi f \approx 314.16 \text{ рад/с}
	\]

	Скольжение:
	\[
		s_n = 1 - \frac{n_n}{n_1} = 1 - \frac{1440}{1500} = 0.04
	\]

	Определим активное сопротивление статора через потери:
	\[
		\Delta P_{1 Cu} = m_1 I_{1N}^2 r_1 = m_1 U_{1N} I_{1N} \cos \phi - \frac{M_n \omega_1}{z_p}
	\]

	Откуда:
	\[
		r_1 = \frac{m_1 U_{1N} I_{1N} \cos \phi - \frac{M_n \omega_1}{z_p}}{m_1 I_{1N}^2} \approx 1.6 \text{ Ом}
	\]

	Здесь принято $z_p = 2$ - число пар полюсов.

	Активное сопротивление ротора:
	\[
		r'_2 = \frac{m_1 z_p U_{1N}^2 s_n}{M_n \omega_1} \approx 1.007 \text{ Ом}
	\]

	Вычислим:
	\[
		a(r'_2) = \frac{r_1}{r'_2} = 1 - 2a(r'_2)s_n(\lambda - 1) \approx 1.589
	\]
	\[
		s_m(r'_2) = s_n(\lambda + \sqrt{\lambda^2 - a(r'_2)}) \approx 0.16
	\]
	\[
		x_{ks}(r'_2) = \sqrt{\left(\frac{r'_2}{s_m(r'_2)}\right)^2 - r_1^2} \approx 6.087
	\]
	\[
		b(r'_2) = \frac{x_{ks}(r'_2)}{(r_1 + \frac{r'_2}{s_n})^2 + (x_{ks}(r'_2))^2} \approx 0.008
	\]
	\[
		x_m(r'_2) = \frac{1}{\frac{I_n\sqrt{1 - cos^2\phi}}{U_{1N}} - b(r'_2)} \approx 45.988
	\]

	Откуда:
	\[
		I'_2(r'_2) = \frac{U_{1N}}{\sqrt{(r_1 + \frac{r'_2}{s_m(r'_2)})^2 + (x_{ks}(r'_2))^2}} \approx 4.7
	\]

	Также имеем:
	\[
		x_{1\sigma} = x_{2\sigma} = x_{ks}(r'_2) / 2 \approx 3.0435
	\]

	\section{Различные характеристики}
	Механические характеристики будем рассчитывать по формуле:
	\[
		M(s) = \frac{m_1 z_p U_{1N}^2 r'_2}{\omega_1 s ((r_1 + \frac{r'_2}{s})^2 + (x_{1\sigma} + x_{2\sigma})^2)}
	\]
	\[
		I_2 = \frac{U_{1N}}{\sqrt{(r_1 + \frac{r'_2}{s})^2 + (x_{1\sigma} + x_{2\sigma})^2}}
	\]

	Электромагнитные характеристики:
	\[
		M_k(s) = \frac{m_1 z_p U_{1N}^2 r'_2 k_r}{\omega_1 s ((r_1 + \frac{r'_2}{s})^2 + (x_{1\sigma} + x_{2\sigma}k_x)^2)}
	\]
	\[
		I'_{2k} = \frac{U_{1N}}{\sqrt{(r_1 + \frac{r'_2}{s})^2 + (x_{1\sigma} + x_{2\sigma}k_x)^2}}
	\]

	Рабочие характеристики АД. Активная мощность:
	\[
		P_2 = m_1 I_2^2 r'_2 \frac{1 - s}{s}
	\]
	\[
		P_1 = P_2 + m_1 I_1^2 r_1 + m_1 I_2^2 r'_2,
	\]

	где
	\[
		I_1 = I'_2 + \frac{U_{1N}}{c_1 x_m}
	\]

	КПД:
	\[
		\eta = \frac{P_2}{P_1}
	\]
	
	Коэффициент мощности:
	\[
		\cos \phi = \frac{P_1}{3U_{1N} I_1}
	\]

	Перейдем к моделированию.
	\begin{figure}[h]
		\centering
		\includegraphics[width=0.7\textwidth]{figures/M_s.png}
		\caption{Механическая характеристика $M(s)$}
	\end{figure}
	
	\begin{figure}
		\centering
		\includegraphics[width=0.7\textwidth]{figures/M_k_s.png}
		\caption{Электромагнитная характеристика $M_k(s)$}
	\end{figure}
	\begin{figure}
		\centering
		\includegraphics[width=0.7\textwidth]{figures/I_2.png}
		\caption{Ток ротора $I_2(s)$}
	\end{figure}
	\begin{figure}
		\centering
		\includegraphics[width=0.7\textwidth]{figures/I_2k.png}
		\caption{Ток ротора $I'_{2k}(s)$}
	\end{figure}
	\begin{figure}
		\centering
		\includegraphics[width=0.7\textwidth]{figures/P_2.png}
		\caption{Активная мощность $P_2(s)$}
	\end{figure}
	\begin{figure}
		\centering
		\includegraphics[width=0.7\textwidth]{figures/P_1.png}
		\caption{Подводимая мощность $P_1(s)$}
	\end{figure}
	\begin{figure}
		\centering
		\includegraphics[width=0.7\textwidth]{figures/eta.png}
		\caption{КПД $\eta(s)$}
	\end{figure}
	\begin{figure}
		\centering
		\includegraphics[width=0.7\textwidth]{figures/cos_phi.png}
		\caption{Коэффициент мощности $\cos \phi(s)$}
	\end{figure}

	\newpage
	\section{Вывод}
	В ходе работы мы получили механические и электромагнитные характеристики АД, а также рабочие характеристики.
	Построенные графики соответствуют теоретическим расчетам, значения на графиках близки к данной справочной информации.




	



	


	
\end{document}