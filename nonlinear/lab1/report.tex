\documentclass[a4paper,hidelinks,14pt]{extarticle}

\usepackage[utf8]{inputenc}
\usepackage[T2A]{fontenc}
\usepackage[english, russian]{babel}
\usepackage{lipsum}
\usepackage{amsmath}
\usepackage{amssymb}
\usepackage{amsfonts}
\usepackage{mathtools}
\usepackage{datetime}
\usepackage[pdftex]{graphicx}
\usepackage{indentfirst}
\usepackage{asymptote}
\usepackage{systeme}
\usepackage[dvipsnames]{xcolor}
\usepackage{lastpage}
\usepackage{fancybox,fancyhdr}
\usepackage{hyperref}
\usepackage[font={small,it}]{caption}
\fancyhead[L]{Лабораторная работа №1}
\fancyhead[C]{}
\fancyhead[R]{\textit{Нелинейные системы}}
\fancyfoot[L]{}
\fancyfoot[C]{\thepage\space}
\fancyfoot[R]{}
\pagestyle{fancy}
\newcommand{\gt}{\textgreater}
\newcommand{\lt}{\textless}
\usepackage{listings}
\usepackage{xcolor}
\lstset{
    basicstyle=\ttfamily\small,
    keywordstyle=\color{blue},
    commentstyle=\color{gray},
    stringstyle=\color{red},
    numbers=left,
    numberstyle=\color[gray]{0.7}\ttfamily\small,
    stepnumber=1,
    numbersep=8pt,
    frame=single,
    showstringspaces=false,
    tabsize=4,
    breaklines=true
}
\usepackage{subcaption}

\begin{document}
	\begin{titlepage}
		\setlength{\parindent}{0ex}
		
		\begin{center}
			\textsc{
				\vspace{1ex}
                Научно-исследовательский университет ИТМО \\
				\vspace{0.5ex}
				Факультет систем управления и робототехники \\
				\vspace{0.5ex}
			}
		\end{center}
		
		\vspace{45mm}
		
		\begin{center}
			Отчет по лабораторной работе №1\\
			Нелинейные системы
			\\
			Точки равновесия и локальные регуляторы \\
		\end{center}
		
		\vspace{50mm}
		
		\begin{minipage}{.4\linewidth}
			Выполнили студенты
            \\
			\\
            \\
			Преподаватель
		\end{minipage}
		\hfill
		\begin{minipage}{.55\linewidth}
			\begin{flushright}
				Мовчан Игорь Евгеньевич
                \\                
				Ибахаев Зубайр Руслан-Бекович
                \\
				Белоус Савва Эрнестович
				\\
				Зименко Константин Александрович
			\end{flushright}
		\end{minipage}
		
		\vfill
		\begin{center}
			Санкт-Петербург
			\\
			2025
		\end{center}
		
	\end{titlepage}

	\tableofcontents
	\clearpage
	
	\section{Точки равновесия систем}

	\subsection{Первая система}
	Пусть система задана следующими уравнениями:
	\[
	\begin{cases}
		\dot{x}_1 = -x_1 + 2x_1^3 + x_2  = f_1\\
		\dot{x}_2 = -x_1 - x_2 = f_2
	\end{cases}
	\]

	Найдем все её точки равновесия, учитывая определение $\dot{x} = 0$:
	\[
	\begin{cases}
		-x_1 + 2x_1^3 + x_2 = 0\\
		-x_1 - x_2 = 0
	\end{cases}
	\Rightarrow
	\begin{cases}
		2x_1(x_1^2 - 1) = 0 \\
		x_2 = - x_1
	\end{cases}
	\]

	Получаем следующие точки:
	\[
		x^*_1 = (0,0), \quad x^*_2 = (1,-1), \quad x^*_3 = (-1,1)
	\]

	Определим типы изолированных состояний равновесия. Для этого дополнительно вычислим матрицу Якоби системы:
	\[
	J(x) = \frac{\partial f}{\partial x} = \begin{bmatrix}
		\dfrac{\partial f_1}{\partial x_1} & \dfrac{\partial f_1}{\partial x_2} \\[2.25ex]
		\dfrac{\partial f_2}{\partial x_1} & \dfrac{\partial f_2}{\partial x_2}
	\end{bmatrix}
	= \begin{bmatrix}
		-1 + 6x_1^2 & 1 \\
		-1 & -1
		\end{bmatrix}
	\]

	Рассмотрим каждую точку в отдельности.
	
	\textbf{Начнём с $x^*_1$.}
	Матрица Якоби:
	\[
	J(x^*_1) = \begin{bmatrix} -1 & 1 \\ -1 & -1 \end{bmatrix}
	\]

	Откуда её собственные числа:
	\[
		\det(\lambda I - J(x_1^*)) = \lambda^2 + 2\lambda + 2 = 0
	\] 
	\[
	\lambda_{12} = -1 \pm i \Rightarrow \Re(\lambda_{12}) = -1 < 0
	\]

	Следовательно, тип состояния равновесия - \textbf{устойчивый фокус}.
	
	\textbf{Далее, возьмём $x^*_2$ и $x^*_3$.}
	Их матрица Якоби:
	\[
	J(x^*_2) = J(x^*_3) = \begin{bmatrix} 5 & 1 \\ -1 & -1 \end{bmatrix}
	\]

	Из которой получаем собственные числа:
	\[
		\det(\lambda I - J(x_2^*)) = \det(\lambda I - J(x_3^*)) = \lambda^2 - 4\lambda - 4 = 0
	\] 
	\[
	\lambda_{12} = 2 \pm 2 \sqrt{2}
	\]

	Они имеют отрицательную и положительную вещественные части, а значит, точки являются \textbf{седловыми}.

	\subsection{Вторая система}
	Рассмотрим систему:
	\[
		\begin{cases}
			\dot{x}_1 = x_1 + x_1x_2 \\
			\dot{x}_2 = -x_2 + x_2^2 + x_1x_2 - x_1^3
		\end{cases}
	\]

	Найдём точки равновесия:
	\[
		\begin{cases}
			x_1 + x_1x_2 = 0 \\
			-x_2 + x_2^2 + x_1x_2 - x_1^3 = 0
		\end{cases}
	\]

	Из первого уравнения: $x_1(1 + x_2) = 0 \Rightarrow x_1 = 0$ или $x_2 = -1$

	\textbf{Случай 1:} $x_1 = 0$, тогда второе уравнение:
	\[
	-x_2 + x_2^2 = 0 \Rightarrow x_2(x_2 - 1) = 0 \Rightarrow x_2 \in \{0; 1\}
	\]

	\textbf{Случай 2:} $x_2 = -1$, тогда второе уравнение:
	\[
	x_1^3 + x_1 - 2 = 0 \Rightarrow (x_1 - 1)(x_1^2 + x_1 + 2) = 0 \Rightarrow x_1 = 1
	\]

	Итого, получаем все изолированные точки равновесия:
	\[
	x_1^* = (0,0), \quad x_2^* = (0,1), \quad x_3^* = (1,-1)
	\]

	Аналогично предыдущему пункту рассмотрим каждую из них в отдельности для определения типа равновесия. Для начала найдем линеаризацию системы через матрицу Якоби:
	\[
	J(x) = \frac{\partial f}{\partial x} = \begin{bmatrix}
	1 + x_2 & x_1 \\
	x_2 - 3x_1^2 & -1 + 2x_2 + x_1
	\end{bmatrix}
	\]

	Теперь рассмотрим каждую точку.
	
	\textbf{Начнём с $x^*_1$.}
	Матрица Якоби:
	\[
	J(x^*_1) = \begin{bmatrix} 1 & 0 \\ 0 & -1 \end{bmatrix}
	\]

	Cобственные числа представлены явно на диагонали:
	\[
	\lambda_{12} = \pm 1
	\]

	Они имеют как положительые действительные части, так и отрицательные, поэтому точка равновесия является \textbf{седлом}.

	\textbf{Далее, возьмём $x_2^*$}. Матрица Якоби:
	\[
	J(x_2^*) = \begin{bmatrix} 2 & 0 \\ 1 & 1 \end{bmatrix}
	\]
	
	Откуда собственные числа:	
	\[
		\det(\lambda I - J(x_2^*)) = \lambda^2 - 3\lambda + 2 = 0
	\]
	\[
		\lambda_1 = 2, \quad \lambda_2 = 1 \Rightarrow \Re (\lambda_{12}) > 0
	\]

	Они имеют положительные вещественные части, поэтому точка имеет тип равновесия - \textbf{неустойчивый узел}.
	
	\textbf{Наконец, примем $x_3^*$.} Матрица Якоби:
	\[
		J(x_2^*) = \begin{bmatrix} 0 & 1 \\ -4 & -2 \end{bmatrix}
	\]

	Собственные числа:
	\[
		\det(\lambda I - J(x_3^*)) = \lambda^2 + 2\lambda + 4 = 0
	\]
	\[
		\lambda_{12} = -1 \pm i \sqrt{3} \Rightarrow \Re (\lambda_{12}) = -1 < 0
	\]

	Действительные части отрицательны, есть мнимые числа, поэтому тип равновесия рассматриваемой точки - \textbf{устойчивый фокус}.

	\subsection{Третья система}
	Рассмотрим систему:
	\[
		\begin{cases}
			\dot{x}_1 = x_2 \\
			\dot{x}_2 = -x_1 + x_2(1 - x_1^2 + 0.1x_1^4)
		\end{cases}
	\]

	Найдем точки равновесия, приравняв производные к нулю:
	\[
	\begin{cases}
		x_2 = 0\\
		-x_1 + x_2(1 - x_1^2 + 0.1x_1^4) = 0
	\end{cases}
	\Rightarrow
	\begin{cases}
		x_2 = 0 \\
		x_1 = 0
	\end{cases}
	\]

	Получаем единственную точку:
	\[
		x_1^* = (0,0)
	\]

	Линеаризуем систему, найдя матрицу Якоби:
	\[
	J(x) = \frac{\partial f}{\partial x} = \begin{bmatrix}
		0 & 1 \\
		-1 - 2x_1x_2 + 0.4x_1^3x_2 & 1 - x_1^2 + 0.1x_1^4
	\end{bmatrix}
	\]

	Для определения типа равновесия изолированного состояния подставим точку в приближение:
	\[
		J(x_1^*) = \begin{bmatrix}
			0 & 1 \\ -1 & 1
		\end{bmatrix}
	\]

	Найдём собственные числа:
	\[
		\det(\lambda I - J(x_1^*)) = \lambda^2 - \lambda + 1 = 0
	\]
	\[
		\lambda_{12} = \frac{1 \pm i\sqrt{3}}{2} \Rightarrow \Re (\lambda_{12}) = \frac{1}{2} > 0
	\]

	Следовательно, точка является \textbf{неустойчивым фокусом}.

	\subsection{Четвёртая система}
	Система:
	\[
		\begin{cases}
		\dot{x}_1 = (x_1 - x_2)(1 - x_1^2 - x_2^2) \\
		\dot{x}_2 = (x_1 + x_2)(1 - x_1^2 - x_2^2)
		\end{cases}
	\]
	
	Найдем точки равновесия, приравняв правые части к нулю:
	\[
	\begin{cases}
	(x_1 - x_2)(1 - x_1^2 - x_2^2) = 0 \\
	(x_1 + x_2)(1 - x_1^2 - x_2^2) = 0
	\end{cases}
	\]

	\textbf{Случай 1:} $1 - x_1^2 - x_2^2 = 0$ (окружность)
	\[
	x_1^2 + x_2^2 = 1
	\]

	Все точки на окружности с центром $x_0 = (0, 0)$ и радиусом $R = 1$ являются точками равновесия.

	\textbf{Случай 2:} $1 - x_1^2 - x_2^2 \neq 0$, тогда:
	\[
	\begin{cases}
		x_1 - x_2 = 0 \\
		x_1 + x_2 = 0
	\end{cases}
	\]

	Решение: $x_1 = 0$, $x_2 = 0$

	Все точки равновесия:
	\[
	x_1^* = (0,0), \quad x_2^* = \text{точки на окружности } x_1^2 + x_2^2 = 1
	\]

	Причем отметим, что первая точка - изолированная, так как в окрестности неё нет других точек равновесия, а вторая - нет.

	Определми тип изолированного состояния равновесия $x_1^*$. Для этого линеаризуем систему, вычислив матрицу Якоби системы в этой точке, предварительно сократив на ненулевой член $1 - x_1^2 - x_2^2 \neq 0$:
	\[
		J(x_1^*) = \begin{bmatrix} 1 & -1 \\ 1 & 1 \end{bmatrix}
	\]

	Откуда собственные числа:
	\[
		\lambda_{12} = 1 \pm i \Rightarrow \Re(\lambda_{12}) = 1 > 0
	\]

	Следовательно, точка равновесия $x_1^*$ - \textbf{неустойчивый фокус}.

	Далее, перейдем к анализу неизолированного состояния равновесия $x_2^*$ через переход к полярной системе координат:
	\[
	\begin{cases}
		x_1 = r \cos (\theta) \\
		x_2 = r \sin (\theta)
	\end{cases}
	\quad 
	\text{или}
	\quad
	\begin{cases}
		r^2 = x_1^2 + x_2^2 \\
		\theta = \arctan\left(\dfrac{x_2}{x_1}\right)
	\end{cases}
	\]

	\textbf{Вычислим производную $\dot{r}$:}
	\[
	2r\dot{r} = 2x_1\dot{x}_1 + 2x_2\dot{x}_2
	\]
	\[
	\dot{r} = \frac{x_1\dot{x}_1 + x_2\dot{x}_2}{r}
	\]

	Подставляем $\dot{x}_1$ и $\dot{x}_2$:
	\[
	x_1\dot{x}_1 = x_1(x_1 - x_2)(1 - r^2) = (x_1^2 - x_1x_2)(1 - r^2)
	\]
	\[
	x_2\dot{x}_2 = x_2(x_1 + x_2)(1 - r^2) = (x_1x_2 + x_2^2)(1 - r^2)
	\]

	Складываем:
	\[
	x_1\dot{x}_1 + x_2\dot{x}_2 = (x_1^2 + x_2^2)(1 - r^2) = r^2(1 - r^2)
	\]

	Таким образом:
	\[
	\dot{r} = \frac{r^2(1 - r^2)}{r} = r(1 - r^2)
	\]

	\textbf{Вычислим производную $\dot{\theta}$:}
	\[
	\dot{\theta} = \dfrac{1}{1 + \left(\dfrac{x_2}{x_1}\right)^2} \cdot \dfrac{x_1\dot{x}_2 - x_2\dot{x}_1}{x_1^2} = \dfrac{x_1\dot{x}_2 - x_2\dot{x}_1}{x_1^2 + x_2^2} = \dfrac{x_1\dot{x}_2 - x_2\dot{x}_1}{r^2}
	\]

	Посчитаем отдельно числитель:
	\[
	x_1\dot{x}_2 = x_1(x_1 + x_2)(1 - r^2) = (x_1^2 + x_1x_2)(1 - r^2)
	\]
	\[
	x_2\dot{x}_1 = x_2(x_1 - x_2)(1 - r^2) = (x_1x_2 - x_2^2)(1 - r^2)
	\]

	Вычитаем:
	\[
	x_1\dot{x}_2 - x_2\dot{x}_1 = (x_1^2 + x_2^2)(1 - r^2) = r^2(1 - r^2)
	\]

	Таким образом:
	\[
	\dot{\theta} = \frac{r^2(1 - r^2)}{r^2} = 1 - r^2
	\]

	Итоговая система в полярной системе координат тогда:
	\[
		\begin{cases}
			\dot{r} = r(1 - r^2) \\
			\dot{\theta} = 1 - r^2
		\end{cases}
	\]

	Проведем небольшой анализ движения системы:

	\begin{itemize}
		\item При $0 < r < 1$: $\dot{r} > 0$ - траектории удаляются
		\item При $r = 1$: $\dot{r} = \dot{\theta} = 0$ - предельный цикл
		\item При $r > 1$: $\dot{r} < 0$ - траектории приближаются к циклу
	\end{itemize}
		
	Значит, предельный цикл $r = 1$ является \textbf{устойчивым}. Все точки цикла - точки равновесия.
	
	Он также соответствует неизолированной системе точек равновесия $x_2^*$, поэтому для них можно сделать те же самые заключения.
	
	\subsection{Пятая система}
	Примем систему:
	\[
	\begin{cases}
	\dot{x}_1 = -x_1^3 + x_2 \\
	\dot{x}_2 = x_1 - x_2^3
	\end{cases}
	\]

	Приравняем производную координат к нулю $\dot{x} = 0$ и решим систему, найдя точки равновесия:
	\[
	\begin{cases}
	-x_1^3 + x_2 = 0 \\
	x_1 - x_2^3 = 0
	\end{cases} \Rightarrow
	\begin{cases}
		x_2 = x_1^3 \\
		x_1(1 - x_1^8) = 0
	\end{cases}
	\]

	Рассматриваем только действительные переменные, поэтому итоговые точки равновесия:
	\[
	x_1^* = (0,0), \quad x_2^* = (1,1), \quad x_3^* = (-1,-1)
	\]

	Найдём их тип, используя линеаризацию через матрицу Якоби:
	\[
	J(x) = \frac{\partial f}{\partial x} = \begin{bmatrix}
	-3x_1^2 & 1 \\
	1 & -3x_2^2
	\end{bmatrix}
	\]

	Рассмотрим каждую точку в отдельности.

	\textbf{Начнём с $x_1^*$.} Матрица Якоби в этой точке:
	\[
		J(x_1^*) = \begin{bmatrix} 0 & 1 \\ 1 & 0 \end{bmatrix}
	\]

	Собственные числа:
	\[
		\det(\lambda I - J(x_1^*)) = \lambda^2 - 1 = 0
	\] 
	\[
		\lambda_{12} = \pm 1
	\]

	Их вещественные части могут быть и положительными, и отрицательными, поэтому тип равновесия - \textbf{седло}.

	\textbf{Перейдём к $x_2^*$.} Матрица Якоби:
	\[
		J(x_2^*) = \begin{bmatrix} -3 & 1 \\ 1 & -3 \end{bmatrix}
	\]

	Собственные числа:
	\[
		\det(\lambda I - J(x_2^*)) = \lambda^2 + 6\lambda + 8 = 0
	\] 
	\[
		\lambda_1 = -2, \quad \lambda_2 = -4 \Rightarrow \Re(\lambda_{12}) < 0
	\]

	Действительные части отрицательны, поэтому тип изолированной точки равновесия - \textbf{устойчивый фокус}.

	\textbf{Наконец, рассмотрим $x_3^*$.}

	Значение матрицы Якоби в ней будет давать те же результаты, что и в предыдущем случае, поэтому можно сделать те же выводы - точка является \textbf{устойчивым фокусом}.

	\subsection{Шестая система.}
	Система:
	\[
	\begin{cases}
	\dot{x}_1 = -x_1^3 + x_2^3 \\
	\dot{x}_2 = x_2^3x_1 - x_2^3
	\end{cases}
	\]
	
	Вычислим точки равновесия, приравняв правые части к 0:
	\[
	\begin{cases}
	-x_1^3 + x_2^3 = 0 \\
	x_2^3x_1 - x_2^3 = 0
	\end{cases} \Rightarrow
	\begin{cases}
		x_1^3 = x_2^3 \\
		x_2^3(x_1 - 1) = 0
	\end{cases} \Rightarrow
	\begin{cases}
		x_1 = x_2 \\
		x_2(x_1 - 1) = 0
	\end{cases}
	\]

	Точки равновесия:
	\[
	x_1^* = (0,0), \quad x_2^* = (1,1)
	\]

	Линеаризуем систему через матрицу Якоби:
	\[
	J(x) = \frac{\partial f}{\partial x}= \begin{bmatrix}
	-3x_1^2 & 3x_2^2 \\
	x_2^3 & 3x_1x_2^2 - 3x_2^2
	\end{bmatrix}
	\]

	Определеим тип каждой точки равновесия по очереди.

	\textbf{Начнём с $x_2^*$.} Матрица Якоби:
	\[
		J(x_2^*) = \begin{bmatrix} -3 & 3 \\ 1 & 0 \end{bmatrix}
	\]

	Её собственные числа:
	\[
		\det(\lambda I - J(x_2^*)) = \lambda^2 + 3\lambda - 3 = 0
	\]
	\[
		\lambda_{12} = \frac{-3 \pm \sqrt{21}}{2}
	\]

	Вещественные части принимают как отрицательные, так и положительные значения, поэтому тип неизолированной точки равновесия - \textbf{седло}.

	\textbf{Далее, рассмотрим $x_1^*$.} Матрица Якоби в этой точке:
	\[
		J(x) = \begin{bmatrix}
			0 & 0 \\
			0 & 0
		\end{bmatrix}
	\]

	По ней невозможно получить какую-либо информацию о типе - нужен более детальный разбор системы через функцию Ляпунова:
	\[
	V(x_1,x_2) = \frac{1}{4}x_1^4 + \frac{1}{4}x_2^4
	\]

	Проверим её производную:
	\[
	\dot{V} = \frac{\partial V}{\partial x_1}\dot{x}_1 + \frac{\partial V}{\partial x_2}\dot{x}_2 = x_1^3(-x_1^3 + x_2^3) + x_2^3(x_1x_2^3 - x_2^3)
	\]

	Раскроем скобки:
	\[
	\dot{V} = -x_1^6 - x_2^6 + x_1^3x_2^3 + x_1x_2^6
	\]

	Оценим знак в малой окрестности начала координат. Используем неравенство Юнга при $p = q = 2$:
	\[
	|x_1^3x_2^3| \leq \frac{1}{2}x_1^6 + \frac{1}{2}x_2^6
	\]

	Также отметим, что $x_1x_2^6$ пренебрежимо мало в сравнении с другими членами более низких порядков. Таким образом:
	\[
	\dot{V} \leq -x_1^6 - x_2^6 + \frac{1}{2}x_1^6 + \frac{1}{2}x_2^6 + O(\|x\|^7) = -\frac{1}{2}x_1^6 - \frac{1}{2}x_2^6 + O(\|x\|^7)
	\]

	В достаточно малой окрестности начала координат при $x \neq 0$:
	\[
	\dot{V} < 0
	\]

	Значит, точка $x_1^*$ является \textbf{асимптотически устойчивой}.

	\subsection{Седьмая система}
	Рассмотрим нелинейную систему с тремя переменными:
	\[
	\begin{cases}
		\dot{x}_1 = -x_1^3 + x_2^3 \\
		\dot{x}_2 = x_1 + 3x_3 - x_2^3 \\
		\dot{x}_3 = x_1x_3 - x_2^3 - \sin x_1
	\end{cases}
	\]

	Найдём точки равновесия через решение системы на $\dot{x} = 0$:
	\[
		\begin{cases}
			-x_1^3 + x_2^3 = 0\\
			x_1 + 3x_3 - x_2^3 =0\\
			x_1x_3 - x_2^3 - \sin x_1 = 0
		\end{cases}
		\Rightarrow
		\begin{cases}
			x_1 = x_2 \\
			x_3 = (x_1^3 - x_1)/3 \\
			x_1x_3 - x_2^3 - \sin x_1 = 0
		\end{cases}
	\]

	Подставляем в третье:
	\[
	\frac{x_1^4 - x_1^2}{3} - x_1^3 - \sin x_1 = 0
	\]
	\[
	x_1^4 - 3x_1^3 - x_1^2 - 3\sin x_1 = 0
	\]

	Откуда точки равновесия:
	\[
	x_1^* = (0, 0, 0), \quad x_2^*\approx(3.2913, 3.2913, 10.7872)
	\]

	Линеаризуем систему через матрицу Якоби:
	\[
		J(x) = \frac{\partial f}{\partial x} =
		\begin{bmatrix}
			-3x_1^2 & 3x_2^2 & 0 \\
			1 & -3x_2^2 & 3 \\
			x_3 - \cos x_1 & -3x_2^2 & x_1
		\end{bmatrix}
	\]

	Рассмотрим каждую точку в отдельности.

	\textbf{Начнём с $x_2^*$.} Матрица Якоби:
	\[
		J(x_2^*) \approx \begin{bmatrix}
			-32.5 & 32.5 & 0 \\
			1 & -32.5 & 3 \\
			11.78 & -32.5 & 3.29
		\end{bmatrix}
	\]

	Численно найдем собственные числа:
	\[
		\lambda_1 \approx -32.7, \quad \lambda_{2} \approx -30.36, \quad \lambda_3 = 1.36
	\]

	Вещественные части принимают как положительные, так и отрицательные значения, поэтому изолированная точка равновесия $x_2^*$ является \textbf{седловой}.

	\textbf{Исследуем также $x_1^*$.} Матрица Якоби:
	\[
		J(x_1^*) \approx \begin{bmatrix}
			0 & 0 & 0 \\
			1 & 0 & 3 \\
			-1 & 0 & 0
		\end{bmatrix}
	\]

	Вычислим её собственные числа:
	\[
	\det(\lambda I - J(x_1^*)) = \lambda^3 = 0 \Rightarrow \lambda_1 = \lambda_2 = \lambda_3 = 0
	\]

	Все значения оказались равны 0, поэтому тип точки равновесия $x_2^*$ не может быть корректно определён с помощью линеаризации.

	\section{Фазовые портреты}
	Проверим полученные результаты с помощью визуализации. Для этого построим фазовые потреты для каждой двумерной системы.
	\begin{figure}[h]
		\centering
		\includegraphics[width=0.61\textwidth]{images/system1_eq.png}
		\caption{Первая система: $(0, 0)$ - устойчивый фокус, $(1, -1)$ и $(-1, 1)$ - седло}
		\label{system1}
	\end{figure}
	\begin{figure}
		\centering
		\includegraphics[width=0.6\textwidth]{images/system2_eq.png}
		\caption{\centering Вторая система: $(0, 0)$ - седло, $(0, 1)$ - неустойчивый узел и $(1, -1)$ - устойчивый фокус}
		\label{system2}
	\end{figure}
	\begin{figure}
		\centering
		\includegraphics[width=0.6\textwidth]{images/system3_eq.png}
		\caption{Третья система: $(0, 0)$ - неустойчивый фокус}
		\label{system3}
	\end{figure}
	\begin{figure}
		\centering
		\includegraphics[width=0.6\textwidth]{images/system4_eq.png}
		\caption{\centering Четвертая система: $(0, 0)$ - неустойчивый фокус и устойчивые точки на окружности}
		\label{system4}
	\end{figure}
	\begin{figure}
		\centering
		\includegraphics[width=0.6\textwidth]{images/system5_eq.png}
		\caption{Пятая система: $(0, 0)$ - седло, $(1, 1)$ и $(-1, -1)$ - устойчивый фокус}
		\label{system5}
	\end{figure}
	\begin{figure}
		\centering
		\includegraphics[width=0.61\textwidth]{images/system6_eq.png}
		\caption{Шестая система: $(0, 0)$ - седло и $(1, 1)$ - устойчивая точка}
		\label{system6}
	\end{figure}

	Можем видеть, что результаты визуализации, представленной на рисунках \ref{system1}-\ref{system6} полностью сходятся с аналитикой - успех!

	\section{Локально стабилизирующие регуляторы}
	\subsection{Первая система}
	Возьмём систему:
	\[
		\begin{cases}
			\dot{x_1} = -x_1 + 2x_1^3 + x_2 + \sin u_1\\
			\dot{x_2} = -x_1 - x_2 + 3 \sin u_2
		\end{cases}
	\]

	При $u = u_{ss} \equiv 0$ она приобретает вид системы, рассматриваемой в первом разделе, поэтому множество точек равновесия уже знаем:
	\[
		x^*_1 = (0,0), \quad x^*_2 = (1,-1), \quad x^*_3 = (-1,1)
	\]

	Напомним, что тип состояния точки равновесия $x_1^*$ - устойчивый фокус, поэтому задача синтеза локально стабилизирующего регулятора для него кажется неактуальной. Однако это всё же может дать более быструю сходимость и большую окрестность устойчивости.
	
	Остальные точки являются седловыми, то есть по какому-то из направлений наблюдается неустойчивость - необходим регулятор, который мы будем строить по линеаризации модели:
	\[
	\dot{x} = Ax + Bu, \quad A = \dfrac{\partial f}{\partial x} \Big|_{x=0,\hspace{1mm} u=0}, \quad B = \dfrac{\partial f}{\partial u} \Big|_{x=0,\hspace{1mm} u=0}
	\]

	Можем видеть, что система линеаризуется возле нуля. Именно поэтому для решения задачи синтеза стабилизирующего регулятора возле \textit{любой} точки равновесия будут вводиться новые координаты $x_\delta = x - x_{ss}$ и $u_\delta = u - u_{ss}$, где $x_{ss}$ соответствует точкам равновесия.

	\textbf{Итак, как и прежде, начнём с $x_1^*=(0, 0)$.}

	Линеаризация даёт следующие результаты:
	\[
		A = \dfrac{\partial f}{\partial x} \Big|_{x=x_1^*,\hspace{1mm} u=0} = 
		\begin{bmatrix}
			-1 & 1 \\
			-1 & -1
		\end{bmatrix}
	\]
	\[
		B = \dfrac{\partial f}{\partial u} \Big|_{x=x_1^*,\hspace{1mm} u=0} = \begin{bmatrix}
			1 & 0 \\
			0 & 3
		\end{bmatrix}
	\]

	Синтезируем регулятор, минимизируя функционал качества при заданных параметрах $Q = R = I$:
	\[
		J = \int_0^{+ \infty} (x^T Q x + u^T R u) dx
	\]

	То есть будем использовать LQR. В итоге получим матрицу обратных связей для линеаризованной системы:
	\[
		K_{x_1^*} = \begin{bmatrix}
			0.3897 & 0.0302 \\
 			0.0907 & 0.7486
		\end{bmatrix} \Rightarrow u = -K_{x_1^*} x_\delta
	\]

	\begin{figure}
		\centering
		\includegraphics[width=0.8\textwidth]{images/part3_1_u_[-0.2, -0.3].png}
		\caption{График управления системы при $x(0) = [-0.2, -0.3]$}
		\label{part3_1_u1}
	\end{figure}
	\begin{figure}
		\centering
		\includegraphics[width=0.8\textwidth]{images/part3_1_[-0.2, -0.3].png}
		\caption{График движения системы при $x(0) = [-0.2, -0.3]$}
		\label{part3_1_x1}
	\end{figure}
	\begin{figure}
		\centering
		\includegraphics[width=0.8\textwidth]{images/part3_1_u_[0.5, 0.4].png}
		\caption{График управления системы при $x(0) = [0.5, 0.4]$}
		\label{part3_1_u2}
	\end{figure}
	\begin{figure}
		\centering
		\includegraphics[width=0.8\textwidth]{images/part3_1_[0.5, 0.4].png}
		\caption{График движения системы при $x(0) = [0.5, 0.4]$}
		\label{part3_1_x2}
	\end{figure}
	\begin{figure}
		\centering
		\includegraphics[width=0.8\textwidth]{images/part3_1_phase.png}
		\caption{Фазовый портрет системы при обратной связе $-K_{x_1^*}$}
		\label{part3_1_phase}
	\end{figure}

	Замоделируем систему при различных начальных условиях системы $x(0) = [-0.2, -0.3]$ и $x(0) = [0.5, 0.4]$ и найденном регуляторе. Все графики приведены на рисунках \ref{part3_1_u1}-\ref{part3_1_phase}.

	Можем видеть, что замкнутая система является устойчивой.
	
	\textbf{Проделаем аналогичные вычисления для $x_2^*=(1, -1)$.}
	
	Но для начала переместим $x_2^*$ в начало координат через сдвиг на этот вектор, получим:
	\[
		x_\delta = x - x_2^*, \quad u_\delta = u - u_{ss} = u
	\]

	Последнее верно, так как все точки равновесия были получены при нулевом входе. Откуда система:
	\[
	\begin{cases} 
		\dot{x}_{\delta 1} = -(x_{\delta 1} + 1) + 2(x_{\delta 1} + 1)^3 + (x_{\delta 2} - 1) + \sin (u_1) \\ 
		\dot{x}_{\delta 2} = -(x_{\delta 1} + 1) - (x_{\delta 2} - 1) + 3 \sin (u_2)
	\end{cases}
	\]
	
	В линейном виде:	
	\[
		x_{\delta} = Ax_{\delta} + Bu
	\]
	
	Где матрицы
	\[
		A = \left. \frac{\partial f}{\partial x_{\delta}} \right|_{x_{\delta} = 0,\hspace{1mm} u = 0} = \left. \frac{\partial f}{\partial x} \right|_{x = x_2^*,\hspace{1mm} u = 0} = 
		\begin{bmatrix}
		5 & 1 \\
		-1 & -1
		\end{bmatrix}
	\]
	\[
	B = \left. \frac{\partial f}{\partial u} \right|_{x_{\delta} = 0,\hspace{1mm} u = 0} = \left. \frac{\partial f}{\partial u} \right|_{x = x_2^*,\hspace{1mm} u = 0} = \begin{bmatrix}
		1 & 0 \\
		0 & 3
	\end{bmatrix}
	\]

	Итак, можем синтезировать локально стабилизирующий регулятор, минимизирующий функционал качества при $Q = R = I$:
	\[
		J = \int_0^{+ \infty} (x^T Q x + u^T R u) dx
	\]

	Получаем матрицу обратной связи:
	\[
		K_{x_2^*} = \begin{bmatrix}
			8.7854 & 1.0331 \\
 			3.0992 & 1.1193
		\end{bmatrix} \Rightarrow u = - K_{x_2^*} x_\delta
	\]

	Теперь можем промоделирать систему при начальных условиях $x_\delta(0) = (0.15, 0.15)$ и $x_\delta(0) = (-0.1, -0.1)$ и найденном регуляторе. Все графики приведены на рисунках \ref{part3_2_u1}-\ref{part3_2_phase} - траектории сходятся к $x_2^*$, являющейся неустойчивой. Регулятор успешно стабилизирует!
	\begin{figure}[h]
		\centering
		\includegraphics[width=0.8\textwidth]{images/part3_2_u_[0.15, 0.15].png}
		\caption{График управления системы при $x_\delta(0) = [0.15, 0.15]$}
		\label{part3_2_u1}
	\end{figure}
	
	\begin{figure}
		\centering
		\includegraphics[width=0.8\textwidth]{images/part3_2_[0.15, 0.15].png}
		\caption{График движения системы при $x_\delta(0) = [0.15, 0.15]$}
		\label{part3_2_x1}
	\end{figure}
	\begin{figure}
		\centering
		\includegraphics[width=0.8\textwidth]{images/part3_2_u_[-0.1, -0.1].png}
		\caption{График управления системы при $x_\delta(0) = [-0.1, -0.1]$}
		\label{part3_2_u2}
	\end{figure}
	\begin{figure}
		\centering
		\includegraphics[width=0.8\textwidth]{images/part3_2_[-0.1, -0.1].png}
		\caption{График движения системы при $x_\delta(0) = [-0.1, -0.1]$}
		\label{part3_2_x2}
	\end{figure}
	\begin{figure}
		\centering
		\includegraphics[width=0.8\textwidth]{images/part3_2_phase.png}
		\caption{Фазовый портрет системы при обратной связе $-K_{x_2^*}$}
		\label{part3_2_phase}
	\end{figure}

	\textbf{Наконец, исследуем точку $x_3^* = (-1, 1)$.}

	Запишем систему в координатах $x_\delta = x - x_3^*$:
	\[
	\begin{cases} 
		\dot{x}_{\delta 1} = -(x_{\delta 1} - 1) + 2(x_{\delta 1} - 1)^3 + (x_{\delta 2} + 1) + \sin u_1 \\ 
		\dot{x}_{\delta 2} = -(x_{\delta 1} - 1) - (x_{\delta 2} + 1) + 3 \sin u_2
	\end{cases}
	\]
		
	В линейном виде:
	\[
		x_{\delta} = Ax_{\delta} + Bu
	\]
	
	Где матрицы
	\[
		A = \left. \frac{\partial f}{\partial x} \right|_{x = x_3^*,\hspace{1mm} u = 0} = 
		\begin{bmatrix}
		5 & 1 \\
		-1 & -1
		\end{bmatrix}
	\]
	\[
	B = \left. \frac{\partial f}{\partial u} \right|_{x = x_3^*,\hspace{1mm} u = 0} = \begin{bmatrix}
		1 & 0 \\
		0 & 3
	\end{bmatrix}
	\]

	Итак, можем синтезировать локально стабилизирующий регулятор, минимизирующий функционал качества при $Q = 10 I$ и $R = I$:
	\[
		J = \int_0^{+ \infty} (x^T Q x + u^T R u) dx
	\]

	Получаем матрицу обратной связи:
	\[
		K_{x_3^*} = \begin{bmatrix}
			8.7854 & 1.0331 \\
 			3.0992 & 1.1193
		\end{bmatrix} \Rightarrow u = -K_{x_3^*} x_\delta
	\]

	Итак, можно промоделировать нелинейную систему при начальных условиях $x_\delta(0) = (0.1, 0.1)$ и $x_\delta(0) = (-0.1, -0.15)$ и синтезированном регуляторе c отрицательной обратной связью $K_{x_3^*}$. Все графики представлены на рисунках \ref{part3_3_u1} - \ref{part3_3_phase}.
	\begin{figure}
		\centering
		\includegraphics[width=0.8\textwidth]{images/part3_3_u_[0.1, 0.1].png}
		\caption{График управления системы при $x_\delta(0) = [0.1, 0.1]$}
		\label{part3_3_u1}
	\end{figure}
	\begin{figure}
		\centering
		\includegraphics[width=0.8\textwidth]{images/part3_3_[0.1, 0.1].png}
		\caption{График движения системы при $x_\delta(0) = [0.1, 0.1]$}
		\label{part3_3_x1}
	\end{figure}
	\begin{figure}
		\centering
		\includegraphics[width=0.8\textwidth]{images/part3_3_u_[-0.1, -0.15].png}
		\caption{График управления системы при $x_\delta(0) = [-0.1, -0.15]$}
		\label{part3_3_u2}
	\end{figure}
	\begin{figure}
		\centering
		\includegraphics[width=0.8\textwidth]{images/part3_3_[-0.1, -0.15].png}
		\caption{График движения системы при $x_\delta(0) = [-0.1, -0.15]$}
		\label{part3_3_x2}
	\end{figure}
	\begin{figure}
		\centering
		\includegraphics[width=0.8\textwidth]{images/part3_3_phase.png}
		\caption{Фазовый ортрет системы при обратной связе $-K_{x_3^*}$}
		\label{part3_3_phase}
	\end{figure}
	
	По рисункам видим, что система успешно стабилизируется вблизи точки неустойчивой точки равновесия $x_3^*$, а значит, найденный регулятор успешно справляется со своей задачей.

	\subsection{Вторая система}
	Итак, возьмем заключительную нелинейную систему:
	\[
		\begin{cases}
			\dot{x}_1 = x_2 + x_1x_2 + u^3\\
			\dot{x}_2 = -x_2 + x_2^2 - x_1^3 + \sin u
		\end{cases}
	\]

	Примем нулевое воздействие и найдем изолированные точки равновесия системы. Для этого приравняем производные к нулю:
	\[
		\begin{cases}
			x_2 + x_1x_2 = 0\\
			-x_2 + x_2^2 - x_1^3 = 0
		\end{cases}
	\]

	Преобразуем первое уравнение:
	\[
	x_2 + x_1x_2 = x_2(1 + x_1) = 0
	\]
	
	Отсюда:
	\[
	x_2 = 0 \quad \text{или} \quad x_1 = -1
	\]

	Подставим $x_2 = 0$ во второе уравнение:
	\[
	x_1^3 = 0 \Rightarrow x_1 = 0
	\]

	Теперь перейдем к \(x_1 = -1\):
	Подставляем во второе уравнение:
	\[
	-x_2 + x_2^2 - (-1)^3 = 0 \Rightarrow -x_2 + x_2^2 + 1 = 0
	\]
	\[
	x_2^2 - x_2 + 1 = 0
	\]
	
	У системы нет действительных корней, так что единственной изолированной точкой равновесия будет:
	\[
		x_1^* = (0, 0)
	\]

	Как прежде, сперва найдем линеаризацию:
	\[
		x = Ax + Bu
	\]
	
	Матрицы системы задаются через частные производные:
	\[
		A = \left. \frac{\partial f}{\partial x} \right|_{x = x_1^*,\hspace{1mm} u = 0} = 
		\begin{bmatrix}
		0 & 1 \\
		0 & -1
		\end{bmatrix}
	\]
	\[
	B = \left. \frac{\partial f}{\partial u} \right|_{x = x_1^*,\hspace{1mm} u = 0} = \begin{bmatrix}
		0 \\
		1
	\end{bmatrix}
	\]

	Аналогично предыдущему пункту синтезируем матрицу обратной связи через LQR при матрицах $Q = I$ и $R = 1$:
	\[
		K_{x_1^*} = \begin{bmatrix}
			1 & 1
		\end{bmatrix} \Rightarrow u = - K_{x_1^*} x
	\]

	Замоделируем изначальную нелинейную систему при начальных условиях $x(0) = (0.1, 0.1)$ и $x(0) = (-0.1, -0.15)$ и регуляторе c матрицей $K_{x_1^*}$. Все графики приведены на рисунках \ref{part3_4_u1} - \ref{part3_4_phase}.
	\begin{figure}
		\centering
		\includegraphics[width=0.8\textwidth]{images/part3_4_u_[0.1, 0.1].png}
		\caption{График управления системы при $x(0) = [0.1, 0.1]$}
		\label{part3_4_u1}
	\end{figure}
	\begin{figure}
		\centering
		\includegraphics[width=0.8\textwidth]{images/part3_4_[0.1, 0.1].png}
		\caption{График движения системы при $x(0) = [0.1, 0.1]$}
		\label{part3_4_x1}
	\end{figure}
	\begin{figure}
		\centering
		\includegraphics[width=0.8\textwidth]{images/part3_4_u_[-0.1, -0.15].png}
		\caption{График управления системы при $x(0) = [-0.1, -0.15]$}
		\label{part3_4_u2}
	\end{figure}
	\begin{figure}
		\centering
		\includegraphics[width=0.8\textwidth]{images/part3_4_[-0.1, -0.15].png}
		\caption{График движения системы при $x(0) = [-0.1, -0.15]$}
		\label{part3_4_x2}
	\end{figure}
	\begin{figure}
		\centering
		\includegraphics[width=0.8\textwidth]{images/part3_4_phase.png}
		\caption{Фазовый ортрет системы при обратной связе $-K_{x_1^*}$}
		\label{part3_4_phase}
	\end{figure}

	Таким образом, в относительно малой окрестности начала координат (при различных начальных условиях) имеем притяжение к точке равновесия $x_1^*$ - синтезированный регулятор справляется с поставленной задачей локальной стабилизации!
	
	\section{Выводы}
	В ходе лабораторной работы успешно исследованы нелинейные системы на наличие точек равновесия и их типы с помощью линеаризации через матрицу Якоби.

	Также были синтезированы локально стабилизирующие регуляторы, позволяющие изменить природу системы, сделав, например, неустойчивые узлы устойчивыми.





	

	


    
	
\end{document}