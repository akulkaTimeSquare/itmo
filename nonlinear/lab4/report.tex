\documentclass[a4paper,hidelinks,14pt]{extarticle}

\usepackage[utf8]{inputenc}
\usepackage[T2A]{fontenc}
\usepackage[english, russian]{babel}
\usepackage{lipsum}
\usepackage{amsmath}
\usepackage{amssymb}
\usepackage{amsfonts}
\usepackage{mathtools}
\usepackage{datetime}
\usepackage[pdftex]{graphicx}
\usepackage{indentfirst}
\usepackage{asymptote}
\usepackage{systeme}
\usepackage[dvipsnames]{xcolor}
\usepackage{lastpage}
\usepackage{fancybox,fancyhdr}
\usepackage{hyperref}
\usepackage[font={small,it}]{caption}
\fancyhead[L]{Лабораторная работа №4}
\fancyhead[C]{}
\fancyhead[R]{\textit{Метод бэкстеппинга}}
\fancyfoot[L]{}
\fancyfoot[C]{\thepage\space}
\fancyfoot[R]{}
\pagestyle{fancy}
\newcommand{\gt}{\textgreater}
\newcommand{\lt}{\textless}
\usepackage{listings}
\usepackage{xcolor}
\lstset{
    basicstyle=\ttfamily\small,
    keywordstyle=\color{blue},
    commentstyle=\color{gray},
    stringstyle=\color{red},
    numbers=left,
    numberstyle=\color[gray]{0.7}\ttfamily\small,
    stepnumber=1,
    numbersep=8pt,
    frame=single,
    showstringspaces=false,
    tabsize=4,
    breaklines=true
}
\usepackage{subcaption}

\begin{document}
	\begin{titlepage}
		\setlength{\parindent}{0ex}
		
		\begin{center}
			\textsc{
				\vspace{1ex}
                Научно-исследовательский университет ИТМО \\
				\vspace{0.5ex}
				Факультет систем управления и робототехники \\
				\vspace{0.5ex}
			}
		\end{center}
		
		\vspace{40mm}
		
		\begin{center}
			Отчет по лабораторной работе №4\\
			\textbf{Метод бэкстеппинга} \\
		\end{center}
		
		\vspace{50mm}
		
		\begin{minipage}{.4\linewidth}
			Выполнили студенты
            \\
			\\
            \\
			Преподаватель
		\end{minipage}
		\hfill
		\begin{minipage}{.55\linewidth}
			\begin{flushright}
				Мовчан Игорь Евгеньевич
                \\                
				Ибахаев Зубайр Руслан-Бекович
                \\
				Белоус Савва Эрнестович
				\\
				Зименко Константин Александрович
			\end{flushright}
		\end{minipage}
		
		\vfill
		\begin{center}
			Санкт-Петербург
			\\
			2025
		\end{center}
	\end{titlepage}

	\tableofcontents
	\clearpage
	
	\section{Первая система}
	Рассмотрим нелинейную систему
	\[
		\begin{cases}
			\dot{x}_1 = x_1^2 + \sin x_1 + x_2 \\
			\dot{x}_2 = x_1^2 + (2 + \sin x_1)u
		\end{cases}
	\]

	Предположим весь вектор состояния измеримым. Задачей поставим стабилизацию центра координат системы регулятором, синтезированным методом бэкстеппинга.

	Для начала стабилизируем подсистему
	\[
		\dot{x}_1 = x_1^2 + \sin x_1 + x_2
	\]

	Она достигается выбором в качестве виртуального выхода переменной состояния $x_2$ и закона управления
	\[
		a(x_1) = x_2 = - x_1^2 - \sin x_1 -k_1 x_1 
	\]

	Так как тогда функция Ляпунова $V_1 = x_1^2/2$ имеет производной
	\[
		\dot{V}_1 = -k_1 x_1^2 \leqslant 0
	\]

	Теперь введем ошибку на переменную $x_2$:
	\[
		z_2 = x_2 - a(x_1) \quad \Rightarrow \quad \dot{x}_1 = -k_1 x_1 + z_2
	\]
	
	При нулевой введенной ошибки $x_1 \to 0$. Для обеспечения этого расширим функцию Ляпунова до
	\[
		V_2 = \frac{x_1^2}{2} + \frac{z_2^2}{2}
	\]
	\[
		\dot{V}_2 = x_1 \dot{x}_1 + z_2 \dot{z}_2
	\]

	Распишем первое слагаемое:
	\[
		x_1 \dot{x}_1 = x_1 (-k_1 x_1 + z_2) = -k_1 x_1^2 + x_1 z_2
	\]

	А также второе:
	\[
		\dot{z}_2 = \dot{x}_2 - \dot{a} = x_1^2 + (2 + \sin x_1 )u - \dot{a}
	\]

	Откуда:
	\[
		\dot{V}_2 = -k_1 x_1^2 + x_1 z_2 + z_2 x_1^2 + z_2 (2 + \sin x_1)u - z_2 \dot{a} =
	\]
	\[
		= -k_1 x_1^2 + z_2 (x_1 + x_1^2 + (2 + \sin x_1)u - \dot{a})
	\]

	Хотим получить отрицательную производную, поэтому
	\[
		x_1 + x_1^2 + (2 + \sin x_1)u - \dot{a} = -k_2 z_2
	\]

	В итоге синтезированное управление:
	\[
		u = \frac{-x_1 - x_1^2  -k_2 z_2  + \dot{a}}{2 + \sin x_1}
	\]

	При этом производная от $a$ по времени и ошибка $z_2$:
	\[
		\dot{a}(x_1) = \frac{\partial a}{\partial x_1} \dot{x}_1 = (-2x_1 -\cos x_1 - k_1) (-k_1 x_1 + z_2)
	\]
	\[
		z_2 = x_2 + x_1^2 + \sin x_1 + k_1 x_1
	\]

	Отлично, стабилизирующее управление найдено! Причем центр координат получилось сделать \textit{глобально} устойчивым, так как расширенная функция Ляпунова $V_2$ радиально неограничена и имеет отрицательную (определенную) производную всюду, кроме 0.
	
	Наконец, проверим управление в действии, проведя моделирование при начальных условиях
	\[
		x(0) = \begin{bmatrix}
			2 & -1
		\end{bmatrix}^T
	\]
	
	и значениях параметров регулятора $k_1 = 1$, $k_2 = 2$. На рисунках \ref{fig:1}-\ref{fig:3} представлены графики состояний системы, ошибки $z_2$, а также синтезированного управляющего воздействия $u$.
	
	По результатам моделирования видим, что все переменные состояния и ошибка $z_2$ сходятся к нулю, а управляющее воздействие остаётся ограниченным. Это подтверждает асимптотическую устойчивость центра координат и корректность синтеза управления.
	\begin{figure}
		\centering
		\includegraphics[width=0.75\textwidth]{images/x_1.png}
		\caption{Графики состояний $x(t)$ нелинейной системы}
		\label{fig:1}
	\end{figure}
	\begin{figure}
		\centering
		\includegraphics[width=0.75\textwidth]{images/z2_1.png}
		\caption{График невязки $z_2 = x_2 - a(x_1)$ нелинейной системы}
		\label{fig:2}
	\end{figure}
	\begin{figure}
		\centering
		\includegraphics[width=0.75\textwidth]{images/u_1.png}
		\caption{Графики синтезированного управления $u$ нелинейной системы}
		\label{fig:3}
	\end{figure}

	\section{Вторая система}
	Аналогично проведем синтез стабилизирующего регулятора для нелинейной системы вида
	\[
		\begin{cases}
			\dot{x}_1 = - x_1^3 + x_2 \\
			\dot{x}_2 = x_1 + u
		\end{cases}
	\]

	Рассмотрим первую подсистему:
	\[
		\dot{x}_1 = -x_1^3 + x_2
	\]

	Будем считать переменную $x_2$ виртуальным управлением, стабилизирующим точку $x_1 = 0$. Здесь заметим, что при
	\[
		a(x_1) = x_2 = 0
	\]

	Система уже является устойчивой, однако нам всё же хочется иметь регулируемую динамику, поэтому проведем чуть более сложные вычисления для виртуального входа:
	\[
		a(x_1) = x_2 = x_1^3 - k_1 x_1 \quad \Rightarrow \quad \dot{V}_1 = -k_1 x_1^2 \leqslant 0
	\]

	В выводе аналогично предыдущему пункту было взято $V_1 = x_1^2/2$.

	Далее введем ошибку виртуального управления, которая со временем должна будет сойтись к нулю:
	\[
		z_2 = x_2 - a(x_1) \quad \Rightarrow \quad \dot{x}_1 = -k_1 x_1 + z_2
	\]

	Для этого примем расширенную функцию Ляпунова $V_2$ и проаналазируем её производную по времени:
	\[
		V_2 = \frac{x_1^2}{2} + \frac{z_2^2}{2} \quad \Rightarrow \quad \dot{V}_2 = x_1 \dot{x}_1 + z_2 \dot{z}_2
	\]
	
	Первая часть:
	\[
		x_1 \dot{x}_1 = -k_1 x_1^2 + x_1 z_2
	\]

	Для ошибки $z_2$:
	\[
		\dot{z}_2 = \dot{x}_2 - \dot{a} = x_1 + u - \dot{a}
	\]

	В итоге полная функция Ляпунова:
	\[
		\dot{V}_2 = -k_1 x_1^2 + 2 x_1 z_2 + z_2 u - z_2 \dot{a} = -k_1 x_1^2 + z_2 (2 x_1 + u - \dot{a})
	\]

	Хотим получить всюду отрицательную производную, поэтому
	\[
		2 x_1 + u - \dot{a} = -k_2 z_2 \quad \Rightarrow \quad u = -2x_1 -k_2 z_2 + \dot{a}
	\]

	При этом производная от $a$ по времени и ошибка $z_2$ равны
	\[
		\dot{a}(x_1) = \frac{\partial a}{\partial x_1} \dot{x}_1 = (3 x_1^2 - k_1) (-k_1 x_1 + z_2)
	\]
	\[
		z_2 = x_2 - x_1^3 + k_1 x_1
	\]

	Также промоделируем замкнутую найденным регулятором систему при параметрах $k_1 = k_2 = 5$ и начальных условиях
	\[
		x(0) = \begin{bmatrix}
			3 & -4
		\end{bmatrix}^T
	\]

	Результаты отображены на рисунках \ref{fig:4}-\ref{fig:6}. На них можем видеть, что обе переменные состояния монотонно сходятся к 0, как и ошибка виртуального управления $z_2$ - управление корректно.

	\begin{figure}
		\centering
		\includegraphics[width=0.75\textwidth]{images/x_2.png}
		\caption{Графики состояний $x(t)$ второй системы}
		\label{fig:4}
	\end{figure}
	\begin{figure}
		\centering
		\includegraphics[width=0.75\textwidth]{images/z2_2.png}
		\caption{График невязки $z_2 = x_2 - a(x_1)$ второй системы}
		\label{fig:5}
	\end{figure}
	\begin{figure}
		\centering
		\includegraphics[width=0.75\textwidth]{images/u_2.png}
		\caption{Графики синтезированного управления $u(t)$ второй системы}
		\label{fig:6}
	\end{figure}

	\section{Третья система}
	Наконец, рассмотрим нелинейную систему четвертого порядка
	\[
		\begin{cases}
			\dot{x}_1 = \cos x_1 - x_2 \\
			\dot{x}_2 = x_1 + x_3 \\
			\dot{x}_3 = x_1 x_3 + (2 - \sin x_3) x_4 \\
			\dot{x}_4 = x_2 x_3 + 2u
		\end{cases}
	\]

	Она, как и прежде, имеет сторого треугольную структуру (каждая подсистема $\dot{x}_{i}$ зависит от $x_{i+1}$ линейно с ненулевым коэффициентом, и не зависит от состояний более высокого порядка) - бэкстеппинг можно применить напрямую.

	Начнём со стабилизации подсистемы для $x_1$:
	\[
		\dot{x}_1 = \cos x_1 - x_2
	\]

	Выберем в качестве виртуального управления
	\[
		a_1(x_1) = x_2 = \cos x_1 + k_1 x_1
	\]

	Тогда для $V_1 = x^2_1/2$:
	\[
		\dot{x}_1 = -k_1 x_1 \quad \Rightarrow \quad \dot{V}_1 = -k_1 x_1^2 \leqslant 0
	\]

	Что даёт глобальную асимптотическую устойчивость в $x_1 = 0$. Далее, выберем ошибкой управления
	\[
		z_2 = x_2 - a_1(x_1)
	\]

	Из этого получаем:
	\[
		\dot{x}_1 = -k_1 x_1 - z_2
	\]

	Для гарантии нулевой ошибки расширим функцию Ляпунова до
	\[
		V_2 = \frac{x_1^2}{2} + \frac{z_2^2}{2}
	\]

	Её производная:
	\[
		\dot{V}_2 = -k_1 x_1^2 - x_1 z_2 + z_2 (\dot{x}_2 - \dot{a}_1) = -k_1 x_1^2 + z_2(x_3 - \dot{a}_1)
	\]

	Мы хотим:
	\[
		x_3 - \dot{a}_1 = -k_2 z_2
	\]

	Поэтому стабилизирующее виртуальное управление $x_3$ для второй подсистемы имеет вид:
	\[
		a_2(x_1, z_2) = x_3 = \dot{a}_1 - k_2 z_2
	\]

	Введем новую ошибку на управление:
	\[
		z_3 = x_3 - a_2 \quad \Rightarrow \quad x_3 = z_3 + a_2
	\]

	Опять же расширим функцию Ляпунова до $V_3$, для краткости используем рекурсивное определение через предыдущее значение $V_2$:
	\[
		V_3 = V_2 + \frac{z_3^2}{2}
	\]
	\[
		\dot{V}_2 = -k_1 x_1^2 + z_2 (z_3 + a_2 -\dot{a}_1) = -k_1 x_1^2 -k_2 z_2^2 + z_2 z_3
	\]

	Её производная:
	\[
		\dot{V}_3 = \dot{V}_2 + z_3 \dot{z}_3 = -k_1 x_1^2 - k_2 z_2^2 + z_2 z_3 + z_3(\dot{x}_3 -\dot{a}_2)
	\]

	Распишем отдельно выражние для динамики $z_3$:
	\[
		\dot{z}_3 = \dot{x}_3 -\dot{a}_2 = x_1 x_3 + (2 - \sin x_3) x_4 - \dot{a}_2
	\]

	Хотим получить:
	\[
		z_2 z_3 + z_3(x_1 x_3 + (2 - \sin x_3) x_4 - \dot{a}_2) = -k_3 z_3^3
	\]

	Поэтому в качестве стабилизирующего виртуального управления третьей подсистемы выберем
	\[
		(2 - \sin x_3) x_4 = -x_1 x_3 + \dot{a}_2 + z_2
	\]
	\[
		x_4 = \frac{-x_1 x_3 + \dot{a}_2 - z_2 - k_3 z_3}{2 - \sin x_3}
	\]
	
	Снова введем ошибку управления
	\[
		z_4 = x_4 - a_3
	\]

	Финальное расширим функцию Ляпунова до
	\[
		V_4 = V_3 + \frac{z_4^2}{2}
	\]
	\[
		\dot{V}_3 = -k_1 x_1^2 -k_2 z_2^2 -k_3 z_3^2 + z_3 z_4
	\]

	Её производная:
	\[
		\dot{V}_4 = -k_1 x_1^2 -k_2 z_2^2 -k_3 z_3^2 + z_3 z_4 + z_4 \dot{z}_4
	\]

	Распишем динамику для
	\[
		\dot{z}_4 = \dot{x}_4 - \dot{a}_3 = x_2 x_3 + 2u - a_3
	\]

	Хотим получить отрицательную производную с дополнительным слагаемым $-k_4 x_4^2$ в функции Ляпунова $V_4$, поэтому
	\[
		z_3 z_4 + z_4 (x_2 x_3 + 2u - \dot{a}_3) = - k_4 z_4^2
	\]

	Отсюда финальное выражение
	\[
		u = \frac{-x_2 x_3 + \dot{a}_3 - z_3 - k_4 z_4}{2}
	\]
	
	Моделирование переходных процессов при начальном состоянии
	\[
		x(0) = \begin{bmatrix}
			2 & -1 & 1 & -2
		\end{bmatrix}
	\]

	и параметрах регулятора $k_1 = 2$, $k_2 = 3$, $k_3 = 4$ и $k_4 = 5$ приведено на рисунках \ref{fig:7}-\ref{fig:9}. Можем видеть, что регулятор сводит все ошибки управления к нулю, как и состояния системы $x_i$ при $i \neq 3$. $x_2$ сходится к единице по причине устройства динамики системы: при $x_1 \to 0$ стабилизирующее управление $a_1(x_1) \to 1$.
	\begin{figure}[h]
		\centering
		\includegraphics[width=0.74\textwidth]{images/x_3.png}
		\caption{Графики состояний $x(t)$ системы четвертого порядка}
		\label{fig:7}
	\end{figure}

	\begin{figure}
		\centering
		\includegraphics[width=0.75\textwidth]{images/z_3.png}
		\caption{Графики невязок $z_i = x - a_i$ системы четвертого порядка}
		\label{fig:8}
	\end{figure}
	\begin{figure}
		\centering
		\includegraphics[width=0.75\textwidth]{images/u_3.png}
		\caption{Графики управления $u(t)$ системы четвертого порядка}
		\label{fig:9}
	\end{figure}

	Также нулевая точка является в целом асимптотически недостижимой, так как тогда должно выполняться одновременное стремление $x_1 \to 0$ и $x_1 \sim t \to \infty$, что, конечно же, противоречиво.

	Отметим также, что точка $x' = \begin{bmatrix}
		0 & 1 & 0 & 0
	\end{bmatrix}^T$ является естественной точкой равновесия, а синтезированное вокруг неё стабилизирующее управление корректно работает \textit{глобально}, так как производная функции Ляпунова $V_4$ отрицательно определена для любых точек, кроме $x'$. Синтез регулятора проведен корректно.

	\section{Выводы}
	В ходе лабораторной работы метод бэкстеппинга успешно применён для синтеза стабилизирующих регуляторов трёх нелинейных систем — от второго до четвёртого порядка. Для каждой системы построена рекурсивная функция Ляпунова и получен закон управления, обеспечивающий асимптотическую устойчивость (во всех случаях — глобальную). Результаты моделирования подтвердили корректность синтеза: ошибки управления сходятся к нулю, а состояния к требуемым (достижимым) точкам равновесия, управляющее воздействие остаётся ограниченным.






	
	


\end{document}