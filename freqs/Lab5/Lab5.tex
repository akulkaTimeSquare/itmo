\documentclass[a4paper,hidelinks,14pt]{extarticle}

\usepackage[utf8]{inputenc}
\usepackage[T2A]{fontenc}
\usepackage[english, russian]{babel}
\usepackage{lipsum}
\usepackage{amsmath}
\usepackage{amssymb}
\usepackage{amsfonts}
\usepackage{mathtools}
\usepackage{datetime}
\usepackage[pdftex]{graphicx}
\usepackage{indentfirst}
\usepackage{asymptote}
\usepackage{systeme}
\usepackage[dvipsnames]{xcolor}
\usepackage{lastpage}
\usepackage{fancybox,fancyhdr}
\usepackage{hyperref}
\usepackage[numbered,framed]{matlab-prettifier}
\usepackage[font={small,it}]{caption}
\fancyhead[L]{Лабораторная работа №5}
\fancyhead[C]{}
\fancyhead[R]{\textit{Cвязь непрерывного и дискретного}}
\fancyfoot[L]{}
\fancyfoot[C]{Страница \thepage\space из \pageref{LastPage}}
\fancyfoot[R]{}
\pagestyle{fancy}
\newcommand{\gt}{\textgreater}
\newcommand{\lt}{\textless}
\usepackage[framed,autolinebreaks,numbered,useliterate]{mcode}

\begin{document}
	\begin{titlepage}
		\setlength{\parindent}{0ex}
		
		\begin{center}
			\textsc{
				\vspace{1ex}
				Научно исследовательский университет ИТМО \\
				\vspace{0.5ex}
				Факультет систем управления и робототехники \\
				\vspace{0.5ex}
			}
		\end{center}
		
		\vspace{50mm}
		
		\begin{center}
			Отчет по лабораторной работе №5 \\
			Cвязь непрерывного и дискретного
		\end{center}
		
		\vspace{50mm}
		
		\begin{minipage}{.48\linewidth}
			Выполнил студент группы R3380
			
			Преподаватели
		\end{minipage}
		\hfill
		\begin{minipage}{.5\linewidth}
			\begin{flushright}
				Мовчан И.Е.
				\\
				Пашенко А.В., Перегудин А.А.
			\end{flushright}
		\end{minipage}
		
		\vfill
		\begin{center}
			Санкт-Петербург
			\\
			2025
		\end{center}
	\end{titlepage}

	\tableofcontents
	\clearpage
	
	\section{Разные преобразования Фурье}
	Прежде, чем приступить к выполнению задания, рассмотрим уже знакомую прямоугольную фунцию П $: \mathbb{R} \rightarrow \mathbb{R}$:
	$$
		\text{П}(t) =
		\begin{cases*}
			1, \ |t| \leq \frac{1}{2}, \\
			0, \ |t| > \frac{1}{2}.
		\end{cases*}	
	$$

	Её Фурье-образом будет являться аналитическое выражение
	$$
	\hat{\text{П}}(\nu) = \int \limits_{-\infty}^{+\infty} \text{П}(t) e^{-2\pi i \nu t} dt = \int \limits_{-1/2}^{1/2} e^{-2\pi i \nu t} dt = \frac{i e^{-2\pi i \nu t}}{2 \pi \nu} \bigg|_{-1/2}^{1/2} = \frac{\sin(\pi \nu)}{\pi \nu}.
	$$
	
	Данный результат уже был нам известен, так что не будем заострять на нём какое-то особое внимание (напомним лишь вид графиков описанных функций - рисунки \ref{1} и \ref{2}).
	\begin{figure}[h]
		\centering
		\includegraphics[scale=0.175]{./images/1.jpg}
		\caption{График функции П$(t)$}
		\label{1}
	\end{figure}
	\begin{figure}
		\centering
		\includegraphics[scale=0.175]{./images/2.jpg}
		\caption{График функции $\hat{\text{П}}(\nu)$}
		\label{2}
	\end{figure}
	
	\subsection{Численное интегрирование}
	
	Прямоугольная волна - одна из самых базовых и фундаментальных функций в теории сигналов, в то же время имеющая достаточно понятный образ Фурье, не вызывающий никаких вопросов. Но что если перед нами будет стоять задача вычисления более сложных объектов? Считать всё вручную - слишком мутная затея, которая в большинстве ситуаций не окупится.

	По этой причине люди прибегают к различных способам оптимизации, одним из которых является численное интегрирование. Давайте переложим долгую и нудную задачу вычисления известных интегралов Фурье преобразования на компьютер, а сами посмотрим, что же в итоге получится.

	Результаты такой работы при промежутке интегрирования по времени $T = 2$, шаге дискретизации $\Delta t = 0.025$, промежутке интегрирования по частоте $V = 10$ и шаге частот $\Delta \nu = 0.1$ приведены на рисунках \ref{3} и \ref{4}.
	\begin{figure}
		\centering
		\includegraphics[scale=0.175]{./images/4.jpg}
		\caption{Восстановление при $T = 2$, $\Delta t = 0.025$, $V = 10$ и $\Delta \nu = 0.1$}
		\label{4}
	\end{figure}
	\begin{figure}
		\centering
		\includegraphics[scale=0.175]{./images/3.jpg}
		\caption{Численный Фурье-образ при $T = 2$, $\Delta t = 0.025$, $V = 10$ и $\Delta \nu = 0.1$}
		\label{3}
	\end{figure}


	Что ж, итоги точно нас не воодушевляют: Фурье-образ явно отличается от истинного, восстановленной функции очень не достаёт плавности, а колебания, которые в ней присутствуют, ещё больше отдаляют нас от исходного сигнала. Посмотрим, можно ли каким-то образом улучшить возникшую ситуацию - для этого проварьируем параметры и исследуем их влияние на получаемые функции.

	Начнём с шага дискретизации $\Delta t$. Он напрямую влияет, с какой точностью строятся графики во временной области (соответственно, чем мельче шаг, тем более гладкими они выходят). Также мелкость $\Delta t$ влияет на ширину правильно находимых амплитуд в частотной области, большие значения $\Delta t$ дают весомые расхождения численно вычисленного образа Фурье от аналитического на высоких частотах. Визуализирующие эти выводы графики при значении $\Delta t = 0.075$ изображены на рисунках \ref{5} и \ref{6} (<<зигзагообразная>> форма восстановленной функции, расхождения образов даже на низких частотах), при значении $\Delta t = 0.001$ - на рисунках \ref{7} и \ref{8} (гармоники сгладились, образ Фурье максимально приближен к исходному на данном $V$).
	\begin{figure}[h]
		\centering
		\includegraphics[scale=0.175]{./images/5.jpg}
		\caption{Восстановление при $T = 2$, $\Delta t = 0.075$, $V = 10$ и $\Delta \nu = 0.1$}
		\label{5}
	\end{figure}

	\begin{figure}
		\centering
		\includegraphics[scale=0.175]{./images/6.jpg}
		\caption{Численный Фурье-образ при $T = 2$, $\Delta t = 0.075$, $V = 10$ и $\Delta \nu = 0.1$}
		\label{6}
	\end{figure}
	\begin{figure}
		\centering
		\includegraphics[scale=0.175]{./images/7.jpg}
		\caption{Восстановление при $T = 2$, $\Delta t = 0.001$, $V = 10$ и $\Delta \nu = 0.1$}
		\label{7}
	\end{figure}
	\begin{figure}
		\centering
		\includegraphics[scale=0.175]{./images/8.jpg}
		\caption{Численный Фурье-образ при $T = 2$, $\Delta t = 0.001$, $V = 10$ и $\Delta \nu = 0.1$}
		\label{8}
	\end{figure}
	
	Рассмотрим далее влияние шага частот $\Delta \nu$. Результаты работы при параметре $\Delta \nu = 0.5$ представлены на рисунках \ref{9} и \ref{10}, при параметре $\Delta \nu = 0.001$ - на рисунках \ref{11} и \ref{12} (в обоих случаях мы расширили захватыемую временную область за счёт увеличения $T$ до 4, чтобы можно было явно отследить все происходящие изменения; остальные параметры при этом остались теми же). Как можно видеть, кратное уменьшение шага по частоте сильно сглаживает вычисляемый Фурье-образ, а также расширяет временную область, в которой исходный сигнал корректно восстанавливается (увеличение $\Delta \nu$ приводит к уменьшению периода восстанавливаемой функции - в итоге получаем расхождения на больших значениях времени; на самом деле похожее происходило и с парой $\Delta t$ и $V$, и в том числе поэтому на высоких частотах при большом шаге по времени выходили ошибки в сравнении с истинными значениями образа Фурье).
	\begin{figure}
		\centering
		\includegraphics[scale=0.175]{./images/9.jpg}
		\caption{Восстановление при $T = 4$, $\Delta t = 0.025$, $V = 10$ и $\Delta \nu = 0.5$}
		\label{9}
	\end{figure}
	\begin{figure}
		\centering
		\includegraphics[scale=0.175]{./images/10.jpg}
		\caption{Численный Фурье-образ при $T = 4$, $\Delta t = 0.025$, $V = 10$ и $\Delta \nu = 0.5$}
		\label{10}
	\end{figure}
	\begin{figure}
		\centering
		\includegraphics[scale=0.175]{./images/11.jpg}
		\caption{Восстановление при $T = 4$, $\Delta t = 0.025$, $V = 10$ и $\Delta \nu = 0.001$}
		\label{11}
	\end{figure}
	\begin{figure}
		\centering
		\includegraphics[scale=0.175]{./images/12.jpg}
		\caption{Численный Фурье-образ при $T = 4$, $\Delta t = 0.025$, $V = 10$ и $\Delta \nu = 0.001$}
		\label{12}
	\end{figure}

	Немаловажными параметрами являются и промежутки интегрирования по времени $T$ и частоте $V$, задающие пределы, в которых мы хотим работать с соответсвующими областями. Комбинации больших $T$ с относительно большими $\Delta \nu$, как мы видели, дают неверно восстановливаемые сигналы на заданном интервале. Похожее можно сказать и про $V$ с $\Delta t$ - результаты при $V = 4$ изображены на рисунках \ref{13} и \ref{14}, при $V = 100$ - на рисунках \ref{15} и \ref{16}. Чрезмерное сужение и расширение частотного отрезка к тому же приводят к неточностям в восстановленной функции, так как на выходе получаем либо недостаток <<вращений>> для приближения сигнала, либо их перебор за счёт повторений и увеличение значений функций (рисунок \ref{15}).
	
	В идеале нужно захватывать как можно большие <<куски>> по времени и частоте на верно вычисляемых промежутках, и в случае появления ломаных линий или других недочётах либо кратно делить шаги $\Delta t$ или $\Delta \nu$, либо сужать области, уменьшая параметры $T$ или $V$. На рисунках \ref{17} и \ref{18} как раз изображен случай хорошо восстанавливаемого сигнала при параметрах $T = 50$, $\Delta t = 0.001$, $V = 100$ и $\Delta \nu = 0.01$. Но и тут не обошлось без трудностей, ведь возле точек разрыва прямоугольной волны ($t = \pm 0.5$) восстанавливамый сигнал бешенно осциллирует, причем с весомой амплитудой.
	\begin{figure}[h]
		\centering
		\includegraphics[scale=0.175]{./images/13.jpg}
		\caption{Восстановление при $T = 2$, $\Delta t = 0.025$, $V = 4$ и $\Delta \nu = 0.1$}
		\label{13}
	\end{figure}

	\begin{figure}
		\centering
		\includegraphics[scale=0.175]{./images/14.jpg}
		\caption{Численный Фурье-образ при $T = 2$, $\Delta t = 0.025$, $V = 4$ и $\Delta \nu = 0.1$}
		\label{14}
	\end{figure}
	\begin{figure}
		\centering
		\includegraphics[scale=0.175]{./images/15.jpg}
		\caption{Восстановление при $T = 2$, $\Delta t = 0.025$, $V = 100$ и $\Delta \nu = 0.1$}
		\label{15}
	\end{figure}
	\begin{figure}
		\centering
		\includegraphics[scale=0.175]{./images/16.jpg}
		\caption{Численный Фурье-образ при $T = 2$, $\Delta t = 0.025$, $V = 100$ и $\Delta \nu = 0.1$}
		\label{16}
	\end{figure}
	\begin{figure}
		\centering
		\includegraphics[scale=0.175]{./images/17.jpg}
		\caption{П$(t)$ при $T = 50$, $\Delta t = 0.001$, $V = 100$, $\Delta \nu = 0.01$ (приближение)}
		\label{17}
	\end{figure}
	\begin{figure}
		\centering
		\includegraphics[scale=0.175]{./images/18.jpg}
		\caption{Численный Фурье-образ при $T = 50$, $\Delta t = 0.001$, $V = 100$ и $\Delta \nu = 0.01$}
		\label{18}
	\end{figure}

	Ко всем вышеперечисленным проблемам добавляется и ключевой недостаток метода - его медлительность. Даже не при самом большом количестве точек мы вынуждены ждать по несколько секунд, а то и более, чтобы получить хоть какие-то удовлетворительные результаты. Численное интегрирование - очень трудоёмкий процесс, обходящийся нам дорого по времени, поэтому возникает естественный вопрос - <<а можно ли быстрее?>>. Оказывается, можно!

	\subsection{Использование DFT и различия}
	
	\begin{figure}
		\centering
		\includegraphics[scale=0.175]{./images/19.jpg}
		\caption{Восстановление при DFT c $T = 4$, $\Delta t = 0.05$ (приближение)}
		\label{19}
	\end{figure}
	\begin{figure}
		\centering
		\includegraphics[scale=0.175]{./images/20.jpg}
		\caption{Фурье-образ при DFT с параметрами $T = 4$, $\Delta t = 0.05$}
		\label{20}
	\end{figure}

	Решить проблему с быстродействием и точностью восстанавливаемой функции помогает, как ни странно, \textit{быстрое преобразование Фурье} (БПФ или fft). Сразу проверим его в действии, установив значения параметраов промежутка по времени $T = 4$ и шага дискретизации $\Delta t = 0.05$, а также соответствующие им $V = 1/ \Delta  t = 20$ (промежуток по частоте) и $\Delta \nu = 1/T = 0.25$ (шаг по частоте). Результаты представлены на рисунках \ref{19} и \ref{20} (стоит отметить, что здесь и везде далее в пункте будет использоваться только преобразование \textit{унитарной} формы).
	
	По итогу прямоугольная функция, если сравнивать с предыдущим методом, используюищм численное интегрирование, восстанавливается очень даже неплохо, хотя и не без проблем, ведь на выходе получилось нечто трапецеидальное. А вот её образ Фурье мало чем напоминает своего аналитически вычисленного аналога: имеются жёсткие пики, различие по амплитудам и смена знаков получаемых значений на каждый шаг (из плюса в минус и обратно).
	
	Всё дело в том, что быстрое преобразование Фурье предполагает работу с дискретными, а не непрерывными, как было до этого, величинами, и по своей сути является ускоренной версией дискретного преобразования Фурье (DFT). Теперь мы не пытаемся приблизить истинные значения интегралов численными средствами (чем-то прерывистым) - мы напрямую работает с тем, что на самом деле записано на компьютере. Это другой подход, поэтому и вычисляемый Фурье-образ отличен от аналитического, ведь имеет иную природу.

	Вышесказанное объясняет и с первого взгляда странно введённую связь между параметрами ($V = 1/ \Delta  t$ и $\Delta \nu = 1/T$). В дискретном случае она явно зашита в формулах: чем шире промежуток, тем ниже шаг, и наоборот (обратное наблюдалось и при численном интегрировании, где мелкость шага по времени расширяла частотный отрезок, где найденное было хорошо приближено к реальному).

	Итак, раз мы получили что-то новое, давайте разберёмся, на что оно способно и как работает - исследуем влияния промежутка по времени $T$ и шага дискретизации $\Delta t$.

	\begin{figure}
		\centering
		\includegraphics[scale=0.175]{./images/21.jpg}
		\caption{Восстановление при DFT c $T = 4$, $\Delta t = 0.1$ (приближение)}
		\label{21}
	\end{figure}
	\begin{figure}
		\centering
		\includegraphics[scale=0.175]{./images/22.jpg}
		\caption{Фурье-образ при DFT с параметрами $T = 4$, $\Delta t = 0.1$}
		\label{22}
	\end{figure}
	\begin{figure}
		\centering
		\includegraphics[scale=0.175]{./images/23.jpg}
		\caption{Восстановление при DFT c $T = 4$, $\Delta t = 0.0075$ (приближение)}
		\label{23}
	\end{figure}
	\begin{figure}
		\centering
		\includegraphics[scale=0.175]{./images/24.jpg}
		\caption{Фурье-образ при DFT с параметрами $T = 4$, $\Delta t = 0.0075$}
		\label{24}
	\end{figure}
	\begin{figure}
		\centering
		\includegraphics[scale=0.175]{./images/25.jpg}
		\caption{Фурье-образ при DFT с параметрами $T = 4$, $\Delta t = 0.0075$}
		\label{25}
	\end{figure}

	Начнём с $\Delta t$. Результаты при параметре $\Delta t = 0.1$ ($V = 10$) - на рисунках \ref{21} и \ref{22}, при $\Delta t = 0.0075$ ($V = 133$)- на рисунках \ref{23}, \ref{24} и \ref{25} ($T$ при этом зафиксирован и равен $4$, откуда $\Delta \nu = 0.25$). Заметно, что мелкий шаг дискретизации сильно <<выпрямляет>> графики за счёт уплотнения временной области, резкие переходы становятся более различимыми. При уменьшении $\Delta t$ также увеличивается амплитуда образа Фурье и расширяется частотная область (за счёт соотношения $V = 1/\Delta t$), в которой он в целом рассматривается (и верно определяется, так как получаем  лучшее восстановление).

	Далее изучим роль промежутка по времени $T$. Графики при параметре $T = 3$ ($\Delta \nu = 0.33$) приведены на рисунках \ref{26} и \ref{27}, при $T = 30$ ($\Delta \nu = 0.033$) - на рисунках \ref{28}, \ref{29} и \ref{30} ($\Delta t = 0.05$ в обоих случаях, $V = 20$). Можно видеть, что временная область, в которой задаётся восстанавливаемая функция, при увеличении $T$ расширяется, однако точность при этом остаётся той же (сравнение рисунков \ref{26} и \ref{29} - они одинаковы). Образ Фурье же становится более частым (так как $\Delta \nu  = 1/T$), а его амплитуда снижается.
	\begin{figure}
		\centering
		\includegraphics[scale=0.175]{./images/26.jpg}
		\caption{Восстановление при DFT c параметрами $T = 3$, $\Delta t = 0.05$}
		\label{26}
	\end{figure}
	\begin{figure}
		\centering
		\includegraphics[scale=0.175]{./images/27.jpg}
		\caption{Фурье-образ при DFT с параметрами $T = 3$, $\Delta t = 0.05$}
		\label{27}
	\end{figure}
	\begin{figure}
		\centering
		\includegraphics[scale=0.175]{./images/28.jpg}
		\caption{Восстановление при DFT c параметрами $T = 30$, $\Delta t = 0.05$}
		\label{28}
	\end{figure}
	\begin{figure}
		\centering
		\includegraphics[scale=0.175]{./images/29.jpg}
		\caption{Восстановление при DFT c $T = 30$, $\Delta t = 0.05$ (приближение)}
		\label{29}
	\end{figure}
	\begin{figure}
		\centering
		\includegraphics[scale=0.175]{./images/30.jpg}
		\caption{Фурье-образ при DFT с параметрами $T = 30$, $\Delta t = 0.05$}
		\label{30}
	\end{figure}
	
	Таким образом, идеальное восстановление сочетает в себе большие промежутки и маленькие шаги дискретизации по времени и частоте. Пример результатов, когда всё работает чуть ли не идеально, приведён на рисунках \ref{31}-\ref{34}, где были выбраны параметры $T = 100$, $\Delta t = 0.001$ и соответствующие им $V = 1000$ и $\Delta \nu = 0.01$. 
	\begin{figure}
		\centering
		\includegraphics[scale=0.175]{./images/31.jpg}
		\caption{Восстановление при DFT c параметрами $T = 100$, $\Delta t = 0.001$}
		\label{31}
	\end{figure}
	\begin{figure}
		\centering
		\includegraphics[scale=0.175]{./images/32.jpg}
		\caption{Восстановление при DFT c $T = 100$, $\Delta t = 0.001$ (приближение)}
		\label{32}
	\end{figure}
	\begin{figure}
		\centering
		\includegraphics[scale=0.175]{./images/33.jpg}
		\caption{Фурье-образ при DFT с параметрами $T = 100$, $\Delta t = 0.001$}
		\label{33}
	\end{figure}
	\begin{figure}
		\centering
		\includegraphics[scale=0.175]{./images/34.jpg}
		\caption{Фурье-образ при DFT с $T = 100$, $\Delta t = 0.001$ (приближение)}
		\label{34}
	\end{figure}

	Отдельного внимания заслуживает быстрота метода. В сравнении с численным интегрированием она действительно космическая: теперь на каждом шаге не требуется вычисление громоздкого интеграла, уменьшается цена одной операции. Кроме того, как было сказано, БПФ само по себе сокращает число операций, так что мы выигрываем вдвойне.

	К тому же дискретное преобразование является более точным при восстановлении, так как мы работаем не с самой функцией, пытаясь приблизить её <<вращениями>>, то есть чем-то непрерывным (что нас сильно сковывает, так как мы не способны корректно передать тот же разрыв), а с отдельными точками, которые мы можем восстанавливать напрямую (и зачастую более точно), поэтому даже когда, например, имеются скачки, мы задаём их более лучшим образом, отчасти потому что они даже не воспринимаются таковыми.

	Однако быстрое преобразование всё ещё даёт серьёзные различия в образах (как было сказано, численное интегрирование при работе как раз стремится приблизить истинный Фурье-образ, заданный через интегралы, от этого у метода и выходит похоже, тогда как дискретное преобразование выдаёт абсолютно другой по своей сущности объект, который даже и не стремится что-либо приближать), что может быть важно. Посмотрим, можно ли это как-то исправить.
	
	\subsection{Приближение непрерывного с помощью DFT}
	
	Давайте попробуем соединить преимущества обоих методов: точность численного интегрирования в частотной области и быстродействие и четкость при восстановлении исходной функции быстрого преобразования Фурье. Для этого используем fft для приближения значений интеграла через суммы Римана
	$$
	\hat{f}(\nu) = \int_{-T/2}^{T/2} f(t) e^{-2\pi i \nu t} dt \approx \sum_{n=0}^{N-1} f(t_n) e^{-2\pi i \nu t_n} \Delta t,
	$$
	где $t_n = -\frac{T}{2} + n\Delta t$, $\Delta t = \frac{T}{N}$, $n = 0, 1, ..., N-1$.

	Заметим, что дискретное преобразование имеет вид
	$$
	F_m = \sum_{n = 0}^{N-1} f_n e^{-2 \pi i \frac{n m}{N}},\hspace{1mm} m = 0, 1, ..., N-1.
	$$

	Теперь положим $\nu_m = \frac{m}{T} = \frac{m}{N \Delta t}$, $t_n = -\frac{T}{2} + n\Delta t$. Откуда
	$$
	\hat{f}(\nu_m) \approx \sum_{n=0}^{N-1} f(t_n) e^{-2 \pi i \nu_m t_n} \Delta t = \Delta t \sum_{n=0}^{N-1} f(t_n) e^{-2 \pi i \frac{m}{N \Delta t} (n \Delta t - \frac{T}{2})} =
	$$
	$$
	 = \Delta t \cdot e^{\pi i m} \sum_{n=0}^{N-1} f(t_n)e^{-2 \pi i \frac{n m}{N}} = \Delta t \cdot e^{\pi i m} F_m = (-1)^m \Delta t F_m = c_m F_m.
	$$

	Выходит, мы ввели некоторые замены и свели нижние суммы Римана, которыми мы приближаем заданное непрерывное преобразование Фурье (так как ещё с математического анализа известно, что при бесконечно малом шаге они задают интегралы), к дискретному (DFT) с некоторым коэффициентом $c_m$. Из этих выкладок также следует, что при $\Delta t \rightarrow 0$ значение полученного метода в точке $\nu_m$ будет всё ближе идти к своему истинному, $\hat{f} (\nu_m)$:
	$$
	c_m F_m = \sum_{n=0}^{N-1} f(t_n) e^{-2 \pi i \nu_m t_n} \Delta t \xrightarrow[\Delta t \to 0]{} \int_{-T/2}^{T/2} f(t) e^{-2\pi i \nu_m t} dt = \hat{f}(\nu_m)
	$$

	Новый метод с введенными коэффициентами $c_m = (-1)^m \Delta t$ проводит масштабирование получаемых от DFT значений на постоянную $\Delta t$, а также устраняет уже упомянутые <<дерганья>> от плюса к минусу образа дискретного преобразования. Итоговая работа происходит по схеме
	$$
	{\cal F} \{f\} \rightarrow \textbf{fftshift(c.*fft(f))},
	$$
	$$
	{\cal F}^{-1} \{\hat{f}\} \rightarrow \textbf{ifft(ifftshift(hatf)\hspace{0.25mm}./c)}.
	$$
	
	Графики при параметрах $T = 4$, $dt = 0.05$ и соответствующих им $V = 1/dt = 20$ и $\Delta \nu = 1/T = 0.25$ изображены на рисунках \ref{35} и \ref{36}. Точность метода при восстановленнии функции по времени остаётся той же, что была и у DFT (из схемы выше ясно, что все изменения были внесены только в представление функций в частотной области, по сути же мы используем тот же метод БПФ, что и был до этого, ведь все введённые домножения на коэффициенты $c_m$, которые лишь проводят некоторую знакопеременную нормировку, снимаются последующим делением, а значит, при обратном преобразовании с помощью ifft берётся тот же образ, что и при дискретном случае), а вот у Фурье-образа заметны серьёзные улучшения: хоть всё ещё и видны отличия от аналитической функции (однако это зависит уже от промежутка по времени и шага дискретизации), но метод сгладил те неровности, которые существовали до этого, - пропала знакопеременность, исчезли различия по амплитудам.
	\begin{figure}
		\centering
		\includegraphics[scale=0.175]{./images/35.jpg}
		\caption{Восстановление улучшенным DFT c $T = 4$, $\Delta t = 0.05$ (приближение)}
		\label{35}
	\end{figure}
	\begin{figure}
		\centering
		\includegraphics[scale=0.175]{./images/36.jpg}
		\caption{Фурье-образ при улучшенном DFT с $T = 4$, $\Delta t = 0.05$}
		\label{36}
	\end{figure}
	
	Что ж, пора провести небольшие исследования. Восстановленная функция и образ Фурье при параметре $\Delta t = 0.1$ ($V = 10$) приведены на рисунках \ref{37} и \ref{38}, при $\Delta t = 0.01$ ($V = 100$) - на рисунках \ref{39} и \ref{40} ($T = 2$ и $\Delta \nu = 0.5$ в обоих случаях). Мелкость шага по прежнему уплотняет временную область, за счёт чего функции становятся более близкими к своим непрерывным (верным) аналогам. В частотной области уменьшение $\Delta t$, как и в случае с непрерывным преобразованием Фурье, расширяет отрезок, в котором хорошо вычисляются амплитудные значения (сравнение рисунков \ref{38} и \ref{41}; так как при стремлении шага дискретизации к 0, как мы выяснили, суммы Римана, которые и считаются, идут к истинным значениям интегралов), рассматриваемая $V = 1/\Delta t$ при этом увеличивается. 
	\begin{figure}
		\centering
		\includegraphics[scale=0.175]{./images/37.jpg}
		\caption{Восстановление при улучшенном DFT c $T = 2$, $\Delta t = 0.1$}
		\label{37}
	\end{figure}
	\begin{figure}
		\centering
		\includegraphics[scale=0.175]{./images/38.jpg}
		\caption{Фурье-образ при улучшенном DFT с $T = 2$, $\Delta t = 0.1$}
		\label{38}
	\end{figure}
	\begin{figure}
		\centering
		\includegraphics[scale=0.175]{./images/39.jpg}
		\caption{Восстановление при улучшенном DFT c $T = 2$, $\Delta t = 0.01$}
		\label{39}
	\end{figure}
	\begin{figure}
		\centering
		\includegraphics[scale=0.175]{./images/40.jpg}
		\caption{Фурье-образ при улучшенном DFT с $T = 2$, $\Delta t = 0.01$}
		\label{40}
	\end{figure}
	\begin{figure}
		\centering
		\includegraphics[scale=0.175]{./images/41.jpg}
		\caption{Фурье-образ при улучшенном DFT с $T = 2$, $\Delta t = 0.01$ (приближение)}
		\label{41}
	\end{figure}

	Рассмотрим теперь реакцию на изменение параметра промежутка времени $T$. Графики при значениях $T = 3$ изображены на рисунках \ref{42} и \ref{43}, при $T = 30$ по времени - на рисунках \ref{44}, \ref{45} и \ref{46}. Как можно видеть, рассматриваемый промежуток по времени прямо связан с гладкостью Фурье-образа (а значит, увеличение одного даёт другое), так как $\Delta \nu = 1/T$. При этом изменение $T$ практически не сказывается на качестве восстановления, кроме расширения временного отрезка, - если не поднять $\Delta t$, то область по прежнему останется разреженной и не удастся добиться должного качества восстановления, функции не придут к своему непрерывному виду на конкретном промежутке.
	\begin{figure}
		\centering
		\includegraphics[scale=0.175]{./images/42.jpg}
		\caption{Восстановление при улучшенном DFT c $T = 3$, $\Delta t = 0.1$}
		\label{42}
	\end{figure}
	\begin{figure}
		\centering
		\includegraphics[scale=0.175]{./images/43.jpg}
		\caption{Фурье-образ при улучшенном DFT с $T = 3$, $\Delta t = 0.1$}
		\label{43}
	\end{figure}
	\begin{figure}
		\centering
		\includegraphics[scale=0.175]{./images/44.jpg}
		\caption{Восстановление при улучшенном DFT c $T = 30$, $\Delta t = 0.1$}
		\label{44}
	\end{figure}
	\begin{figure}
		\centering
		\includegraphics[scale=0.175]{./images/45.jpg}
		\caption{П$(t)$ при улучшенном DFT c $T = 30$, $\Delta t = 0.1$ (приближение)}
		\label{45}
	\end{figure}
	\begin{figure}
		\centering
		\includegraphics[scale=0.175]{./images/46.jpg}
		\caption{Фурье-образ при улучшенном DFT с $T = 30$, $\Delta t = 0.1$}
		\label{46}
	\end{figure}

	Из всего сказанного можно сделать вывод - идеальная работа метода включает в себя расширение временного отрезка до необходимого уровня за счёт увеличения $T$, а также кратное деление шага времени до позволимой нормы (по времени и памяти). При $T = 100$, $\Delta t = 0.001$, $V = 1000$ и $\Delta \nu = 0.01$ достигается весомая точность как в Фурье-образе, так и в востановленной функции. На рисунках \ref{47}, \ref{48} показаны графики во временной области, на рисунках \ref{49}, \ref{50} - в частотной (всё при указанных выше параметрах).
	\begin{figure}
		\centering
		\includegraphics[scale=0.175]{./images/47.jpg}
		\caption{Восстановление при улучшенном DFT c $T = 100$, $\Delta t = 0.001$}
		\label{47}
	\end{figure}
	\begin{figure}
		\centering
		\includegraphics[scale=0.175]{./images/48.jpg}
		\caption{П$(t)$ при улучшенном DFT c $T = 100$, $\Delta t = 0.001$ (приближение)}
		\label{48}
	\end{figure}
	\begin{figure}
		\centering
		\includegraphics[scale=0.175]{./images/49.jpg}
		\caption{Фурье-образ при улучшенном DFT с $T = 100$, $\Delta t = 0.001$}
		\label{49}
	\end{figure}
	\begin{figure}
		\centering
		\includegraphics[scale=0.175]{./images/50.jpg}
		\caption{Фурье-образ с улучшенным DFT и $T = 100$, $\Delta t = 0.001$ (приближение)}
		\label{50}
	\end{figure}

	Сравним новый метод с предыдущими. Как уже было сказано, улучшенное fft всегда выигрывает в точности у обычного в приближении истинного образа Фурье, так как последний даже не пытается что-либо приближать, при этом остаётся та же точность по времени, опять-таки потому что все внесенные в новый метод изменения относятся только к частотной области. 
	
	Однако с численным интегрированием возникают серьёзные различия: новый метод проигрывает в точности Фурье-образов при плохой дискретизации (малых $\Delta t$ и $T$), ведь trapz использует трапецеидальное приближение значений итегралов, которое (известно из лабороторных по математическому анализу) даёт погрешность порядка $O(h^2)$, тогда как суммы Римана, которые мы и используем в улучшенном fft для приближения значений по частотам, даёт погрешности порядка $O(h)$, $h=\Delta \nu$ - шаг интегрирования. В восстановлении функции же дискретное преобразование явно выигрывает.

	Все слова выше про сравнения методов подтверждают и графики, представленные с параметрами $T = 1.15$, $\Delta t = 0.05$, $V = 20$ и $\Delta \nu \approx 0.87$ на рисунках \ref{51},  \ref{52} и \ref{53}, а с $T = 10$, $\Delta t = 0.001$, $V = 1000$ и $\Delta \nu = 0.1$ - на рисунках \ref{54} и \ref{55}. В первом случае плохая дискретизация, и метод численного интегрирования даёт более близкие значения образов Фурье (его общая сумма ошибок $\approx 0.21$, у улучшенного fft - $\approx 0.8$), но при лучшей ситуации (при меньшем $\Delta t$) всё уже примерно равно, новый метод даёт очень точные значения.

	Улучшенное fft также чуть менее быстрое относительно обычного быстрого преобразования Фурье, так как теперь необходимо считать и совершать операции с коэффициентами $c_m$, однако всё ещё молниеносно обрабатывает данные в сравнении с методом численного интегрирования, ведь использует все те же ускорения fft.
	\begin{figure}
		\centering
		\includegraphics[scale=0.175]{./images/51.jpg}
		\caption{Сравнение методов во временной области c $T = 1.15$, $\Delta t = 0.05$}
		\label{51}
	\end{figure}
	\begin{figure}
		\centering
		\includegraphics[scale=0.175]{./images/52.jpg}
		\caption{Сравнение методов в частотной области с $T = 1.15$, $\Delta t = 0.05$}
		\label{52}
	\end{figure}
	\begin{figure}
		\centering
		\includegraphics[scale=0.175]{./images/53.jpg}
		\caption{Сравнение Фурье-образов с $T = 1.15$, $\Delta t = 0.05$ (приближение)}
		\label{53}
	\end{figure}

	\begin{figure}
		\centering
		\includegraphics[scale=0.175]{./images/54.jpg}
		\caption{Сравнение методов во временной области c $T = 10$, $\Delta t = 0.001$}
		\label{54}
	\end{figure}
	\begin{figure}
		\centering
		\includegraphics[scale=0.175]{./images/55.jpg}
		\caption{Сравнение Фурье-образов с $T = 10$, $\Delta t = 0.001$ (приближение)}
		\label{55}
	\end{figure}

	Важно отметить и то, что БПФ при исследовании гармоник будет давать те же результаты (причем за скорое время), что и методы выше, так как в этом случае Фурье-образ - набор дельта-функций, локализованных в определённых точках, соотвественно, нет смысла в нормиках (бесконечно большие значения в теории), а также в исследованиях на всём промежутке.

	Таким образом, мы получили метод получения Фурье-образа и восстановления исходной функции, работающий очень быстро и дающий точные результаты при больших $T$, увеличивающих, кроме рассматриваемого промежутка по времени, ещё и гладкость образа, и малых $\Delta t$, делающих представление дискретизированного сигнала схожим с непрерывным и расширяющих частотную область, в которой все значения амплитуд вычисляются более точно.

	\section{Сэмплирование}
	Попробуем изученную на лекциях теорему Найквиста-Шеннона-Котельникова для восстановления непрерывного сигнала по его дискретизации в действии. Для этого зададимся параметрами $a_1 = 1$, $a_2 = 0.5$, $w_1 = 10 \pi = 2\pi \cdot 5 = 5$ Гц, $w_2 = 20 \pi = 2\pi\cdot10 = 10$ Гц, $\varphi_1 = 0$, $\varphi_2 = \pi/4$ и $b = 4\pi$ и рассмотрим функции $y_1(t)$ и $y_2(t)$:
	\[
		y_1(t) = a_1 \sin(\omega_1 t + \varphi_1) + a_2 \sin(\omega_2 t + \varphi_2), \quad 
		y_2(t) = \mathrm{sinc}(bt);
	\]
	\[
		y_1(t) = \sin(10 \pi t) + 0.5 \sin(20 \pi t + \pi/4), \quad 
		y_2(t) = \mathrm{sinc}(20t).
	\]
	\begin{figure}
		\centering
		\includegraphics[scale=0.175]{./images/56.jpg}
		\caption{График функции $y_1(t)$}
		\label{56}
	\end{figure}
	\begin{figure}
		\centering
		\includegraphics[scale=0.175]{./images/57.jpg}
		\caption{График функции $y_2(t)$}
		\label{57}
	\end{figure}

	Их графики изображены на рисунках \ref{56} и \ref{57} соотвественно.
	\subsection{Гармоники}
	Исследуем каждую функцию $y_i(t)$ по отдельности. Начнём с $y_1(t)$, как видим она представляет из себя сумму гармоник, двух синусоид разной частоты, амплитуды и начальной фазы, а её образ Фурье полностью содержится на отрезке (рисунок \ref{58}).
	\begin{figure}
		\centering
		\includegraphics[scale=0.175]{./images/58.jpg}
		\caption{Образ Фурье функции $y_1(t)$}
		\label{58}
	\end{figure}

	Теорема Найквиста-Шеннона-Котельникова говорит о том, что можно полностью восстановить изначально непрерывную функцию $f(t)$ по её дискретизации по формуле, если образ Фурье $\hat{f}(\nu)$ полностью содержится в отрезке $[-B/2; B/2]$, а шаг дискретизации $\Delta t < 1/B$, $B$ - промежуток нитегрирования по частоте:
	$$
	f(t) = \sum_{n=-\infty}^{+\infty} f(t_n) \mathrm{sinc}\Big( \frac{\pi}{\Delta t}(t - t_n) \Big).
	$$

	Но так как мы физически не способны на бесконечное число интераций на компьютере, то рассматривается определённый конечный промежуток по времени $T$, на котором восстанавливается и дискретизируется с шагом $\Delta t$ функция $f(t)$, откуда имеем:
	$$
	f(t) = \sum_{n=0}^{N} f(t_n) \mathrm{sinc}\Big( \frac{\pi}{\Delta t}(t - t_n) \Big), 
	$$
	где $t_n = -T/2 + \Delta t/N$, $\Delta t = T/N < 1/B$.

	\begin{figure}
		\centering
		\includegraphics[scale=0.175]{./images/59.jpg}
		\caption{Теорема Котельникова при $T = 0.5$, $\Delta t = 1/30$ (время)}
		\label{59}
	\end{figure}
	\begin{figure}
		\centering
		\includegraphics[scale=0.175]{./images/60.jpg}
		\caption{Теорема Котельникова при $T = 0.5$, $\Delta t = 1/30$ (частоты)}
		\label{60}
	\end{figure}
	\begin{figure}
		\centering
		\includegraphics[scale=0.175]{./images/61.jpg}
		\caption{Использование теоремы с $T = 0.5$, $\Delta t = 1/30$ (частоты, приближение)}
		\label{61}
	\end{figure}
	Что ж, проверим теорему на деле. Зададимся параметрами рассматриваемого промежутка по времени $T = 0.5$ и шага дискретизации $\Delta t = 1/(B+10) = 1/30$, так как $B=20$ - удвоенная максимальная частота ненулевой амплитуды, содержащейся в образе Фурье $y_2(t)$, тем самым создадим сэмплированный сигнал ($V = 30$, а $\Delta \nu = 2$). Используя приведенную выше формулу, восстановим исходный сигнал через интерполяцию. На рисунках \ref{59}, \ref{60} и \ref{61} изображены графики исходного сигнала, интерполяции и сэмплирования во временной области, а также их Фурье-образы, строящиеся с помощью улучшенного fft. Получаем чуть ли не идеальное сходство восстановления с исходной функцией, однако в спектре заметны проблемы - образ интерполяции хоть и имеет некоторое сходство с оригиналом (дельта-выступы в $\nu = \pm 10$), но всё ещё ошибается, не давая, например, нужные пики в $\nu = \pm 5$ (что связано, в том числе с большими значениями $\Delta \nu = 1/T$). Уменьшение частоты дискретизации при этом даёт, как и подразумевается, неточности в восстановлении исходной функции (возникают искажения за счёт явления алиасинга - наложения высоких частот на низкие, рисунок \ref{64}; частотного промежутка $V = 1/\Delta t <B$ как бы становится недостаточно, чтобы покрыть все необходимые значения, верно применив теорему) - демонстрирующие данное графики при параметре $\Delta t=1/15$ показаны на рисунках \ref{62}, \ref{63} и \ref{64} (при этом $T = 0.5$, $V = 15$, а $\Delta \nu = 2$).
	\begin{figure}[h]
		\centering
		\includegraphics[scale=0.175]{./images/62.jpg}
		\caption{Теорема Котельникова при $T = 0.5$, $\Delta t = 1/15$ (время)}
		\label{62}
	\end{figure}
	\begin{figure}
		\centering
		\includegraphics[scale=0.175]{./images/63.jpg}
		\caption{Теорема Котельникова при $T = 0.5$, $\Delta t = 1/15$ (частоты)}
		\label{63}
	\end{figure}
	\begin{figure}
		\centering
		\includegraphics[scale=0.175]{./images/64.jpg}
		\caption{Использование теоремы с $T = 0.5$, $\Delta t = 1/15$ (частоты, приближение)}
		\label{64}
	\end{figure}

	Промежуток по времени $T$ задаёт отрезок времени, в котором реализуется дискретизация и интерполяция. Соотвественно, выбирается тот промежуток, в котором хочется проводить исследования и видеть итоги. Увеличение параметра за счёт связи $\Delta \nu = 1/T$ также уплотняет частотную область, приближая графики к своему истинному виду и сглаживая их. Результаты восстановления при $T = 5$ и $\Delta t = 1/30$ ($V = 30$, $\Delta \nu = 0.1$) представлены на рисунке \ref{65}, Фурье-образы - на рисунках \ref{66} и \ref{67}. 
	\begin{figure}
		\centering
		\includegraphics[scale=0.175]{./images/65.jpg}
		\caption{Теорема Котельникова при $T = 5$, $\Delta t = 1/30$ (время)}
		\label{65}
	\end{figure}
	\begin{figure}
		\centering
		\includegraphics[scale=0.175]{./images/66.jpg}
		\caption{Теорема Котельникова при $T = 5$, $\Delta t = 1/30$ (частоты)}
		\label{66}
	\end{figure}
	\begin{figure}
		\centering
		\includegraphics[scale=0.175]{./images/67.jpg}
		\caption{Использование теоремы с $T = 5$, $\Delta t = 1/30$ (частоты, приближение)}
		\label{67}
	\end{figure}
	\begin{figure}
		\centering
		\includegraphics[scale=0.175]{./images/68.jpg}
		\caption{Образ Фурье функции $y_2(t)$}
		\label{68}
	\end{figure}

	\subsection{Кардинальный синус}
	Проверим теперь работу уже рассмотренной теоремы на функции кардинального синуса $y_2 (t) = \mathrm{sinc}(b t) = \frac{\sin(b t)}{b t}$ (её образ Фурье изображен на рисунке \ref{68}; как видим, он полностью содержится в отрезке $[-B/2, B/2] = [-2, 2]$, остальное - зануляется).

	Зададимся шагом дискретизации $\Delta t = 1/10$ и промежутком интегрирования $T = 5$ (им соответствуют $V = 10$ и $\Delta \nu = 0.2$), тем самым проведем сэмплироване исходного сигнала. Далее попытаемся его восстановить. По теореме Котельникова при шаге $\Delta t < 1/B = 1/4$, сигнал можно восстановить с полной точностью, что мы и наблюдаем на рисунке \ref{69}, где представлено само восстановление, и \ref{70}, где видны образы Фурье всех используемых функций. Увеличение шага, как и ожидается, нарушает условия теоремы, и образы верной интерполяции не происходит (демонстрирующие рисунки \ref{71} и \ref{72} при параметрах $T = 5$, $\Delta t = 1/3$, $V = 3$ и $\Delta \nu = 0.2$). Причем заметно, что при малой частоте дискретизации восстановленная функция и Фурье-образы интерполяции дают всё тот же кардинальный синус, только большей амплитуды и меньшей частоты. Всё дело в том, что исследуемый метод работает схожим образом с обратным преобразованием Фурье - он берёт спектр сэмплированного сигнала, урезает его до нужного отрезка частот $[-B/2, B/2]$, в котором и хранится вся информация исходного сигнала, а после проводит восстановление; а так как в данном случае на отрезке наблюдается вся та же прямоугольная функция, только чуть иной длины, то и интерполяция пытается приблизить не исходный кардинальный синус, а его <<сородича>>, тоже $\mathrm{sinc}$, имеющего Фурье образ, содержащийся в промежутке $[-V/2, V/2]$.
	
	Параметр $T$, как и в предыдущих примерах, задаёт используемый промежуток времени и сглаживает спектры (демонстрирующие рисунки \ref{73} и \ref{74} с $T = 15$, $\Delta t = 1/10$, $V = 10$ и $\Delta \nu = 1/15$).
	\begin{figure}
		\centering
		\includegraphics[scale=0.175]{./images/69.jpg}
		\caption{Теорема Котельникова при $T = 5$, $\Delta t = 1/10$ (время)}
		\label{69}
	\end{figure}
	\begin{figure}
		\centering
		\includegraphics[scale=0.175]{./images/70.jpg}
		\caption{Теорема Котельникова при $T = 5$, $\Delta t = 1/10$ (частоты)}
		\label{70}
	\end{figure}
	\begin{figure}
		\centering
		\includegraphics[scale=0.175]{./images/71.jpg}
		\caption{Теорема Котельникова при $T = 5$, $\Delta t = 1/3$ (время)}
		\label{71}
	\end{figure}
	\begin{figure}
		\centering
		\includegraphics[scale=0.175]{./images/72.jpg}
		\caption{Теорема Котельникова при $T = 5$, $\Delta t = 1/3$ (частоты)}
		\label{72}
	\end{figure}
	\begin{figure}
		\centering
		\includegraphics[scale=0.175]{./images/73.jpg}
		\caption{Теорема Котельникова при $T = 15$, $\Delta t = 1/10$ (время)}
		\label{73}
	\end{figure}
	\begin{figure}
		\centering
		\includegraphics[scale=0.175]{./images/74.jpg}
		\caption{Теорема Котельникова при $T = 15$, $\Delta t = 1/10$ (частоты)}
		\label{74}
	\end{figure}

	Таким образом, теорема Найквиста-Шеннона-Котельникова работает, с её помощью мы можем восстанавливать непрерывный сигнал по его дискретизации со сколь угодно большой точностью при тех условиях, что образ функции задан на определённом промежутке по частотам $[-B/2, B/2]$, а шаг дискретизации сэмплированного сигнала $\Delta t < 1/B$. Однако стоит сказать и то, что эти условия весьма жёсткие: мало для каких функций существует ограниченный спектр, а значит, его уже придётся обрезать, даже если Фурье-образ достаточно хорошо убывает, что уже рушит нашу идеализацию. Кроме того, сэмплирования с бесконечной точностью не существует - с помощью теоремы мы можем лишь улучшать точность хранящегося на памяти сэпмлирования, но сигнал по-прежнему останется в памяти дискретным (впрочем, другим храниться он и не может).

	\section{Общие выводы}
	В ходе лабораторной работы были исследованы различные методы нахождения Фурье-образов функций, включая численное интегрирование, быстрое дискретное преобразование Фурье (БПФ) и его модифицированную версию с использованием приближённых коэффициентов вида \(c_m = \Delta t \cdot (-1)^m \). Было показано, что точность образов существенно зависят от выбора временного отрезка ширины \( T \) и шага дискретизации \( \Delta t \), а также от (связанных с ними) промежутка частот $V$ и шага частот $\Delta \nu$. Также была рассмотрена теорема Найквиста-Шеннона-Котельникова на примерах функций \( y_1 = a_1 \sin(\omega_1 t + \varphi_0) + a_2 \sin(\omega_2 t + \varphi_1) \) и \( y_2 = \mathrm{sinc}(bt) \). Установлено, что при нарушении условия теоремы (частота дискретизации ниже удвоенной максимальной частоты сигнала $B$) возникают искажения спектра и ошибки интерполяции во временной области. Анализ Фурье-образов исходного, сэмплированного и интерполяционного сигналов позволил более точно понять закулисье изученного способа восстановления функций по их дискретизации.
	
	\newpage
	\section{Приложение}
	\begin{lstlisting}[caption={Код для метода численного интегрирования}]
T = 25;
dt = 0.001;
t = -T/2:dt:T/2;
p = zeros(size(t));
p(abs(t) <= 1/2) = 1;
		
nt = -T/2:0.001:T/2;
np = zeros(size(nt));
np(abs(nt) <= 1/2) = 1;
		
V = 100;
dnu = 0.01;
nu = -V/2:dnu:V/2;
nnu = -V/2:0.01:V/2;
hatp = sinc(nnu);
		
Pi_nu = zeros(size(nu));
for k = 1:length(nu)
	Pi_nu(k) = trapz(t, p .* exp(-1j*2*pi*nu(k)*t));
end
		
Pi_t_rec = zeros(size(t));
for k = 1:length(t)
	Pi_t_rec(k) = trapz(nu, Pi_nu .* exp(1j*2*pi*nu*t(k)));
end
		
Pi_t_rec = real(Pi_t_rec);
	\end{lstlisting}

	\newpage
	\begin{lstlisting}[caption={Код для метода fft}]
T = 4;
dt = 0.0075;
t = -T/2:dt:T/2;

nt = -T/2:0.001:T/2;
np = zeros(size(nt));
np(abs(nt) <= 1/2) = 1;

Pi_t = double(abs(t) <= 0.5);
N = length(Pi_t);
Pi_f = fftshift(fft(Pi_t)) / sqrt(N);

dnu = 1/T;
V = 1/dt;
nu = -V/2:dnu:V/2;

nnu = -V/2:0.01:V/2;
hatp = sinc(nnu);

Pi_t_rec = ifft(ifftshift(Pi_f)) * sqrt(N);
	\end{lstlisting}
	\begin{lstlisting}[caption={Код для метода модифицированного fft}]
T = 4;
dt = 0.05;
N = T/dt;
t = linspace(-T/2, T/2-dt, N);
func = double(abs(t) <= 0.5);

nt = -T/2:0.0001:T/2;
np = zeros(size(nt));
np(abs(nt) <= 1/2) = 1;

m = linspace(-N/2, N/2-1, N);
c = dt * (-1).^m;
V = 1/dt;
dnu = 1/T;
nu = m / T;

F_fft = fftshift(c .* fft(func));

nnu = -V/2:0.001:V/2;
hatp = sinc(nnu);

f_recovered = real(ifft(ifftshift(F_fft) ./ c));
	\end{lstlisting}

	\newpage
	\begin{lstlisting}[caption={Код для сравнения различных методов нахождения образов}]
T = 1.15;
dt = 0.05;
N = T/dt;
t = linspace(-T/2, T/2, N);
p = double(abs(t) <= 0.5);

nt = -T/2:0.0001:T/2;
np = zeros(size(nt));
np(abs(nt) <= 1/2) = 1;

m = 0:N-1;
c = dt * (-1).^m;
V = 1/dt;
dnu = 1/T;
nu = linspace(-V/2, V/2-dnu, N);

F_fft = fftshift(c .* fft(p));

nnu = -V/2:0.001:V/2;
hatp = sinc(nnu);

f_recovered = real(ifft(ifftshift(F_fft) ./ c));

F_plain = fftshift(fft(p)) / sqrt(N);
f_recovered_with_ifft = ifft(ifftshift(F_plain)) * sqrt(N);


hdnu = 1/T;
hV = 1/dt;
hnu = nu;

Pi_nu = zeros(size(hnu));
for k = 1:length(hnu)
    Pi_nu(k) = trapz(t, p .* exp(-1j*2*pi*hnu(k)*t));
end

Pi_t_rec = zeros(size(t));
for k = 1:length(t)
    Pi_t_rec(k) = trapz(hnu, Pi_nu .* exp(1j*2*pi*hnu*t(k)));
end

Pi_t_rec = real(Pi_t_rec);

sum(abs(abs(sinc(nu)) - abs(F_fft)))
sum(abs(abs(sinc(hnu)) - abs(Pi_nu)))
	\end{lstlisting}
	
	\newpage
	\begin{lstlisting}[caption={Код для исследования теоремы Котельникова (гармоники)}]
a1 = 1;
a2 = 0.5;
w1 = 2*pi*5;
w2 = 2*pi*10; 
phi1 = 0;
phi2 = pi/4;
B1 = 20;

T = 1e3;
T1 = 5;
dt = 1e-3;
t = -T/2:dt:T/2;
N = length(t);
m = 0:N-1;
c = dt * (-1).^m;

y1 = a1*sin(w1*t + phi1) + a2*sin(w2*t + phi2);

y1_fft = fftshift(c .* fft(y1));

Fs = 30;
dt_sample = 1/Fs;
t_sample = -T1/2:dt_sample:T1/2;

y1_sample = a1*sin(w1*t_sample + phi1) + a2*sin(w2*t_sample + phi2);

y1_interp = zeros(size(t));
for k = 1:length(t_sample)
    y1_interp = y1_interp + y1_sample(k)*sinc(Fs*(t - t_sample(k)));
end

y1_fft_interp = fftshift(c .* fft(y1_interp));

c1_sample = dt_sample * (-1).^(0:length(t_sample)-1);
y1_sample_fft = fftshift(c1_sample .* fft(y1_sample));


V = 1/dt_sample;
dnu = 1/T1;
nu = -1/dt/2:1/T:1/dt/2;
N_sample = T1/dt_sample+1;
nu_sample = linspace(-V/2, V/2, N_sample);
	\end{lstlisting}

	\newpage
	\begin{lstlisting}[caption={Код для исследования теоремы Котельникова ($\mathrm{sinc}$)}]
b = 4*pi;
B2 = 4;

T = 1e2;
T2 = 5;
dt = 1e-3;
N = T/dt;
t = linspace(-T/2, T/2, N);
N = length(t);
m = 0:N-1;
c = dt * (-1).^m;

y2 = sinc(b*t/pi);

y2_fft = fftshift(c .* fft(y2));

Fs = 3; 
dt_sample = 1/Fs;
N_sample = T2/dt_sample+1;
t_sample = linspace(-T2/2, T2/2, N_sample);

y2_sample = sinc(b*t_sample/pi);

y2_interp = zeros(size(t));
for k = 1:length(t_sample)
    y2_interp = y2_interp + y2_sample(k)*sinc(Fs*(t - t_sample(k)));
end

y2_fft_interp = fftshift(c .* fft(y2_interp));

c2_sample = dt_sample * (-1).^(0:length(t_sample)-1);
y2_sample_fft = fftshift(c2_sample .* fft(y2_sample));

V = 1/dt_sample;
dnu = 1/T2;
nu = linspace(-1/dt/2, 1/dt/2, N);
nu_sample = linspace(-V/2, V/2, N_sample);
	\end{lstlisting}






\end{document}