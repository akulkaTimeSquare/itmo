\documentclass[a4paper,hidelinks,14pt]{extarticle}

\usepackage[utf8]{inputenc}
\usepackage[T2A]{fontenc}
\usepackage[english, russian]{babel}
\usepackage{lipsum}
\usepackage{amsmath}
\usepackage{amssymb}
\usepackage{amsfonts}
\usepackage{mathtools}
\usepackage{datetime}
\usepackage[pdftex]{graphicx}
\usepackage{indentfirst}
\usepackage{asymptote}
\usepackage{systeme}
\usepackage[dvipsnames]{xcolor}
\usepackage{lastpage}
\usepackage{fancybox,fancyhdr}
\usepackage{hyperref}
\usepackage[numbered,framed]{matlab-prettifier}
\usepackage[font={small,it}]{caption}
\fancyhead[L]{Лабораторная работа №4}
\fancyhead[C]{}
\fancyhead[R]{\textit{Линейная фильтрация}}
\fancyfoot[L]{}
\fancyfoot[C]{Страница \thepage\space из \pageref{LastPage}}
\fancyfoot[R]{}
\pagestyle{fancy}
\newcommand{\gt}{\textgreater}
\newcommand{\lt}{\textless}


\begin{document}
	\begin{titlepage}
		\setlength{\parindent}{0ex}
		
		\begin{center}
			\textsc{
				\vspace{1ex}
				Научно исследовательский университет ИТМО \\
				\vspace{0.5ex}
				Факультет систем управления и робототехники \\
				\vspace{0.5ex}
			}
		\end{center}
		
		\vspace{50mm}
		
		\begin{center}
			Отчет по лабораторной работе №4 \\
			Линейная фильтрация
		\end{center}
		
		\vspace{50mm}
		
		\begin{minipage}{.48\linewidth}
			Выполнил студент группы R3380
			
			Преподаватели
		\end{minipage}
		\hfill
		\begin{minipage}{.5\linewidth}
			\begin{flushright}
				Мовчан И.Е.
				\\
				Пашенко А.В., Перегудин А.А.
			\end{flushright}
		\end{minipage}
		
		\vfill
		\begin{center}
			Санкт-Петербург
			\\
			2025
		\end{center}
	\end{titlepage}

	\tableofcontents
	\clearpage

	\section{Линейные фильтры}
	Зададимся числами $a = 4$, $t_1 = 3$, $t_2 = 8$ и возьмём сигнал в виде прямоугольной волны:
	$$
	g(t) = 
		\begin{cases}
			a, \hspace{1mm} t \in [t_1, t_2],\\
			0, \hspace{1mm} t \notin [t_1, t_2];
		\end{cases}
		= 
		\begin{cases}
			4, \hspace{1mm} t \in [3, 8],\\
			0, \hspace{1mm} t \notin [3, 8];
		\end{cases}
	$$
	и его зашумлённую версию
	$$
	u(t) = g(t) + b\xi (t) + c \sin (d t),
	$$
	где $\xi \sim U[-1, 1]$ - равномерное распределение, представляющее белый шум, а $b$, $c$, $d$ - параметры возмущений.

	Чистый и зашумлённый сигналы при параметрах $b  = 0.5$, $c = 0.3$, $d = 5$ представлены на рисунке ниже. Слагаемое $b \xi(t)$ в результате дало стабильные резкие скачки по времени, отображающиеся, как мы увидим далее, по большей части на высоких частотах, $c \sin (d t)$ же - стабильное гармоническое воздействие.
	\begin{figure}[h]
		\centering
		\includegraphics[scale=0.175]{./images/1.jpg}
	\end{figure}

	\subsection{Фильтры первого порядка}
	Пусть у нас не существует синусоидального возмущения в замушлённом сигнале $u(t)$, то есть примем $c = 0$, при этом оставим $b = 0.5$ (имеем, пусть это будет и не очень верно сказано, высокочастотный шум, а $u(t) = g(t) + 0.5\xi (t)$). Рассмотрим линейный фильтр, заданный передаточной функцией ($T = 0.1$)
	$$
	W_1(p) = \frac{1}{Tp + 1} = \frac{1}{0.1 p + 1}.
	$$

	Его амплитудно-частотная характеристика (АЧХ) представлена на рисунке ниже
	\begin{figure}[h]
		\centering
		\includegraphics[scale=0.175]{./images/2.jpg}
	\end{figure}

	Так как АЧХ напрямую влияет на способности фильтра к подавлению, то, например, в нашем случае все амплитуды образа Фурье обрабатываемого сигнал ($\hat{u}(t)$), имеющие частоту менее 10 рад/c, будут хорошо подавлены ($|W_1(i w)|$ убывает и равна $1/ \sqrt{2}$ при частоте среза $w_0 = 10$ рад/c). Упомянем также, что глушиться будут в целом все амплитуды в частотной области, так как  $|W_1(i w)|$ < 1 везде, кроме точки 0.

	Настало время немного поэкспериментировать, попробуем применить его к нашей зашумлённой функции $u(t)$ (все вычисления далее проводятся в среде matlab, там же строятся и графики, для задания передаточной функции и применения её к сигналу используются команды \textbf{tf} и \textbf{lsim} соответственно):
	\begin{figure}[h]
		\centering
		\includegraphics[scale=0.1675]{./images/3.jpg}
		\caption{Сравнение сигналов при $a = 4$, $T = 0.1$}
	\end{figure}

	В общем, наш фильтр неплохо выполняет свою работу, нивелируя воздействие шума, но можно ли лучше? Для этого немного покрутим параметры и посмотрим, на что же влияет постоянная времени $T$ (зафиксируем при этом $a = 4$).
	
	На рисунках \ref{4} и \ref{5} исследованы случаи $T = 0.6$ и $T = 0.025$. По результатам понятно, что если мы увеличиваем параметр $T$, то фильтрация становится более <<жёсткой>>, давящей всякое стороннее воздействие, однако появляется и запаздывание по фазе, связанная с тем, что применяя фильтр, мы также влияем и на фазовые характеристики нашего сигнала. И наоборот, если мы уменьшаем параметр $T$, то фильтрация <<смягчается>>, а обработанный сигнал практически не запаздывает по времени в сравнении с исходным. Получается, выиграла правда <<больше - сильнее, меньше - слабее>>.
	\begin{figure}
		\centering
		\includegraphics[scale=0.1675]{./images/4.jpg}
		\caption{Сравнение сигналов при $a = 4$, $T = 0.6$}
		\label{4}
	\end{figure}
	\begin{figure}
		\centering
		\includegraphics[scale=0.1675]{./images/5.jpg}
		\caption{Сравнение сигналов при $a = 4$, $T = 0.025$}
		\label{5}
	\end{figure}

	В целом, нашей задачей как раз и является выбор оптимального $T$, при котором всё ещё хорошо фильтруется, при этом практически отсутствует запаздывание. Исходная передаточная функция с $T = 0.1$ как раз достаточно хорошо с этим справляется.


	Отметим также, что фильтрация напрямую связана с АЧХ, так как частота среза $w_0$ уменьшается при увеличении $T$, то диапазон глушащихся частот увеличивается. Наглядно можно видеть в сравнении на рисунке 
	\begin{figure}[h]
		\centering
		\includegraphics[scale=0.1675]{./images/8.jpg}
		\caption{Сравнение АЧХ фильтров}
	\end{figure}

	Влияние параметра $a$ на качество фильтрации представлено на рисунках \ref{6} и \ref{7} (в данных примерах $a = 15$ и $0.5$ соответственно, а $T$ фиксировано и равно $0.1$).
	\begin{figure}
		\centering
		\includegraphics[scale=0.1675]{./images/6.jpg}
		\caption{Сравнение сигналов при $a = 15$, $T = 0.1$}
		\label{6}
	\end{figure}

	\begin{figure}
		\centering
		\includegraphics[scale=0.1675]{./images/7.jpg}
		\caption{Сравнение сигналов при $a = 0.5$, $T = 0.1$}
		\label{7}
	\end{figure}
	
	Можно заметить, что чем больше $a$, тем как бы меньший вклад шум вносит в наш сигнал (амплитуда стороннего становится много меньшей по сравнению с исходным), и тем лучше работает фильтрация (рисунок \ref{6}). Чем же ближе наша кривая к оси $Ox$ - тем хуже фильтр справляется, в том числе и потому, что весь наш сигнал в целом становится тяжело отличимым от шума, для которого <<центром>> как раз и является ось $Ox$ (рисунок \ref{7}).
	
	Итак, с параметрами фильтрации и сигнала разобрались, теперь посмотрим, что же у нас в действительности происходит внутри частотной области. Для начала взглянем на поведение модулей Фурье-образов исходного, зашумлённого и отфильтрованного сигналов, при этом для большей наглядности сильно увеличим амплитуду присутствующих возмущений, подняв параметр $b$ до $2$ ($a = 4$, $T = 0.1$). Результаты показаны на рисунке \ref{9}.
	
	\begin{figure}
		\centering
		\includegraphics[scale=0.1675]{./images/12.jpg}
		\caption{Сравнение модулей Фурье-образов}
		\label{9}
	\end{figure}


	Видим, что основное различие зашумлённого модуля образа Фурье сигнала от исходного проявляется на высоких частотах с низкой амплитудой (необработанный сигнал не идёт к 0 на бесконечностях, то и дело появляются скачки!) - именно они и вносят основной шумовой вклад (так как на больших значениях влияние мелких выбросов мало), и именно с этими частотами мы работаем, устремляя их к 0 к исходному, незашумлённому сигналу.


	Для полноты картины проверим выполнение теоремы о свёртке в двух её представлениях: фильтрованный сигнал $y$ должен совпадать с обратным преобразованием Фурье от $W(iw) \hat{u} (w)$, или что его образ Фурье $\hat{y} = W(iw) \hat{u}(w)$ (в данном случае будем исследовать модули для большей наглядности). Записывая то же, но используя свёртку: $$y = \mathcal{F}^{-1} \{ W(i w) \hat{u} \} = w(t) * u,$$
	где $u$ - зашумлённый вход, $w(t) = \mathcal{F}^{-1} \{ W(i w) \}/\sqrt{2 \pi}$ - весовая функция. Результаты представлены на рисунках \ref{13} и \ref{14}
	
	\begin{figure}
		\centering
		\includegraphics[scale=0.1675]{./images/13.jpg}
		\caption{Проверка теоремы о свёртке (временная область)}
		\label{13}
	\end{figure}
	\begin{figure}
		\centering
		\includegraphics[scale=0.1675]{./images/14.jpg}
		\caption{Проверка теоремы о свёртке (частотная область)}
		\label{14}
	\end{figure}
	
	Как можно видеть, всё действительно работает. Тогда, казалось бы, зачем мы до этого как-то динамически применяли передаточную функцию, строили графики, анализировали их, если в итоге всё свелось к тем же преобразованиям Фурье и свёрткам, что и в прошлой работе, просто чуть иной природы? Дело в том, что это лишь одно из представлений линейных фильтров, подразумевающее, как и жёсткая фильтрация, использование дорогих операций, связанных с постобработкой сигнала, тогда как в реальности мы нечасто прибегаем к таким методам их реализации. Отметим также, что жёсткие фильтры - математический идеал (их импульсная характеристика $w(t)$ существует и в отрицательной полуоси), к которому в жизни мы можем лишь приблизиться, например, с помощью легкодоступных линейных моделей (именно поэтому первых и не существует в виде динамических по природе передаточных функций).
	
	Не будем исключать и роль плавности в процессе перехода от пропускаемых частот к подавляемым, жёсткие фильтры отсекают резко (и это может приводить к артефактам), исследованные нами - плавно, в задачах же практически всегда более предпочтительным является второй вариант, пусть он и чуть менее <<эффективный>> (отметим однако, что мы также можем и понизить эту плавность, увеличив крутизну их АЧХ добавлением порядка).

	Что ж, линейные фильтры первого порядка, конечно, хороши при подавлении высокочастотного шума, изменяя параметры, мы можем с приемлимой точностью добиваться необходимой в задаче жёсткости фильтрации, они также имеют и свою специфику работы, описанную выше, отличающую их от обычных низкочастотных жёстких фильтров. Но что если перед нами стоит задача иного рода (например, у нас имеется стабильный синусоидальный шум, который низкочастотный фильтр не заденет), и, используя фильтр, мы хотим воздействовать только на определённый частотный отрезок? Здесь нам пригодятся режекторные фильтры, рассмотрим их подробнее.

	\subsection{Режекторный полосовой фильтр}

	Итак, пусть у нас отсутствует белый шум, но присутствует стабильный синусоидальный шум частоты $d$ и амплитуды $c$ (то есть сигнал $u(t) = g(t) + b\xi (t) + c \sin (d t) = g(t) + c \sin (d t)$, $b = 0$, $c = 0.3$). Рассмотрим режекторный фильтр

	$$
	W_2(p) = \frac{p^2 + a_1 p + a_2}{p^2 + b_1 p + b_2}
	$$
	и выберем числа
	$$
	a_1 = 0,\hspace{1mm} a_2 = 16,\hspace{1mm} b_1 = 10,\hspace{1mm} b_2 = 16,
	$$
	при которых выполняются условия:
	\begin{enumerate}
		\item Фильтр является устойчивым, так как корни полинома знаменателя $p_1=-2$, $p_2=-8$ имеют строго отрицательные вещественные части;
		\item На низких частотах ($w \rightarrow 0$) и высоких частотах ($w \rightarrow \pm \infty$) АЧХ фильтра $|W_2(iw)| \rightarrow 1$;
		\item При некоторой частоте $w_0 = 4$ амплитуда $|W_2(iw)| = 0$.
	\end{enumerate}
	
	Амплитудно-частотная характеристика фильтра $W_2(p)$ представлена на рисунке \ref{15}. Можно видеть, что всё перечисленное действительно выполняется.
	
	Попробуем созданное в действии. Исходный, зашумлённый и обработанный сигналы изображены на рисунке \ref{16}. После низкочастотных фильтров работа режекторных кажется немного странной (присутствуют весомые скачки, связанные с резкими изменениями в самом сигнале, всё та же задержка), однако результат говорит сам за себя - все синусоидальные воздействия были подавлены, причем до общей амплитуды, связано это с тем, что шум в сигнале - стабильный, в образе Фурье возникает в виде одного и того же точечного выброса, на который мы и воздействуем при фильтрации (причем, как увидим далее, тем больше, чем мы ближе к $w_0 = 4$; у нас $d = 5$, поэтому возникает серьёзное, но не полное приглушение).
	
	\begin{figure}
		\centering
		\includegraphics[scale=0.175]{./images/15.jpg}
		\caption{АЧХ полосового фильтра}
		\label{15}
	\end{figure}

	\begin{figure}
		\centering
		\includegraphics[scale=0.175]{./images/16.jpg}
		\caption{Сравнение сигналов при $d = 5$, $b_1 = 10$}
		\label{16}
	\end{figure}

	Что ж, убедимся в сказанном на деле, попробуем проварьировать частоту синусоиды, меняя параметр $d$. Результаты при $d = 15$, $d = 2$ на рисунках \ref{17}, \ref{18} соответственно. И действительно, чем мы дальше от $w_0$, тем фильтрация синусоиды становится хуже, и это работает в двух направлениях (и при уменьшении $d$, и при её увеличении), всё согласно вышеприведённой АЧХ фильтра.
	\begin{figure}
		\centering
		\includegraphics[scale=0.175]{./images/17.jpg}
		\caption{Сравнение сигналов при $d = 15$, $b_1 = 10$}
		\label{17}
	\end{figure}
	\begin{figure}
		\centering
		\includegraphics[scale=0.175]{./images/18.jpg}
		\caption{Сравнение сигналов при $d = 2$, $b_1 = 10$}
		\label{18}
	\end{figure}
	\begin{figure}
		\centering
		\includegraphics[scale=0.175]{./images/19.jpg}
		\caption{Сравнение сигналов при $d = w_0 = 4$, $b_1 = 10$}
		\label{19}
	\end{figure}
	
	Особого внимания заслуживает случай $d = w_0 = 4$, где происходит <<чудо>> и глушение идеально, всё сторонее воздействие сходит на нет (рисунок \ref{19})! Отсюда вывод - чем лучше мы понимаем наш шум, тем легче нам с ним работать (поэтому в задачах фильтрации имеет смысл сначала подумать про источники воздействия, а только потом про саму фильтрацию). Однако почти всегда это исследование трудоёмко, а точечно определить частоту синусоиды невозможно. Отсюда возникает резонный вопрос - можем ли как-то повлиять на ширину приглушаемых частот? Ответ - да, с помощью $b_1$ знаменателя $W_2$!

	Параметр $b_1$ влияет на жёсткость фильтрации, при его увеличении <<полоса подавления>> становится шире, при уменьшении - уже (на рисунке \ref{20} сравнения трёх различных АЧХ чётко прослеживается данная идея), мы начинаем воздействовать точечно на определённую частоту $w_0$ (это может пригодиться, например, в редкой задаче, где мы достоверно знаем, с какой частотой задан шум).
	\begin{figure}
		\centering
		\includegraphics[scale=0.175]{./images/20.jpg}
		\caption{Сравнение АЧХ при $b_1 = 2$, $b_1 = 10$, $b_1 = 18$}
		\label{20}
	\end{figure}

	Результаты работы фильтра при $b_1 = 2$ и $b_2 = 18$ ($d$ фиксировано и равно 5) представлены на рисунках \ref{21} и \ref{22} соответственно. Сравнивая графики, можно подтвердить слова выше про <<силу>> фильтрации: уменьшение $b_1$ действительно приводит к тому, что любое воздействие, c отличной от $w_0$ частотой, остаётся практически нетронутым и подавляется слабо; увеличение же приводит к весомой задержке в изменениях, однако работа происходит лучшим образом. В реальной жизни опять-таки стоит стремиться к балансу: так как точно неизвестно, какое природа <<задала>> $w_0$, то часто выбирают именно хорошее среднее, при котором и запаздывание незначимо, и фильтрация находится на должном уровне.
	\begin{figure}
		\centering
		\includegraphics[scale=0.175]{./images/21.jpg}
		\caption{Сравнение сигналов при $d = 5$, $b_1 = 2$}
		\label{21}
	\end{figure}
	\begin{figure}
		\centering
		\includegraphics[scale=0.175]{./images/22.jpg}
		\caption{Сравнение сигналов при $d = 5$, $b_1 = 18$}
		\label{22}
	\end{figure}

	Посмотрим также, что же у нас происходит в частотной области (рисунок \ref{23}). Видим выброс в образе Фурье $u(t)$ - это и есть та самая сторонняя гармоника в сигнале, которую мы пытаемся заглушить режекторным фильтром. Суть его работы - зануление самой $w_0$ и уменьшение по модулю тех частот, которых располагаются рядом, причем тем больше, чем ближе они к настроенной частоте фильтрации (в нашем случае $w_0$). Это же можно наблюдать и на графике - у образа Фурье обработанного сигнала значение при $w_0$ равно 0, а всё, что находится рядом, прижато к оси $Ox$. Именно поэтому при исследовании влияния $d$ на фильтрацию мы получали полную очистку от всякого шума при $d = w_0$, а при исследуемой сейчас (и других, отличных от равенства, вариациях) - лишь частичную, фильтр просто снизил их влияние в частотной области, от этого уменьшилась и их амплитуда, однако не полностью, так как они не были так близки к настроенной частоте.
	\begin{figure}
		\centering
		\includegraphics[scale=0.175]{./images/23.jpg}
		\caption{Сравнение модулей Фурье-образов}
		\label{23}
	\end{figure}

	\begin{figure}
		\centering
		\includegraphics[scale=0.175]{./images/24.jpg}
		\caption{Сравнение сигналов при $d = 5$, $b_1 = 10$}
		\label{24}
	\end{figure}
	\begin{figure}
		\centering
		\includegraphics[scale=0.175]{./images/25.jpg}
		\caption{Сравнение сигналов при $d = 5$, $b_1 = 10$}
		\label{25}
	\end{figure}

	Наконец, проверим теорему о свёртке. Результаты представлены на рисунках \ref{24} и \ref{25}. Как можно заметить, всё сходится друг к другу и теорема действительно работает, так что можем ей пользоваться, если считать сигнал динамически ненужно и слишком затратно в сравнении с методами, использующими преобразования Фурье.

	По итогу режекторные полосовые фильтры хороши в задачах присутствия стороннего гармонического воздействия какой-то (желательнее известной) частоты, мы сможем точечно погасить существующий выброс в частотной оси, который те создают, причем также адаптивно настроить и ширину его подавления. Однако, например, в ситуации присутствия как синусоидального шума, так и высокочастотного или же белого шумов использовать только полосовой фильтр будет очевидно неуместено, ведь он никак не обработает резкие скачки по времени. В данном случае более уместным будет скомбинировать работу двух фильтров исследованных фильтров (один погасит высокие, другой - гармоническое). На выходе получаем вполне удовлетворительный результат общей фильтрации, учитывая при этом, что настроенная частота фильтрации не совпадала с синусоидальной (рисунок \ref{26}).
	\begin{figure}[h]
		\centering
		\includegraphics[scale=0.175]{./images/26.jpg}
		\caption{Фильтрация сигнала двумя фильтрами при $b = 0.5$, $c = 0.3$, $d = 5$}
		\label{26}
	\end{figure}

	\section{Сглаживание биржевых данных}

	Пришло время применить изученное на практике. Представим, что мы разрабатываем инвестиционное 
	приложение, в котором должна присутствовать функция представления сглаженных графиков 
	котировок акций, при этом степень сглаживания должна зависеть от рассматриваемого пользователем
	временного периода.
	
	Найдём данные о стоимости акций Сбербанка за интервал времени с 11 апреля 2020 года по 10 апреля 2025 года с периодичностью в 1 день и применим к ним изученный линейный фильтр первого порядка ($W_1(p) = 1/(Tp + 1)$) с T: 1 день, 1 неделя, 1 месяц, 3 месяца, 1 год. Результаты работы сглаживания приведены на рисунках \ref{27}-\ref{31} соответственно (для фильтров также были найдены необходимые начальные состояния, чтобы решить возникающую проблему провала в начале временного интервала у \textbf{lsim}).

	Сделаем небольшую выжимку из полученного: если мы хотим получить как можно более точные результаты, отражающие все важные краткосрочные изменения в данных, можем использовать $T = $ день или неделя, такие фильтры хорошо передадут всю краткосрочную динамику, чуть сгладя её от случайных выбросов для лучшего отражения общих дневных-недельных изменений; при ненадобности краткосрочных эффектов можем полностью их устранить, используя $T = 1$ месяц; если же мы хотим отразить долгосрочные тренды, то однозначно нужно использовать $T = 3$ месяца или $1$ год.

	В итоге всё сводится к нашим нуждам. Отсюда идея - давайте будем применять сглаживание в зависимости от текущего масштаба: при малом захватываемом временном периоде мы интуитивно хотим больше анализировать краткосрочные эффекты, поэтому будем применять фильтры с малым $T$ (день или неделя), тогда как при большом окне логичнее будет использовать сильное сглаживание ($T = $ 3 месяца - целый год), чтобы устранить всякий <<шум>> и сделать акцент на долгосрочных трендах. Изменение масштаба, соответственно, будет приводить и к изменениям в наблюдаемых на экране данных, подстраивая их под наши интуитивные человеческие <<хотелки>>.

	\begin{figure}
		\centering
		\includegraphics[scale=0.175]{./images/27.jpg}
		\caption{Сглаживание котировок (T = 1 день)}
		\label{27}
	\end{figure}
	\begin{figure}
		\centering
		\includegraphics[scale=0.175]{./images/28.jpg}
		\caption{Сглаживание котировок (T = 1 неделя)}
		\label{28}
	\end{figure}
	\begin{figure}
		\centering
		\includegraphics[scale=0.175]{./images/29.jpg}
		\caption{Сглаживание котировок (T = 1 месяц)}
		\label{29}
	\end{figure}
	\begin{figure}
		\centering
		\includegraphics[scale=0.175]{./images/30.jpg}
		\caption{Сглаживание котировок (T = 3 месяца)}
		\label{30}
	\end{figure}
	\begin{figure}
		\centering
		\includegraphics[scale=0.175]{./images/31.jpg}
		\caption{Сглаживание котировок (T = 1 год)}
		\label{31}
	\end{figure}

	\section{Общие выводы}
	В ходе выполнения работы были исследованы линейные фильтры первого порядка и полосовые режекторные фильтры второго порядка, а также применение низкочастотных для сглаживания биржевых данных. Фильтры первого порядка дают простую реализацию и хорошую эффективность в снижении шумов, однако вносят задержки в сигнал и мало <<разбираются>>, что же они в действительности подавляют, из-за плавности действия в частотной области. Режекторные фильтры дали точечное подавление определённого отрезка частот, что делает их полезными для устранения локализованных помех. При обработке биржевых данных линейные фильтры позволили сгладить кратковременные колебания, а также выделить общие месячные и годовые тренды. В целом, динамическая фильтрация показала себя с лучшей стороны в обработке сигналов, несмотря на все возникшие ограничения и трудности.

	\textbf{На этом всё, спасибо!}
	


\end{document}