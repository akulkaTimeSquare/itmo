\documentclass[a4paper,hidelinks,14pt]{extarticle}

\usepackage[utf8]{inputenc}
\usepackage[T2A]{fontenc}
\usepackage[english, russian]{babel}
\usepackage{lipsum}
\usepackage{amsmath}
\usepackage{amssymb}
\usepackage{amsfonts}
\usepackage{mathtools}
\usepackage{datetime}
\usepackage[pdftex]{graphicx}
\usepackage{indentfirst}
\usepackage{asymptote}
\usepackage{systeme}
\usepackage[dvipsnames]{xcolor}
\usepackage{lastpage}
\usepackage{fancybox,fancyhdr}
\usepackage{hyperref}
\usepackage[framed,autolinebreaks,useliterate,numbered]{mcode}
\usepackage[font={small,it}]{caption}
\fancyhead[L]{Лабораторная работа №3}
\fancyhead[C]{}
\fancyhead[R]{\textit{Жёсткая фильтрация}}
\fancyfoot[L]{}
\fancyfoot[C]{Страница \thepage\space из \pageref{LastPage}}
\fancyfoot[R]{}
\pagestyle{fancy}
\newcommand{\gt}{\textgreater}
\newcommand{\lt}{\textless}

\begin{document}
	\begin{titlepage}
		\setlength{\parindent}{0ex}
		
		\begin{center}
			\textsc{
				\vspace{1ex}
				Научно исследовательский университет ИТМО \\
				\vspace{0.5ex}
				Факультет систем управления и робототехники \\
				\vspace{0.5ex}
			}
		\end{center}
		
		\vspace{50mm}
		
		\begin{center}
			Отчет по лабораторной работе №3 \\
			Жёсткая фильтрация
		\end{center}
		
		\vspace{50mm}
		
		\begin{minipage}{.48\linewidth}
			Выполнил студент группы R3380
			
			Преподаватели
		\end{minipage}
		\hfill
		\begin{minipage}{.5\linewidth}
			\begin{flushright}
				Мовчан И. Е.
				\\
				Пашенко А.В., Перегудин А. А.
			\end{flushright}
		\end{minipage}
		
		\vfill
		\begin{center}
			Санкт-Петербург
			\\
			2025
		\end{center}
	\end{titlepage}
	\tableofcontents
	\clearpage
	
	\section{Жёсткие фильтры}
	
	Перед выполнением зададимся числами $a = 1$, $t_1=-3$, $t_2=3$ (отметим, что $t_1 < t_2$), составив тем самым прямоугольную волну $$g(t) = \left\{ 
	\begin{array}{l}
		a,\ t \in [t_1, t_2], \\
		0,\ \text{иначе};
	\end{array} 
	\right.
	\Rightarrow
	g(t) = 
	\left\{ 
	\begin{array}{l}
		1,\ t \in [-3, 3], \\
		0,\ \text{иначе}.
	\end{array} 
	\right.$$
	
	Кроме этого, выберем времени большой интервал по времени $T$ и достаточно малым шагом дискретизации $\Delta t$ (для улучшения частотных значений), задав массив времени $t$ с данными параметрами и значения функции $g(t)$ на этом же отрезке.

	Рассмотрим зашумлённую версию функции:
	$$
	u(t) = g(t) + b \xi(t) + c \sin(d t),
	$$
	где $\xi \sim U[-1, 1]$ - равномерное распределение, представляющее белый шум, а $b$, $c$, $d$ - параметры возмущений. Сигнал $u(t)$ мы и будем очищать от этих сторонних воздействий различной природы.
	\begin{figure}[h]
		\centering
		\includegraphics[width=0.9\textwidth]{./images/1.jpg}
		\caption{Исходный и зашумленный сигналы при $b = 0.35$, $c = 0.1$, $d = 3$}
		\label{1}
	\end{figure}

	\subsection{Убираем высокие частоты}
	Положим $c = 0$ (уберем стабильную синусоидальную помеху), а $b = 0.35$ (то есть будем исследовать то, как можно подавить белый шум; \textit{для личного удобства далее в качестве параметра $b$ будет рассматриваться максимально возможный <<разнос>> случайного возмущения, то есть параметр в два раза больший = $2b$ исходного}). Найдём Фурье-образ зашумлённого сигнала $\hat{u}(\nu)$ и обнулим его значения для всего диапазона частот вне отрезка $[-\nu_0, \nu_0]$, где $\nu_0=2$, сделаем обратное преобразование и посмотрим, что изменится. Результаты работы представлен на рисунках \ref{2} и \ref{3}.
	\begin{figure}[h]
		\centering
		\includegraphics[width=\textwidth]{./images/2.jpg}
		\caption{Модули образов Фурье сигналов при $b = 0.35$, $\nu_0 = 2$}
		\label{2}
	\end{figure}
	\begin{figure}[h]
		\centering
		\includegraphics[width=\textwidth]{./images/3.jpg}
		\caption{Восстановление отфильтрованной функции при $b = 0.35$, $\nu_0 = 2$}
		\label{3}
	\end{figure}
	
	Как видим, модуль зашёмлённой функции на высоких частотах представляет из себя еле разборчивое <<месиво>>, не вносящее никакого положительного вклада в сигнал. Это признаки проявления белого шума, который мы и пытаемся подавить, сведя его проявление в образах к 0. Можно заметить, что стороннее воздействие также провляется и на низких частотах, только тут оно менее значимо, так как его амплитуда в сравнении с истинным сигналом минимальна, соотвественно, минимальны и их вклад с искажениями (ровно противоположная ситуация наблюдается с высокими частотами, обладающими малыми амплитудами, для которых сторонние выбросы достаточно значимы). Итак, идея метода фильтрации белого шума - оставить неизменными низкие частоты, так как там он не мало что меняет, и подавить до 0 высокие, которые и так не несут в себе никакого смысла, так как полностью зашумлены. То есть мы применяем \textit{фильтр низких частот} (на рисунке \ref{3} как раз продемонстрирована его хорошая работа - случайные частые выбросы исчезли, превратясь в плавные сдвиги возле истинных значений сигналов, представление функции явно сдвинулось в лучшую сторону, став более гладким).

	Параметр $b$ влияет на уровень разноса шума, $\nu_0$ - на <<грубость>> его отсечения в частотной области. Соответственно, чем выше $b$, тем более мелкие отрезки нам приходится использовать, уменьшая $\nu_0$, так как шум начинает проявляться на всё более большом диапазоне высоких частот (доходя и до низких), портя их амплитуды. Увеличение $\nu_0$ также приводит к более жёсткой фильтрации, его уменьшение - к мягкости (зашумлённый сигнал практически мало трогается, так как урезаются не все шумовые компоненты в частотной облатси). Докажем сказанное, проварьировав каждый из параметров. Результаты работы при параметрах $b = 0.35$ и $\nu_0 = 8$ представлены на рисунках \ref{4} и \ref{5} (мягкая фильтрация, значима роль сторонних воздействий), при $b = 2$, $\nu_0 = 2$ - на рисунках \ref{6} и \ref{7}, а при $b = 1$, $\nu_0 = 1$ на рисунках \ref{8} \ref{9}. Можно видеть, что сказанное действительно подтвердилось. Также отметим, что чем больше влияние рассматриваемого шума, тем проблематичнее выделить главный вид сигнала (вплоть до практически полной невозможности восстановления).
	\begin{figure}[h]
		\centering
		\includegraphics[width=0.95\textwidth]{./images/4.jpg}
		\caption{Модули образов Фурье сигналов при $b = 0.35$, $\nu_0 = 8$}
		\label{4}
	\end{figure}
	\begin{figure}
		\centering
		\includegraphics[width=0.95\textwidth]{./images/5.jpg}
		\caption{Восстановление отфильтрованной функции при $b = 0.35$, $\nu_0 = 8$}
		\label{5}
	\end{figure}
	\begin{figure}
		\centering
		\includegraphics[width=0.95\textwidth]{./images/6.jpg}
		\caption{Модули образов Фурье сигналов при $b = 1$, $\nu_0 = 2$}
		\label{6}
	\end{figure}
	\begin{figure}
		\centering
		\includegraphics[width=0.95\textwidth]{./images/7.jpg}
		\caption{Восстановление отфильтрованной функции при $b = 1$, $\nu_0 = 2$}
		\label{7}
	\end{figure}
	\begin{figure}
		\centering
		\includegraphics[width=0.95\textwidth]{./images/8.jpg}
		\caption{Модули образов Фурье сигналов при $b = 2$, $\nu_0 = 1$}
		\label{8}
	\end{figure}
	\begin{figure}
		\centering
		\includegraphics[width=0.95\textwidth]{./images/9.jpg}
		\caption{Восстановление отфильтрованной функции при $b = 2$, $\nu_0 = 1$}
		\label{9}
	\end{figure}
	
	\subsection{Убираем специфические частоты}
	Примем теперь все параметры $b = 0.35$, $c = 0.1$ и $d = 3$ ненулевыми (получим график зашумлённого сигнала, как на рисунке \ref{1}). Подберём такой совмещенный фильтр, который независимым образом подавляет как специфические помехи в виде стороннего синусоидального воздействия (гармоники), так и случайные шумы. Данное достигается при последовательном использовании уже изученного \textit{фильтра низких частот} с параметром $\nu_0$ и \textit{полосового фильтра}, давящий конкретные симметричные относительно 0 отрезки $[\pm\nu_1-h, \pm\nu_1+h]$, где $h$ - ширина подавления, а $\pm\nu_1$ - точки, возле которых уменьшение амплитуд и происходит.

	Всё дело в том, что гармоника при преобразовании Фурье превращается в дельта-функцию, разнесенную на соответствующие частоты, симметричные относительно 0 (дельта-функции локализованны в одной точке, поэтому их нужно подавлять таким же локализированным на конкретных отрезках полосовым фильтром), а случайный шум, как уже было сказано, проявляется в основном на высоких частотах (и их мы будем гасить с помощью изученного в предыдущем пункте метода). Именно поэтому последовательное заглушение сначала одного воздействия, а после другого (без разницы, в каком порядке, ведь итог один) как раз даёт необходимую нам фильтрацию.
	
	Результаты работы при параметрах фильтра $\nu_0 = 2$, $\nu_1 = 0.5$, $h = 0.1$ и параметрах сигнала $b = 0.35$, $c = 0.1$ и $d = 3$ приведены на рисунках \ref{10} и \ref{11}. Видим, что подавились частота, находящиеся за пределами диапазона $[-\nu_0, \nu_0] = [-2, 2]$ либо частоты, входящие в отрезки $[\pm\nu_1-h, \pm\nu_1+h] = [\pm 0.5 - 0.1, \pm 0,5 + 0.1]$. Соответственно, те дельта-пики, которые давало гармоническое воздействие были полностью подавлены (в ширине $h$), а испорченные амплитуды высоких частот были полностью занулены. На рисунке \ref{11} вышла хоть и не лучшая, но вполне неплохая фильтрация.
	\begin{figure}[h]
		\centering
		\includegraphics[width=0.95\textwidth]{./images/10.jpg}
		\caption{Образы сигналов при $\nu_0 = 2$, $\nu_1 = 0.5$, $h = 0.1$, $b = 0.35$, $c = 0.1$ и $d = 3$}
		\label{10}
	\end{figure}
	\begin{figure}
		\centering
		\includegraphics[width=0.95\textwidth]{./images/11.jpg}
		\caption{Фильтрация при $\nu_0 = 2$, $\nu_1 = 0.5$, $h = 0.1$, $b = 0.35$, $c = 0.1$ и $d = 3$}
		\label{11}
	\end{figure}
	\begin{figure}
		\centering
		\includegraphics[width=0.95\textwidth]{./images/12.jpg}
		\caption{Образы сигналов при $\nu_0 = 0.75$, $\nu_1 = 0.5$, $h = 0.1$, $b = 0.35$, $c = 0.1$, $d = 3$}
		\label{12}
	\end{figure}
	\begin{figure}
		\centering
		\includegraphics[width=0.95\textwidth]{./images/13.jpg}
		\caption{Фильтрация при $\nu_0 = 0.75$, $\nu_1 = 0.5$, $h = 0.1$, $b = 0.35$, $c = 0.1$ и $d = 3$}
		\label{13}
	\end{figure}
	
	Давайте разберёмся с каждым из параметров по отдельности. Начнём с уже знакомого нам $\nu_0$. Как и прежде, он задаёт отрезок низких частот, который мы оставляем неизменным, и отвечает за силу, с которой будут подавляться случайные шумы, присутствующие в обрабатываемом сигнале. Уменьшение параметра приводит к более сильной фильтрации, а также большему урезания важных низкочастотных компонент, дающих общий вид функции, что и наблюдаем на рисунках \ref{12} и \ref{13}, построенные при параметрах $\nu_0 = 0.75$, $\nu_1 = 0.5$, $h = 0.1$, $b = 0.35$, $c = 0.1$ и $d = 3$ (исчезают свойственные резкие пики сигнала, однако влияние белого шума практически незаметно). Стоит также отметить, что с помощью низкочастотных фильтров мы можем <<зайти на территорию полосовых>>, убрав дельта-скачки в частотной области за счёт увеличения подавляемой области высоких частот (однако этот метод хоть и имеет место на существование, но является уж слишком варварским и грубым).
	
	Далее на очереди - параметры $\nu_1$ и $h$. С их помощью мы способны регулировать область работы полосового фильтра, расширяя её за счёт увеличения $h$ и двигая изменением $\nu_1$. Может возникнуть ситуация, где мы знаем частоту сторонней гармоники, но с какой-то определенной погрешностью (то есть имеем $x \pm \Delta x$), - изменениями параметров $\nu_1$ и $h$ можно точно добиться подавления именно той частотной области, в которой со 100-процентной вероятностью находится стороннее воздействие, дельта-функция (например, выбрав ширину $h$ чуть выше $\Delta x$, а $\nu_1 = x$). Существует некоторый неизбежный недостаток метода в виде нежелательного подавления важных низких частот (имеет места при хоть сколько-то большой ширине). Избежать его возможно, только если мы по какой-то случайности чётко знаем частоту гармоники (в этом случае можно взять очень малое $h$, которое оставит практически все амлпитуды неизменными, $\nu_1$, которое в точности задаёт точку дельта-пика образа синусоиды, и точечно в спектре очистить функцию). Итак, в идеале необходимо выбирать маленькую ширину $h$ и точечно находить $\nu_1 = \frac{d}{2 \pi}$, где $d$ - частота синусоиды $\sin(dt)$ (у нас в идеале должно быть $\nu_1$ = $\frac{3}{2\pi} \approx 0.48$), а в случае проблем увеличивать $h$ либо двигать $\nu_1$. 
	

	Подкрепим наши слова графиками: результаты работы при $\nu_1 = 1.5$ изображены на рисунках \ref{14} и \ref{15} (произошло смещение подавления - частоты, в которых находится образ Фурье гармоники, остаются нетронутыми; на выходе имеем ту же характерную синусоиду, не измененную ни по амплитуде, ни по фазе). На рисунках \ref{16} и \ref{17} находятся графики модулей образов исследуемых сигналов и сравнения их во временной области при параметрах $\nu_0 = 2$, $\nu_1 = 0.5$, $h = 0.25$, $b = 0.35$, $c = 0.1$ и $d = 3$ (увеличение $h$ привело к захвату важных низких частот, ожидаемо возникли искажениям функции в общем виде функции - пропали пики, в целом потерялась структура). Также отдельно была рассмотрена ситуация, когда $\nu_1 = \frac{d}{2 \pi}$, а $h = 0.01$ (при этом $\nu_0 = 2$, $b = 0.35$, $c = 0.1$ и $d = 3$), то есть происходит точечное уничтожение выброса в частотной обалсти. Результаты на рисунках \ref{18} и \ref{19}. Заметно, что результат практически идентичен, полученному на рисунке \ref{3} при тех же прочих параметрах, где лишние гармоники попросту отсутствовали.
	\begin{figure}
		\centering
		\includegraphics[width=0.95\textwidth]{./images/14.jpg}
		\caption{Образы сигналов при $\nu_0 = 2$, $\nu_1 = 0.5$, $h = 0.1$, $b = 0.35$, $c = 0.1$ и $d = 3$}
		\label{14}
	\end{figure}
	\begin{figure}
		\centering
		\includegraphics[width=0.95\textwidth]{./images/15.jpg}
		\caption{Фильтрация при $\nu_0 = 2$, $\nu_1 = 0.5$, $h = 0.1$, $b = 0.35$, $c = 0.1$ и $d = 3$}
		\label{15}
	\end{figure}
	\begin{figure}
		\centering
		\includegraphics[width=0.95\textwidth]{./images/16.jpg}
		\caption{Образы сигналов при $\nu_0 = 2$, $\nu_1 = 0.5$, $h = 0.35$, $b = 0.35$, $c = 0.1$ и $d = 3$}
		\label{16}
	\end{figure}
	\begin{figure}
		\centering
		\includegraphics[width=0.95\textwidth]{./images/17.jpg}
		\caption{Фильтрация при $\nu_0 = 2$, $\nu_1 = 0.5$, $h = 0.35$, $b = 0.35$, $c = 0.1$ и $d = 3$}
		\label{17}
	\end{figure}
	\begin{figure}
		\centering
		\includegraphics[width=0.95\textwidth]{./images/18.jpg}
		\caption{Образы при $\nu_0 = 2$, $\nu_1 \approx 0.48$, $h = 0.015$, $b = 0.35$, $c = 0.1$ и $d = 3$}
		\label{18}
	\end{figure}
	\begin{figure}
		\centering
		\includegraphics[width=0.95\textwidth]{./images/19.jpg}
		\caption{Фильтрация при $\nu_0 = 2$, $\nu_1 \approx 0.48$, $h = 0.01$, $b = 0.35$, $c = 0.1$ и $d = 3$}
		\label{19}
	\end{figure}

	Дальше исследуем влияние каждого из параметров, задающих шумовое воздействие. Увеличение $b$, как уже было изучено, влечёт большие амплитуды белого шума (графики образов сигналов и их вида во временной области при изменении параметра до $b = 2$ представлены на рисунках \ref{20} и \ref{21} соотвественно; прочие параметры при этом равны $\nu_0 = 2$, $\nu_1 = 0.5$, $h = 0.1$, $c = 0.1$ и $d = 3$). Всё как раньше: при мелких $b$ бы вольны задавать малые диапазоны глушащихся частот, так как шум вносит свой вклад лишь на самых мелких амплитудах, при больших - вынуждены расширяться (рассмотрено ранее). Как и прежде, при высоких значениях шумов восстановление явно хромает (страдают даже низкие частоты с высокой амплитудой).
	\begin{figure}[h]
		\centering
		\includegraphics[width=0.95\textwidth]{./images/20.jpg}
		\caption{Образы при $\nu_0 = 2$, $\nu_1 = 0.5$, $h = 0.1$, $b = 2$, $c = 0.1$ и $d = 3$}
		\label{20}
	\end{figure}
	\begin{figure}
		\centering
		\includegraphics[width=0.95\textwidth]{./images/21.jpg}
		\caption{Фильтрация при $\nu_0 = 2$, $\nu_1 = 0.5$, $h = 0.1$, $b = 2$, $c = 0.1$ и $d = 3$}
		\label{21}
	\end{figure}
	\begin{figure}
		\centering
		\includegraphics[width=0.95\textwidth]{./images/22.jpg}
		\caption{Образы при $\nu_0 = 2$, $\nu_1 \approx 0.48$, $h = 0.01$, $b = 0$, $c = 0.1$ и $d = 3$}
		\label{22}
	\end{figure}
	\begin{figure}
		\centering
		\includegraphics[width=0.95\textwidth]{./images/23.jpg}
		\caption{Фильтрация при $\nu_0 = 2$, $\nu_1 \approx 0.48$, $h = 0.01$, $b = 0$, $c = 0.1$ и $d = 3$}
		\label{23}
	\end{figure}

	Отдельного внимания заслуживает случай $b = 0$. В этой ситуации случайный шум отсутствует (а значит, низкочастотных фильтр нам ни к чему), и можно практически неизменно восстановить исходынй сигнал, точечно подавив присутствующую гармонику. Результаты при параметрах $\nu_0 = 2$, $\nu_1 \approx 0.48$, $h = 0.01$, $b = 0$, $c = 0.1$ и $d = 3$ - на рисунках \ref{22} и \ref{23}.

	
	Перейдём теперь к исследованию гармоники. Параметр $d$ задаёт её частоту, $c$ - амплитуду. Уже было получно, что синусоидальное воздействие отображается в частотной области в виде дельта-пика с аргументом, равным частоте гармоники, то есть $d$. Получается, изменение $d$ <<двигает>> эту вершину, а при большом значении этого параметра мы можем даже перейти в случай высокочастотной помехи, которую будет подавлять уже не полосовой фильтр, а низкочастотный. Итак, графики при $d = 6$ представлены на рисунках \ref{24} и \ref{25} (на образе произошло смещение дельта-функции, поэтому полосовой фильтр не смог подавить возникающий выброс; синусоидальное воздействие при фильтрации сохранилось). Также зададимся достаточно высоким $d = 15$, и попробуем использовать только низкочастотный фильтр вместе с параметрами $\nu_0 = 2$, $b = 0.35$, $c = 0.1$ (рисунки \ref{26} и \ref{27}). Видим, что пики задаются уже на высоких частотах, а значит, использование полосового фильтра не так уж и необходима.
	\begin{figure}
		\centering
		\includegraphics[width=0.95\textwidth]{./images/24.jpg}
		\caption{Образы при $\nu_0 = 2$, $\nu_1 = 0.5$, $h = 0.1$, $b = 0.35$, $c = 0.1$ и $d = 6$}
		\label{24}
	\end{figure}
	\begin{figure}
		\centering
		\includegraphics[width=0.95\textwidth]{./images/25.jpg}
		\caption{Фильтрация при $\nu_0 = 2$, $\nu_1 = 0.5$, $h = 0.1$, $b = 0.35$, $c = 0.1$ и $d = 6$}
		\label{25}
	\end{figure}
	\begin{figure}
		\centering
		\includegraphics[width=0.95\textwidth]{./images/26.jpg}
		\caption{Образы при $\nu_0 = 2$, $b = 0.35$, $c = 0.1$ и $d = 15$}
		\label{26}
	\end{figure}
	\begin{figure}
		\centering
		\includegraphics[width=0.95\textwidth]{./images/27.jpg}
		\caption{Фильтрация при $\nu_0 = 2$, $b = 0.35$, $c = 0.1$ и $d = 15$}
		\label{27}
	\end{figure}
	\begin{figure}
		\centering
		\includegraphics[width=0.95\textwidth]{./images/28.jpg}
		\caption{Образы при $\nu_0 = 2$, $\nu_1 = 0.5$, $h = 0.1$, $b = 0.35$, $c = 0.75$ и $d = 3$}
		\label{28}
	\end{figure}
	\begin{figure}
		\centering
		\includegraphics[width=0.95\textwidth]{./images/29.jpg}
		\caption{Фильтрация при $\nu_0 = 2$, $\nu_1 = 0.5$, $h = 0.1$, $b = 0.35$, $c = 0.75$ и $d = 3$}
		\label{29}
	\end{figure}
	\begin{figure}
		\centering
		\includegraphics[width=0.95\textwidth]{./images/30.jpg}
		\caption{Образы при $\nu_0 = 2$, $\nu_1 = 0.5$, $h = 0.1$, $b = 0.35$, $c = 0.05$ и $d = 3$}
		\label{30}
	\end{figure}
	\begin{figure}
		\centering
		\includegraphics[width=0.95\textwidth]{./images/31.jpg}
		\caption{Фильтрация при $\nu_0 = 2$, $\nu_1 = 0.5$, $h = 0.1$, $b = 0.35$, $c = 0.05$ и $d = 3$}
		\label{31}
	\end{figure}
	
	Параметр $c$ влияет на амплитуду сторонней гармоники. По рисункам \ref{28} и \ref{29}, построенных при параметрах $\nu_0 = 2$, $\nu_1 = 0.5$, $h = 0.1$, $b = 0.35$, $c = 0.75$ и $d = 3$, и рисунках \ref{30} и \ref{31}, построенных при параметрах $\nu_0 = 2$, $\nu_1 = 0.5$, $h = 0.1$, $b = 0.35$, $c = 0.05$ и $d = 3$. Видим, что меньшие амплитуды дают и меньшие пики в частотной области, которые легче и в подавлении (то есть при больших $c$ мы можем быть вынуждены расширять и параметр ширины полосового фильтра $h$, из-за приближений \textbf{fft} дельта-функций). В остальном работа фильтров при вариации параметра $c$ остаётся той же. Можно также заметить, что вариация $c$ не влечёт никакого изменнеия в белом шуме, что подтвержает его независимость с гармоникой.

	\subsection{Убираем низкие частоты?}
	Рассмотрим фильтр, который обнуляет Фурье-образ на всех частотах в некоторой окрестности $[-\hat{\nu}, \hat{\nu}]$ точки $\nu = 0$. Зададимся всё теми же параметрами обрабатываемого сигнала $b = 0.35$, $c = 0.05$ и $d = 3$ и $\hat{\nu} = 1$. Посмотрим, что из этого выйдет. Результаты представлены на рисунках \ref{32} и \ref{33}.
	\begin{figure}[h]
		\centering
		\includegraphics[width=0.95\textwidth]{./images/32.jpg}
		\caption{Образы при $\hat{\nu} = 1$, $b = 0.35$, $c = 0.1$ и $d = 3$}
		\label{32}
	\end{figure}
	\begin{figure}
		\centering
		\includegraphics[width=0.95\textwidth]{./images/33.jpg}
		\caption{Фильтрация при $\hat{\nu} = 1$, $b = 0.35$, $c = 0.1$ и $d = 3$}
		\label{33}
	\end{figure}
	\begin{figure}
		\centering
		\includegraphics[width=0.95\textwidth]{./images/34.jpg}
		\caption{Образы при $\hat{\nu} = 0.1$, $b = 0.35$, $c = 0.1$ и $d = 3$}
		\label{34}
	\end{figure}
	\begin{figure}
		\centering
		\includegraphics[width=0.95\textwidth]{./images/35.jpg}
		\caption{Фильтрация при $\hat{\nu} = 0.1$, $b = 0.35$, $c = 0.1$ и $d = 3$}
		\label{35}
	\end{figure}
	\begin{figure}
		\centering
		\includegraphics[width=0.95\textwidth]{./images/36.jpg}
		\caption{Образы при $\hat{\nu} = 0.25$, $b = 0.35$, $c = 0.1$ и $d = 3$}
		\label{36}
	\end{figure}
	\begin{figure}
		\centering
		\includegraphics[width=0.95\textwidth]{./images/37.jpg}
		\caption{Фильтрация при $\hat{\nu} = 0.25$, $b = 0.35$, $c = 0.1$ и $d = 3$}
		\label{37}
	\end{figure}

	Как видим, выделяются только высокочастотные составляющие сигнала (в нашем случае выделились белый шум и резкие переходы). Вариация параметра $\hat{\nu}$ от $0.1$ до $5$ на рисунках \ref{34} и \ref{35}, а также \ref{36} и \ref{37} (при прочих неизменных). Соотвественно, чем больше $\hat{\nu}$, тем больший диапазон подавления низких частот применяется. Данный фильтр может использоваться в случае наличия низкочастотных помех (расположения самого сигнала при этом должно находиться на более высоких частотах, так как иначе теряется информация и появляются искажения в исходных составляющих) или же просто нежелательных компонент, которые мы хотим подавить, получив некоторую обработку.

	Таким образом, показано, что фильтр нижних частот эффективно устраняет высокочастотный шум, при этом параметр $\nu_0$ критически влияет на степень сглаживания (увеличение влечёт большую силу фильтрации), а параметр $b$ — на форму обрабатываемого сигнала. Применение полосовых фильтров вместе с низкочастотными позволило подавить все существующие помехи, включая гармонику и случайный шум. Отдельный случай $b=0$ в данном контексте также продемонстрировал отсутствие шумовой составляющей и подчеркнул необходимость точной настройки фильтра (так как в этом случае мы смогли практически без искажений восстановить исходный сигнал). Фильтрация же низких частот показала, что даже удаление небольшого диапазона около нуля может сильно исказить общий вид сигнала, особенно при наличии медленно меняющихся компонент.
	
	\section{Фильтрация звука}
	Скачаем файл \textbf{MUHA.wav}. При его прослушивании был замечен голос и шумы, одним из которых был явный низкий гул. Сделаем так, чтобы остался только голос. Для этого для начала найдём образ Фурье аудиофайла:
	\begin{figure}[h]
		\centering
		\includegraphics[width=0.95\textwidth]{./images/38.jpg}
		\caption{Модуль Фурье-образа аудиозаписи}
		\label{38}
	\end{figure}

	В центре видны явные отклонения - голос такими амплитудами частот не обладает. Рисунок \ref{39} показывает, что данные помехи находятся на отрезке $[-300, 300]$ ($\hat{\nu} = 300$), то есть мы можем их подавить изученным в предыдущем пункте фильтром верхних частот. Сделав данное, мы уберём тот самый постоянный низкий гул.
	\begin{figure}
		\centering
		\includegraphics[width=0.95\textwidth]{./images/39.jpg}
		\caption{Модуль Фурье-образа аудиозаписи (приближение в центре)}
		\label{39}
	\end{figure}
	\begin{figure}
		\centering
		\includegraphics[width=0.95\textwidth]{./images/42.jpg}
		\caption{График исходной аудиозаписи}
		\label{42}
	\end{figure}

	Далее, на записи были слышны высокочастотные помехи (что-то вроде жужжания сверчков). Подтвердают это и графики: на исходная запись явно сильно колеблется (рисунок \ref{42}). Экспериментально было получено, что фильтр низких частот при $\nu_0 = 5000$ успешно их уничтожает, поэтому будем использовать его.

	По итогу, обработка аудиозаписи свелась к последовательному применению фильтра низких частот ($\hat{\nu} = 300$), а после фильтра верхних частот ($\nu_0 = 5000$). В конце остался только чистый голос мужчины, говорящего: <<муха - это маленькая птичка>>. Сравнение образов фильтрованной записи и исходной приведено на рисуне \ref{40}, сравнение фильтрации во временной области - на рисунке \ref{41}.
	\begin{figure}[h]
		\centering
		\includegraphics[width=0.95\textwidth]{./images/40.jpg}
		\caption{Сравнение модулей Фурье-образов}
		\label{40}
	\end{figure}
	\begin{figure}
		\centering
		\includegraphics[width=0.95\textwidth]{./images/41.jpg}
		\caption{Сравнение сигналов во временной области}
		\label{41}
	\end{figure}

	\section{Общие выводы}
	В ходе лабораторной работы было проведено исследование различных методов жесткой фильтрации сигналов на основе преобразования Фурье. Были изучены эффекты удаления высоких, специфических и низких частот, а также проанализировано влияние параметров фильтров на качество восстановления сигнала. На каждом этапе также выполнено сравнение образов исходного, зашумленного и отфильтрованного сигналов, что позволило более точно выявить оптимальные значения параметров фильтрации. Кроме того, успешно выполнена фильтрация звукового сигнала, при которой удалось отделить голос от шума высокого и низкого шумов. В общем, работа показала эффективность жёсткой фильтрация для подавления помех самой разной природы.

	\newpage
	\section{Приложение}
	\begin{lstlisting}[caption={Код для жёсткой фильтрации}]
a = 1;
t1 = -3;
t2 = 3;
T = 100;
dt = 0.01;
t = -T/2:dt:T/2;
N = length(t);
g = zeros(size(t));
g(t >= t1 & t <= t2) = a;

c = 0.1;
b = 0.35;
d = 3;
u = g + b*(rand(size(t))-0.5) + c*sin(d*t);

%%
G = fftshift(fft(g));
U = fftshift(fft(u));
V = 1/dt;
dnu = 1/T;
nu = -V/2:dnu:V/2;
U_filt = U;

%%
n0 = 2;
U_filt(nu < -n0 | nu > n0 ) = 0;

%%
n1 = 0.5;
h = 0.1;
U_filt(nu < n1 + h & nu > n1 - h ) = 0;
U_filt(nu < -n1 + h & nu > -n1 - h ) = 0;

%%
hatnu = 5;
U_filt(nu > -hatnu & nu < hatnu) = 0;

%%
u_filtered = ifft(ifftshift(U_filt));
	\end{lstlisting}

	\newpage
	\begin{lstlisting}[caption={Код для фильтрации аудиозаписи}]
[y, f] = audioread('MUHA.wav');
%sound(y, f);

dt = 1/f;
T = length(y)*dt;
t = 0:dt:T-dt;

Y = fftshift(fft(y));
magY = abs(Y);
N = length(Y);
nu = (-N/2:N/2-1)*(f/N);

hatnu = 5;
n1 = 300;
n2 = 5000;
Y_filtered = Y;
Y_filtered(nu > -n1 & nu < n1) = 0;
Y_filtered(nu < -n2 | nu > n2) = 0;

y_filtered = real(ifft(ifftshift(Y_filtered)));
sound(y_filtered, f);
	\end{lstlisting}


	 
	
	
	
	
\end{document}