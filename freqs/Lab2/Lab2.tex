\documentclass[a4paper,hidelinks,14pt]{extarticle}

\usepackage[utf8]{inputenc}
\usepackage[T2A]{fontenc}
\usepackage[english, russian]{babel}
\usepackage{lipsum}
\usepackage{amsmath}
\usepackage{amssymb}
\usepackage{amsfonts}
\usepackage{mathtools}
\usepackage{datetime}
\usepackage[pdftex]{graphicx}
\usepackage{indentfirst}
\usepackage{asymptote}
\usepackage{systeme}
\usepackage[dvipsnames]{xcolor}
\usepackage{lastpage}
\usepackage{fancybox,fancyhdr}
\usepackage{hyperref}
\usepackage[framed,autolinebreaks,useliterate,numbered]{mcode}
\usepackage[font={small,it}]{caption}
\fancyhead[L]{Лабораторная работа №2}
\fancyhead[C]{}
\fancyhead[R]{\textit{Преобразование Фурье}}
\fancyfoot[L]{}
\fancyfoot[C]{Страница \thepage\space из \pageref{LastPage}}
\fancyfoot[R]{}
\pagestyle{fancy}
\newcommand{\gt}{\textgreater}
\newcommand{\lt}{\textless}

\begin{document}
	\begin{titlepage}
		\setlength{\parindent}{0ex}
		
		\begin{center}
			\textsc{
				\vspace{1ex}
				Научно исследовательский университет ИТМО \\
				\vspace{0.5ex}
				Факультет систем управления и робототехники \\
				\vspace{0.5ex}
			}
		\end{center}
		
		\vspace{50mm}
		
		\begin{center}
			Отчет по лабораторной работе №2 \\
			Преобразование Фурье
		\end{center}
		
		\vspace{50mm}
		
		\begin{minipage}{.48\linewidth}
			Выполнил студент группы R3380
			
			Преподаватели
		\end{minipage}
		\hfill
		\begin{minipage}{.5\linewidth}
			\begin{flushright}
				Мовчан И. Е.
				\\
				Пашенко А.В., Перегудин А. А.
			\end{flushright}
		\end{minipage}
		
		\vfill
		\begin{center}
			Санкт-Петербург
			\\
			2025
		\end{center}
	\end{titlepage}
	\tableofcontents
	\clearpage
	
	\section{Вещественное}
	Будем рассматривать функции $f:\mathbb{R}\rightarrow\mathbb{R}$, а использовать унитарное преобразование Фурье к угловой частоте $\omega$, то есть
	$$
	f(t) = \frac{1}{\sqrt{2\pi}}\int_{-\infty}^{\infty}\hat{f}(\omega) e^{i\omega t} d\omega,
	$$
	$$
	\hat{f}(\omega) = \frac{1}{\sqrt{2\pi}}\int_{-\infty}^{\infty}f(t) e^{-i\omega t} dt,
	$$
	где $\hat{f}(\omega)$ задаёт Фурье-образ. Рассмотрим каждую по отдельности.
	
	\subsection{Прямоугольная функция}
	Функция имеет следующий вид ($a$, $b > 0$):
	$$
	f(t) =
	\left\{ 
	\begin{array}{l}
		a,\ |t| \le b, \\
		0,\ |t| > b.
	\end{array} 
	\right.
	$$

	Её вид представляет из себя П-образный симметричный <<горбик>> (собственно, именно поэтому она так и называется). Но об этом всём после, для начала же найдём Фурье-образ нашей функции:
	$$
	\hat{f}(\omega) = \frac{1}{\sqrt{2\pi}}\int_{-b}^{b} a e^{-i\omega t}dt = -\frac{a}{\sqrt{2\pi}}\cdot\frac{e^{-i\omega t}}{i\omega} \bigg|_{t=-b}^{t=b}= \frac{a}{\sqrt{2\pi}}\cdot\frac{e^{i\omega b}-e^{-i\omega b}}{i\omega} = 
	$$
	
	$$
	= a\sqrt{\frac{2}{\pi}}\frac{\sin(\omega b)}{\omega} = ab\sqrt{\frac{2}{\pi}}\text{sinc}(\omega b) = \frac{2 a b}{\sqrt{2\pi}}\mathrm{sinc}(wb).
	$$

	\begin{figure}
		\centering
		\includegraphics[width=0.95\textwidth]{./images/1.jpg}
		\caption{Сравнение прямоугольных функций}
		\label{1}
	\end{figure}
	\begin{figure}
		\centering
		\includegraphics[width=0.95\textwidth]{./images/2.jpg}
		\caption{Сравнение образов Фурье прямоугольных функций}
		\label{2}
	\end{figure}
	Параметр $ab$ в нашем случае отвечает за амплитуду колебаний графика Фурье-образа (чем он больше, тем выше <<размах>>, например, в 0), $b$ - за частоту (чем больше $b$, тем чаще происходят сами колебания, присутствующие в знаменателе, синусе). Данное подтверждает и графиками на рисунках \ref{1} и \ref{2} при параметрах $a = 1$ и $b = 1$, $a = 1$ и $b = 2$, $a = 2$ и $b = 1$. При вариации $a$ вдвое увеличись значения как во временной оси, так и в частотной (при этом частота Фурье-образа, как и ожидалось, никак не изменилась), та же вариация, но параметра $b$, увеличила как амплитуду, так и частоту спектра, а также ширину задаваемого сигнала по времени.
	
	С помощью свойств преобразования Фурье данные явления можно объяснить следующим образом (вторая формула демонстрирует принцип неопределённости и говорит о том, что сужение временной области в $b$ раз даёт расширение частотной с тем же коэффициентом, к тому же увеличивает ещё амплитуду образа; и наоборот, расширение временной области даёт сужение частотной, то есть в нашем случае увеличением параметра $b$ мы добились расширения по времени, а значит и сужения частототной области, увеличения частоты $\mathrm{sinc}$ и его амплитуды; $F$ - оператор преобразования Фурье):
	$$
	F\{af(t)\} = a F\{f(t)\}, \hspace{2mm} F\{f(bt)\} = \frac{1}{b} \hat{f}\Big(\frac{\omega}{b}\Big), \hspace{2mm} \hat{f}(\omega) = F\{f(t)\}.
	$$

	Проверим также равенство Парсеваля, то есть выполняется ли
	$$
		E_{time} = \int_{-\infty}^{\infty} |x(t)|^2 \, dt = \int_{-\infty}^{\infty} \left| \mathcal{F}\{x(t)\} \right|^2 \, d\omega = E_{freq}.
	$$
	
	Будем делать это численными способами, для каждого из наборов параметров, используя функцию \textbf{trapz} в matlab:
	\begin{itemize}
		\item $a = 1$, $b = 1$: $E_{time} = 2.0021$, $E_{freq} = 1.9975$;
		\item $a = 1$, $b = 2$: $E_{time} = 4.0042$, $E_{freq} = 3.9975$;
		\item $a = 1$, $b = 1$: $E_{time} = 8.0083$, $E_{freq} = 7.9901$.
	\end{itemize}

	Можно видеть, что всё выполнилось, а значит было действительно использовано унитарное преобразование Фурье (так как оно сохраняет <<энергию>> без необходимости дополнительного масштабирования). Небольшие различия можно скинуть на аппаратную часть, так как исследовались лишь приближения интегралов, ограниченные как по шагу, так и по исследуемому промежутку, а не они сами.
	
	\subsection{Треугольная функция}
	Пусть задана функция ($a$, $b > 0$)
	$$
	f(t) =
	\begin{cases}
		a - |at/b|, & |t| \leq b, \\
		0, & |t| > b;
		\end{cases}
	\hspace{3mm}
	=
	\begin{cases}
	a - a|t|/b, & |t| \leq b, \\
	0, & |t| > b.
	\end{cases}
	$$

	Как и прежде, найдём её преобразование. Поскольку исследуемая функция чётная, её образ Фурье выражается через только через преобразование, использующее косинус, а рассматривать можем только положительную временную ось, предварительно умножив на 2:
	\[
	\hat{f}(\omega) = \frac{2}{\sqrt{2\pi}} \int_{0}^{b} \left(a - \frac{a}{b} t\right) \cos(\omega t) = \frac{2a}{\sqrt{2\pi}} \int_{0}^{b} \left(1 - \frac{t}{b} \right) \cos(\omega t) dt.
	\]
	
	Раскроем интеграл как
	\[
	\int_{0}^{b} \left(1 - \frac{t}{b} \right) \cos(\omega t) dt
	= \int_{0}^{b} \cos(\omega t) dt - \frac{1}{b} \int_{0}^{b} t \cos(\omega t) dt
	\]

	Первый случай совсем простой:
	\[
	\int_{0}^{b} \cos(\omega t) dt = \frac{\sin(\omega t)}{\omega} \bigg|_0^b = \frac{\sin(\omega b)}{\omega}
	\]

	А вот со вторым придётся повозиться, используя интегрирование по частям (пусть \( u = t \), \( dv = \cos(\omega t) dt \), тогда $du = dt, v = \sin(\omega t)/\omega$):
	\[
	\int_0^b t \cos(\omega t) dt = \frac{t \sin(\omega t)}{\omega}\bigg|_0^b - \int_0^b \frac{\sin(\omega t)}{\omega} dt = 
	\frac{t \sin(\omega t)}{\omega}\bigg|_0^b + \frac{\cos(\omega t)}{\omega^2}\bigg|_0^b
	\]

	Откуда:
	\[
	\int_0^b t \cos(\omega t) dt = \left(\frac{t \sin(\omega t)}{\omega} + \frac{\cos(\omega t)}{\omega^2}\right)\bigg|_0^b = \frac{b \sin(\omega b)}{\omega} + \frac{\cos(\omega b) - 1}{\omega^2}
	\]

	В итоге интеграл сводится к следующему:
	$$
	\int_{0}^{b} \left(1 - \frac{t}{b} \right) \cos(\omega t) dt = \frac{\sin(\omega b)}{\omega} - \frac{1}{b}\Big(\frac{b \sin(\omega b)}{\omega} + \frac{\cos(\omega b) - 1}{\omega^2}\Big) =
	$$
	$$
	= \frac{1 - \cos(\omega b)}{b \omega^2} = \frac{2\sin^2(\omega b/2)}{b \omega^2} = \frac{\sin^2(\omega b/2)}{\omega^2 b/2} = \frac{b}{2} \cdot \frac{\sin^2(\omega b/2)}{\omega^2 b^2/4} =
	$$
	$$
	= \frac{b}{2}\left(\mathrm{sinc}^2(\omega b/2)\right).
	$$

	И тогда конечный интеграл принимает форму
	$$
	\hat{f}(\omega) = \frac{2a}{\sqrt{2\pi}} \cdot \frac{b}{2}\left(\mathrm{sinc}^2(\omega b/2)\right) = \frac{ab}{\sqrt{2 \pi}} \mathrm{sinc}^2(wb/2).
	$$
	
	Действие параметров $a$ и $b$ схоже с ситуацией прямоугольных функций. Посмотрим на это, построив графики треугольных во временной области и соответствующих им Фурье-образов в частотной при все тех же наборах параметров $a = 5$ и $b = 5$, $a = 5$ и $b = 6$, $a = 6$ и $b = 5$. Результаты показаны на рисунках \ref{3} и \ref{4}.
	\begin{figure}[h]
		\centering
		\includegraphics[width=0.95\textwidth]{./images/3.jpg}
		\caption{Сравнение треугольных функций}
		\label{3}
	\end{figure}
	\begin{figure}
		\centering
		\includegraphics[width=0.95\textwidth]{./images/4.jpg}
		\caption{Сравнение образов Фурье треугольных функций}
		\label{4}
	\end{figure}
	
	Можем видеть, что параметр \( a \) масштабирует амплитуду функции и её образа (возрастание одного влечёт возрастание другого). Параметр $b$ же влияет на ширину треугольной функции: при увеличении \( b \): \( f(t) \) становится шире и ниже по высоте, а \(\hat{f}(\omega)\)— уже и выше, а при уменьшении \( f(t) \) становится уже и выше, спектр \(\hat{f}(\omega)\) — шире и ниже (здесь аналогично предыдущему пункту работает принцип неопределённости - сжатие временной области влечёт растягивание частотной и наоборот). Данное можно также записать через уже знакомые нам свойства преобразований:
	$$
	F\{af(t)\} = a F\{f(t)\}, \hspace{2mm} F\{f(bt)\} = \frac{1}{b} \hat{f}\Big(\frac{\omega}{b}\Big), \hspace{2mm} \hat{f}(\omega) = F\{f(t)\}.
	$$

	Теперь проверим равенство Парсеваля, формула уже была дана, так что приводить её заново не будем:
	\begin{itemize}
		\item $a = 5$, $b = 5$: $E_{time} = 83.3332$, $E_{freq} = 83.3333$;
		\item $a = 5$, $b = 6$: $E_{time} = 99.9999$, $E_{freq} = 100.0000$;
		\item $a = 6$, $b = 5$: $E_{time} = 119.9998$, $E_{freq} = 120.0000$.
	\end{itemize}
	
	Значения очень близки, так что теорема о сохранении норм, квадраты которых мы и сравниваем, работает, и равенство достигается!

	\subsection{Кардинальный синус}
	Теперь рассмотрим функцию (по прежнему $a$, $b > 0$)
	\[
	f(t) = a \cdot \text{sinc}(bt) = a \cdot \frac{\sin(bt)}{bt}.
	\]

	Известно, что образ Фурье кардинального синуса (в этом определении, без обычно вводящихся коэффициентов $\pi$ в аргументе) имеет следующий вид:
	\[
	F\left\{b \sqrt{\frac{2}{\pi}} \mathrm{sinc}(bt) \right\} = \text{П}\left( \frac{\omega}{2b} \right),
	\]
	где прямоугольная функция $\text{П}$ определяется как:
	\[
	\text{П}\left( \frac{\omega}{2b} \right) =
	\begin{cases}
	1, & |\omega| \leq b, \\
	0, & |\omega| > b.
	\end{cases}
	\]

	Откуда по свойствам вынесения постоянной можно получить:
	$$
	F\left\{\mathrm{sinc}(bt) \right\} = \frac{1}{b} \cdot \sqrt{\frac{\pi}{2}} \text{П}\left( \frac{\omega}{2b} \right),
	$$

	А учитывая амплитудный коэффициент $a$, имеем:
	\[
	\hat{f}(\omega) = \frac{a}{b} \cdot \sqrt{\frac{\pi}{2}} \text{П}\left( \frac{\omega}{2b} \right).
	\]

	\begin{figure}
		\centering
		\includegraphics[width=0.95\textwidth]{./images/5.jpg}
		\caption{Сравнение кардинальных синусов во временной области}
		\label{5}
	\end{figure}
	\begin{figure}
		\centering
		\includegraphics[width=0.95\textwidth]{./images/6.jpg}
		\caption{Сравнение образов Фурье кардинальных синусов}
		\label{6}
	\end{figure}
	Ситуация с параметрами здесь чуть иная (преимущественно в Фурье-образе), поэтому давайте сперва зададимся ими и проанализируем графики при вариации каждого из них, чтобы точно понять, что же в нашем случае происходит. Итак, графики самих функций и их спектров при параметрах $a = 5$ и $b = 5$, $a = 5$ и $b = 8$, $a = 8$ и $b = 5$ изображены на рисунках \ref{5} и \ref{6}.

	Получается, параметр $a$ масштабирует амплитуду функции $f(t)$ и $\hat{f}(\omega)$ линейно. $b$ же обратно влияет на ширину функции во времени (увеличивает её частоту при возрастании) и прямо пропорционально в спектре (опять-таки сужение одного растягивает другое), также при увеличении параметра $b$ снижается амплитуда получающейся при преобразовании прямоугольной функции. Всё это опять-таки можно описать с помощью формул свойств (второе здесь даже работает более явно, чем в предыдущих случаях, где нужно было ещё покопаться, чтобы найти его):
	$$
	F\{af(t)\} = a F\{f(t)\}, \hspace{2mm} F\{f(bt)\} = \frac{1}{b} \hat{f}\Big(\frac{\omega}{b}\Big), \hspace{2mm} \hat{f}(\omega) = F\{f(t)\}.
	$$

	Проверка равенства Парсеваля осуществаляется всё тем же способом, результаты ниже (всё сошлось до машинной точности):
	\begin{itemize}
		\item $a = 5$, $b = 5$: $E_{time} = 15.6980$, $E_{freq} = 15.7411$;
		\item $a = 5$, $b = 8$: $E_{time} = 9.8136$, $E_{freq} = 9.8112$;
		\item $a = 8$, $b = 5$: $E_{time} = 40.1868$, $E_{freq} = 40.2971$.
	\end{itemize}
	
	Вообще, вся эта ситуация очень связана с прямоугольной функцией, рассмотренной ранее, так как одно переходит в другое и обратно. Соответственно, наблюдаются и похожие общие для всех преобразований явления и проявления свойств.

	\subsection{Функция Гаусса}
	Пусть дана функция при положительных параметрах $a$ и $b$
	\[
	f(t) = a e^{-b t^2}.
	\]
	
	Подставим $f(t)$ в формулу унитарного преобразования Фурье к угловой частоте, приведенную в начале (также вынесем $a$):
	\[
	\hat{f}(\omega) = \frac{a}{\sqrt{2\pi}} \int_{-\infty}^{\infty} e^{-b t^2} e^{-i \omega t} dt = \frac{a}{\sqrt{2\pi}} \int_{-\infty}^{\infty} e^{-b t^2 - i \omega t} dt.
	\]
	
	Преобразуем выражение, заданное в экспоненте:
	\[
	-b t^2 - i \omega t = -b \left(t^2 + \frac{i \omega}{b} t\right) = -b \left(t + \frac{i \omega}{2b} \right)^2 + \frac{\omega^2}{4b}.
	\]
	
	Тогда имеем:
	\[
	\hat{f}(\omega) = \frac{a}{\sqrt{2\pi}} e^{-\frac{\omega^2}{4b}} \int_{-\infty}^{\infty} e^{-b \left(t + \frac{i \omega}{2b} \right)^2} dt.
	\]
	
	Теперь заметим, что смещение на комплексное число никак не влияет на значение оставшегося интеграла (так как мы всегда можем занести его в $dt$ в виде постоянной), и следовательно, пользуясь знаниями из теории вероятности, получаем:
	\[
	\int_{-\infty}^{\infty} e^{-b \left(t + \frac{i \omega}{2b} \right)^2} dt = \int_{-\infty}^{\infty} e^{-b t^2} dt = \sqrt{\frac{\pi}{b}}.
	\]
	
	А итоговое перепишется в виде:
	\[
	\hat{f}(\omega) = \frac{a}{\sqrt{2\pi}} e^{-\frac{\omega^2}{4b}} \cdot \sqrt{\frac{\pi}{b}} = \frac{a}{\sqrt{2b}} e^{-\frac{\omega^2}{4b}} = \frac{a}{\sqrt{2b}} e^{-\left(\frac{\omega}{2\sqrt{b}}\right)^2}.
	\]
	
	Рассмотрим три набора параметров: $a = 1$ и $b = 1$, $a = 1$ и $b = 4$, $a = 2$ и $b = 1$. Соответствующие им функция задаются уж очень красиво (по моим меркам), так что приведём их отдельно:
	\begin{itemize}
		\item при $a = 1$ и $b = 1$: $f(t) = e^{-t^2}$, а $\hat{f}(\omega) = \frac{1}{\sqrt{2}} e^{-\omega^2/4}$;
		\item при $a = 1$ и $b = 4$: $f(t) = e^{-4t^2}$, а $\hat{f}(\omega) = \frac{1}{2} e^{-\omega^2/16}$;
		\item при $a = 2$ и $b = 1$: $f(t) = 2 e^{-t^2}$, а $\hat{f}(\omega) = \sqrt{2} e^{-\omega^2/4}$.
	\end{itemize}

	\begin{figure}
		\centering
		\includegraphics[width=0.95\textwidth]{./images/7.jpg}
		\caption{Сравнение Гауссовых функций во временной области}
		\label{7}
	\end{figure}
	\begin{figure}
		\centering
		\includegraphics[width=0.95\textwidth]{./images/8.jpg}
		\caption{Сравнение образов Фурье функций Гаусса}
		\label{8}
	\end{figure}
	Их графики по времени и образа по угловой частоте изображены на рисунках \ref{7} и \ref{8}. Сравнение наборов 1 и 2 даёт типичное проявление принципа неопределённости Фурье - при увеличении $b$ получаем более узкую по времени функцию, зато широкую по частоте (а также с меньшей амплитудой), то есть $b$ управляет шириной функции. Сравнение же 1 и 3 позволяет понять роль параметра $a$ - он лишь увеличивает амплитуды, никак не воздействую на быстроту затухания. В общем, всё согласуется с уже приведёнными свойствами Фурье: широкие по времени функции переходят в сжатые по частоте, а амплитуда может быть вынесена по свойствам интеграла, увеличив его значение соответствуетствующим образом.

	Проверим равенство Парсеваля для всех случаев:
	\begin{itemize}
		\item $a = 1$, $b = 1$: $E_{time} = 1.2533$, $E_{freq} = 1.2533$;
		\item $a = 1$, $b = 4$: $E_{time} = 0.6267$, $E_{freq} = 0.6267$;
		\item $a = 2$, $b = 1$: $E_{time} = 5.0133$, $E_{freq} = 5.0133$.
	\end{itemize}
	
	Все значения получились равны (конечно, с точностью до округления)! Скорее всего, это связано с тем, что функция очень гладкая, быстро затухает, а её основные не близкие к нуля значения расположены возле 0 (соотвественно, необходимо брать меньшую область по времени для интегрирования).

	Вообще, исследуемая функция известна нам с теории вероятности. Её замечательное свойство (проявляющееся у очень немногих) состоит в том, что образ Фурье является её же <<сородичем>>, то есть тоже Гауссовой функцией. А при определённых параметрах (например, при $a = 1$, а $b = 1/2$) она может перейти сама в себя! 

	\subsection{Двустороннее затухание}
	Наконец зададимся функцией при $a > 0$, $b > 0$
	\[
	f(t) = ae^{-b|t|}.
	\]

	Найдём её Фурье-образ:
	\[
	\hat{f}(\omega) = \frac{1}{\sqrt{2\pi}} \int_{-\infty}^{\infty} a e^{-b|t|} e^{-i\omega t} dt
	\]

	Раскрывая модуль, разделим интеграл по областям:
	\[
	\hat{f}(\omega) = \frac{a}{\sqrt{2\pi}} \left( \int_{-\infty}^0 e^{bt} e^{-i\omega t} dt + \int_0^{\infty} e^{-bt} e^{-i\omega t} dt \right)
	\]

	Преобразуем каждый из кусочков и найдём спектр:
	\[
	\hat{f}(\omega) = \frac{a}{\sqrt{2\pi}} \left( \int_{-\infty}^0 e^{(b - i\omega)t} dt + \int_0^{\infty} e^{-(b + i\omega)t} dt \right) =
	\]
	$$
	=\frac{a}{\sqrt{2\pi}}\left(\left( \frac{e^{(b - i\omega)t}}{b - i\omega} \right)\bigg|_{-\infty}^{0} + \left( \frac{-e^{-(b + i\omega)t}}{b + i\omega} \right)\bigg|_{0}^{\infty} \right) =
	$$
	$$
	= \frac{a}{\sqrt{2\pi}} \left( \frac{1}{b - i\omega} + \frac{1}{b + i\omega} \right) = \frac{a}{\sqrt{2\pi}} \cdot \frac{(b + i\omega) + (b - i\omega)}{b^2 + \omega^2} =
	$$
	$$
	=\frac{a}{\sqrt{2\pi}} \cdot \frac{2b}{b^2 + \omega^2} = \frac{2ab}{\sqrt{2\pi}(b^2 + \omega^2)}.
	$$

	\begin{figure}
		\centering
		\includegraphics[width=0.95\textwidth]{./images/9.jpg}
		\caption{Сравнение функций двустороннего затухания во временной области}
		\label{9}
	\end{figure}
	\begin{figure}
		\centering
		\includegraphics[width=0.95\textwidth]{./images/10.jpg}
		\caption{Сравнение образов Фурье функций двустороннего затухания}
		\label{10}
	\end{figure}
	Теперь аналогично всем предыдущим пунктам зададимся положительными наборами чисел: $a = 1$ и $b = 1$, $a = 1$ и $b = 4$, $a = 2$ и $b = 1$. Все необходимые графики приведены на рисунках \ref{9} и \ref{10}. При увеличении $b$ функция $f(t)$ становится уже (более сконцентрированной вокруг $t = 0$), спектр становится шире. Параметр $a$ влияет только на амплитуду причем прямо и линейно (соотвественно, чем большие значения $a$ мы берём, тем большие амплитуд выходят).

	Результат интересен также и тем, что из прерывной функции получилась гладкая (так как интегралу как математическому объекту безразлично на значение в одной точке, важна её окрестность). Если же не брать во внимание точку $0$, то $f(t)$ получается абсолютно гладкой, соотвественно, её образ Фурье убывает очень быстро (ровно такая же ситуация возникала и при остальных функциях - это связь гладкости и скорости убывания, и данное работает в две стороны).

	Таким образом, были исследованы влияния параметров при каждом случае задания функции $f(t)$, проверены выполнения некоторых свойств преобразования Фурье, доказано их прямое существование на примерах.

	Проверка равенства Парсеваля для случая двустороннего затухания приведена ниже (всё выполнилось, причём с большой точностью, аналогично ситуации с Гауссовой функцией):
	\begin{itemize}
		\item $a = 1$, $b = 1$: $E_{time} = 1.0000$, $E_{freq} = 1.0000$;
		\item $a = 1$, $b = 4$: $E_{time} = 0.2499$, $E_{freq} = 0.2500$;
		\item $a = 2$, $b = 1$: $E_{time} = 3.9999$, $E_{freq} = 4.0000$.
	\end{itemize}


	\section{Комплексное}
	Выберем Гауссову функцию $f(t)$ с параметрами $a = 1$ и $b = 1$ и рассмотрим сдвинутую функцию $g(t) = f(t+c)$ с параметром $c$, то есть зададимся
	$$
	g(t) = e^{-(t+c)^2}.
	$$

	Аналитический вывод её образа следующий (обозначим $u = t + c$, тогда $t = u - c$, и $dt = du$ и сделаем замену переменной):
	\[
	\hat{g}(\omega) = \frac{1}{\sqrt{2\pi}}\int_{-\infty}^{\infty} e^{-(t + c)^2} e^{-i \omega t} \, dt = 
	\frac{1}{\sqrt{2\pi}} \int_{-\infty}^{\infty} e^{-u^2} e^{-i \omega (u - c)} \, du =
	\]
	\[
	= \frac{e^{i \omega c}}{\sqrt{2\pi}} \int_{-\infty}^{\infty} e^{-u^2} e^{-i \omega u} \, du = e^{i \omega c} \cdot \hat{f}(\omega) = \frac{e^{i \omega c}}{\sqrt{2}}e^{-\omega^2/4} =
	\]
	$$
	= \frac{e^{-\omega^2/4}}{\sqrt{2}}(\cos(\omega c) + i \sin(\omega c)).
	$$

	То есть результат представляет из себя уже комплексную функцию, хотя работали мы с вещественной. Для более детального изучения поэкспериментируем с параметром $c \neq 0$: выберем 4 значения $c_1 = -3$, $c_2 = -1$, $c_3 = 2$, $c_4 = 3$ и построим графики оригиналов $g(t)$, компонент Фурье-образов Re$(\hat{g}\omega)$, Im$(\hat{g}(\omega))$ и модулей $|\hat{g}(\omega)|$(соответственно, приведены на рисунках \ref{11}, \ref{12}, \ref{13} и \ref{14}).
	\begin{figure}
		\centering
		\includegraphics[width=0.95\textwidth]{./images/11.jpg}
		\caption{Сравнение функций $g(t)$ при разных $c$}
		\label{11}
	\end{figure}
	\begin{figure}
		\centering
		\includegraphics[width=0.95\textwidth]{./images/12.jpg}
		\caption{Сравнение вещественных частей образа $g(t)$ при разных $c$}
		\label{12}
	\end{figure}
	\begin{figure}
		\centering
		\includegraphics[width=0.95\textwidth]{./images/13.jpg}
		\caption{Сравнение мнимых частей образа $g(t)$ при разных $c$}
		\label{13}
	\end{figure}
	\begin{figure}
		\centering
		\includegraphics[width=0.95\textwidth]{./images/14.jpg}
		\caption{Сравнение модулей образов $g(t)$ при разных $c$}
		\label{14}
	\end{figure}

	Можно видеть, что параметр $c$ сдвигает функцию по оси времени влево (при $c > 0$) или вправо ($c < 0$). Причем тем больше, чем выше значение. Амплитуда и форма при этом остаются теми же.
	
	Вещественные и мнимые части образа Фурье, как можно было понять, тоже зависят от введённого параметра. При больших $|c|$ наблюдаем более сильные колебания в обоих частях, ведь частота синуса и косинуса прямо зависит от $c$ (мнимая экспонента вносит более серьёзный вклад около 0). Также, так как вещественная часть чётная (произведение косинуса на чётную функцию), то и результаты при $c = \pm 3$ получились идентичными, а вот в мнимой части наоборот - противоположными по знаку.
	
	Модули функций при всём этом остались теми же, ведь основное тригонометрическое тождество никуда не пропадало, а амплитуда экспоненты $e^{i \omega c}$ равна 1 (то есть модуль не зависит от $c $ и всегда будет равен исходно взятой $\hat{f}(\omega) = e^{-\omega^2/4}/\sqrt{2}$).

	\section{Музыкальное}
	Пришло время практики! Загрузим запись музыкального аккорда из предложенных данных (рисунок	\ref{15}).
	\begin{figure}
		\centering
		\includegraphics[width=0.95\textwidth]{./images/15.jpg}
		\caption{Запись аккорда}
		\label{15}
	\end{figure}
	\begin{figure}
		\centering
		\includegraphics[width=0.95\textwidth]{./images/16.jpg}
		\caption{Модуль образа Фурье аккорда}
		\label{16}
	\end{figure}
	
	А после найдём его спектр (рисунок \ref{16}). Как известно, преобразование Фурье помогает разложить сложное на гармоники - а в нашем случае поможет понять, что за аккорд сейчас играет. Итак, в модуле образа найдём пики (чем они больше, тем большее участие нота приняла в образовании), аргументы которых как раз и будут давать нужные частоты, соответствующие нотам. Выделенные значения, их частоты и приближенные ноты:
	\begin{itemize}
		\item $v_1$ = 130, $a_1$ = 0.0006, С3;
		\item $v_2$ = 164, $a_2$ = 0.0014, E3;
		\item $v_3$ = 196, $a_3$ = 0.0036, G3;
		\item $v_4$ = 232, $a_4$ = 0.0009, ля-диез малой октавы;
		\item $v_5$ = 262, $a_5$ = 0.0015, C4;
		\item $v_6$ = 294, $a_6$ = 0.0018, D4;
		\item $v_7$ = 330, $a_7$ = 0.0029, E4;
		\item $v_8$ = 350, $a_8$ = 0.0011, F4;
		\item $v_9$ = 440, $a_9$ = 0.0044, A4;
		\item $v_{10}$ = 588, $a_{10}$ = 0.0008, D5;
	\end{itemize}
    
	Сопоставление с таблицей нот-частот показало, что сигнал представляет собой Dm11 аккорд (D–F–A–C–E–G) с возможной добавкой из C мажора (C–E–G).

	\section{Общие выводы}
	В ходе лабораторной работы были исследованы преобразования Фурье для ряда вещественных функций: прямоугольной, треугольной, кардинального синуса, функции Гаусса и функции с двусторонним экспоненциальным затуханием. Для каждой из них были получены образы, построены графики при различных наборах параметров $a$ и $b$, исследовано их влияние, а также проверено выполнение равенства Парсеваля. Во втором задании было изучено влияние временного сдвига на спектр функции, проанализированы изменения модуля, вещественной и мнимой частей образа при изменении параметра $c$. В музыкальной части по построенному спектр записи, были получены основные ноты и аккорды, которые её образуют.

	\newpage
	\section{Приложение}
	\begin{lstlisting}[caption={Код для вещественного}]
T = 100;
N = 16384;
t = linspace(-T, T, N);
dt = t(2) - t(1);
omega = 2*pi*(-N/2:N/2-1)/(N*dt);

params = [1, 1; 1, 4; 2, 1];
colors = ['b', 'r', 'g'];
widths = [13, 9, 4];

...

a = params(i,1);
b = params(i,2);
%f_t = a * double(abs(t) <= b);
    
%f_t = zeros(size(t));
%f_t(abs(t) <= b) = a * (1 - abs(t(abs(t) <= b)) / b);
    
%f_t = a * sinc(b * t/pi);
    
%f_t = a * exp(-b * t.^2);

f_t = a * exp(-b * abs(t));

...

a = params(i,1);
b = params(i,2);
%F_w = 2*a*b/(sqrt(2*pi))*sinc(omega*b/pi);

%F_w = (a*b/sqrt(2*pi))*(sinc(omega*b/(2*pi))).^2;

%F_w = (a/b)*sqrt(pi/2)*double(abs(omega) <= b);

%F_w = a / sqrt(2*b)*exp(-omega.^2 / (4*b));

F_w = (2*a*b) ./ (sqrt(2*pi)*(b^2 + omega.^2));

...

a = 2;
b = 1;
%f_t = a * double(abs(t) <= b);
%F_w = 2*a*b/(sqrt(2*pi))*sinc(omega*b/pi);

%f_t = zeros(size(t));
%f_t(abs(t) <= b) = a * (1 - abs(t(abs(t) <= b)) / b);
%F_w = (a * b / sqrt(2*pi)) * (sinc(omega * b / (2*pi))).^2;

%f_t = a * sinc(b * t/pi);
%F_w = (a / b) * sqrt(pi / 2) * double(abs(omega) <= b);

%f_t = a * exp(-b * t.^2);
%F_w = a * sqrt(1 / (2*b)) * exp(-omega.^2 / (4*b));

f_t = a * exp(-b * abs(t));
F_w = (2*a*b) ./ (sqrt(2*pi)*(b^2 + omega.^2));

E_time = trapz(t, abs(f_t).^2)
E_freq = trapz(omega, abs(F_w).^2)

	\end{lstlisting}
	\newpage
	\begin{lstlisting}[caption = {Код для комплексного}]
T = 100;
N = 16384;
t = linspace(-T, T, N);
dt = t(2) - t(1);
omega = 2*pi*(-N/2:N/2-1)/(N*dt);         

c_values = [-3, -1, 2, 3];
colors = ['b', 'r', 'g', 'cyan'];
widths = [13, 9, 4, 3];

...

c = c_values(i);
g = exp(-(t + c).^2);

...

c = c_values(i);
hatg = exp(1i*omega*c).*exp(-omega.^2/4)/sqrt(2);
rhatg = real(hatg);

...

c = c_values(i);
hatg = exp(1i*omega*c).*exp(-omega.^2/4)/sqrt(2);
imhatg = imag(hatg);

...

c = c_values(i);
hatg = exp(1i*omega*c).*exp(-omega.^2/4)/sqrt(2);
abshatg = abs(hatg)
	\end{lstlisting}
	
	\newpage
	\begin{lstlisting}[caption={Код для музыкального}]
[y, Fs] = audioread('ac.mp3');
y = y(:,1);
y = y(:).';
factor = 2;
y = y(1:factor:end);
Fs = Fs / factor;
t = (0:length(y)-1)/Fs;

V = 1000;
dv = 2;
v = 0:dv:V;

Y = zeros(size(v));
for k = 1:length(v)
    Y(k) = trapz(t, y .* exp(-1i*2*pi*v(k)*t));
end

[peaks, locs] = findpeaks(abs(Y), v, 'MinPeakHeight', 0.125*max(abs(Y)));

peaks
locs

magY = abs(Y);
	\end{lstlisting}

\end{document}