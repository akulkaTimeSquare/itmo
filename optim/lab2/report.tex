\documentclass[a4paper,hidelinks,14pt]{extarticle}

\usepackage[utf8]{inputenc}
\usepackage[T2A]{fontenc}
\usepackage[english, russian]{babel}
\usepackage{lipsum}
\usepackage{amsmath}
\usepackage{amssymb}
\usepackage{amsfonts}
\usepackage{mathtools}
\usepackage{datetime}
\usepackage[pdftex]{graphicx}
\usepackage{indentfirst}
\usepackage{asymptote}
\usepackage{systeme}
\usepackage[dvipsnames]{xcolor}
\usepackage{lastpage}
\usepackage{fancybox,fancyhdr}
\usepackage{hyperref}
\usepackage[font={small,it}]{caption}
\fancyhead[L]{Лабораторная работа №2}
\fancyhead[C]{}
\fancyhead[R]{\textit{Принцип максимума}}
\fancyfoot[L]{}
\fancyfoot[C]{\thepage\space}
\fancyfoot[R]{}
\pagestyle{fancy}
\newcommand{\gt}{\textgreater}
\newcommand{\lt}{\textless}
\usepackage{listings}
\usepackage{xcolor}
\lstset{
    basicstyle=\ttfamily\small,
    keywordstyle=\color{blue},
    commentstyle=\color{gray},
    stringstyle=\color{red},
    numbers=left,
    numberstyle=\color[gray]{0.7}\ttfamily\small,
    stepnumber=1,
    numbersep=8pt,
    frame=single,
    showstringspaces=false,
    tabsize=4,
    breaklines=true
}
\usepackage{subcaption}

\begin{document}
	\begin{titlepage}
		\setlength{\parindent}{0ex}
		
		\begin{center}
			\textsc{
				\vspace{1ex}
                Научно-исследовательский университет ИТМО \\
				\vspace{0.5ex}
				Факультет систем управления и робототехники \\
				\vspace{0.5ex}
			}
		\end{center}
		
		\vspace{45mm}
		
		\begin{center}
			Отчет по лабораторной работе №2\\
			\textbf{Синтез оптимального управления \\
			Принцип максимума} \\
			Вариант 21
		\end{center}
		
		\vspace{50mm}
		
		\begin{minipage}{.45\linewidth}
			Выполнил студент
            \\
			Проверил преподаватель
		\end{minipage}
		\hfill
		\begin{minipage}{.52\linewidth}
			\begin{flushright}
				Мовчан Игорь Евгеньевич, R3480
                \\
				Парамонов Алексей Владимирович
			\end{flushright}
		\end{minipage}
		
		\vfill
		\begin{center}
			Санкт-Петербург
			\\
			2025
		\end{center}
		
	\end{titlepage}

	\tableofcontents
	\clearpage
	
	\section{Построение оптимального регулятора}
	Пусть имеется система:
	\begin{equation}
		\begin{cases}
			\dot{x_1} = x_2 \\
			\dot{x_2} = -3 x_1 - 6 x_2 + u \\
		\end{cases}
	\end{equation}

	Начальное и конечное (в момент времени $t = 1$) состояния:
	\[
		x(0) = \begin{bmatrix}
			0 \\
			0
		\end{bmatrix}, \quad
		x(1) = \begin{bmatrix}
			10 \\
			0
		\end{bmatrix}
	\]

	Минимизируемый критерий качества:
	\[
		J = \int_0^1 u^2(\tau) d\tau
	\]

	Задачей синтеза оптимального регулятора поставим нахождение такого управления $u(t)$, которое минимизирует критерий качества при заданных начальном и конечном состояниях. Назовем его оптимальным управлением и обозначим $u^*(t)$.

	Для решения задачи сперва составим функцию Гамильтона:
	\[
		H = u^2 + \lambda_1 x_2 + \lambda_2 (-3 x_1 - 6 x_2 + u)
	\]

	Запишем условие оптимальности:
	\[
		\frac{\partial H}{\partial u} = 2u + \lambda_2 = 0 \quad \Rightarrow \quad u^*(t) = -\frac{\lambda_2(t)}{2}
	\]

	А также уравнения сопряженной системы:
	\[
		\begin{cases}
			\dot{\lambda}_1 = -\dfrac{\partial H}{\partial x_1} = 3 \lambda_2 \\[2ex]
			\dot{\lambda}_2 = -\dfrac{\partial H}{\partial x_2} = \lambda_1 - 6 \lambda_2
		\end{cases}
	\]

	Откуда:
	\[
		\ddot{\lambda}_1 + 6 \dot{\lambda}_1 - 3 \lambda_1 = 0
	\]

	Соответствующее характеристическое уравнение имеет решения
	\[
		\alpha = -3 + \sqrt{12}, \quad \beta = -3 - \sqrt{12}
	\]

	Значит, решение общее решение первой части
	\[
		\lambda_1(t) = C_1 e^{\alpha t} + C_2 e^{\beta t}
	\]

	Теперь можем найти решение и для второй части
	\[
		\lambda_2(t) = \frac{\dot{\lambda}_1}{3} = \frac{\alpha}{3} C_1 e^{\alpha t} + \frac{\beta}{3} C_2 e^{\beta t}
	\]

	Осталось отыскать константы $C_i$ из начальных и конечных условий системы.
	
\end{document}