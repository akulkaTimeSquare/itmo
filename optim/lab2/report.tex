\documentclass[a4paper,hidelinks,14pt]{extarticle}

\usepackage[utf8]{inputenc}
\usepackage[T2A]{fontenc}
\usepackage[english, russian]{babel}
\usepackage{lipsum}
\usepackage{amsmath}
\usepackage{amssymb}
\usepackage{amsfonts}
\usepackage{mathtools}
\usepackage{datetime}
\usepackage[pdftex]{graphicx}
\usepackage{indentfirst}
\usepackage{asymptote}
\usepackage{systeme}
\usepackage[dvipsnames]{xcolor}
\usepackage{lastpage}
\usepackage{fancybox,fancyhdr}
\usepackage{hyperref}
\usepackage[font={small,it}]{caption}
\fancyhead[L]{Лабораторная работа №2}
\fancyhead[C]{}
\fancyhead[R]{\textit{Принцип максимума}}
\fancyfoot[L]{}
\fancyfoot[C]{\thepage\space}
\fancyfoot[R]{}
\pagestyle{fancy}
\newcommand{\gt}{\textgreater}
\newcommand{\lt}{\textless}
\usepackage{listings}
\usepackage{xcolor}
\lstset{
    basicstyle=\ttfamily\small,
    keywordstyle=\color{blue},
    commentstyle=\color{gray},
    stringstyle=\color{red},
    numbers=left,
    numberstyle=\color[gray]{0.7}\ttfamily\small,
    stepnumber=1,
    numbersep=8pt,
    frame=single,
    showstringspaces=false,
    tabsize=4,
    breaklines=true
}
\usepackage{subcaption}

\begin{document}
	\begin{titlepage}
		\setlength{\parindent}{0ex}
		
		\begin{center}
			\textsc{
				\vspace{1ex}
                Научно-исследовательский университет ИТМО \\
				\vspace{0.5ex}
				Факультет систем управления и робототехники \\
				\vspace{0.5ex}
			}
		\end{center}
		
		\vspace{45mm}
		
		\begin{center}
			Отчет по лабораторной работе №2\\
			\textbf{Синтез оптимального управления \\
			Принцип максимума} \\
			Вариант 21
		\end{center}
		
		\vspace{50mm}
		
		\begin{minipage}{.45\linewidth}
			Выполнил студент
            \\
			Проверил преподаватель
		\end{minipage}
		\hfill
		\begin{minipage}{.52\linewidth}
			\begin{flushright}
				Мовчан Игорь Евгеньевич, R3480
                \\
				Парамонов Алексей Владимирович
			\end{flushright}
		\end{minipage}
		
		\vfill
		\begin{center}
			Санкт-Петербург
			\\
			2025
		\end{center}
		
	\end{titlepage}

	\tableofcontents
	\clearpage
	
	\section{Построение оптимального регулятора}
	Пусть имеется система:
	\[
		\begin{cases}
			\dot{x}_1 = x_2 \\
			\dot{x}_2 = -3 x_1 - 6 x_2 + u \\
		\end{cases}
	\]

	Начальное и конечное (в момент времени $t = 1$) состояния:
	\[
		x(0) = \begin{bmatrix}
			0 \\
			0
		\end{bmatrix}, \quad
		x(1) = \begin{bmatrix}
			10 \\
			0
		\end{bmatrix}
	\]

	Минимизируемый критерий качества:
	\[
		J = \int_0^1 u^2(\tau) d\tau
	\]

	Задачей синтеза оптимального регулятора поставим нахождение такого управления $u(t)$, которое минимизирует критерий качества при заданных начальном и конечном состояниях. Назовем его оптимальным управлением и обозначим $u^*(t)$.

	Для решения задачи сперва составим Гамильтониан:
	\[
		H = -u^2 + \varphi_1 x_2 + \varphi_2 (-3 x_1 - 6 x_2 + u)
	\]

	Теперь запишем систему уравнений Эйлера-Лагранжа:
	\[
		\begin{cases}
			\dot{\varphi}_i = -\dfrac{\partial H}{\partial x_i} \\[2ex]
			\dfrac{\partial H}{\partial u} = 0
		\end{cases} \Rightarrow \quad 
		\begin{cases}
			\dot{\varphi}_1 = 3 \varphi_2 \\
			\dot{\varphi}_2 = -\varphi_1 + 6 \varphi_2 \\
			-2u + \varphi_2 = 0
		\end{cases} \Rightarrow \quad
		\begin{cases}
			\dot{\varphi}_1 = 3 \varphi_2 \\
			\dot{\varphi}_2 = -\varphi_1 + 6 \varphi_2 \\
			u = \dfrac{\varphi_2}{2}
		\end{cases}
	\]

	Подставим управление в уравнения объекта и расширим систему:
	\[
		\begin{cases}
			\dot{\varphi}_1 = 3 \varphi_2 \\
			\dot{\varphi}_2 = -\varphi_1 + 6 \varphi_2 \\
			\dot{x}_1 = x_2 \\
			\dot{x}_2 = -3 x_1 - 6 x_2 + \dfrac{\varphi_2}{2}
		\end{cases} \Rightarrow \quad
		\begin{bmatrix}
			\dot{\varphi}_1 \\
			\dot{\varphi}_2 \\
			\dot{x}_1 \\
			\dot{x}_2
		\end{bmatrix} = \begin{bmatrix}
			0 & 3 & 0 & 0 \\
			-1 & 6 & 0 & 0 \\
			0 & 0 & 0 & 1 \\
			0 & \dfrac{1}{2} & -3 & -6
		\end{bmatrix}
		\begin{bmatrix}
			\varphi_1 \\
			\varphi_2 \\
			x_1 \\
			x_2
		\end{bmatrix}
	\]

	Полученную расширенную систему бессмысленно и нерационально решать аналитически руками, поэтому прибегнем к численным вычислениям. Итоговое управление тогда:
	\[
		u^*(t)
		= \dfrac{\varphi_2(t)}{2} =
		\frac{180\, e^{-t(\sqrt{6}-3)}}{D}
		\sum_{i=1}^{3} A_i
		-
		\frac{60\, e^{t(\sqrt{6}+3)}}{D}
		\sum_{i=1}^{3} B_i
	\]

	Здесь коэффициенты $A_i$:
	\[
		A_1 =
		  2e^{\sqrt{6}+3}
		+ 2e^{\sqrt{6}+9}
		+ 2e^{\sqrt{6}+15}
	\]
	\[
		A_2 =
		- 6e^{3-\sqrt{6}}
		+ 6e^{9-3\sqrt{6}}
		- 6e^{15-\sqrt{6}}
	\]
	\[
		A_3 = \sqrt{6} (- 2e^{3-\sqrt{6}} - 2e^{9-\sqrt{6}} + 3e^{9-3\sqrt{6}} - 2e^{15-\sqrt{6}} + 3e^{\sqrt{6}+9})
	\]

	Коэффициенты $B_i$:
	\[
		B_1 =
		  18e^{\sqrt{6}+3}
		- 30e^{3-\sqrt{6}}
		- 12e^{9-\sqrt{6}}
	\]
	\[
	B_2 =
		  36e^{9-3\sqrt{6}}
		+ 6e^{15-\sqrt{6}}
		- 18e^{15-3\sqrt{6}}
	\]
	\[
		B_3 =
		- 12\sqrt{6}\,e^{3-\sqrt{6}}
		- 6\sqrt{6}\,e^{9-\sqrt{6}}
		+ 15\sqrt{6}\,e^{9-3\sqrt{6}} -
	\]
	\[
		- 6\sqrt{6}\,e^{15-3\sqrt{6}}
		+ 6\sqrt{6}\,e^{\sqrt{6}+3}
		+ 3\sqrt{6}\,e^{\sqrt{6}+9}
	\]

	Знаменатель:
	\[
		D = D_1 D_2
	\]

	Коэффициенты $D_i$:
	\[
		D_1 = \sqrt{6}\,e^{6}
		- 6e^{6-2\sqrt{6}}
		- 3\sqrt{6}\,e^{6-2\sqrt{6}}
		+ 2\sqrt{6}
		+ 6
	\]
	\[
		D_2 = 2e^{6}
		+ 2e^{12}
		- 3e^{6-2\sqrt{6}}	
		- 3e^{2\sqrt{6}+6}
		+ 2
	\]

	\begin{figure}
		\centering
		\includegraphics[width=0.8\textwidth]{images/u.png}
		\caption{График управляющего воздействия при оптимальном регуляторе}
		\label{fig:u}
	\end{figure}
	\begin{figure}
		\centering
		\includegraphics[width=0.8\textwidth]{images/x.png}
		\caption{Графики переменных состояния объекта при оптимальном регуляторе}
		\label{fig:x}
	\end{figure}
	\begin{figure}
		\centering
		\includegraphics[width=0.8\textwidth]{images/J_cum.png}
		\caption{График изменения минимизируемого критерия качества}
		\label{fig:jcum}
	\end{figure}

	Результаты моделирования замкнутой найденным регулятором системы объекта приведены на рисунках \ref{fig:u}-\ref{fig:jcum}. При этом получаем приближенное значение минимизируемого критерия $J \approx 6829.24$.

	Можем видеть, что управление действительно переводит состояния из начальных условий в заданные конечные. Успех!

	\section{Подтверждение оптимальности}
	Для подтверждения оптимальности управления $u^*(t)$, введенного для начального и конечного состояний $x(0)$ и $x(1)$, сравним его с другим управлением, удовлетворяющим тем краевым условиям. При этом данное управление должно дать большее значение минимизированного $u^*(t)$ критерия качества $J$. Продемонстрируем это, незначительно отклонив параметры регулятора от оптимальных:
	\[
		u_1(t) = u^*(t) - 25
		= \dfrac{\varphi_2(t)}{2} - 25, \quad 
		u_2(t) = u^*(t) + 50
		= \dfrac{\varphi_2(t)}{2} + 50
	\]

	При будем решаеть уже введенную расширенную систему:
	\[
		\begin{cases}
			\dot{\varphi}_1 = 3 \varphi_2 \\
			\dot{\varphi}_2 = -\varphi_1 + 6 \varphi_2 \\
			\dot{x}_1 = x_2 \\
			\dot{x}_2 = -3 x_1 - 6 x_2 + u(t)
		\end{cases}
	\]

	Итак, проведем моделирование замкнутых систем этими регуляторами. Результаты представлены на рисунках \ref{fig:um}-\ref{fig:jcumm}. При этом были достигнуты следующие значения критериев качества:
	\[
		J^* = 6829.4, \quad J = 6834.08 > J^*, \quad J = 6847.69 > J^*
	\]
	\begin{figure}[h]
		\centering
		\includegraphics[width=0.8\textwidth]{images/um.png}
		\caption{Графики управления при отклонении параметров регулятора}
		\label{fig:um}
	\end{figure}
	\begin{figure}
		\centering
		\includegraphics[width=0.8\textwidth]{images/x1m.png}
		\caption{Графики состояния $x_1$ при отклонении параметров регулятора}
		\label{fig:x1m}
	\end{figure}
	\begin{figure}
		\centering
		\includegraphics[width=0.8\textwidth]{images/x2m.png}
		\caption{Графики состояния $x_2$ при отклонении параметров регулятора}
		\label{fig:x2m}
	\end{figure}
	\begin{figure}
		\centering
		\includegraphics[width=0.8\textwidth]{images/Jm.png}
		\caption{Графики критериев при отклонении параметров регулятора}
		\label{fig:jcumm}
	\end{figure}

	Можем видеть, что относительно малые отклонения значений регулятора заметно сказываются на виде графиков как состояний, так и управлений, однако значения функционалов $J$ отклоняются слабо, но всегда в большую сторону от $J^*$. Данное и подтверждает оптимальность синтезированного $u^*(t)$ - \textit{любые} регуляторы, которые могут браться для решения поставленной задачи терминального управления, будут давать большие значения введенного критерия качества
	\[
		J = \int_0^1 u^2(\tau) d\tau
	\]

	\section{Выводы}
	В ходе лабораторной работы был синтезирован оптимальный регулятор, позволяющий перевести динамическую систему из начального положения в заданное конечное состояние за фиксированное время при минимизации квадратичного критерия. Для нахождения закона управления использовался метод, основанный на построении Гамильтониана и решении системы дифференциальных уравнений Эйлера-Лагранжа. Полученные результаты моделирования подтвердили, что найденное управление $u^*(t)$ обеспечивает точное выполнение краевых условий, а сравнение с возмущенными управлениями доказало минимальность значения функционала качества. Таким образом, задача терминального управления была успешно решена, а оптимальность синтезированного регулятора доказана!
	

\end{document}