\documentclass[a4paper,hidelinks,14pt]{extarticle}

\usepackage[utf8]{inputenc}
\usepackage[T2A]{fontenc}
\usepackage[english, russian]{babel}
\usepackage{lipsum}
\usepackage{amsmath}
\usepackage{amssymb}
\usepackage{amsfonts}
\usepackage{mathtools}
\usepackage{datetime}
\usepackage[pdftex]{graphicx}
\usepackage{indentfirst}
\usepackage{asymptote}
\usepackage{systeme}
\usepackage[dvipsnames]{xcolor}
\usepackage{lastpage}
\usepackage{fancybox,fancyhdr}
\usepackage{hyperref}
\usepackage[font={small,it}]{caption}
\fancyhead[L]{Лабораторная работа №1}
\fancyhead[C]{}
\fancyhead[R]{\textit{Поиск минимума}}
\fancyfoot[L]{}
\fancyfoot[C]{\thepage\space}
\fancyfoot[R]{}
\pagestyle{fancy}
\newcommand{\gt}{\textgreater}
\newcommand{\lt}{\textless}
\usepackage{listings}
\usepackage{xcolor}
\lstset{
    basicstyle=\ttfamily\small,
    keywordstyle=\color{blue},
    commentstyle=\color{gray},
    stringstyle=\color{red},
    numbers=left,
    numberstyle=\color[gray]{0.7}\ttfamily\small,
    stepnumber=1,
    numbersep=8pt,
    frame=single,
    showstringspaces=false,
    tabsize=4,
    breaklines=true
}
\usepackage{subcaption}

\begin{document}
	\begin{titlepage}
		\setlength{\parindent}{0ex}
		
		\begin{center}
			\textsc{
				\vspace{1ex}
                Научно-исследовательский университет ИТМО \\
				\vspace{0.5ex}
				Факультет систем управления и робототехники \\
				\vspace{0.5ex}
			}
		\end{center}
		
		\vspace{45mm}
		
		\begin{center}
			Отчет по лабораторной работе №1\\
			\textbf{Поиск минимума с помощью методов \\
			статический оптимизации} \\
			Вариант 21
		\end{center}
		
		\vspace{50mm}
		
		\begin{minipage}{.45\linewidth}
			Выполнил студент
            \\
			Проверил преподаватель
		\end{minipage}
		\hfill
		\begin{minipage}{.52\linewidth}
			\begin{flushright}
				Мовчан Игорь Евгеньевич, R3480
                \\
				Парамонов Алексей Владимирович
			\end{flushright}
		\end{minipage}
		
		\vfill
		\begin{center}
			Санкт-Петербург
			\\
			2025
		\end{center}
		
	\end{titlepage}

	\tableofcontents
	\clearpage
	
	\section{Поиск глобального минимума}
	Рассмотрим функция двух переменных:
	\[
		J(x, u) = 3x^2 + 2u^2 + 3xu + 4x + 5u - 20
	\]

	Будем проводить поиск глобального минимума $J(x, u)$ на основе необходимого и достаточного условий.
	\subsection{Поиск минимума без ограничений}
	Для начала примем, что все ограничения отсутствуют. Найдем градиент функции:
	




	
\end{document}