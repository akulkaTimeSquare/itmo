\documentclass[a4paper,hidelinks,14pt]{extarticle}

\usepackage[utf8]{inputenc}
\usepackage[T2A]{fontenc}
\usepackage[english, russian]{babel}
\usepackage{lipsum}
\usepackage{amsmath}
\usepackage{amssymb}
\usepackage{amsfonts}
\usepackage{mathtools}
\usepackage{datetime}
\usepackage[pdftex]{graphicx}
\usepackage{indentfirst}
\usepackage{asymptote}
\usepackage{systeme}
\usepackage[dvipsnames]{xcolor}
\usepackage{lastpage}
\usepackage{fancybox,fancyhdr}
\usepackage{hyperref}
\usepackage[font={small,it}]{caption}
\fancyhead[L]{Лабораторная работа №3}
\fancyhead[C]{}
\fancyhead[R]{\textit{LQR-регулятор}}
\fancyfoot[L]{}
\fancyfoot[C]{\thepage\space}
\fancyfoot[R]{}
\pagestyle{fancy}
\newcommand{\gt}{\textgreater}
\newcommand{\lt}{\textless}
\usepackage{listings}
\usepackage{xcolor}
\lstset{
    basicstyle=\ttfamily\small,
    keywordstyle=\color{blue},
    commentstyle=\color{gray},
    stringstyle=\color{red},
    numbers=left,
    numberstyle=\color[gray]{0.7}\ttfamily\small,
    stepnumber=1,
    numbersep=8pt,
    frame=single,
    showstringspaces=false,
    tabsize=4,
    breaklines=true
}
\usepackage{subcaption}

\begin{document}
	\begin{titlepage}
		\setlength{\parindent}{0ex}
		
		\begin{center}
			\textsc{
				\vspace{1ex}
                Научно-исследовательский университет ИТМО \\
				\vspace{0.5ex}
				Факультет систем управления и робототехники \\
				\vspace{0.5ex}
			}
		\end{center}
		
		\vspace{45mm}
		
		\begin{center}
			Отчет по лабораторной работе №3\\
			\textbf{Синтез оптимального регулятора \\
			для линейного стационарного объекта} \\
			Вариант 21
		\end{center}
		
		\vspace{50mm}
		
		\begin{minipage}{.45\linewidth}
			Выполнил студент
            \\
			Проверил преподаватель
		\end{minipage}
		\hfill
		\begin{minipage}{.52\linewidth}
			\begin{flushright}
				Мовчан Игорь Евгеньевич, R3480
                \\
				Парамонов Алексей Владимирович
			\end{flushright}
		\end{minipage}
		
		\vfill
		\begin{center}
			Санкт-Петербург
			\\
			2025
		\end{center}
		
	\end{titlepage}

	\tableofcontents
	\clearpage
	
	\section{Построение оптимального регулятора}
	Рассмотрим линейный объект с начальными условиями $x(0)$
	\[
		\dot{x} = Ax + bu, \quad x(0) = \begin{bmatrix}
			1 & 0
		\end{bmatrix}^T
	\]

	Матрицы возьмём согласно варианту:
	\[
		A = \begin{bmatrix}
			0 & 1 \\
			5 & -4
		\end{bmatrix}, \quad
		b = \begin{bmatrix}
			1 \\
			1
		\end{bmatrix}
	\]

	Будем строить регулятор с отрицательной обратной связью вида
	\[
		u = -Kx
	\]

	Расчет будем производить на основе уравнения Риккати относительно положительно определенной матрицы $P \succ 0$:
	\[
		A^T P + PA + Q - P b r^{-1} b^T P = 0
	\]

	Матрицу $Q \succcurlyeq 0$ и параметр $r > 0$ примем из варианта
	\[
		Q = \begin{bmatrix}
			2 & 0 \\
			0 & 1
		\end{bmatrix}, \quad
		r = 5
	\]

	А матрицу $K$ обратной связи будем расчитывать через
	\[
		K = r^{-1} b^T P
	\]

	При такой постановке задачи минимизируется критерий качества
	\[
		J = \int_0^\infty x^T(\tau) Q x(\tau) + r u^2(\tau) d\tau
	\]

	Из его вида можем предположить, что чем больше $Q$ относительно $r$, тем более важно скорее <<заглушить>> $x(\tau)$ (достигается большее быстродействие системы), а чем больше $r$ относительно $Q$, тем это быстродействие меньше, как и максимальные значения управления.

	Итак, произведём расчёт регулятора при принятых матрицах и параметрах. Решение уравнения Риккати даёт матрицу
	\[
		P = \begin{bmatrix}
			7.806 & 1.551  \\
			1.551 & 0.416 
		\end{bmatrix} \succ 0
	\]

	Из этого находим обратную связь оптимального регулятора
	\[
		K = r^{-1} b^T P = 
		\begin{bmatrix}
			1.8715 & 0.3934 
		\end{bmatrix}
	\]

	Проведем моделирование с найденным управлением и начальными условиями $x(0) = \begin{bmatrix}
		1 & 0
	\end{bmatrix}^T$. Результаты приведены на рисунках \ref{fig:u}-\ref{fig:j}. Установившееся значения критерия качества $J_{min} \approx 7.8062$.
	\begin{figure}[h]
		\centering
		\includegraphics[width=0.8\textwidth]{images/u.png}
		\caption{График управляющего воздействия при оптимальном регуляторе}
		\label{fig:u}
	\end{figure}
	\begin{figure}
		\centering
		\includegraphics[width=0.8\textwidth]{images/x.png}
		\caption{Графики переменных состояния объекта при оптимальном регуляторе}
		\label{fig:x}
	\end{figure}
	\begin{figure}
		\centering
		\includegraphics[width=0.8\textwidth]{images/j.png}
		\caption{График изменения минимизируемого критерия качества}
		\label{fig:j}
	\end{figure}

	Можем видеть, что регулятор успешно стабилизирует систему с ненулевыми начальными условиями, а критерий качества со временем достигает минимально возможной оценки $J_{min} = x(0)^T P x(0)$.

	Чтобы доказать минимальность установившегося значения критерия качества, стоит незначительно отклонить расчётную матрицу обратной связи $K$ регулятора и убедиться, что в этом случае предельное значение функционала $J$ больше $J_{min}$. Сделаем это.

	\section{Оптимальность найденного регулятора}

	Итак, немного отклоним матрицу обратной связи регулятора так, чтобы система сохранила устойчивость, до
	\[
		K' = \begin{bmatrix}
			2 & 0.525
		\end{bmatrix} = K + \begin{bmatrix}
			0.1285 & 0.1316
		\end{bmatrix}
	\]

	Повторим моделирование с условиями $x(0) = \begin{bmatrix}
		1 & 0
	\end{bmatrix}^T$. Результаты приведены на рисунках \ref{fig:u1}-\ref{fig:j1}. Установившееся значение функционала качества $J_{stab} \approx 7.8604 > J_{min} \approx 7.8062$.
	\begin{figure}[h]
		\centering
		\includegraphics[width=0.8\textwidth]{images/u1.png}
		\caption{График управляющего воздействия при измененном $K$}
		\label{fig:u1}
	\end{figure}
	\begin{figure}
		\centering
		\includegraphics[width=0.8\textwidth]{images/x1.png}
		\caption{Графики переменных состояния при измененном $K$}
		\label{fig:x1}
	\end{figure}
	\begin{figure}
		\centering
		\includegraphics[width=0.8\textwidth]{images/j1.png}
		\caption{График критерия качества при измененном $K$}
		\label{fig:j1}
	\end{figure}

	Из графиков видим, что система всё ещё является устойчивой, но установившееся значение функционала качества выше, чем у замкнутой оптимальным регулятором. Для убедительности повторим эксперименты ещё раз, но уже с матрицей
	\[
		K'' = \begin{bmatrix}
			1.7 & 0.35
		\end{bmatrix} = K + \begin{bmatrix}
			-0.1715 & -0.0434
		\end{bmatrix}
	\]

	Результаты продемонстрированы на рисунках \ref{fig:u2}-\ref{fig:j2}. Установившееся значение $J_{stab} \approx 7.8803 > J_{min} \approx 7.8062$.
	\begin{figure}[h]
		\centering
		\includegraphics[width=0.8\textwidth]{images/u2.png}
		\caption{График управляющего воздействия при связи $K''$}
		\label{fig:u2}
	\end{figure}
	\begin{figure}
		\centering
		\includegraphics[width=0.8\textwidth]{images/x2.png}
		\caption{Графики переменных состояния при связи $K''$}
		\label{fig:x2}
	\end{figure}
	\begin{figure}
		\centering
		\includegraphics[width=0.8\textwidth]{images/j2.png}
		\caption{График критерия качества при связи $K'$}
		\label{fig:j2}
	\end{figure}

	Таким образом, продемонстрировано, что оптимальный регулятор с матрицей обратной связи $K$ действительно достигает минимального значения $J_{min}$ критерия качества
	\[
		J = \int_0^\infty x^T(\tau) Q x(\tau) + r u^2(\tau) d\tau
	\]

	При положительно (полу-)определенных матрицах
	\[
		Q = \begin{bmatrix}
			2 & 0 \\
			0 & 1
		\end{bmatrix} \succcurlyeq 0, \quad
		r = 5 > 0
	\]

	Посмотрим теперь, как матрица $Q$ и параметр $r$ влияют на качество переходных процессов в замкнутой системе.

	\section{Влияние параметров на стабилизацию}
	Итак, зададимся тремя различными значениями параметра $r$ и трёмя разными матрицами $Q$ при условии, что $r > 0$, $Q = k Q^*$ при положительном коэффициенте $k$ и матрице $Q^*\succcurlyeq 0$, взятой в соответствии с вариантом задания
	\[
		Q^* = \begin{bmatrix}
			2 & 0 \\
			0 & 1
		\end{bmatrix} \succcurlyeq 0
	\]

	В итоге получаем пары:
	\[
		r_1 = 15,\, k_1 = 3 \qquad
		r_2 = 2,\, k_2 = 10 \qquad
		r_3 = 10,\, k_3 = 2
	\]

	При этом для сравнения с уже вычисленным оптимальным регулятором добавим в рассмотрение пару
	\[
		r_0 = 5,\quad k_0 = 1
	\]

	Найденные обратные связи для введенных пар $(r_i, k_i)$ тогда:
	\[
		K_0 = \begin{bmatrix}
			1.8715 & 0.3934 
		\end{bmatrix}, \quad
		K_1 = \begin{bmatrix}
			1.8715 & 0.3934
		\end{bmatrix}
	\]
	\[
		K_2 = \begin{bmatrix}
			2.1130 & 0.4682
		\end{bmatrix}, \quad
		K_3 = \begin{bmatrix}
			2.9073 & 0.7384
		\end{bmatrix}
	\]

	Моделирование фомируемого управления регуляторов приведено на рисунке \ref{fig:um}, переменных состояния - на рисунках \ref{fig:x1m} и \ref{fig:x2m}, а графиков критериев качества с установившемися значениями
	\[
		J_{min0} \approx 7.8062, \quad J_{min1} = \dfrac{J_{min0}}{3} \approx 23.4185
	\]
	\[
		J_{min2} \approx 7.0661, \quad J_{min3} \approx 4.9242
	\]

	На рисунках \ref{fig:j_diff}. Из всего выведенного можем сделать вывод, что на характер переходных процессов влияет именно соотношение между $Q$ и $r$. Соответственно, чем <<больше>> $Q$, тем управление более агрессивное, затрачивает большие управления и быстрее стабилизирует систему. И наоборот, чем больше $r$, тем управляющее воздействие более мягкое, но переходные процессы длятся дольше.
	\begin{figure}
		\centering
		\includegraphics[width=0.8\textwidth]{images/u_m.png}
		\caption{График управляющего воздействия при различных парах $(r_i, k_i)$}
		\label{fig:um}
	\end{figure}
	\begin{figure}
		\centering
		\includegraphics[width=0.8\textwidth]{images/x1_m.png}
		\caption{График переменных состояния $x_1$ при различных парах $(r_i, k_i)$}
		\label{fig:x1m}
	\end{figure}
	\begin{figure}
		\centering
		\includegraphics[width=0.8\textwidth]{images/x2_m.png}
		\caption{График переменных состояния $x_2$ при различных парах $(r_i, k_i)$}
		\label{fig:x2m}
	\end{figure}
	\begin{figure}
		\centering
		\includegraphics[width=0.8\textwidth]{images/j_diff.png}
		\caption{Графики критериев качества при различных парах $(r_i, k_i)$}
		\label{fig:j_diff}
	\end{figure}

	\section{Выводы}
	В ходе лабораторной работы был синтезирован оптимальный линейный квадратичный регулятор для линейной системы на основе уравнения Риккати и отрицательной обратной связи по состоянию. Показано, что полученный регулятор обеспечивает устойчивость замкнутой системы и минимизирует квадратичный функционал качества, причём минимальное значение критерия совпадает с теоретической оценкой \( J_{\min} = x(0)^T P x(0) \). Численные эксперименты с отклонёнными матрицами обратной связи подтвердили оптимальность решения, так как значение функционала при этом возрастает. Также исследовано влияние параметров \(Q\) и \(r\), показавшее, что их соотношение определяет компромисс между быстродействием системы и величиной управляющего воздействия.

	
\end{document}