\documentclass[a4paper,hidelinks,14pt]{extarticle}

\usepackage[utf8]{inputenc}
\usepackage[T2A]{fontenc}
\usepackage[english, russian]{babel}
\usepackage{lipsum}
\usepackage{amsmath}
\usepackage{amssymb}
\usepackage{amsfonts}
\usepackage{mathtools}
\usepackage{datetime}
\usepackage[pdftex]{graphicx}
\usepackage{indentfirst}
\usepackage{asymptote}
\usepackage{systeme}
\usepackage[dvipsnames]{xcolor}
\usepackage{lastpage}
\usepackage{fancybox,fancyhdr}
\usepackage{hyperref}
\usepackage[font={small,it}]{caption}
\usepackage[framed,autolinebreaks,numbered,useliterate]{mcode}


\usepackage{minted, xcolor} 
\usemintedstyle{monokai}
\definecolor{bg}{HTML}{282828}
\setminted{bgcolor=bg}
\setminted{fontsize=\scriptsize}
\setminted{}

\fancyhead[R]{\textit{Влияние нулей и полюсов}}
\fancyhead[L]{Лабораторная работа №6}
\fancyfoot[C]{Страница \thepage\space из \pageref{LastPage}}
\fancyfoot[R]{}
\fancyfoot[L]{}
\pagestyle{fancy}
\setlength{\headheight}{17.0pt}

\newcommand{\anonsection}[1]{\section*{#1}\addcontentsline{toc}{section}{#1}}

\begin{document}
	\begin{titlepage}
		\setlength{\parindent}{0ex}
		
		\begin{center}
			\textsc{
				\vspace{1ex}
				Научно исследовательский университет ИТМО \\
				\vspace{0.5ex}
				Факультет систем управления и робототехники \\
				\vspace{0.5ex}
			}
		\end{center}
		
		\vspace{50mm}
		
		\begin{center}
			Отчет по лабораторной работе №6 \\
			Анализ влияния нулей и полюсов передаточной функции на динамические свойства
		\end{center}
		
		\vspace{50mm}
		
		\begin{minipage}{.48\linewidth}
			Выполнил студент группы R3380
			
			Преподаватели
		\end{minipage}
		\hfill
		\begin{minipage}{.5\linewidth}
			\begin{flushright}
				Мовчан И.Е.
				\\
				Лопарев А.В., Золотаревич В.П.
			\end{flushright}
		\end{minipage}
		
		\vfill
		\begin{center}
			Санкт-Петербург
			\\
			2024
		\end{center}
		
	\end{titlepage}
	
	\section{Цель работы}
	Изучить связь характера переходной характеристики, ди
намических свойств системы с размещением на комплексной плоскости нулей и
полюсов.
	\section{От постоянных к системе (Баттерворт)}
	Пусть заданы значения постоянных $n = 6$, $t_\text{П} = 8$, $k = 5$ (исходя из варианта 11). Хотим построить систему с этими характеристиками с нулями передаточной функции, распределенными по системе Баттерворта.
	
	Напишем некоторую автоматизацию для вычисления корней и полинома:
	\begin{lstlisting}
n = 6;
k = 5;
tp = 8;
alphas = [1, ];
for j = 1:n
	p = [1, -exp(1i*(pi/2 + (2*j-1)/(2*n)*pi))];
	alphas = conv(alphas, p);
end
alphas = real(alphas)
	\end{lstlisting}

	Получаем коэфиициенты полинома:
	$$
	alphas = [1.0000,\hspace{1mm} 3.8637,\hspace{1mm} 7.4641,\hspace{1mm} 9.1416, \hspace{1mm} 7.4641, \hspace{1mm}3.8637, \hspace{1mm}1.0000]
	$$

	Вычислим время нормированного переходного процесс (рисунок \ref{1}).
	\begin{figure}[h]
		\centering
		\includegraphics[scale=0.55]{./images/1.png}
		\caption{Переходный процесс при полиноме Баттерворта}
		\label{1}
	\end{figure}

	Далее найдём всё необходимое, используя:
	\begin{lstlisting}
a = zeros(n);
tpn = 14.1;
omega = tpn/tp;
for j = 1:n
	a(j) = alphas(j)*omega^(n-j+1);
end
a = cat(1, [1, ], wrev(a(:, 1)))
b = a(end)*k
r = roots(a);
rp = real(r)
ip = imag(r)
nu = abs(rp(1));
t = 1/nu*log(1/0.05)
	\end{lstlisting}

	Откуда получаем:
$$
a = [1.0000,\hspace{1mm} 6.8098\hspace{1mm} , 23.1865\hspace{1mm} , 50.0507\hspace{1mm} , 72.0268\hspace{1mm} , 65.7127\hspace{1mm}, 29.9761]
$$
$$
b = 149.8804
$$
$$
rp = [-0.4562,\hspace{1mm} -0.4562\hspace{1mm} , -1.2463\hspace{1mm} , -1.2463\hspace{1mm} , -1.7024\hspace{1mm} , -1.7024]
$$
$$
ip = [1.7024,\hspace{1mm} -1.7024\hspace{1mm} , 1.2463\hspace{1mm} , -1.2463\hspace{1mm} , 0.4562\hspace{1mm} , -0.4562]
$$
$$
t = 6.5672
$$
	
	$a$ - коэффициенты при $y$, $b$ - при входном воздействии, $rp$, $ip$ - действительные и мнимые части корней соответственно, $t$ - время переходного процесса, вычисленного по данной формуле.

	Схема моделирования и переходные процессы:
	\begin{figure}[h]
		\centering
		\includegraphics[scale=0.55]{./images/2.png}
		\caption{Cхема моделирования при полиноме Баттерворта}
		\label{2}
	\end{figure}
	\begin{figure}[h]
		\centering
		\includegraphics[scale=0.55]{./images/3.png}
		\caption{Переходный процесс при полиноме Баттерворта}
		\label{3}
	\end{figure}

	\section{От постоянных к системе (бином)}
	Аналогично можем вычислить соответствующие величины при биномиальном распределении корней:
	\begin{lstlisting}
n = 6;
k = 5;
tp = 8;
alphas = [1, ];
for j = 1:n
	p = [1, 1];
	alphas = conv(alphas, p);
end
alphas
a = zeros(n);
tpn = 12;
omega = tpn/tp;
for j = 1:n
	a(j) = alphas(j)*omega^(n-j+1);
end
a = cat(1, [1, ], wrev(a(:, 1)))
b = a(end)*k
r = roots(a);
rp = real(r);
ip = imag(r);
nu = abs(rp(3));
t = 1/nu*log(1/0.05);
	\end{lstlisting}
	
	Откуда:
$$
	alphas = [1,\hspace{1mm}     6,\hspace{1mm}    15,\hspace{1mm}    20,\hspace{1mm}    15,\hspace{1mm}     6,\hspace{1mm}     1]
$$
$$
a = [1.0000,\hspace{1mm} 9.0000,\hspace{1mm} 33.7500,\hspace{1mm} 67.5000,\hspace{1mm} 75.9375,\hspace{1mm} 45.5625,\hspace{1mm} 11.3906]   
$$
$$
b = 56.9531
$$
$$
r =\begin{bmatrix}
  -1.5052 + 0.0030i \\
  -1.5052 - 0.0030i \\
  -1.5000 + 0.0060i\\
  -1.5000 - 0.0060i \\
  -1.4948 + 0.0030i \\
  -1.4948 - 0.0030i \\	
\end{bmatrix}
$$
$$
t = 1.9972
$$

	Схемы моделирования и переходные процессы представлены на рис. \ref{4}, \ref{5}, \ref{6}, \ref{7}
	\begin{figure}[h]
		\centering
		\includegraphics[scale=0.75]{./images/4.png}
		\caption{Cхема моделирования (нормированное) при биномиальном распределении}
		\label{4}
	\end{figure}
	\begin{figure}
		\centering
		\includegraphics[scale=0.45]{./images/5.png}
		\caption{Переходный процесс (нормированное) при биномиальном распределении}
		\label{5}
	\end{figure}
	\begin{figure}
		\centering
		\includegraphics[scale=0.55]{./images/6.png}
		\caption{Cхема моделирования при биномиальном распределении}
		\label{6}
	\end{figure}
	\begin{figure}
		\centering
		\includegraphics[scale=0.45]{./images/7.png}
		\caption{Переходный процесс при биномиальном распределении}
		\label{7}
	\end{figure}

	\section{Переходные процессы различных систем}
	Добавим теперь к входному сигналу линейное воздействие (согласно варианту 11):
	$$
	W(s) = \frac{2.75s+56.9531}{s^6+9s^5+33.75s^4+67.5s^3+75.9375s^2+45.5625s+11.3906}
	$$
	
	Откуда
	\begin{figure}[h]
		\centering
		\includegraphics[scale=0.55]{./images/8.png}
		\caption{Переходный процесс при линейной входном воздействии}
		\label{8}
	\end{figure}

	Используем также следующую передаточную функцию:
	$$
	W(s) = \frac{2.5s^6+0.3s^5+0.2s^4+0.3s^3+0.1s^2+0.3s+56.9531}{s^6+9s^5+33.75s^4+67.5s^3+75.9375s^2+45.5625s+11.3906}
	$$

	Переходный процесс данной системы:
	\begin{figure}[h]
		\centering
		\includegraphics[scale=0.55]{./images/9.png}
		\caption{Переходный процесс при полиномиальном входном воздействии}
		\label{9}
	\end{figure}

	\section{Заданное входное воздействие}
	Пусть заданое входное воздействие $g(t)=4\cos (4t)$, а передаточная функция системы имеет вид
	$$
	W(s)=\frac{0.5s^2 + 8}{s^6+9s^5+33.75s^4+67.5s^3+75.9375s^2+45.5625s+11.3904}.
	$$

	Смоделируем систему:
	\begin{figure}[h]
		\centering
		\includegraphics[scale=0.45]{./images/10.png}
		\caption{Схема моделирования системы при заданном входном воздействии}
		\label{10}
	\end{figure}
	\newpage

	Графики:
	\begin{figure}[h]
		\centering
		\includegraphics[scale=0.32]{./images/11.png}
		\caption{График y(t) системы при заданном входном воздействии}
		\label{11}
	\end{figure}

	\begin{figure}[h]
		\centering
		\includegraphics[scale=0.32]{./images/12.png}
		\caption{График входного воздействия}
		\label{12}
	\end{figure}
	\section{Выводы}
	В результате выполнения лабораторной работы была изучена связь характера переходной характеристики, динамических свойств системы с размещением на комплексной плоскости нулей и
	полюсов (при распределении по Баттерворту и биному).
\end{document}