\documentclass[a4paper,hidelinks,14pt]{extarticle}

\usepackage[utf8]{inputenc}
\usepackage[T2A]{fontenc}
\usepackage[english, russian]{babel}
\usepackage{lipsum}
\usepackage{amsmath}
\usepackage{amssymb}
\usepackage{amsfonts}
\usepackage{mathtools}
\usepackage{datetime}
\usepackage[pdftex]{graphicx}
\usepackage{indentfirst}
\usepackage{asymptote}
\usepackage{systeme}
\usepackage[dvipsnames]{xcolor}
\usepackage{lastpage}
\usepackage{fancybox,fancyhdr}
\usepackage{hyperref}
\usepackage[font={small,it}]{caption}

\usepackage{minted, xcolor}
\usemintedstyle{monokai}
\definecolor{bg}{HTML}{282828} 
\setminted{bgcolor=bg}
\setminted{fontsize=\scriptsize}
\setminted{}

\fancyhead[R]{\textit{Анализ точности систем}}
\fancyhead[L]{Лабораторная работа №7}
\fancyfoot[C]{Страница \thepage\space из \pageref{LastPage}}
\fancyfoot[R]{}
\fancyfoot[L]{}
\pagestyle{fancy}
\setlength{\headheight}{17.0pt}

\newcommand{\anonsection}[1]{\section*{#1}\addcontentsline{toc}{section}{#1}}

\begin{document}
	\begin{titlepage}
		\setlength{\parindent}{0ex}
		
		\begin{center}
			\textsc{
				\vspace{1ex}
				Научно исследовательский университет ИТМО \\
				\vspace{0.5ex}
				Факультет систем управления и робототехники \\
				\vspace{0.5ex}
			}
		\end{center}
		
		\vspace{50mm}
		
		\begin{center}
			Отчет по лабораторной работе №7 \\
			Анализ точности систем управления
		\end{center}
		
		\vspace{50mm}
		
		\begin{minipage}{.48\linewidth}
			Выполнил студент группы R3380
			
			Преподаватели
		\end{minipage}
		\hfill
		\begin{minipage}{.5\linewidth}
			\begin{flushright}
				Мовчан И.Е.
				\\
				Лопарев А.В., Золотаревич В.П.
			\end{flushright}
		\end{minipage}
		
		\vfill
		\begin{center}
			Санкт-Петербург
			\\
			2024
		\end{center}
		
	\end{titlepage}
	
	\section{Цель работы}
	Исследование точностных свойств систем управления.
	\section{Структурные схемы и графики}
	Исходя из задания варианта ($W(s) = \frac{1}{0.5s^2+s+1}$, $A = 2$) построим схему моделирования системы:
	\begin{figure}[h]
		\centering
		\includegraphics[scale=0.55]{./images/1.png}
		\caption{Структурная схема при $g(t)=A$}
		\label{1}
	\end{figure}

	Графики ошибок и переходных процессов соответственно:
	\begin{figure}[h]
		\centering
		\includegraphics[scale=0.35]{./images/2.png}
		\caption{Графики ошибок и переходного процесса при $g(t)=A$, $k = 1$}
		\label{2}
	\end{figure}
	\begin{figure}
		\centering
		\includegraphics[scale=0.35]{./images/3.png}
		\caption{Графики ошибок и переходного процесса при $g(t)=A$, $k = 5$}
		\label{3}
	\end{figure}

	\begin{figure}
		\centering
		\includegraphics[scale=0.35]{./images/4.png}
		\caption{Графики ошибок и переходного процесса при $g(t)=A$, $k = 10$}
		\label{4}
	\end{figure}

	Установившиеся ошибки в данном случае равны $\frac{A}{1+k}$.

	Аналогично для воздействия $g(t)=Vt=2t$:

	\begin{figure}[h]
		\centering
		\includegraphics[scale=0.35]{./images/5.png}
		\caption{Графики ошибок и переходного процесса при $g(t)=2t$, $k = 1$}
		\label{5}
	\end{figure}

	\begin{figure}
		\centering
		\includegraphics[scale=0.35]{./images/6.png}
		\caption{Графики ошибок и переходного процесса при $g(t)=2t$, $k = 5$}
		\label{6}
	\end{figure}

	\begin{figure}
		\centering
		\includegraphics[scale=0.35]{./images/7.png}
		\caption{Графики ошибок и переходного процесса при $g(t)=2t$, $k = 10$}
		\label{7}
	\end{figure}

	\newpage

	Ошибка улетает в бесконечность, так как астатизм нулевого порядка.

	Зададим теперь схему следующим образом:
	\begin{figure}[h]
		\centering
		\includegraphics[scale=0.55]{./images/8.png}
		\caption{Структурная схема системы с астатизмом первого порядка}
		\label{8}
	\end{figure}

	Посмотрим на её переходные характеристики и ошибки при линейном воздействии:
	\begin{figure}[h]
		\centering
		\includegraphics[scale=0.35]{./images/9.png}
		\caption{Графики ошибок и переходного процесса при $g(t)=2t$, $k = 1$}
		\label{9}
	\end{figure}
	\begin{figure}
		\centering
		\includegraphics[scale=0.35]{./images/10.png}
		\caption{Графики ошибок и переходного процесса при $g(t)=2t$, $k = 5$}
		\label{10}
	\end{figure}
	\begin{figure}
		\centering
		\includegraphics[scale=0.35]{./images/11.png}
		\caption{Графики ошибок и переходного процесса при $g(t)=2t$, $k = 10$}
		\label{11}
	\end{figure}

	\newpage
	Установившаяся ошибка равна $\frac{V}{k}$.

	При квадратичном воздействии однако наблюдаем всё тот же рост ошибок:
	\begin{figure}[h]
		\centering
		\includegraphics[scale=0.35]{./images/12.png}
		\caption{Графики ошибок и переходного процесса при $g(t)= 0.45t^2$, $k = 1$}
		\label{12}
	\end{figure}

	\section{Возмущенные системы}
	Соберём схему согласно варианту 11 (рис.\ref{13})
	\begin{figure}[h]
		\centering
		\includegraphics[scale=0.55]{./images/13.png}
		\caption{Возмёщенная система}
		\label{13}
	\end{figure}

	Ошибки и переходные характеристики (рис.\ref{14},\ref{15})
	\begin{figure}[h]
		\centering
		\includegraphics[scale=0.55]{./images/14.png}
		\caption{Графики возмёщенной системы при $f_1 = 0$}
		\label{14}
	\end{figure}
	\begin{figure}[h]
		\centering
		\includegraphics[scale=0.55]{./images/15.png}
		\caption{Графики возмёщенной системы при $f_2 = 0$}
		\label{15}
	\end{figure}

	\section{Произвольное входное воздействие}
	Пусть $g(t) = 0.3t+2\sin (0.8t)$, соберём следующую схему (рис.\ref{16})
	\begin{figure}[h]
		\centering
		\includegraphics[scale=0.55]{./images/16.png}
		\caption{Моделирование произвольного входного воздействия}
		\label{16}
	\end{figure}

	Тогда фактическая ошибка (рис.\ref{17})
	\begin{figure}
		\centering
		\includegraphics[scale=0.55]{./images/17.png}
		\caption{Фактическая ошибка при произвольном входном воздействии}
		\label{17}
	\end{figure}

	А приближенная (рис.\ref{18})
	\begin{figure}
		\centering
		\includegraphics[scale=0.55]{./images/18.png}
		\caption{Приближенная ошибка при произвольном входном воздействии}
		\label{18}
	\end{figure}

	\newpage

	\section{Выводы}
	В результате выполнения лабораторной работы были исследование точностные свойства систем управления.

\end{document}