\documentclass[a4paper,hidelinks,14pt]{extarticle}

\usepackage[utf8]{inputenc}
\usepackage[T2A]{fontenc}
\usepackage[english, russian]{babel}
\usepackage{lipsum}
\usepackage{amsmath}
\usepackage{amssymb}
\usepackage{amsfonts}
\usepackage{mathtools}
\usepackage{datetime}
\usepackage[pdftex]{graphicx}
\usepackage{indentfirst}
\usepackage{asymptote}
\usepackage{systeme}
\usepackage[dvipsnames]{xcolor}
\usepackage{lastpage}
\usepackage{fancybox,fancyhdr}
\usepackage{hyperref}
\usepackage[font={small,it}]{caption}

\usepackage{minted, xcolor}
\usemintedstyle{monokai}
\definecolor{bg}{HTML}{282828}
\setminted{bgcolor=bg}
\setminted{fontsize=\scriptsize}
\setminted{}

\fancyhead[R]{\textit{Моделирование систем}}
\fancyhead[L]{Лабораторная работа №1}
\fancyfoot[C]{Страница \thepage\space из \pageref{LastPage}}
\fancyfoot[R]{}
\fancyfoot[L]{}
\pagestyle{fancy}
\setlength{\headheight}{17.0pt}

\newcommand{\anonsection}[1]{\section*{#1}\addcontentsline{toc}{section}{#1}}

\begin{document}
	\begin{titlepage}
		\setlength{\parindent}{0ex}
		
		\begin{center}
			\textsc{
				\vspace{1ex}
				Научно исследовательский университет ИТМО \\
				\vspace{0.5ex}
				Факультет систем управления и робототехники \\
				\vspace{0.5ex}
			}
		\end{center}
		
		\vspace{50mm}
		
		\begin{center}
			Отчет по лабораторной работе №1 \\
			Моделирование линейных динамических систем
		\end{center}
		
		\vspace{50mm}
		
		\begin{minipage}{.48\linewidth}
			Выполнил студент группы R3380
			
			Преподаватели
		\end{minipage}
		\hfill
		\begin{minipage}{.5\linewidth}
			\begin{flushright}
				Мовчан И.Е.
				\\
				Лопарев А.В., Золотаревич В.П.
			\end{flushright}
		\end{minipage}
		
		\vfill
		\begin{center}
			Санкт-Петербург
			\\
			2024
		\end{center}
		
	\end{titlepage}
	
	\section{Цель работы}
	Ознакомление с пакетом прикладных программ SIMULINK и основными приемами моделирования линейных динамических систем.
	
	\section{Модель вход-выход}
	Согласно данным из 11 варианта, построим схему моделирования динамической системы вида
	$$
	y^{(2)} + 0.8y' + 30y = 3u' + 30u ,
	$$
	переписываемую через оператор дифференцирования $s = \frac{d}{dt}$:
	$$
	y = \frac{1}{s}(3u-0.8y) + \frac{1}{s^2}(30u - 30y).
	$$
	Вводя новые переменные $z_1 = y$ и $z_2 = y' - 3u + 0.8y$ (выходы интеграторов), можем смоделировать систему в simulink:
	\begin{figure}[h]
		\centering
		\includegraphics[scale=0.5]{./images/1.png}
		\caption{Схема модели вход-выход}
	\end{figure}

	Теперь, если задать вход $u = 1(t)$, то нулевым начальным условиям по непрерывности слева входа будут соответствовать условия $z_1(0) = y(0) = 0$, а $z_2(0) = y'(0) - 3u(0-) + 0.8y(0) = 0$ (аналогично для входа $2\sin (t)$).

	Графики $u(t)$ и $y(t)$ (при интервале времени от 0 до 15) будут следующими:
	\begin{figure}[h]
		\centering
		\includegraphics[scale=0.5]{./images/2.png}
		\caption{Графики модели вход-выход при $u = 1(t)$}
	\end{figure}

	Аналогичные рассуждения для $u = 2\sin (t)$ дают рисунок \ref{3} (схема представлена на рисунке \ref{4}):
	\begin{figure}
		\centering
		\includegraphics[scale=0.5]{./images/3.png}
		\caption{Графики модели вход-выход при $u = 2\sin (t)$}
		\label{3}
	\end{figure}

	\begin{figure}
		\centering
		\includegraphics[scale=0.5]{./images/4.png}
		\caption{Схема модели вход-выход при $u = 2\sin (t)$}
		\label{4}
	\end{figure}

	Попробуем теперь поставить ненулевые начальные условия (согласно варианту) $y(0) = 1, y'(0) = 0.5$, а в качестве входной сигнал оставить нулевым. Тогда $z_1(0) = y(0) = 1$, а $z_2(0) = y'(0) - 3u(0) + 0.8y(0) = 0.5 + 0.8 = 1.3$; график $y(t)$ c интервалом времени от о до 15 секунд представлен на рисунке \ref{5}.
	
	\begin{figure}
		\centering
		\includegraphics[scale=0.65]{./images/5.png}
		\caption{График выходного сигнала с ненулевыми начальными условиями}
		\label{5}
	\end{figure}

	\newpage
	\newpage
	\newpage
	
	\section{Модель вход-состояние-выход}
	Модель задаётся системой: 
	$$
	\begin{cases*}
		\dot{x}_1 = \phantom{-}0\cdot x_1 +1\cdot x_2 + 1\cdot x_3 + 0\cdot u \\
		\dot{x}_2 = -4\cdot x_1 - 1\cdot x_2 + 2\cdot x_3 + 2\cdot u \\
		\dot{x}_3 = \phantom{-}0\cdot x_1 + 1\cdot x_2 - 2\cdot x_3 + 1 \cdot u \\
		y = _1 + 0\cdot x_2 + 0.5 \cdot x_3
	\end{cases*}	
	$$

	Соответствующая схема моделирования:
	\begin{figure}[h]
		\centering
		\includegraphics[scale=0.45]{./images/6.png}
		\caption{Схема модели вход-выход при $u=1(t)$}
		\label{6}
	\end{figure}
	
	Графики $u(t)$ и $y(t)$ при $u = 1(t)$ и $u = 2\sin (t)$ и нулевых начальных условиях представлены на рисунках \ref{7} и \ref{8}, а для ненулевых начальных условий $x_1(0) = 0.5, x_2(0) = -2, x_3(0) = 0$ и константном нулевом на входе на рисунке \ref{9}

	\begin{figure}
		\centering
		\includegraphics[scale=0.45]{./images/7.png}
		\caption{Графики модели вход-состояние-выход при $u = 1(t)$}
		\label{7}
	\end{figure}

	\begin{figure}
		\centering
		\includegraphics[scale=0.45]{./images/8.png}
		\caption{Графики модели вход-состояние-выход при $u=2\sin (t)$}
		\label{8}
	\end{figure}

	\newpage
	\begin{figure}[h]
		\centering
		\includegraphics[scale=0.45]{./images/9.png}
		\caption{График выхода модели при ненулевых начальных условиях}
		\label{9}
	\end{figure}
	\section{Выводы}
	В ходе выполнения лабораторной работы было изучено построение схем моделей вход-состояние-выход и вход-выход, изучен инструмент вывода графиков в simulink.

\end{document}