\documentclass[a4paper,hidelinks,14pt]{extarticle}

\usepackage[utf8]{inputenc}
\usepackage[T2A]{fontenc}
\usepackage[english, russian]{babel}
\usepackage{lipsum}
\usepackage{amsmath}
\usepackage{amssymb}
\usepackage{amsfonts}
\usepackage{mathtools}
\usepackage{datetime}
\usepackage[pdftex]{graphicx}
\usepackage{indentfirst}
\usepackage{asymptote}
\usepackage{systeme}
\usepackage[dvipsnames]{xcolor}
\usepackage{lastpage}
\usepackage{fancybox,fancyhdr}
\usepackage{hyperref}
\usepackage[font={small,it}]{caption}

\usepackage{minted, xcolor} 
\usemintedstyle{monokai}
\definecolor{bg}{HTML}{282828}
\setminted{bgcolor=bg}
\setminted{fontsize=\scriptsize}
\setminted{}

\fancyhead[R]{\textit{Канонические формы}}
\fancyhead[L]{Лабораторная работа №2}
\fancyfoot[C]{Страница \thepage\space из \pageref{LastPage}}
\fancyfoot[R]{}
\fancyfoot[L]{}
\pagestyle{fancy}
\setlength{\headheight}{17.0pt}

\newcommand{\anonsection}[1]{\section*{#1}\addcontentsline{toc}{section}{#1}}

\begin{document}
	\begin{titlepage}
		\setlength{\parindent}{0ex}
		
		\begin{center}
			\textsc{
				\vspace{1ex}
				Научно исследовательский университет ИТМО \\
				\vspace{0.5ex}
				Факультет систем управления и робототехники \\
				\vspace{0.5ex}
			}
		\end{center}
		
		\vspace{50mm}
		
		\begin{center}
			Отчет по лабораторной работе №2 \\
			Канонические формы представления динамических систем
		\end{center}
		
		\vspace{50mm}
		
		\begin{minipage}{.48\linewidth}
			Выполнил студент группы R3380
			
			Преподаватели
		\end{minipage}
		\hfill
		\begin{minipage}{.5\linewidth}
			\begin{flushright}
				Мовчан И.Е.
				\\
				Лопарев А.В., Золотаревич В.П.
			\end{flushright}
		\end{minipage}
		
		\vfill
		\begin{center}
			Санкт-Петербург
			\\
			2024
		\end{center}
		
	\end{titlepage}
	
	\section{Цель работы}
	Ознакомление с методами взаимного перехода между моделями
	вход-выход и вход-состояние-выход, а также с каноническими формами представления
	моделей вход-состояние-выход.	
	\section{От вход-выход к вход-состояние-выход}
	Модель задаётся (согласно варианту 11) уравнением
	$$
	y'' + 0.8y' + 30y = 3u' + 30u,
	$$

	откуда получаем канонические представления.

	Математическая модель вход-состояние-выход в канонической управляемой форме ($a_0 = 30,\hspace{1mm} a_1 = 0.8,\hspace{1mm} b_0 = 30,\hspace{1mm} b_1 = 3$):
	$$
	A = 
	\begin{bmatrix}
		0 & 1 \\
		-a_0 & -a_1
	\end{bmatrix} =
	\begin{bmatrix}
		0 & 1 \\
		-30 & -0.8
	\end{bmatrix},\hspace{1mm}
	B = \begin{bmatrix}
		0 \\
		1
	\end{bmatrix},
	\hspace{1mm}
	C^{\text{T}} = \begin{bmatrix}
		b_0 \\
		b_1
	\end{bmatrix} =
	\begin{bmatrix}
		30 \\
		3
	\end{bmatrix}
	$$
	Математическая модель вход-состояние-выход в канонической наблюдаемой форме:
	$$
	A = \begin{bmatrix}
		0 & -a_0 \\
		1 & -a_1
	\end{bmatrix} = 
	\begin{bmatrix}
		0 & -30 \\
		1 & -0.8
	\end{bmatrix},\hspace{1mm}
	B = \begin{bmatrix}
		b_0 \\
		b_1
	\end{bmatrix} = 
	\begin{bmatrix}
		30 \\
		3
	\end{bmatrix},
	\hspace{1mm}
	C^{\text{T}} = \begin{bmatrix}
		0 \\
		1
	\end{bmatrix}
	$$

	Передаточная функция системы имеет следующий вид:
	$$
	W(s) = \frac{3s + 30}{s^2 + 0.8s + 30}.
	$$

	При моделировании систем в системе SIMULINK ожидаем получить одни и те же графики, посмотрим, действительно ли это так.
	
	Моделирование при нулевых начальных условиях представлено на рисунке \ref{1}, графики на рисунке \ref{2}.
	\begin{figure}
		\centering
		\includegraphics[scale=0.75]{./images/2.png}
		\caption{Моделирование системы}
		\label{1}
	\end{figure}

	\begin{figure}
		\centering
		\includegraphics[scale=0.45]{./images/1.png}
		\caption{Графики при ступенчатом воздействии}
		\label{2}
	\end{figure}

	\newpage

	\section{От вход-состояние-выход к вход-выход}
	Согласно варианту 11 модель вход-состояние-выход задаётся:
	$$
	A = \begin{bmatrix}
		-1 & 2 \\
		-10 & -3
	\end{bmatrix},\hspace{1mm}
	B = \begin{bmatrix}
		0.5 \\
		1
	\end{bmatrix},\hspace{1mm}
	C^{\text{T}} = 
	\begin{bmatrix}
		6 \\
		1.5
	\end{bmatrix}
	$$

	Выведем передаточную функцию системы:
	$$
	W(s) = C (sI - A)^{-1}B = 
	\begin{bmatrix}
		6.5 & 1.5
	\end{bmatrix} \cdot
	\begin{bmatrix}
		s+1 & -2 \\
		10 & s+3
	\end{bmatrix}^{-1} \cdot
	\begin{bmatrix}
		0.5 \\
		1
	\end{bmatrix} =
	$$
	$$
	=
	\frac{1}{s^2 + 4s + 23}\cdot
	\begin{bmatrix}
		6 & 1.5
	\end{bmatrix}\cdot
	\begin{bmatrix}
		s+3 & 2 \\
		-10 & s+1
	\end{bmatrix}\cdot
	\begin{bmatrix}
		0.5 \\
		1
	\end{bmatrix} =
	$$
	$$
	=
	\frac{1}{s^2 + 4s + 23}\cdot
	\begin{bmatrix}
		6 & 1.5
	\end{bmatrix}\cdot
	\begin{bmatrix}
		\frac{s+3}{2} + 2 \\
		-4 + s
	\end{bmatrix} =
	\frac{4.5s + 15}{s^2 + 4s + 23}
	$$

	Отсюда можем получить канонические формы.

	Математическая модель вход-состояние-выход в канонической управляемой форме:
	$$
	A = 
	\begin{bmatrix}
		0 & 1 \\
		-23 & -4
	\end{bmatrix},\hspace{1mm}
	B = \begin{bmatrix}
		0 \\
		1
	\end{bmatrix},
	\hspace{1mm}
	C^{\text{T}} =
	\begin{bmatrix}
		15 \\
		4.5
	\end{bmatrix}
	$$

	Математическая модель вход-состояние-выход в канонической наблюдаемой форме:
	$$
	A =
	\begin{bmatrix}
		0 & -23 \\
		1 & -4
	\end{bmatrix},\hspace{1mm}
	B =
	\begin{bmatrix}
		15 \\
		4.5
	\end{bmatrix},
	\hspace{1mm}
	C^{\text{T}} = \begin{bmatrix}
		0 \\
		1
	\end{bmatrix}
	$$

	Матрица преобразования исходной модели к канонической наблюдаемой форме вычисляется следующим образом:
	$$
	M = N_y \hat{N}_y^{-1},
	$$
	где $N_y = [b\hspace{2mm}Ab]$, $\hat{N}_y = [\hat{b}\hspace{2mm} \hat{A}\hat{b}]$ - матрицы управляемости исходной и преобразованной модели соответственно.
	
	Отсюда:
	$$
	M = \begin{bmatrix}
		0.5 & 1.5 
		1 & -8
	\end{bmatrix}
	\cdot
	\begin{bmatrix}
		0 & 1 \\
		1 & -4
	\end{bmatrix}^{-1} = 
	\begin{bmatrix}
		0.5 & 1.5 
		1 & -8
	\end{bmatrix}
	\cdot
	\begin{bmatrix}
		4 & 1 \\
		1 & 0
	\end{bmatrix} =
	\begin{bmatrix}
		3.5 & 0.5 \\
		-4 & 1
	\end{bmatrix}
	$$

	Матрица преобразования исходной модели к канонической управляемой форме вычисляется аналогично и имеет вид:
	$$
	M = \begin{bmatrix}
		0.5 & 1.5 \\
		1 & -8
	\end{bmatrix}\cdot
	\begin{bmatrix}
		1.5 & -103.5 \\
		4.5 & -3
	\end{bmatrix}^{-1}
	=
	\begin{bmatrix}
		\frac{1}{51} & & -\frac{3}{17} \\
		\\
		\frac{4}{51} & &-\frac{2}{51}
	\end{bmatrix}
	$$

	Моделирование и соответствующие графики (нулевые начальные условия):
	\begin{figure}[h]
		\centering
		\includegraphics[scale=0.75]{./images/3.png}
		\caption{Моделирование систем}
		\label{3}
	\end{figure}

	\begin{figure}
		\centering
		\includegraphics[scale=0.45]{./images/4.png}
		\caption{Графики при ступенчатом воздействии}
		\label{4}
	\end{figure}

	\section{Замена базиса}
	Матрицы преобразования координат $M$ из варианта 11 имеет следующий вид:
	$$
	M = \begin{bmatrix}
		1 & 2 \\
		0 & 2
	\end{bmatrix}
	$$

	Сделаем преобразование изначальной системы:
	$$
	A = \begin{bmatrix}
		-1 & 2 \\
		-10 & -3
	\end{bmatrix},\hspace{1mm}
	B = \begin{bmatrix}
		0.5 \\
		1
	\end{bmatrix}, \hspace{1mm}
	C^{\text{T}} = \begin{bmatrix}
		6.5 \\
		1.5
	\end{bmatrix}
	$$
	$$
	\hat{A} = M^{-1}A M = \begin{bmatrix}
		1 & -1 \\
		0 & 0.5
	\end{bmatrix} \cdot
	\begin{bmatrix}
		-1 & 2 \\
		-10 & -3
	\end{bmatrix} \cdot
	\begin{bmatrix}
		1 & 2 \\
		0 & 2
	\end{bmatrix} =
	$$
	$$
	= \begin{bmatrix}
		9 & 5 \\
		-5 & -1.5
	\end{bmatrix} \cdot
	\begin{bmatrix}
		1 & 2 \\
		0 & 2
	\end{bmatrix} =
	\begin{bmatrix}
		9 & 28 \\
		-5 & -13
	\end{bmatrix}
	$$
	$$
	\hat{B} = M^{-1}B = \begin{bmatrix}
		1 & -1 \\
		0 & 0.5
	\end{bmatrix} = 
	\begin{bmatrix}
		-0.5 \\
		0.5
	\end{bmatrix}
	$$
	$$
	\hat{C} = C M = \begin{bmatrix}
		6.5 & 1.5
	\end{bmatrix} \cdot
	\begin{bmatrix}
		1 & 2\\
		0 & 2
	\end{bmatrix} =
	\begin{bmatrix}
		6.5 & 16
	\end{bmatrix}
	$$

	Осуществим моделирование:
	\begin{figure}[h]
		\centering
		\includegraphics[scale=0.75]{./images/5.png}
		\caption{Моделирование систем}
		\label{5}
	\end{figure}

	\begin{figure}[h]
		\centering
		\includegraphics[scale=0.44]{./images/6.png}
		\caption{Графики при ступенчатом воздействии}
		\label{6}
	\end{figure}
	\section{Выводы}
	В ходе выполнения лабораторной работы были изучены методы перехода между разными представлениями модели, а также их каноническими формами в среде SIMULINK.
\end{document}