\documentclass[a4paper,hidelinks,14pt]{extarticle}

\usepackage[utf8]{inputenc}
\usepackage[T2A]{fontenc}
\usepackage[english, russian]{babel}
\usepackage{lipsum}
\usepackage{amsmath}
\usepackage{amssymb}
\usepackage{amsfonts}
\usepackage{mathtools}
\usepackage{datetime}
\usepackage[pdftex]{graphicx}
\usepackage{indentfirst}
\usepackage{asymptote}
\usepackage{systeme}
\usepackage[dvipsnames]{xcolor}
\usepackage{lastpage}
\usepackage{fancybox,fancyhdr}
\usepackage{hyperref}
\usepackage[font={small,it}]{caption}
\fancyhead[L]{ЛР №6}
\fancyhead[C]{}
\fancyhead[R]{\textit{Адаптивный наблюдатель состояния}}
\fancyfoot[L]{}
\fancyfoot[C]{\thepage\space}
\fancyfoot[R]{}
\pagestyle{fancy}
\newcommand{\gt}{\textgreater}
\newcommand{\lt}{\textless}
\usepackage{listings}
\usepackage{xcolor}
\lstset{
    basicstyle=\ttfamily\small,
    keywordstyle=\color{blue},
    commentstyle=\color{gray},
    stringstyle=\color{red},
    numbers=left,
    numberstyle=\color[gray]{0.7}\ttfamily\small,
    stepnumber=1,
    numbersep=8pt,
    frame=single,
    showstringspaces=false,
    tabsize=4,
    breaklines=true
}
\usepackage{subcaption}

\begin{document}
	\begin{titlepage}
		\setlength{\parindent}{0ex}
		
		\begin{center}
			\textsc{
				\vspace{1ex}
                Научно-исследовательский университет ИТМО \\
				\vspace{0.5ex}
				Факультет систем управления и робототехники \\
				\vspace{0.5ex}
			}
		\end{center}
		
		\vspace{40mm}
		
		\begin{center}
			Отчет по лабораторной работе №6\\
			\textbf{Синтез адаптивного наблюдателя состояния}\\
			\textbf{линейного объекта}\\
			Вариант 9
		\end{center}
		
		\vspace{40mm}
		
		\begin{minipage}{.45\linewidth}
			Выполнили студенты
            \\
			\\[5mm]
			Преподаватель
		\end{minipage}
		\hfill
		\begin{minipage}{.52\linewidth}
			\begin{flushright}
				Мовчан Игорь Евгеньевич 
				\\
				Копылов Андрей Михайлович
				\\[5mm]
				Парамонов Алексей Владимирович
			\end{flushright}
		\end{minipage}
		
		\vfill
		\begin{center}
			Санкт-Петербург
			\\
			2025
		\end{center}
		
	\end{titlepage}

	\tableofcontents
	\clearpage
	
	\section{Цель работы}
	Освоение процедуры построения адаптивного наблюдателя для линейного объекта управления.
	
	\section{Постановка задачи}
	Пусть дан асимптотически устойчивый объект при начальном состоянии $x(0)$ и недоступном к прямому измерению векторе $x$:
	\[
		\begin{cases*}
			\dot{x} = Ax + bu\\
			y = Cx
		\end{cases*}
	\]

	Здесь $u$, $y$ - входной и выходной сигналы объекта, уже доступные к прямым измерениям - на них и будем основываться.
	
	Также задаются матрица системы $A$, вектор управления $b$ и наблюдения $C$:
	\[
		A = 
		\begin{bmatrix}
		-a_{n-1} & 1       & 0& \ldots & 0 \\
		-a_{n-2} & 0       & 1 & \ldots & 0 \\
		\vdots   & \vdots  & \vdots & \ddots & \vdots \\
		-a_1     & 0       & 0 &\ldots & 1 \\
		-a_0     & 0       & 0 &\ldots & 0
		\end{bmatrix}, \quad 
		b = 
		\begin{bmatrix}
		0 \\
		\vdots \\
		0 \\
		b_m \\
		\vdots \\
		b_0
		\end{bmatrix}, \quad 
		C = \begin{bmatrix}
		1 & 0 & \ldots & 0
		\end{bmatrix}
	\]

	Коэффициенты $a_i$ и $b_j$ являются неизвестными.
	
	Итак, рассматривается задача построения оценки вектора состояния $\hat{x}$ такой, что
	\[
		\lim_{t \to \infty} ||x(t) - \hat{x}(t)|| = 0
	\]

	Синтезируемый адаптивный наблюдатель должен одновременно оценивать неизвестные параметры объекта управления $\theta$ и генерировать оценку вектора состояния $\hat{x}$.

	Отметим, что в задаче мы ограничены допущением, называемым \textit{условием согласования}:
	\[
		A_0 = A - \overline{\theta}C, \quad \overline{\theta}^T = \begin{bmatrix}
			k_{n-1} - a_{n-1} & \ldots & k_0 - a_0
		\end{bmatrix}
	\]

	\section{Одногармонический вход}
	Выберем сигнал $u(t)$, подаваемый на вход системы в виде
	\[
		u(t) = 10 \sin (t)
	\]

	Зададим также коэффициенты для матрицы $A$ и вектора $b$ управления модели согласно выбранному варианту:
	\[
		a_1 = 4, \quad a_0 = 2, \quad b_1 = 2, \quad b_0 = 3 
	\]

	Для решения поставленной задачи построения адаптивного наблюдателя сформируем настраиваемую модель объекта, получаемую применением \textit{гурвицевой} передаточной функции $H(s)$ к левой и правой частям объекта в форме вход-выход, а также последующей заменой $\theta$ на оценку $\hat{\theta}$
	\begin{equation}
		\label{1}
		\hat{y} = \hat{\theta}^T \omega, \quad H(s) = \frac{1}{K(s)} = \frac{1}{s^n + k_{n-1}s^{n-1} + \ldots + k_0}
	\end{equation}

	Здесь $\hat{y}$ и $\hat{\theta}$ являются оценками соответствующих переменных, а параметры $\theta$ и $\omega$ задаются как:
	\[
		\theta^{T} = [\,k_{0}-a_{0},\, k_{1}-a_{1},\, \ldots,\, k_{n-1}-a_{n-1},\, b_{0},\, b_{1},\, \ldots,\, b_{m}\,]
	\]
	\[
		\omega^{T} =
	\Bigl[
	\frac{1}{K(s)}[y],\;
	\frac{s}{K(s)}[y],\;
	\ldots,\;
	\frac{s^{\,n-1}}{K(s)}[y],\;
	\frac{1}{K(s)}[u],\;
	\ldots,\;
	\frac{s^{\,m}}{K(s)}[u]
	\Bigr]
	\]

	Введем в рассмотрение ошибку идентификации с $\hat{y}$ - оценка переменной $y$:
	\[
		\epsilon = y - \hat{y} = \theta^T \omega - \hat{\theta}^T \omega = \tilde{\theta}^T \omega
	\]

	Тогда можно построить следующий алгоритм адаптации:
	\begin{equation}
		\label{2}
		\dot{\hat{\theta}} = \gamma \omega \epsilon, \quad \gamma > 0
	\end{equation}

	Который при выбранной оценке $\hat{\theta}$ для функции Ляпунова
	\[
		V = \frac{\tilde{\theta}^T \tilde{\theta}}{2\gamma}
	\]

	Будет давать отрицательность её производной:
	\[
		\dot{V} = -\frac{\theta^T \dot{\hat{\theta}}}{\gamma} = - \tilde{\theta}^T \omega \epsilon = -\epsilon^2 < 0
	\]
	
	Оценка самого вектора состояния $x$ строится с использованием выведенной оценки $\theta$ и применением оператора матричной передаточной функции $\Phi(s)$ к правой и левой частям системы в форме вход-состояние-выход:
	\[
		\Phi(s) = (Is - A_0)^{-1}, \quad A_0 = \begin{bmatrix}
			-k_{n-1} & 1       & 0& \ldots & 0 \\
			-k_{n-2} & 0       & 1 & \ldots & 0 \\
			\vdots   & \vdots  & \vdots & \ddots & \vdots \\
			-k_1     & 0       & 0 &\ldots & 1 \\
			-k_0     & 0       & 0 &\ldots & 0
			\end{bmatrix}
	\]

	После проведения алгебраических преобразований и замены вектора $\theta$ на его оценку $\hat{\theta}$ можно получить:
	\begin{equation}
		\label{3}
		\hat{x} =
	\sum_{i=0}^{n-1} \hat{\theta}_{i+1} (sI - A_0)^{-1} e_{n-i}[y]
	\;+\;
	\sum_{j=0}^{m} \hat{\theta}_{j+1+n} (sI - A_0)^{-1} e_{n-j}[u]	
	\end{equation}

	Важно отметить, что $e_i^{T} = [\,0,\ldots,0,\underset{i}{1},0,\ldots,0\,].$

	Итак, примем коэффициенты применяемых передаточных функций из данных в варианте:
	\[
		k_1 = 1, \quad k_0 = 0.25
	\]

	И проверим построенный алгоритм, основанный на выражениях \refeq{1}-\refeq{3}, при выбранном входном сигнале $u(t)$ и $\gamma = 1$. Модель представлена на рисунке \ref{fig:model}, а графики нормы разности $||x(t) - \hat{x}(t)||$ и параметрической ошибки $\tilde{\theta}$ на рисунках \ref{fig:2} и \ref{fig:3} соответсвенно.
	
	\begin{figure}
		\centering
		\includegraphics[width=0.95\textwidth]{images/model.png}
		\caption{Модель наблюдателя}
		\label{fig:model}
	\end{figure}
	\begin{figure}
		\centering
		\includegraphics[width=0.8\textwidth]{images/u1_ex.png}
		\caption{График нормы ошибки оценки вектора состояния $||x(t) - \hat{x}(t)||$}
		\label{fig:2}
	\end{figure}
	\begin{figure}[h!]
		\centering
		\includegraphics[width=0.8\textwidth]{images/u1_theta.png}
		\caption{График ошибки оценки параметра $\tilde{\theta} = \theta - \hat{\theta}$}
		\label{fig:3}
	\end{figure}

	Теория говорит, что оценка $\hat{\theta}$ должна сойтись к $\theta$ ($\hat{x}$ к $x$), если входной сигнал достаточно гармонически насыщен - при 2 порядке системы для экспоненциальной сходимости ошибки $\tilde{\theta}$ и оценки 4 параметров одной гармоники во входном сигнале не хватает, поэтому в результатах можем видеть ограниченную ошибку $\tilde{\theta}$.

	\section{Четыре гармоники}
	Повторим эксперимент при входном сигнале, состоящем уже из четырех гармоник \textit{разной частоты}
	\[
		u(t) = 10 \sin(t) + 5 \cos(2t) + 4 \cos(4t) + 3 \cos(8t)
	\]

	Результаты моделирования при $\gamma = 1$ и обогащенном входной сигнале представлен на рисунках \ref{fig:4} и \ref{fig:5}.
	
	Можем видеть, что все ошибки сходятся к нулю, а значит, задача построения оценки выполнена успешно!

	Стоит отметить, что в общем случае построенным алгоритмом наблюдения гарантируется лишь схождение $\epsilon = y - \hat{y}$ к нулю, а значит, для построения адаптивных наблюдателей для систем высоких порядков необходимо брать целое множество \textit{разнообразных} на частоты гармоник для входа $u(t)$, тогда как для систем малых порядков будет достаточно и нескольких гармонических сигналов.
	\begin{figure}
		\centering
		\includegraphics[width=0.8\textwidth]{images/u2_ex.png}
		\caption{График ошибки оценки состояния $||x(t) - \hat{x}(t)||$ при добавлении гармоник}
		\label{fig:4}
	\end{figure}
	\begin{figure}
		\centering
		\includegraphics[width=0.8\textwidth]{images/u2_theta.png}
		\caption{График ошибки оценки параметра $\tilde{\theta} = \theta - \hat{\theta}$ при добавлении гармоник}
		\label{fig:5}
	\end{figure}

	\section{Общие выводы}
	В ходе лабораторной работы был реализован адаптивный наблюдатель для линейного объекта с неизвестными параметрами. На входе с одной гармоникой продемонстрирована недостаточность возбуждения системы: параметры не сходятся, а ошибка оценки остаётся ограниченной. Тогда как при использовании входного сигнала, содержащего уже 4 гармоник различных частот, уже обеспечена необходимая информативность регрессора. В результате достигнута сходимость параметров и хорошая оценка вектора состояния.
\end{document}