\documentclass[a4paper,hidelinks,14pt]{extarticle}

\usepackage[utf8]{inputenc}
\usepackage[T2A]{fontenc}
\usepackage[english, russian]{babel}
\usepackage{lipsum}
\usepackage{amsmath}
\usepackage{amssymb}
\usepackage{amsfonts}
\usepackage{mathtools}
\usepackage{datetime}
\usepackage[pdftex]{graphicx}
\usepackage{indentfirst}
\usepackage{asymptote}
\usepackage{systeme}
\usepackage[dvipsnames]{xcolor}
\usepackage{lastpage}
\usepackage{fancybox,fancyhdr}
\usepackage{hyperref}
\usepackage[font={small,it}]{caption}
\fancyhead[L]{ЛР №6}
\fancyhead[C]{}
\fancyhead[R]{\textit{Адаптивный наблюдатель состояния}}
\fancyfoot[L]{}
\fancyfoot[C]{\thepage\space}
\fancyfoot[R]{}
\pagestyle{fancy}
\newcommand{\gt}{\textgreater}
\newcommand{\lt}{\textless}
\usepackage{listings}
\usepackage{xcolor}
\lstset{
    basicstyle=\ttfamily\small,
    keywordstyle=\color{blue},
    commentstyle=\color{gray},
    stringstyle=\color{red},
    numbers=left,
    numberstyle=\color[gray]{0.7}\ttfamily\small,
    stepnumber=1,
    numbersep=8pt,
    frame=single,
    showstringspaces=false,
    tabsize=4,
    breaklines=true
}
\usepackage{subcaption}

\begin{document}
	\begin{titlepage}
		\setlength{\parindent}{0ex}
		
		\begin{center}
			\textsc{
				\vspace{1ex}
                Научно-исследовательский университет ИТМО \\
				\vspace{0.5ex}
				Факультет систем управления и робототехники \\
				\vspace{0.5ex}
			}
		\end{center}
		
		\vspace{40mm}
		
		\begin{center}
			Отчет по лабораторной работе №6\\
			\textbf{Синтез адаптивного наблюдателя состояния}\\
			\textbf{линейного объекта}\\
			Вариант 9
		\end{center}
		
		\vspace{40mm}
		
		\begin{minipage}{.45\linewidth}
			Выполнили студенты
            \\
			\\[5mm]
			Преподаватель
		\end{minipage}
		\hfill
		\begin{minipage}{.52\linewidth}
			\begin{flushright}
				Мовчан Игорь Евгеньевич 
				\\
				Копылов Андрей Михайлович
				\\[5mm]
				Парамонов Алексей Владимирович
			\end{flushright}
		\end{minipage}
		
		\vfill
		\begin{center}
			Санкт-Петербург
			\\
			2025
		\end{center}
		
	\end{titlepage}

	\tableofcontents
	\clearpage
	
	\section{Цель работы}
	Освоение процедуры построения адаптивного наблюдателя для линейного объекта управления.
	
	\section{Постановка задачи}
	Пусть дан асимптотически устойчивый объект при начальном состоянии $x(0)$ и недоступным к прямому измерению вектор $x$:
	\[
		\begin{cases*}
			\dot{x} = Ax + bu\\
			y = Cx
		\end{cases*}
	\]

	Здесь $u$, $y$ - входной и выходной сигналы объекта, уже доступные к прямым измерениям.
	
	Также задаются матрица системы $A$, вектор управления $b$ и наблюдения $C$:
	\[
		A = 
		\begin{bmatrix}
		-a_{n-1} & 1       & 0& \ldots & 0 \\
		-a_{n-2} & 0       & 1 & \ldots & 0 \\
		\vdots   & \vdots  & \vdots & \ddots & \vdots \\
		-a_1     & 0       & 0 &\ldots & 1 \\
		-a_0     & 0       & 0 &\ldots & 0
		\end{bmatrix}, \quad 
		b = 
		\begin{bmatrix}
		0 \\
		\vdots \\
		0 \\
		b_m \\
		\vdots \\
		b_0
		\end{bmatrix}, \quad 
		C = \begin{bmatrix}
		1 & 0 & \ldots & 0
		\end{bmatrix},
	\]

	Коэффициенты $a_i$ и $b_j$ являются неизвестными.
	
	Итак, рассматривается задача построения оценки вектора состояния $\hat{x}$ такой, что
	\[
		\lim_{t \to \infty} ||x(t) - \hat{x}(t)|| = 0
	\]

	Синтезируемый адаптивный наблюдатель должен одновременно оценивать неизвестные параметры объекта управления $\theta$ и генерировать оценку вектора состояния $\hat{x}$.

	Отметим, что в задаче мы ограничены допущением, называемым \textit{условием согласования}:
	\[
		A_0 = A - \overline{\theta}C, \quad \overline{\theta}^T = \begin{bmatrix}
			k_{n-1} - a_{n-1} & \ldots & k_0 - a_0
		\end{bmatrix}
	\]

	\section{Одногармонический вход}

	





	

	



	






	
\end{document}