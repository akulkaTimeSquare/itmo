\documentclass[a4paper,hidelinks,14pt]{extarticle}

\usepackage[utf8]{inputenc}
\usepackage[T2A]{fontenc}
\usepackage[english, russian]{babel}
\usepackage{lipsum}
\usepackage{amsmath}
\usepackage{amssymb}
\usepackage{amsfonts}
\usepackage{mathtools}
\usepackage{datetime}
\usepackage[pdftex]{graphicx}
\usepackage{indentfirst}
\usepackage{asymptote}
\usepackage{systeme}
\usepackage[dvipsnames]{xcolor}
\usepackage{lastpage}
\usepackage{fancybox,fancyhdr}
\usepackage{hyperref}
\usepackage[font={small,it}]{caption}
\fancyhead[L]{ЛР №11}
\fancyhead[C]{}
\fancyhead[R]{\textit{Адаптивная компенсация возмущения}}
\fancyfoot[L]{}
\fancyfoot[C]{\thepage\space}
\fancyfoot[R]{}
\pagestyle{fancy}
\newcommand{\gt}{\textgreater}
\newcommand{\lt}{\textless}
\usepackage{listings}
\usepackage{xcolor}
\lstset{
    basicstyle=\ttfamily\small,
    keywordstyle=\color{blue},
    commentstyle=\color{gray},
    stringstyle=\color{red},
    numbers=left,
    numberstyle=\color[gray]{0.7}\ttfamily\small,
    stepnumber=1,
    numbersep=8pt,
    frame=single,
    showstringspaces=false,
    tabsize=4,
    breaklines=true
}
\usepackage{subcaption}

\begin{document}
	\begin{titlepage}
		\setlength{\parindent}{0ex}
		
		\begin{center}
			\textsc{
				\vspace{1ex}
                Научно-исследовательский университет ИТМО \\
				\vspace{0.5ex}
				Факультет систем управления и робототехники \\
				\vspace{0.5ex}
			}
		\end{center}
		
		\vspace{40mm}
		
		\begin{center}
			Отчет по лабораторной работе №11\\
			\textbf{Адаптивная компенсация внешнего возмущения}\\
			Вариант 9
		\end{center}
		
		\vspace{45mm}
		
		\begin{minipage}{.45\linewidth}
			Выполнили студенты
            \\
			\\[5mm]
			Преподаватель
		\end{minipage}
		\hfill
		\begin{minipage}{.52\linewidth}
			\begin{flushright}
				Мовчан Игорь Евгеньевич 
				\\
				Копылов Андрей Михайлович
				\\[5mm]
				Парамонов Алексей Владимирович
			\end{flushright}
		\end{minipage}
		
		\vfill
		\begin{center}
			Санкт-Петербург
			\\
			2025
		\end{center}
		
	\end{titlepage}

	\tableofcontents
	\clearpage
	
	\section{Цель работы}
	Освоение принципа адаптивной компенсации возмущения на примере задачи стабилизации многомерного линейного объекта.
	
	\section{Постановка задачи}
	Рассмотрим задачу компенсации внешнего возмущения, действующего на объект с начальными условиями $x(0)$
	\begin{equation}
		\begin{cases}
			\dot{x} = Ax + bu + df \\ \tag{1}
			y = Cx
		\end{cases}	
	\end{equation}

	Здесь $x \in \mathbb{R}^n$ - измеряемый вектор состояния, $u$, $y$ - измеряемые вход и выход объекта (скаляры), $A$, $b$, $C$ и $d$ - известные матрицы соответствующих размерностей, $f$ - \textit{неизмеряемое} мультисинусоидальное возмущение с \textit{априори неизвестными} амплитудами, частотами и фазами гармоник. 
	
	Предполагается, что возмущение $f$ может быть промоделировано как выход автономного генератора:
	\[
		f^{(r)} + l_{r-1}f^{(r-1)} + l_{r-2}f^{(r-2)} + \ldots + l_0 f = 0 \tag{2}
	\]

	При этом параметры модели $l_i$ неизвестны, как и начальные условия сигнала и его производных $f^{(i)}(0)$, а корни характеристического полинома модели являются чисто мнимыми и некратными.

	Примем и допущение, что сигналы $u$, $f$ согласованы и $b = d^4$.

	Цель задачи заключается в построении управления, компенсирующего неизвестное возмущение так, чтобы
	\begin{equation}
		\lim_{t \to \infty} \|x(t)\| = 0. \tag{3}
	\end{equation}

	\section{Построение модели}
	Для начала зададим матрицу $A$ систему и вектор управляемости $b$, а также желаемое время переходного процесса $t_n$ и максимальное перерегулирование $\bar{\sigma}$ согласно выбранному варианту:
	\[
		A = \begin{bmatrix}
			1 & 0 \\
			0 & 0
		\end{bmatrix}, \quad b = \begin{bmatrix}
			2 \\ 4
		\end{bmatrix}, \quad t_n = 0.9, \quad \bar{\sigma} = 0
	\]

	Прежде чем приступить к построению управления необходимо проверить пару $(A, b)$ на управляемость (можем ли мы воздействовать на все моды системы) с помощью матрицы управляемости $U$:
	\[
		U = \begin{bmatrix}
			b & Ab
		\end{bmatrix} = \begin{bmatrix}
			2 & 2\\
			4 & 0
		\end{bmatrix}
	\]

	Так как её ранг полон, то система полностью управляема.

	
\end{document}