\documentclass[a4paper,hidelinks,14pt]{extarticle}

\usepackage[utf8]{inputenc}
\usepackage[T2A]{fontenc}
\usepackage[english, russian]{babel}
\usepackage{lipsum}
\usepackage{amsmath}
\usepackage{amssymb}
\usepackage{amsfonts}
\usepackage{mathtools}
\usepackage{datetime}
\usepackage[pdftex]{graphicx}
\usepackage{indentfirst}
\usepackage{asymptote}
\usepackage{systeme}
\usepackage[dvipsnames]{xcolor}
\usepackage{lastpage}
\usepackage{fancybox,fancyhdr}
\usepackage{hyperref}
\usepackage[font={small,it}]{caption}
\fancyhead[L]{ЛР №1}
\fancyhead[C]{}
\fancyhead[R]{\textit{Управление многомерным объектом}}
\fancyfoot[L]{}
\fancyfoot[C]{\thepage\space}
\fancyfoot[R]{}
\pagestyle{fancy}
\newcommand{\gt}{\textgreater}
\newcommand{\lt}{\textless}
\usepackage{listings}
\usepackage{xcolor}
\lstset{
    basicstyle=\ttfamily\small,
    keywordstyle=\color{blue},
    commentstyle=\color{gray},
    stringstyle=\color{red},
    numbers=left,
    numberstyle=\color[gray]{0.7}\ttfamily\small,
    stepnumber=1,
    numbersep=8pt,
    frame=single,
    showstringspaces=false,
    tabsize=4,
    breaklines=true
}
\usepackage{subcaption}

\begin{document}
	\begin{titlepage}
		\setlength{\parindent}{0ex}
		
		\begin{center}
			\textsc{
				\vspace{1ex}
                Научно-исследовательский университет ИТМО \\
				\vspace{0.5ex}
				Факультет систем управления и робототехники \\
				\vspace{0.5ex}
			}
		\end{center}
		
		\vspace{45mm}
		
		\begin{center}
			Отчет по лабораторной работе №1\\
			\textbf{Адаптивное управление многомерным} \\
			\textbf{объектом по состоянию} \\
			Вариант 9
		\end{center}
		
		\vspace{50mm}
		
		\begin{minipage}{.45\linewidth}
			Выполнили студенты
            \\
			\\
			Преподаватель
		\end{minipage}
		\hfill
		\begin{minipage}{.52\linewidth}
			\begin{flushright}
				Мовчан Игорь Евгеньевич
				\\
				Копылов Андрей Михайлович
                \\
				Парамонов Алексей Владимирович
			\end{flushright}
		\end{minipage}
		
		\vfill
		\begin{center}
			Санкт-Петербург
			\\
			2025
		\end{center}
		
	\end{titlepage}

	\tableofcontents
	\clearpage
	
	\section{Формирование эталонной модели}
	Пусть дан объект
	\[
	\begin{cases}
		\dot{x}(t) = A x(t) + B u(t), \\
		y(t) = C x(t)
	\end{cases}
	\]

	Согласно варианту задания, матрицы объекта имеют вид:
	\[
		A = \begin{bmatrix}
			0 & 1 \\
			0 & 0
		\end{bmatrix}, \quad
		B = \begin{bmatrix}
			0 \\
			b_0
		\end{bmatrix} =
		\begin{bmatrix}
			0 \\
			9
		\end{bmatrix}, \quad
		C = \begin{bmatrix}
			1 & 0
		\end{bmatrix}
	\]

	К тому же известны параметры времени переходного процесса $t_n$ по $5\%$-критерию и максимального перерегулирования $\overline{\sigma}$:
	\[
		t_n = 0.9, \quad \overline{\sigma} = 0
	\]

	Эталонная модель задается следующей системой:
	\[
	\begin{cases}
		\dot{x}_M(t) = A_M x_M(t) + B_M g(t), \\
		y_M(t) = C_M x_M(t)
	\end{cases}
	\]

	Здесь $g(t)$ - вход модели, а матрицы $A_M$, $B_M$, $C_M$ имеют вид:
	\[
		A_M = \begin{bmatrix}
			0 & 1 \\
			-a_{M0} & -a_{M1}
		\end{bmatrix}, \quad
		B_M = \begin{bmatrix}
			0 \\
			a_{M0}
		\end{bmatrix}, \quad
		C_M = \begin{bmatrix}
			1 & 0
		\end{bmatrix}
	\]

	Запишем стандартный характеристический полином для двумерного случая $n = 2$ и нулевого перерегулирования:
	\[
		s^2 + a_{M1} s + a_{M0} = s^2 + 2 \omega_n s + \omega_n^2, \quad \omega_n = \frac{4.8}{t_n}
	\]
	
	Откуда можно найти коэффициенты эталонной модели:
	\[
		a_{M0} = \omega_n^2 = 11 \frac{1}{9}, \quad
		a_{M1} = 2 \omega_n = 6 \frac{2}{3}
	\]
	
	Отлично! Теперь построим графики переходного процесса модели при единичном воздействии $g(t) = 1(t)$ и убедимся, что они соответствуют заданным параметрам времени переходного процесса $t_n$ и перерегулирования $\overline{\sigma}$. Результат изображен на рисунке \ref{fig:model_step_response}.
	\begin{figure}[h!]
		\centering
		\includegraphics[width=0.9\textwidth]{images/etalon_response.png}
		\caption{Переходный процесс эталонной модели}
		\label{fig:model_step_response}
	\end{figure}

	Можем видеть, что всё соответствует требованиям - можем использовать эталонную модель для синтеза необходимого управления.

	\section{Формирование адаптивного управления}

	Адаптивное управление представляет собой метод, позволяющий системе автоматически подстраиваться под изменения во внешней среде и внутреннем состоянии. В нашем случае мы будем использовать эталонную модель для реализации адаптивного управления многомерным объектом.

	Для начала определим структуру адаптивного контроллера. Он будет состоять из следующих компонентов:
	\begin{itemize}
		\item \textbf{Модель объекта} - эталонная модель, полученная на предыдущем этапе.
		\item \textbf{Детектор ошибок} - система, отслеживающая отклонения реального поведения объекта от эталонного.
		\item \textbf{Адаптивный закон управления} - алгоритм, который на основе информации от детектора ошибок корректирует параметры управления.
	\end{itemize}

	В качестве первого шага реализуем детектор ошибок, который будет сравнивать выходные сигналы реального объекта и эталонной модели. Для этого нам потребуется определить функцию ошибки:
	\[
		e(t) = y(t) - y_M(t)
	\]
	где $y(t)$ - выход реального объекта, $y_M(t)$ - выход эталонной модели.

	Далее, на основе функции ошибки, мы можем разработать адаптивный закон управления. Например, можно использовать пропорциональный контроллер с учетом ошибки:
	\[
		u(t) = K_p e(t)
	\]
	где $K_p$ - коэффициент пропорциональности, который может быть адаптирован в процессе работы системы.

	Таким образом, мы получаем основу для реализации адаптивного управления многомерным объектом на основе эталонной модели. В следующем разделе мы рассмотрим конкретные примеры применения данного подхода.
	
\end{document}