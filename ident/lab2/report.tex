\documentclass[a4paper,hidelinks,14pt]{extarticle}

\usepackage[utf8]{inputenc}
\usepackage[T2A]{fontenc}
\usepackage[english, russian]{babel}
\usepackage{lipsum}
\usepackage{amsmath}
\usepackage{amssymb}
\usepackage{amsfonts}
\usepackage{mathtools}
\usepackage{datetime}
\usepackage[pdftex]{graphicx}
\usepackage{indentfirst}
\usepackage{asymptote}
\usepackage{systeme}
\usepackage[dvipsnames]{xcolor}
\usepackage{lastpage}
\usepackage{fancybox,fancyhdr}
\usepackage{hyperref}
\usepackage[font={small,it}]{caption}
\fancyhead[L]{ЛР №2}
\fancyhead[C]{}
\fancyhead[R]{\textit{Динамические методы идентификации}}
\fancyfoot[L]{}
\fancyfoot[C]{\thepage\space}
\fancyfoot[R]{}
\pagestyle{fancy}
\newcommand{\gt}{\textgreater}
\newcommand{\lt}{\textless}
\usepackage{listings}
\usepackage{xcolor}
\lstset{
    basicstyle=\ttfamily\small,
    keywordstyle=\color{blue},
    commentstyle=\color{gray},
    stringstyle=\color{red},
    numbers=left,
    numberstyle=\color[gray]{0.7}\ttfamily\small,
    stepnumber=1,
    numbersep=8pt,
    frame=single,
    showstringspaces=false,
    tabsize=4,
    breaklines=true
}
\usepackage{subcaption}

\begin{document}
	\begin{titlepage}
		\setlength{\parindent}{0ex}
		
		\begin{center}
			\textsc{
				\vspace{1ex}
                Научно-исследовательский университет ИТМО \\
				\vspace{0.5ex}
				Факультет систем управления и робототехники \\
				\vspace{0.5ex}
			}
		\end{center}
		
		\vspace{40mm}
		
		\begin{center}
			Отчет по лабораторной работе №2\\
			\textbf{Динамические методы идентификации}\\
			Вариант 23
		\end{center}
		
		\vspace{40mm}
		
		\begin{minipage}{.45\linewidth}
			Выполнили студенты
            \\
			\\
			\\[5mm]
			Преподаватель
		\end{minipage}
		\hfill
		\begin{minipage}{.52\linewidth}
			\begin{flushright}
				Мовчан Игорь Евгеньевич
				\\
				Соколов Дмитрий Алексеевич
				\\
				Тенишев Алексей Николаевич
				\\[5mm]
				Ведяков Алексей Алексеевич
			\end{flushright}
		\end{minipage}
		
		\vfill
		\begin{center}
			Санкт-Петербург
			\\
			2025
		\end{center}
		
	\end{titlepage}

	\tableofcontents
	\clearpage
	
	\section{Дискретная система первого порядка}
	\subsection{Задание системы}
	Зададим дискретную систему с выходом $y(k)$ через передаточную функцию и вход $u(t) = \sin (\omega t)$, который по ходу дискретизируем:
	\[
		y(k) = W(z) u(k), \quad W(z) = \frac{b}{z + a}, \quad u(k) = \sin (\omega k T_d)
	\]

	Здесь переменная $T_d = 0.1$ отвечает за интервал дискретизации, $a$ и $b$ - параметры модели, $\omega$ - частота гармонического входа $u(t)$. Значения брались из варианта, в нашем случае это:
	\[
		a = 0.97, \quad b = 2.8, \quad \omega = 5.65
	\]

	Проведем моделирование входа $u(k)$ и выхода $y(k)$ при выбранных параметрах системы. Результат представлен на рисунке \ref{fig:model_response}.
	\begin{figure}[h!]
		\centering
		\includegraphics[width=0.825\textwidth]{images/task1.png}
		\caption{Дискретизированный вход и выход системы}
		\label{fig:model_response}
	\end{figure}

	Для дальнейшей работы к тому же необходимо перевести систему в форму линейной регрессии, для нашего случая это:
	\[
		y(k) = \phi^T(k) \theta = \begin{bmatrix}
			-y(k-1) & u(k)
		\end{bmatrix} \begin{bmatrix}
			a \\
			b
		\end{bmatrix} = -a y(k-1) + b u(k)
	\]

	\subsection{Идентификации с нормировкой шага}
	Попробуем теперь, зная только $y(k)$ и $u(k)$, оценить \textit{постоянные} параметры системы $\theta = \begin{bmatrix}
		a & b
	\end{bmatrix}^T$, используя градиентный алгоритм:
	\begin{equation}
		\label{extended}
		\hat{\theta}(k) = \hat{\theta}(k-1) - \gamma \nabla_{\hat{\theta}} J_{\text{SE}}(k) = \hat{\theta}(k-1) + \gamma \frac{\phi(k) e^0(k)}{1 + \gamma \phi^T(k)\phi(k)}	
	\end{equation}
	
	В методе $e^0(k) := y(k) - \phi^T(k)\hat{\theta}(k-1)$ - ошибка оценивания $y(k)$ со <<знаниями>> о параметре на предыдущем шаге, а минимизируемый критерий качества основан на квадрате ошибки:
	\[
		J_{\text{SE}}(k) := \frac{e^2(k)}{2} = \frac{(y(k) - \phi^T \hat{\theta}(k))^2}{2}
	\]

	Знаменатель, присутствующий во втором слагаемом алгоритма \ref{extended}, служит некой нормировкой шага и позволяет стабильнее оценивать параметры дискретной системы.

	Итак, проведем численное моделирование процесса идентификации при значениях $\gamma = 1$, $\gamma = 3$ и $\gamma = 10$. На рисунках \ref{fig:gamma1}-\ref{fig:all_gamma_error} приведены результаты работы алгоритма \ref{extended} с данными $\gamma$.
	\begin{figure}
		\centering
		\includegraphics[width=0.825\textwidth]{images/task1_1.png}
		\caption{Процесс идентификации параметров при значении $\gamma = 1$}
		\label{fig:gamma1}
	\end{figure}
	\begin{figure}
		\centering
		\includegraphics[width=0.825\textwidth]{images/task1_2.png}
		\caption{Процесс идентификации параметров при значении $\gamma = 3$}
		\label{fig:gamma3}
	\end{figure}
	\begin{figure}
		\centering
		\includegraphics[width=0.825\textwidth]{images/task1_3.png}
		\caption{Процесс идентификации параметров при значении $\gamma = 10$}
		\label{fig:gamma10}
	\end{figure}
	\begin{figure}
		\centering
		\includegraphics[width=0.825\textwidth]{images/task1_all_gammas.png}
		\caption{Сравнение процессов идентификации при различных $\gamma$}
		\label{fig:all_gamma}
	\end{figure}
	\begin{figure}
		\centering
		\includegraphics[width=0.825\textwidth]{images/task1_error_norms.png}
		\caption{Сравнение норм параметрических ошибок при различных $\gamma$}
		\label{fig:all_gamma_error}
	\end{figure}

	Можем видеть, что с увеличением $\gamma$ повышается скорость сходимости адаптации, но одновременно ухудшается и стабильность - появляются более резкие скачки, падает качество.

	\subsection{Упрощенный алгоритм идентификации}
	Воспользуемся также упрощенным градиентным алгоритмом:
	\begin{equation}
		\label{simpled}
		\hat{\theta}(k) = \hat{\theta}(k-1) + \gamma \phi(k) e^0(k)
	\end{equation}

	Здесь исчезает нормировка по величине $\gamma$ и $\phi(k)$, поэтому при больших значениях $\gamma$ система идентификации может стать неустойчивой. Проверим это, проведя моделирование при $\gamma = 1$ и $\gamma = 10$. Результаты представлены на рисунках \ref{fig:gamma0.5_sim} и \ref{fig:gamma10_sim}.
	\begin{figure}
		\centering
		\includegraphics[width=0.825\textwidth]{images/task1_4.png}
		\caption{Идентификации при значении $\gamma = 0.5$ и упрощенном алгоритме}
		\label{fig:gamma0.5_sim}
	\end{figure}
	\begin{figure}
		\centering
		\includegraphics[width=0.825\textwidth]{images/task1_5.png}
		\caption{Идентификации при значении $\gamma = 10$ и упрощенном алгоритме}
		\label{fig:gamma10_sim}
	\end{figure}

	Сильное увеличение $\gamma$ при использовании алгоритма идентификации \ref{simpled}, как и было сказано, приводит к неустойчивости оценок.
	
	Также в сравнении с предыдущими результатами для идентификации с нормировкой на $\gamma$ процесс сходимости менее стабилен - так, при оценке параметра $a$ присутствуют резкие скачки на величину выше самого $a$. 
	
	Таким образом, для получения качественных результатов при использовании упрощенного метода необходимо выбирать мелкие $\gamma$, что, в свою очередь, увеличивает ещё и время сходимости.
	
	Здесь важно, что за все вышесказанные проблемы мы получаем более легкое вычисление шагов градиента, что в большинстве случаев, пожалуй, не так уж и полезно :)

	\section{Дискретная система второго порядка}
	\subsection{Задание системы}
	Зададимся теперь дискретной системой второго порядка с выходом $y(k)$ и передаточной функцией $W(z)$:
	\[
		y(k) = W(z) u(k), \quad W(z) = \frac{b}{z^2 + a_1 z + a_2}
	\]

	Также примем интервал дискретизации $T_d = 0.1$, а параметры системы $a_1$, $a_2$ и $b$ возьмем из варианта:
	\[
		a_1 = -1.89, \quad a_2 = 0.8928, \quad b = 2.5
	\]

	И переведем систему в форму линейной регрессии:
	\[
		y(k) = \phi^T(k) \theta = \begin{bmatrix}
			-y(k-1) & y(k-2) & u(k)
		\end{bmatrix} \begin{bmatrix}
			a_1 \\
			a_2 \\
			b
		\end{bmatrix} =
	\]
	\[
		= -a_1 y(k-1) -a_2 y(k-2) + b u(k)
	\]

	\subsection{Простой гармонический вход}
	Посмотрим, как пройдет идентификация параметров при различных гармонических входах $u(t)$. Для начала возьмем уже использовавшийся нами ранее синус с заданной из варианта частотой $\omega$:
	\[
		u(t) = \sin(\omega t), \quad u(k) = \sin(\omega k T_d), \quad \omega = 45.86
	\]

	Моделирование входа $u(k)$ и выхода $y(k)$ при выбранных параметрах системы приведено на рисунке \ref{fig:model_response2}.
	\begin{figure}
		\centering
		\includegraphics[width=0.825\textwidth]{images/task2.png}
		\caption{Вход и выход дискретной системы второго порядка при $u = \sin(\omega t)$}
		\label{fig:model_response2}
	\end{figure}

	Для оценки параметров будем использовать градиентный алгоритм \ref{extended} при $\gamma = 1$, так как он показал себя в предыдущем случае лучшим образом. Результаты численного моделирования процесса идентификации представлены на рисунке \ref{fig:task2_1}.
	\begin{figure}
		\centering
		\includegraphics[width=0.825\textwidth]{images/task2_1.png}
		\caption{Идентификации параметров при входе $u = \sin(\omega t)$}
		\label{fig:task2_1}
	\end{figure}

	Заметим, что увеличение частоты входа привело к более частым колебаниям и в оценках $\hat{\theta}$ набора параметров $\theta = \begin{bmatrix}
		a_1 & a_2 & b
	\end{bmatrix}^T$. При всем этом оценка сошлась достаточно точно, хоть и <<застряла>> на некотором уровне от истинных значений из-за малого $\gamma$.

	\subsection{Двухсоставной гармонический вход}
	Изменим вход, добавив синус меньшей амплитуды и частоты:
	\[
		u(t) = \sin(\omega t) + 0.2\sin(0.5 \omega t), \quad u(k) = \sin(\omega k T_d) + 0.2\sin(0.5 \omega k T_d)
	\]

	Аналогично проведем моделирование входа $u(k)$ и выхода $y(k)$, а также применим идентификацию параметров при $\gamma = 1$. На рисунках \ref{fig:task2_2}-\ref{fig:task2_4} изображены все соответствующие графики.
	\begin{figure}[h!]
		\centering
		\includegraphics[width=0.825\textwidth]{images/task2_2.png}
		\caption{Вход и выход системы второго порядка при $u = \sin(\omega t) + 0.2 \sin(0.5 \omega t)$}
		\label{fig:task2_2}
	\end{figure}
	\begin{figure}
		\centering
		\includegraphics[width=0.825\textwidth]{images/task2_3.png}
		\caption{Идентификации параметров при входе $u = \sin(\omega t) + 0.2 \sin(0.5 \omega t)$}
		\label{fig:task2_3}
	\end{figure}
	\begin{figure}
		\centering
		\includegraphics[width=0.825\textwidth]{images/task2_4.png}
		\caption{Сравнение процессов при входах $u = \sin(\omega t)$ и $u = \sin(\omega t) + 0.2 \sin(0.5 \omega t)$}
		\label{fig:task2_4}
	\end{figure}
	\begin{figure}
		\centering
		\includegraphics[width=0.825\textwidth]{images/task2_4_errors.png}
		\caption{Сравнение норм параметрических ошибок при различных входах}
		\label{fig:task2_4_errors}
	\end{figure}

	Для наглядности на рисунках \ref{fig:task2_4} и \ref{fig:task2_4_errors} также представлено сравнение процессов идентификации при входах, использующих только одну гармонику - $u(t) = \sin(\omega t)$, и две - $u(t) = \sin(\omega t) + 0.2 \sin(0.5 \omega t)$.

	Видим, что добавление дополнительной гармоники в общем оставило динамику той же, однако \textit{немного} ускорило адаптацию за счёт обогащения входного сигнала слабым синусом, обладающим меньшей амплитудой и частотой.
	
	В общем случае добавление такого рода гармоники уменьшает вырожденность входа и улучшает способность алгоритмов различать вклад отдельных параметров $a_1$, $a_2$ и $b$.

	\section{Непрерывная система}
	\subsection{Задание системы}
	Пусть дана непрерывная линейная система первого порядка с гармоническим входом $u(t)$ и выходом $y(t)$:
	\[
		y(t) = \frac{b}{p + a}u(t), \quad u(t) = \sin(\omega t)
	\]

	Перепишем модель в форме линейной регрессии, считая производную $\dot{y}(t)$ доступной к измерению:
	\[
		\dot{y}(t) = \begin{bmatrix}
			-y(t) & u(t)
		\end{bmatrix}
		\begin{bmatrix}
			a \\
			b
		\end{bmatrix} = -a y(t) + b u(t)
	\]

	Также примем параметры системы и входа $a$, $b$ и $\omega$ из варианта:
	\[
		a = 0.5, \quad b = 3.2, \quad \omega = 5.65
	\]

	Отлично, всё задано! На рисунке изображено моделирование входа $u(t)$ и выхода $y(t)$ данной непрерывной системы.
	\begin{figure}[h!]
		\centering
		\includegraphics[width=0.825\textwidth]{images/task3.png}
		\caption{Вход и выход непрерывной системы}
		\label{fig:task3}
	\end{figure}

	\subsection{Непрерывный алгоритм идентификации}
	Теперь проведем идентификацию параметров $a$ и $b$ на основе градиентного алгоритма для непрерывных систем:
	\begin{equation}
		\label{cont}
		\hat{\theta} = -\gamma \nabla_{\hat{\theta}} J_{\text{SE}}(t) = \gamma \phi e(t)
	\end{equation}

	Здесь $e(t) := y(t) - \phi^T(t) \hat{\theta}(t)$ - ошибка на выход $y(t)$.

	Численное моделирование процессов идентификации параметров $a$ и $b$ при значениях $\gamma = 1$, $\gamma = 3$ и $\gamma = 10$ и заданном алгоритме \ref{cont}, а также их сравнение представлено на рисунках \ref{fig:task3_1}-\ref{fig:task3_errors}.
	\begin{figure}[h!]
		\centering
		\includegraphics[width=0.825\textwidth]{images/task3_1.png}
		\caption{Идентификации параметров непрерывной системы при $\gamma = 1$}
		\label{fig:task3_1}
	\end{figure}
	\begin{figure}
		\centering
		\includegraphics[width=0.825\textwidth]{images/task3_2.png}
		\caption{Идентификации параметров непрерывной системы при $\gamma = 3$}
		\label{fig:task3_3}
	\end{figure}
	\begin{figure}
		\centering
		\includegraphics[width=0.825\textwidth]{images/task3_3.png}
		\caption{Идентификации параметров непрерывной системы при $\gamma = 10$}
		\label{fig:task3_10}
	\end{figure}
	\begin{figure}
		\centering
		\includegraphics[width=0.825\textwidth]{images/task3_4.png}
		\caption{Сравнение процессов непрерывной системы при различных $\gamma$}
		\label{fig:task3_4}
	\end{figure}
	\begin{figure}
		\centering
		\includegraphics[width=0.825\textwidth]{images/task3_errors.png}
		\caption{Сравнение норм ошибок в непрерывной системе при различных $\gamma$}
		\label{fig:task3_errors}
	\end{figure}

	Можем видеть, что увеличение $\gamma$, как и для дискретной системы, ведет к ускорению процессов адаптации, однако снижает их качество и стабильность - например, на рисунке \ref{fig:task3_10} появляются некоторые колебания возле истинных значений параметров $a$ и $b$ системы.

	\section{Выводы}
	В ходе лабораторной работы исследованы градиентные алгоритмы идентификации для дискретных и непрерывных систем. Установлено, что для дискретных моделей использование алгоритма с нормировкой шага критически важно для обеспечения устойчивости, тогда как увеличение коэффициента адаптации $\gamma$ ускоряет сходимость ценой появления возможных нестабильностей.
	
	Эксперименты с системой второго порядка подтвердили, что обогащение спектра входного сигнала дополнительными гармониками немного повышает скорость адаптации.
	
	В конечном итоге, получено, что эффективность идентификации определяется величиной шага адаптации и насыщенностью входного сигнала.


	





	

	



	






	
\end{document}